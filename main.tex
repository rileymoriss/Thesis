\documentclass[12pt, openany]{book}

%\documentclass[12pt]{report}
\usepackage[utf8]{inputenc}
\usepackage[a4paper]{geometry}
\usepackage{graphicx}
\usepackage{tikz-cd}
\usepackage{mathrsfs}
\usepackage{amsfonts}
\usepackage{soul}
\usepackage{ stmaryrd }
\usepackage{amsthm}
\usepackage{amsmath}
\usepackage{amssymb}
\usepackage{mathtools}
\usepackage{extarrows}
\usepackage{interval}
\usepackage{bbm}
\usepackage[utf8]{inputenc}
\usepackage[english]{babel}
\usepackage{romannum}
\usepackage{setspace}
\usepackage{listings}
\usepackage{pdfpages}
\usepackage{hyperref}
\usepackage{enumitem}
\usepackage{ragged2e}
\title{A Paraphrase of a Paraphrase}
\author{Riley Moriss}
		\author{Riley Moriss}
	
		%images
		\usepackage{graphicx}
		\usepackage{quiver} %Tikz diagram generator
		\usetikzlibrary{fit}
		
		
		%Bibliography
		\usepackage{hyperref} %Content is clickable
		
		%Math
		\usepackage{amsmath}
		
		%Formatting
		\usepackage{comment} %comment out code blocks
		\usepackage{csquotes} %Quote environment
		\usepackage{amsfonts}
		\usepackage{mathrsfs} %Script fonts

		%Todos	
		\usepackage[colorinlistoftodos,prependcaption,textsize=scriptsize, color=red!20, shadow]{todonotes}
			
		%Math shortcuts
		%Sets
		\newcommand{\R}{\ensuremath{\mathbb{R}}\phantom{ }}
		\newcommand{\C}{\ensuremath{\mathbb{C}}\phantom{ }}
		\newcommand{\Z}{\ensuremath{\mathbb{Z}}\phantom{ }}
		\newcommand{\N}{\ensuremath{\mathbb{N}}\phantom{ }}
		\newcommand{\Q}{\ensuremath{\mathbb{Q}}\phantom{ }}
		\newcommand{\F}{\ensuremath{\mathbb{F}}\phantom{ }}
		\newcommand{\G}{\ensuremath{\mathbb{G}}\phantom{ }}

		%Symbols
		\newcommand{\hilb}{\ensuremath{\mathfrak{h}}\phantom{ }}
		\newcommand{\tensor}{\ensuremath{\otimes}}
		\newcommand{\inner}[1]{\ensuremath{\langle #1\rangle}}
		\newcommand{\norm}[1]{\lVert #1 \rVert}
		\newcommand{\inv}{^{-1}}
		\newcommand{\model}[1]{\llbracket #1\rrbracket}
		\newcommand{\Par}{\rotatebox[origin=c]{180}{\&}}
		\newcommand{\Dif}[2]{\ensuremath{\frac{\partial #1}{\partial #2}}}
		\newcommand{\comp}{\circ}
		\newcommand*{\defeq}{\hspace{2mm}\mathrel{\vcenter{\baselineskip0.5ex \lineskiplimit0pt
					\hbox{\scriptsize.}\hbox{\scriptsize.}}}%
			=\hspace{2mm}}

		%Operators
		\DeclareMathOperator{\Ind}{Ind}
		\DeclareMathOperator{\Spec}{Spec}
		\DeclareMathOperator{\Res}{Res}
		\DeclareMathOperator{\Hom}{Hom}
		\DeclareMathOperator{\Maps}{Maps}
		\DeclareMathOperator{\Aut}{Aut}
		\usepackage{mathtools}
\DeclarePairedDelimiter{\ceil}{\lceil}{\rceil}
\DeclarePairedDelimiter{\floor}{\lfloor}{\rfloor}

		\DeclareMathOperator{\dif}{d}
		\newcommand{\A}{\ensuremath{\mathbb{A}}\phantom{ }}
		\renewcommand{\P}{\ensuremath{\mathbb{P}}\phantom{ }}
		\newcommand{\Struc}{\ensuremath{\mathcal{O}}\phantom{ }}
		\newcommand{\Gcov}{\ensuremath{\mathbf{G}}\phantom{ }}
		
		\DeclareMathOperator{\Sp}{Sp}
		\DeclareMathOperator{\SL}{SL}
		\DeclareMathOperator{\GL}{GL}
		\DeclareMathOperator{\Mp}{Mp}

		%Fonts
		\newcommand{\Cal}[1]{\ensuremath{\mathcal{#1}}}
		\newcommand{\Cat}[1]{\ensuremath{\mathscr{#1}}}
		%Format
		\newcommand{\gap}{\vspace{4mm}}
		\newcommand\scalemath[2]{\scalebox{#1}{\mbox{\ensuremath{\displaystyle #2}}}}
		\usepackage{ragged2e}


		
		%Environments
		\usepackage{amsthm} %No numbering
		
		\newtheorem{theorem}{Theorem}[chapter]
		\newtheorem{lemma}[theorem]{Lemma}
		\newtheorem{corollary}[theorem]{Corollary}
		\newtheorem{proposition}[theorem]{Proposition}

		%\newtheorem{example}[theorem]{Example}
		\newtheorem{problem}[theorem]{Problem}

		\newtheorem{Lemma}[theorem]{Lemma}
		\newtheorem{Theorem}[theorem]{Theorem}
		\newtheorem{Remark}[theorem]{Remark}
		\newtheorem{Exercise}[theorem]{Exercise}
		\newtheorem{Notation}[theorem]{Notation}
		\newtheorem{Definition}[theorem]{Definition}
		\newtheorem{example}[theorem]{Example}	
		\newtheorem{cor}[theorem]{Corollary}
		\newtheorem{Prop}[theorem]{Proposition}
		\newtheorem{Conj}[theorem]{Conjecture}
		
		\theoremstyle{remark}
		\newtheorem{remark}[theorem]{Remark}
		\theoremstyle{definition}
		\newtheorem{definition}[theorem]{Definition}


		\usepackage{framed}
		%\usepackage{mdframed}
		\newcommand{\proofbar}[1]{
			\begin{leftbar}
				\textit{\textbf{Proof.}} 	#1
			\end{leftbar}
			}

		%Highlighting
		\newcommand{\notice}[1]{\hl{\textbf{\textit{#1}}}}
		\DeclareRobustCommand{\hlcyan}[1]{{\sethlcolor{cyan}\hl{#1}}}
		\DeclareRobustCommand{\hlgreen}[1]{{\sethlcolor{green}\hl{#1}}}
			
			
			
			
			
%No title page.
\makeatletter
\newcommand*{\toccontents}{\@starttoc{toc}}
\makeatother

\let\oldphi\phi 
\let\phi\varphi 
\let\varphi\oldphi

% As an example:
%   \intof[0]{1}{x^2}{x}
\newcommand*\intof[4][]{\int^{#1}_{#2}#3\,d#4}

% As an example:
%   \setof[\Big]{ n\in\N \st n < \binom n 2 }
\newcommand\setof[2][]{\mathord{\mathopen#1\{\,#2\,\mathclose#1\}}}

%\usepackage{multicol}
%\usepackage{geometry}
%\usepackage{soul}
%\usepackage{ stmaryrd }
%\geometry{ margin=3cm}

%%%%%%%%%%%%%%%%%%%%%%%%%%%%%%%%%%%%%%%%%%%%%%%%%%%%%%%%%%%%%%%%%%%%%%
\setcounter{tocdepth}{1}%%%%% Stops subsections in TOC

\numberwithin{equation}{section}
\renewcommand{\baselinestretch}{1.25}
\geometry{
 left=30mm,
 right=30mm,
 top=30mm,
 bottom=30mm,
 includefoot
}
\allowdisplaybreaks
\geometry{ margin=3cm}
\usepackage{cite}

%%%%%%%%%%%%%% 
%Type Writer Font
%\usepackage{courier}
%\renewcommand*\familydefault{\ttdefault} %% Only if the base font of the document is to be typewriter style
%\usepackage[T1]{fontenc}

%%%%%%%%%%%%%%%%%%%%%%%%%%%%%%%%%%%%%%%%%%%%%%%%%%%%%%%%% BEGIN
\begin{document}

    \begin{titlepage}
    \begin{center}
    \vspace*{1cm}
    \huge
    \textbf{A Paraphrase of a Paraphrase}\\
    \vspace{2cm}
    \Large
    \text{Riley Moriss}\\
    \vspace{0.5cm}
    \text{Supervisor: Dr. Chenyan Wu }\\    
 
    A thesis submitted in partial fulfillment of the\\
    requirements for the degree of\\
    Master of Science\\
    in the\\
    School of Mathematics and Statistics\\
    at\\
    The University of Melbourne\\
    \vspace{1,5cm}
    October 2024
\end{center}
\end{titlepage}

\pagebreak


\pagenumbering{roman}


\pagenumbering{roman}

\begin{center}
	\textbf{Abstract}
\end{center}
The classical Eisenstein series is the first concrete example of a modular form. Their automorphic analogue plays a similarly central role to the theory of automorphic forms and automorphic representations. In fact it was whilst computing the constant terms of the Eisenstein series that Langlands formed his famous conjectures. In 2013 Dihua Jiang, Baiying Liu and Lei Zhang in their paper "Poles of certain residual Eisenstein series of classical groups" \cite{jiangPolesCertainResidual2013} prove some theorems about the possible locations of poles of Eisenstein series (as the name suggests) associated to the so called classical matrix groups. Our thesis builds up the theory of automorphic representations in order to explain these results.



\chapter*{Acknowledgements}
I need to thank Chenyan Wu,  Alex Ghitza,  Chengjing Zhang, Bowan Hafey, Oliver, Fei and Miscellaneous lecturers
for the math that they taught me. 

Thanks to my bros for being bros. Thanks to Fei Peng for the thesis template. Arun Ram for helping track down the name for parabolic. Thank you to my proof readers Yuhan, Chengjing, Kwan, Barb, Adam and Donna.

\chapter*{Introduction}
\section*{Motivation}
The goal of this thesis is to exposit some of the results in \cite{jiangPolesCertainResidual2013}. We aim our exposition at the other masters students in our cohort.
To explain the results on poles of Eisenstein series to students in other disciplines there is a fair amount of background to be covered.

Here we attempt to put down what we understand of the ``big picture''. It could be argued that we spent too much time trying to understand the motivation for the results in \cite{jiangPolesCertainResidual2013} and not enough time on the results themselves and so this section is to ensure that time was not (very) wasted. 

We should point out that there are many surveys on the Langlands program, class field theory and modern topics in number theory that this introduction is indebted to. Some exemplars are ----------.

\subsection{From Ancient to Modern}
We follow the wonderful exposition in \cite{weinsteinReciprocityLawsGalois2015}. A problem that Euclid could have understood is ``which positive integers are the sum of two squares''. In 1640 Fermat answered this question, he first reduces the question to when is a prime the sum of two squares. Thus the problem is immediately reformulated as a problem about congruences mod a prime \(p\), ``when does there exist a solution to \(a^2  +b^2 \equiv 0 \;\;(\text{mod }p) \)'', or whats the same, by dividing out \(b^2\), ``when is there a solution to \(x^2 + 1 \equiv 0 \;\;(\text{mod }p)\)''. The famously has the solution 
\begin{Theorem}
	Let \(p\) be an odd prime. Then \(x^2 + 1 \equiv 0 \;\;(\text{mod }p)\) has a solution if and only if \(p\equiv 1 \;\;(\text{mod }4) \).
\end{Theorem}

Recall the Legendre symbol, for \(p, q\) odd and non-equal primes we have 
\[\left(\frac{q}{p}\right) \defeq \begin{cases}
	1 , & \text{there is a solution to }x^2 - q \equiv 0 \;\;(\text{mod }p)\\
	-1 , & \text{else.}
\end{cases}.\]
Then the theorem of Fermat was ``upgraded'' by Gauss to his famous reciprocity law.

\begin{Theorem}
	For \(p, q\) odd and non-equal primes,
	\[\left(\frac{p}{q}\right)\left(\frac{q}{p}\right) = (-1)^{\frac{(p-1)(q-1)}{2}}.\]
\end{Theorem}

Having a solution mod a prime is the same as asking whether the polynomial splits mod that prime. The natural question is then: \textbf{Q1.} Given a monic irreducible polynomial with integral coefficients can we determine by congruences whether it splits mod a prime. Gausses reciprocity is a complete solution to this problem for polynomials of the form \(f(x) = x^2 - q\) for \(q\) odd prime. 
\begin{remark}
	The odd limitation is for brevity here and of course can be lifted. Moreover the solution for primes can be leveraged for a solution for other integers. 
\end{remark}
Recall that if \(f(x)\in \Z[x]\) is monic and irreducible then there is a unique minimal field \(F\) in which it factors as linear polynomials, called the splitting field. The Galois group of \(f(x)\) is then defined to be \(\mathrm{Gal}(F/\Q)\). Class field theory is a solution to problem \textbf{Q1.} when this Galois group is \textit{Abelian}. To explain we need to introduce the standard algebraic number theory setup.

Let \(\Q \subseteq K\) be an extension of number fields, with respective rings of integers \(\Z \subseteq \mathcal{O}_K\) and let \(p\) be a prime in \(\Z\) hence \((p)\) is a prime ideal of \(\Z\) and let 
\[\mathcal{O}_K(p) = \prod_i \mathfrak{P}_i^{e_i},\]
be the prime decomposition in \(\mathcal{O}_K\). Then \((p)\) \textbf{splits} in \(\mathcal{O}_K\) if for every \(i\) we  have \(e_i = 1\) (this is being unramified) and \(\mathcal{O}_K/\mathfrak{P}_i \cong \Z/(p)\). The splitting of primes is related to the splitting of polynomials by the following theorem
\begin{Theorem}[\cite{langAlgebraicNumberTheory1994}, Prop. 26]
	If \(f\in \Z[x]\) monic and irreducible and \(f(\alpha) = 0\) then for \(K = \Q(\alpha)\) we have with finitely many exceptions that \(f\) is split mod \(p\) if and only if \((p)\) splits in \(\mathcal{O}_K\).
\end{Theorem}
\begin{example}
	
\end{example}
So to answer \textbf{Q1.} we now want to solve by congruences when prime ideals split. Every field extension \(K/L\) has a Galois closure, that is an extension \(L'/K\) of minimal degree such that \(L\subseteq L'\) and \(L'\) is Galois over \(K\). 
\begin{Lemma}
	A prime ideal of \(\mathcal{O}_K\) is split in \(\mathcal{O}_L\) if and only if it is split in \(\mathcal{O}_{L'}\).
\end{Lemma}
Thus we lose nothing by considering only Galois extensions of fields. Thus we have ``the main theorem'' of class field theory:
\begin{Theorem}[\cite{weinsteinReciprocityLawsGalois2015}, Thm. 3.2.1]\label{thm:reciprocity}
	Let \(K/\Q\) be an abelian and Galois extension. There is an ideal \(\mathfrak{f} = (m)\subseteq \mathcal{O}_\Q = \Z\) such that for a prime \(p\in \Z\) the ideal \((p)\) is split in \(\mathcal{O}_K\) if \(p\equiv 1 \;\; (\text{mod }m)\)
\end{Theorem} 
Thus we have a solution to the splitting of primes via congruence relations. 

\begin{example}
	
\end{example}

This we hope motivates class field theory, now we will follow \cite{conradHISTORYCLASSFIELD} for some more detail on class field theory. Class field theory is over a hundred years old with a storied past and many incarnations of the main theorem above. To see the Langlands program as a generalisation of this theory we want to trace the development to where Langlands picked up. 

Class field theory begins with Kronecker in 1853, who constructed an extension of number fields \(K'/K\) whose Galois group is isomorphic to the ideal class group of \(K\), a so called (by Weber) ``class field'' for \(K\). Kronecker would go on to make several conjectures that would form the heart of class field theory, for instance he conjectured that a Galois extension of \Q is determined by the primes of \Z that split over that extension. In fact this was solved by Bauer in 1916
\begin{Theorem}(Bauer)
	Let \(L_1, L_2\) be finite extensions of a number field \(K\), then \(L_1 = L_2\) if and only if the primes of \(\mathcal{O}_K\) that split in \(\mathcal{O}_{L_1}\) is equal to the set of primes that split in \(\mathcal{O}_{L_2}\).
\end{Theorem}
However there was no systematic way of finding \textit{which} primes split over the extension. Takagi was to supply something very close to theorem \ref{thm:reciprocity} in 1920 and it was to be made even more explicit finally by Artin in 1927. Thus \textit{global} class field theory was ``solved'', immediately the natural question was raised, what happens in the \textit{non-abelian} extensions of number fields. The (global) Langlands conjectures (amongst other things) can be viewed as an attempt to answer this question. 

Another direction that people were interested in was extensions of local fields, as opposed to number fields. It was Hilbert who introduced in 1897 the use of the p-adic numbers, in spirit if not in name, he wrote congruences of arbitrary powers of primes. Let \(\nu\) be a place of \(\Q\), then define the \(\nu\)-adic Hilbert symbol for \(a,b\in \Q^\times\)
\[(a,b)_\nu \defeq \begin{cases}
	1, & a= x^2-by^2 \text{ has a solution in }\Q_\nu\\
	-1, &\text{else}
\end{cases} .\]
\begin{Theorem}[Hilbert's Quadratic Reciprocity]
	For all \(	a,b\in \Q^\times\) 
	\[\prod_\nu (a,b)_\nu = 1.\]
\end{Theorem}
This is equivalent to Gauss's reciprocity law, however much more uniform to state, treating odd and even primes in the same way, and not requiring any co-prime conditions. This moreover treats finite and infinite places uniformly. Building on this work and using Artin reciprocity Hasse, after introducing the p-adic numbers in 1927, proved the first versions of \textit{local} class field theory in 1930, that is reciprocity for extensions of the local fields \(\Q_\nu\). The statements here are too technical for a motivational introduction however replacing all the global fields in the above statements with local fields is not far off. 

Note that the definition and proof of local class field theory \textit{depends logically } on global class field theory. Hasse was able to prove later in 1933 the main results again but without recourse to global class field theory. It lacked the explicit construction of the class fields however which was finally supplied in 1965 by Lubin and Tate.

What remained to do was supply a proof of \textit{global} class field theory from local class field theory. In persuit of this task the machinery of the ideles and adeles was introduced. In this language (part of) \textit{global} class field theory can be restated as 
\begin{Theorem}[\cite{neukirchGlobalClassField1999}, Prop. 1.3]
	Let the ideal class group of a number field \(K\) be denoted \(\mathrm{Cl}_K\). Then there is a surjection
	\[ \A^*/ K^* \xrightarrow{A} \mathrm{Cl}_K \cong \mathrm{Gal}(K'/K).\]
	Where \(K'\) is the class field of \(K\).
\end{Theorem}
If we think about representations of these groups then this surjection gives a relation between characters \(\chi\) of \(\A^*/K^*\) and characters \(\chi'\) of \(\mathrm{Gal}(K'/K)\) by pulling back along \(A\).

% https://q.uiver.app/#q=WzAsMyxbMCwwLCJcXEFeKi9LXioiXSxbMiwwLCJcXG1hdGhybXtDbH1fSyJdLFsxLDIsIlxcQ14qIl0sWzAsMiwiXFxjaGkiLDFdLFsxLDIsIlxcY2hpJyIsMV0sWzAsMSwiQSIsMV1d
\[\begin{tikzcd}[cramped]
	{\A^*/K^*} && {\mathrm{Gal}(K'/K)} \\
	\\
	& {\C^*}
	\arrow["A"{description}, from=1-1, to=1-3]
	\arrow["\chi"{description}, from=1-1, to=3-2]
	\arrow["{\chi'}"{description}, from=1-3, to=3-2]
\end{tikzcd}\]
Thus \(A\) can be thought of as generating a correspondence
\[\{\text{Maps }\A^*/K^*\to \C^*\} \to\{ \text{Maps }\mathrm{Gal}(K'/K)\to \C^*\}. \]
One then observes that this can be rewritten as 
\[\{\text{Maps }\GL(\A)/\GL_1(K)\to \GL_1(\C)\} \to\{ \text{Maps }\mathrm{Gal}(K'/K)\to \GL_1(\C)\} .\]
This suggests the generalisation to 
\[\{\text{Certain reps of }\GL_n(\A)/\GL_n(K)\} \to\{ \text{Certain reps of  }\mathrm{Gal}(\bar{K}/K)\text{ on } \GL_n\}   .\]
But according to Langlands \cite{langlandsREPRESENTATIONTHEORYITS}, who was inspired by the philosophy of Harish-Chandra, we should treat all reductive groups the same, and so Langlands conjectures that for any reductive linear algebraic group \(G\) there is some correspondence
\[\{\text{Certain reps of }G(\A)/G(K)\} \to\{ \text{Certain reps of  }\mathrm{Gal}(\bar{K}/K)\text{ on } G\} .\]


\subsection{Harmonic Analysis}
As we mentioned the work of Langlands was inspired by the work of Harish-Chandra in harmonic analysis of Lie groups. Here we want to say something about the precursors to Langlands work in this respect. 

The story starts with the two Fourier transforms, the one for functions on the circle and functions on the real line. Both appearing in the work of Fourier himself around 1822. 

The circle is a compact topological group and for this we have the generalisation of the Fourier transform given by the Peter-Weyl theorem in 1927.

\subsection{The Work of Langlands}

\subsection{Poles of Residual Eisenstein Series}

\section*{Outline of Content}
Chapter one deals with the generalities of linear algebraic groups, the objects whose representation theory is the subject of discussion. First we define them and then look at the important subgroups that are used in the study of automorphic forms arising on the adelic points of these groups. We focus on the classical groups.

Chapter two deals with automorphic forms. We define automorphic forms in both the Archimedean and adelic places. Finally we give the details of how to view modular forms as automorphic forms. 

Chapter three is a discussion of the concept of the constant term in the Archimedean place. First we define the constant term of an automorphic form (Archimedean) and then we show how it is related to the constant term of the Fourier series of a modular form. Finally we show how the classical Siegel Phi operator can be realised as a constant term.

Chapter four is dedicated to automorphic representations. We define them and specify some important constructions that are needed in the final section.

In chapter five we define adelic Eisenstein series and show how they generalise the classical modular forms also known as Eisenstein series.

Chapter six is dedicated to the constant term in the adelic setting. We first define them and then go through the process of computing them in great detail for Eisenstein series. 

Chapter seven is for defining L-functions, the analytic invariants that are central to the Langlands program. We will give several of the special cases that appear through out history and the literature. 

Finally chapter eight contains some exposition of recent work on the poles of residual Eisenstein series.




\tableofcontents
\pagenumbering{arabic}

\chapter{Classical Groups}
We will recall a small amount of the theory of linear algebraic groups to fix conventions, for a more detailed treatment one should consult the litany of sources on this matter: For a full treatment see \cite{milneAlgebraicGroupsTheory2017}\cite{milneLieAlgebrasAlgebraic}\cite{milneBasicTheoryAffine}\cite{springerLinearAlgebraicGroups1998}. Excellent example computations can also be found in \cite{BuildingsClassicalGroups}\cite{makisumiStructureTheoryReductive}\cite{malleLinearAlgebraicGroups}\cite{NotesClassAlgebraic}. Or for a brief brush up on the main facts consult \cite[I.I.1]{borelAutomorphicFormsRepresentations1979}. The purpose of this section is to treat the classical groups and more specifically \(\Sp_{2n}\) as an example and work out some of the details of the general theory, in order to "get our hands dirty" and have some familiarity with this object, to become fundamental in what follows. Because this theory is made up of simple ideas that can often be obscured by the generality we will make several restrictive assumptions for ease of exposition. 

\section{Definition}
An algebraic group is for us a group scheme that is reduced, of finite type and defined over a field. A linear algebraic group (LAG) is an affine algebraic group.
By \cite[2.3.7(i)]{springerLinearAlgebraicGroups1998} then every LAG is (isomorphic) to a (Zariski) closed subgroup of \(\GL_n\). A group scheme is called linear if it is (isomorphic to) a closed subgroup for \(\GL_n\). Hence every linear algebraic group is a \textit{linear} algebraic group. Moreover it is a basic fact that a Zariski closed sub-scheme of an affine scheme is affine \cite[II.5.T3]{mumfordRedBookVarieties1999}.

As Milne points out \cite[Abstract]{milneAlgebraicGroupsTheory2017} these are supposed to be matrix groups defined by polynomials, which are somehow the natural combinations of symbols that matrix multiplication will lead to. This means that they come with the powerful but cumbersome (for the beginner) technology of algebraic geometry. In particular one must be adept at moving between the following equivalences
\begin{Theorem}[\cite{milneBasicTheoryAffine}, II.6, III.4]
    For k a field then the following categories are equivalent
    \begin{itemize}
        \item Group objects in \(\mathrm{Alg}_k^{opp}\)
        \item Representable (in the category of groups) functors \(\mathrm{Alg}_k \to \mathrm{Group}\)
        \item Group object in the category of affine schemes over \(k\)
        \item Commutative Hopf algebras
    \end{itemize}
\end{Theorem}

The exact groups that an author might mean by classical may vary. Here we will follow \cite[\S 13]{CliffordAlgebrasClassical} or indeed many other places as defining them as the automorphisms of a vector space with a bilinear form. First let V be an F vector space with a bilinear form \(\inner{,}\). An automorphism of this form is a map \(\alpha\in Aut(V)\) such that 
\[\inner{\alpha(x), \alpha(y)} = \inner{x,y}\]
Therefore we can consider the space of automorphisms of \(Aut(V, \inner{,})\). This space, depending on the properties of the bilinear form will define our classical groups. 

If the form is degenerate, \(\forall x,y \;\; \inner{x, y} = 0\) then we define 
\[\GL(V) \defeq Aut(V, \inner{,}) = Aut(V)\]
If the form is non-degenerate and symmetric \(\forall x,y \;\; \inner{x, y} = \inner{y,x}\) then we define
\[\mathrm{O}(V) \defeq Aut(V, \inner{,})\]
Finally if the form is non-degenerate and skew symmetric \(\forall x,y \;\; \inner{x, y} = -\inner{y,x}\) then 
\[\Sp(V) \defeq Aut(V, \inner{,})\]
There are the further classical groups given by the determinant one subgroups, \(\SL(V), \mathrm{SO}(V)\) respectively (\(\Sp(V)\) one can show already implies that the determinant is one). Moreover every \(F\) algebra is an \(F\) vector space and so we have defined a functor from \(F\)-algebras to groups. This is what we will refer to as "classical groups" although we should note the omission of what the unitary groups that  would usually be included.

Another way of motivating the importance of these groups is to consider the classification theory of the split reductive groups. Reductive is a representation theoretic condition that means, worse than semi-simple but still manageable, for more detail see \cite[22.138]{milneAlgebraicGroupsTheory2017}, this is only for motivation. (Split) Reductive groups over fields have a classification in terms of root datum \cite[22.48]{milneAlgebraicGroupsTheory2017} which is a strong parallel of the classification of semi-simple Lie algebras, and in fact is proved largely by bootstrapping from that theory to a more general setting. These correspond to Dynkin diagrams that are listed in \cite[Appendix A,B]{shahidiEisensteinSeriesAutomorphic2010}\todo{is this right} and of which there are four infinite families, which correspond to the classical groups (plus the unitary groups). This is in the way of motivating the notion of classical groups as "almost all" reductive groups.


\section{Subgroups}
From now on we restrict to split reductive LAG because these are the natural adjectives for our classical groups over a number field (in particular characteristic 0)

Subgroups with special properties allow us to reduce and break up problems into smaller ones. Here we will briefly review and compute some examples of special subgroups, with a particular eye on those of \(\Sp\). The point of these subgroups is two fold. Some of them will help us perform "induction" from smaller simpler groups to larger ones. Others are there essentially as a part of the combinatorial data that classifies the groups we are working with. In particular we will have the decomposition's (not direct)
\[G(\A) = M(\A)U(\A) K = T(\A)U(\A)K\]

\subsection{Parabolics, Levis and Unipotents}
Parabolic subgroups have two equivalent formulations, both useful.\todo{Do I need to give a reference for definitions?}
\begin{definition}
    A subgroup \(P\subseteq G\) is called parabolic if the following equivalent conditions hold
    \begin{itemize}
        \item \(G/P\) is a complete variety
        \item \(P\) contains a Borel (see below)
    \end{itemize}
\end{definition}

Completeness is the algebro-geometric analogue of compact, which is always a desirable property. The fact that they contain a Borel gives us an algebraic "parametrisation" of these subgroups, in the case of the classical groups through the use of flags. 

\begin{example}[\(\GL_n\)]
    A flag for \(F^n\) is a sequence of subspaces
    \[0\subset W_1 \subset \cdots \subset W_r = F^n\]
    note the strict inclusion. When \(n = r\) it is a complete flag. \(\GL_n(F)\) acts on a flag component wise i.e. 
    \[g.(W_1, ..., W_r) \defeq (g.W_1, ..., g.W_r)\]
    \todo{check that this is an action, g is a bijection but why does the inclusion still hold?}
    If \(F^n\) is given the standard basis of \(e_i = (\delta_i^j)_j\) then the standard (complete) flag is 
    \[0 \subset Fe_1 \subset Fe_1 \oplus Fe_2 \subset \cdots \subset \oplus_i Fe_i = F^n\]
    From now on we consider only subflags of the standard flag. 
    \begin{Remark}
        This is becuase the stabiliser of a flag is always conjugate to a stabiliser of a sub-flag of the standard flag. i.e. stabilisers of flags are up to conjugacy stabilisers of the standard flag.

        It is worth noting that what we do here by fixing a basis of \(F\) is the same as fixing a Borel and then considering only standard parabolics. Thus when we are talking about standard parabolics we are really working up "up to conjugacy".
    \end{Remark}
     Stabilisers of such subflags are "staircases":
    \[\begin{pmatrix}
        A_{11} && \ast\\
         & \ddots & \\
         && A_{rr} 
    \end{pmatrix}\]

    \begin{example}
        Consider the flag
        \[0 \subset Fe_1 \subset Fe_1\oplus Fe_2 \oplus Fe_3 \subset F^4\]
        lets find the stabiliser under the action of \(\GL_4(F)\). We need to send \(Fe_1\) to itself
        \[\forall x\in F, \;\; \begin{pmatrix}
            a & b&c &d\\
            e&f&g &h \\
            i&j&k&l \\
            m&n&o&p
        \end{pmatrix}\begin{pmatrix}
            x \\
            \\
            \\
            \\
        \end{pmatrix} = \begin{pmatrix}
            ax\\
            ex\\
            ix\\
            mx 
        \end{pmatrix} \in F\begin{pmatrix}
            1 \\
            \\
            \\
            \\
        \end{pmatrix} \]
        and so \(e = i = m = 0\). Next we need to send \(Fe_1\oplus Fe_2 \oplus Fe_3\) to itself (as a subspace)
        \[\forall x,y,z\in F, \;\;\begin{pmatrix}
            a & b&c &d\\
            e&f&g &h \\
            i&j&k&l \\
            m&n&o&p
        \end{pmatrix}\begin{pmatrix}
            x \\
            y\\
            z\\
            \\
        \end{pmatrix} = \begin{pmatrix}
            ax+by+cz\\
            ex+fy+gz\\
            iz+jy+kz\\
            mx+ny+oz
        \end{pmatrix} \in Span_F\left\{\begin{pmatrix}
            1 \\
            \\
            \\
            \\
        \end{pmatrix}, \begin{pmatrix}
             \\
            1\\
            \\
            \\
        \end{pmatrix}, \begin{pmatrix}
             \\
            \\
            1\\
            \\
        \end{pmatrix}\right\} =\begin{pmatrix}
            *\\
            *\\
            *\\
            \\
        \end{pmatrix} \]
        So we conclude that \(m = n = o = 0\). Thus we are left with elements in \(\GL_4(F)\) that look like
        \[\begin{pmatrix}
            * & *&*&*\\
            & *&*&*\\
            &*&*&*\\
            &&&*
        \end{pmatrix}\]

    \end{example}
    So what is happening here, for each peice of the flag you complete the diagonal that the basis vectors are in into the box that they span and you say that everything \textit{directly} below that box must be zero. 
    



    \tikzset{every picture/.style={line width=0.75pt}} %set default line width to 0.75pt        

    \begin{tikzpicture}[x=0.75pt,y=0.75pt,yscale=-1,xscale=1]
    %uncomment if require: \path (0,300); %set diagram left start at 0, and has height of 300
    
    %Straight Lines [id:da514748463323927] 
    \draw    (172.15,169.8) -- (257.15,169.8) ;
    \draw [shift={(259.15,169.8)}, rotate = 180] [color={rgb, 255:red, 0; green, 0; blue, 0 }  ][line width=0.75]    (10.93,-3.29) .. controls (6.95,-1.4) and (3.31,-0.3) .. (0,0) .. controls (3.31,0.3) and (6.95,1.4) .. (10.93,3.29)   ;
    %Shape: Rectangle [id:dp7553422026549674] 
    \draw   (281,127) -- (352.15,127) -- (352.15,188.8) -- (281,188.8) -- cycle ;
    %Straight Lines [id:da9263382380660232] 
    \draw    (403.15,173.8) -- (488.15,173.8) ;
    \draw [shift={(490.15,173.8)}, rotate = 180] [color={rgb, 255:red, 0; green, 0; blue, 0 }  ][line width=0.75]    (10.93,-3.29) .. controls (6.95,-1.4) and (3.31,-0.3) .. (0,0) .. controls (3.31,0.3) and (6.95,1.4) .. (10.93,3.29)   ;
    %Shape: Rectangle [id:dp7811323439998978] 
    \draw   (509,127) -- (580.15,127) -- (580.15,188.8) -- (509,188.8) -- cycle ;
    %Shape: Rectangle [id:dp26317671145628463] 
    \draw  [color={rgb, 255:red, 255; green, 0; blue, 0 }  ,draw opacity=1 ] (510,195) -- (580,195) -- (580,222.8) -- (510,222.8) -- cycle ;
    %Straight Lines [id:da61956528286507] 
    \draw [color={rgb, 255:red, 255; green, 0; blue, 0 }  ,draw opacity=1 ]   (510,195) -- (580,222.8) ;
    %Straight Lines [id:da8927414476865059] 
    \draw [color={rgb, 255:red, 255; green, 0; blue, 0 }  ,draw opacity=1 ]   (510,222.8) -- (580,195) ;
    
    % Text Node
    \draw (39,126.4) node [anchor=north west][inner sep=0.75pt]    {$\begin{pmatrix}
    1 &  &  &  & \\
     & \ddots  &  &  & \\
     &  & 1 &  & \\
     &  &  &  & \\
     &  &  &  & 
    \end{pmatrix}$};
    % Text Node
    \draw (272,126.4) node [anchor=north west][inner sep=0.75pt]    {$\begin{pmatrix}
    * &  & * &  & \\
     & \ddots  &  &  & \\
    * &  & * &  & \\
     &  &  &  & \\
     &  &  &  & 
    \end{pmatrix}$};
    % Text Node
    \draw (196,149.4) node [anchor=north west][inner sep=0.75pt]    {$span$};
    % Text Node
    \draw (427,153.4) node [anchor=north west][inner sep=0.75pt]    {$delete$};
    % Text Node
    \draw (500,125.4) node [anchor=north west][inner sep=0.75pt]    {$\begin{pmatrix}
    * &  & * &  & \\
     & \ddots  &  &  & \\
    * &  & * &  & \\
     &  &  &  & \\
     &  &  &  & 
    \end{pmatrix}$};
    
    
    \end{tikzpicture}

    It is a fact that these are the parabolic subgroups of \(\GL\) up to conjugacy, \cite[Exercise 3.2.16, 6.2.11]{springerLinearAlgebraicGroups1998}\cite{conradStandardParabolicSubgroups}
\end{example}

Parabolics also have the nice property that they split into a semi-direct product 
where one of the factors is a reductive group \(M\). For this recall the definition
\begin{definition}
    A subgroup is unipotent if all its elements are unipotent.
    The maximal closed, connected, unipotent subgroup \(U\subseteq G\) is the unipotent radical of \(G\). 
\end{definition}
Then we have the following fact / definition:

\begin{Lemma}[\cite{borelLinearAlgebraicGroups1991} 11.22]
    There is a split exact sequence 
    \[0 \to U \to P \to M \to 0\]
    where \(U\) is the unipotent of P, and \(M\) is a reductive group known as a Levi (unique up to conjugacy).
\end{Lemma}

\begin{example}
     The staircase 
    \[\begin{pmatrix}
        A_{11} && \ast\\
         & \ddots & \\
         && A_{rr}
    \end{pmatrix}\]
    has unipotent \(N\) and Levi \(M\) given by 
    \[U = \begin{pmatrix}
        I_{11} && \ast\\
         & \ddots & \\
         && I_{rr}
    \end{pmatrix}, \quad M = \begin{pmatrix}
        A_{11} && 0\\
         & \ddots & \\
         && A_{rr}
    \end{pmatrix}\]
    Then 
    \[P = M\ltimes U\]
\end{example}

Thus doing things on a parabolic allows us to induce said actions up to the whole group, whist maintaining the nice property of being reductive. 

\begin{Remark}[Bad Etymology]
    The origin of the name parabolic is a mystery. Borel in his history \cite[VI.\S 2]{EssaysHistoryLie} attributes it to R. Godement in \cite{godementGroupesLineairesAlgebriques}. Godement conjectures that the quotient \(G(\A) / G(\Q)\) is compact if and only if every element of \(G(\Q)\) is semi-simple, as is the case in classical groups. \todo{this is probably known by now.} He says that 
    \begin{quote}
        Lorsque n'est pas compact, il est non moins facile de conjecturer qu’on doit pouvoir définir quelque chose d’analogue aux classiques "pointes paraboliques", lesquelles doivent correspondre à des  sous-groupes unipotents non triviaux de \(G_\Q\)
    \end{quote}
    which roughly (google) translates to that one can also conjecture that non-trivial unipotent elements should correspond to "parabolic points" in a fundamental domain.

    In the case of modular forms the fundamental domain is \(\mathcal{H} = \SL_2(\R)/SO_2(\R)\) (using orbit stabiliser theorem). We have the classification of elements of  \(\SL_2(\R) -\{\pm 1\}\) as in \cite[3.5]{borelAutomorphicFormsSL21997} via their trace
    \[g\text{ is of type } \;\;\; 
    \begin{cases}
        \text{Elliptic } & \frac{1}{2}|tr(g)| < 1 \\
        \text{Parabolic } & \frac{1}{2}|tr(g)| = 1 \\
        \text{Hyperbolic} & \frac{1}{2}|tr(g)| > 1 \\
    \end{cases}
    \]
    Being parabolic is equivilent to having eigenvalue 1 hence by the Jordan decomposition we know that parabolics in \(\SL_2\) are conjugate (over \(\C\)) to 
    \[\begin{pmatrix}
        1 & 1\\
        0 & 1
    \end{pmatrix},\;\;\; \pm\begin{pmatrix}
        1 & 0\\
        0 & 1
    \end{pmatrix}\]
    Clearly the standard parabolic 
    \[\begin{pmatrix}
        a & b \\
         & a\inv
    \end{pmatrix} \subseteq \SL_2(\R)\]
    contains these matricies, and moreover all parabolics are \textit{conjugate} to this parabolic. Hence all parabolic elements are contained in a parabolic subgroup. This classification it seems relies entrirely on the \textit{aesthetic} connection with the classification of the sections of conics via eccentricity.

    To connect this to Godements concept we have two facts from classical geometry. Proper parabolic subgroups of \(\SL_2(\R)\) can be realised as the stabilisers of lines in \(\R^2\) under the standard action of \(\SL_2\) on \(\R^2\) \cite[2.6]{borelAutomorphicFormsSL21997} and moreover some an element of \(\SL_2(\R)\) is parabolic if and only if it has one fixed point on \(\partial\bar{\mathcal{H}}\) and none on \(\mathcal{H}\) \cite[3.5]{borelAutomorphicFormsSL21997}. 

    The take away is that perhaps the folklore of the name being for "para-Borelic", as in kind of a Borel, is probably a better way of thinking of them.
\end{Remark}

\subsubsection{\(\Sp_{2n}\)}
The case of \(\Sp_{2n}\) is very similar to that of \(\GL_n\), but we will use this later and so we present the details here. Following \cite{conradStandardParabolicSubgroups} and the very explicit calculations in  \cite[\S 8]{BuildingsClassicalGroups}. We let \((V, \inner{,})\) be a symplectic space, hence \(Sp(V)\) is the automorphisms preserving the form. A subspace is said to be isotropic if the form is constantly zero on it (in both variables). A flag is isotropic if the proper subspaces in it are isotropic subspaces. A maximal isotropic flag is one with exactly \(n\) components (we elaborate later). The action of \(\Sp\) preserves isotropic flags i.e. it sends an isotropic flag to an isotropic flag. Stabilisers of isotropic flags give parabolics of \(\Sp\) and moreover all parabolics arise in this way (see the above exercises in Springer).

\begin{example}
        First we remind ourselves why the two notions of the symplectic group agree
        \[\Sp_{2n} = \{M\in \GL_{2n} :\forall a,b \;\; \inner{Ma, Mb} = \inner{a,b}\}\]
        if we make the form into a matrix (by setting the entries to be \(A_{ij} = \inner{e_i, e_j}\)) then we get 
        \[\forall a,b \;\; \inner{Ma, Mb} = (Ma)^T A (Mb) = a^T M^T A M b= a^TAb \]
        and becuase this is so for all \(a,b\) we get that 
        \[M^T A M = A\]

        So if we fix a basis of V such that the form is given by the matrix
        \[\begin{pmatrix}
            0 & I_n \\
            -I_n & 0\\
        \end{pmatrix}\]
        then we can see what the form does on the standard basis vectors
        \[(e_{n+1})^T\begin{pmatrix}
            0 & I_n \\
            -I_n & 0\\
        \end{pmatrix}e_{n+1} = (e_{n+1})^T\begin{pmatrix}
            0 & I_n \\
            -I_n & 0\\
        \end{pmatrix}\begin{pmatrix}
            0\\
            \vdots\\
            0\\
            1\\
            0\\
            \vdots\\
            0\\
        \end{pmatrix} = (e_{n+1})^T\begin{pmatrix}
            0\\
            \vdots\\
            0\\
            1\\
            0\\
            \vdots\\
            0\\
        \end{pmatrix} = 1\]
        \[ (e_1)^T\begin{pmatrix}
            0 & I_n \\
            -I_n & 0\\
        \end{pmatrix}e_1 = (e_1)^T\begin{pmatrix}
            0 & I_n \\
            -I_n & 0\\
        \end{pmatrix}\begin{pmatrix}
            1\\
            0\\
            \vdots\\
            0\\
        \end{pmatrix} = (e_1)^T\begin{pmatrix}
            -1\\
            0\\
            \vdots\\
            0\\
        \end{pmatrix} = -1\]
        hence 
        \begin{equation*}
            \begin{aligned}
                (e_1+e_{n+1})^T\begin{pmatrix}
                    0 & I_n \\
                    -I_n & 0\\
                \end{pmatrix}(e_1+e_{n+1}) &=(e_1+e_{n+1})^T\left(\begin{pmatrix}
                    0 & I_n \\
                    -I_n & 0\\
                \end{pmatrix}e_1 + \begin{pmatrix}
                    0 & I_n \\
                    -I_n & 0\\
                \end{pmatrix}e_{n+1}\right) \\
                &= (e_1+e_{n+1})^T(-e_1+e_{n+1})  \\
                &=0 \\
            \end{aligned}
        \end{equation*} 
        so the vectors \(e_i^+ \defeq e_i + e_{n+i}\) give a basis for an \(n\) dimensional isotropic subspace. In our reference it is stated more abstractly that an \(n\) dimensional isotropic subspace is maximal and that any flag of isotropic subspaces can be considered as a subflag of a complete flag of a maximal isotropic subspace (it is beleivable from inspecting the matrix of the form that if your space has a dimension of greater than n then it cannot be isotropic).
        \todo{This might all be wrong, garretts book is giving something different.}

        Now we have a maximal isotropic flag we can consider subflags and find their stabiliser. Lets look at \(\Sp_4\) and the following flag
        \[0 \subset Fe_1^+ \subset Fe_1^+ \oplus Fe^+_2 \subset F^4\]

        If we change our basis to these \(e_i^+\) then we can re-use our \(\GL\) computations to see that the stabiliser is matricies in \(\Sp\) of the form (in this basis)
        \[\begin{pmatrix}
            *&*&*&* \\
             &*&*&* \\
             && *&* \\
             && *& *
        \end{pmatrix}\]
        \begin{Remark}
            Note that the fact that we change basis here doesnt change that the subspaces are isotypic, this is basis independent. What it does change is the representation of the form, it will no longer be represented by the matrix given above. 
        \end{Remark}
        Because these are in \(\Sp\) we can find more relations among the entries than in the \(\GL\) case, we will persue this further in the maximal case below. 
    \end{example}

    In particular maximal parabolics of \(\Sp\) are stabilizers of \textit{minimal} (non-trivial flags), i.e. stabilisers of non-zero isotropic subspaces.
    \[0 \subset V_\ell \subset V\]
    where \(V_\ell = span_F(e_1, ..., e_\ell)\). Then the stabilizer is 
    \[\begin{pmatrix}
        * &*&*&* \\
        0 &*&*&* \\
        0 &*&*&* \\
        0 &*&*&* \\
    \end{pmatrix}\]
    with the sizes of the diagonal blocks being (these numbers square)
    \[\begin{pmatrix}
        \ell &*&*&* \\
        0 &n-\ell&*&* \\
        0 &*&\ell&* \\
        0 &*&*&n-\ell \\
    \end{pmatrix}\]
    these sizes clearly determine the sizes of the rest of the matrix. This has Levi
    \[\begin{pmatrix}
        A &&& \\
         &a&&b \\
         &&(A^T)\inv& \\
         &c&&d \\
    \end{pmatrix}\]
    such that \(A\in \GL_\ell(F)\) and 
    \[\begin{pmatrix}
        a & b\\
        c & d \\
    \end{pmatrix} \in \Sp_{2(n-\ell)}(F)\]

    and unipotent 
    \[\begin{pmatrix}
        1 &*&*&* \\
        & 1&*& \\
        && 1& \\
        &&*&1
    \end{pmatrix}\]
    with relations among the entries. 

    \subsection{Borel and Torus}
    One may find it helpful to understand these subgroups to understand the analogous story for Lie groups and their classification \cite{hallLieGroupsLie2015}, however it is not necissary. 

    \begin{definition}
        A split torus is an algebraic group that is isomorphic to some products of \(\GL_1^b\).
    \end{definition}

    \begin{example}[Bad Etymology]
        \(\GL^2_1\) is a split torus. Notice that 
        \[\GL^2_1(\C) = \C^\ast\times \C^\ast\]
        is isomorphic as abstract groups to \(U(1)\times U(1)\) which when \(U(1)\) is realised as \(\{z\in \C : |z| = 1\}\) is topologically equivilent to \(\mathbb{T}^2 = \mathbb{S}^1\times \mathbb{S}^1 \) which is a torus. Note that it is clear that 
        \[\GL^2_1(\C) \not\cong T^2 \]
        as topological groups, as the right had side is compact whilst the left is not.
    \end{example}
    
     Perhaps a more compelling reason to call these Tori is that they play the same role in the classification as the genuine tori in the theory of Lie groups. \todo{example maybe idk}

    \begin{definition}
        A Borel is a maximal closed solvable connected subgroup of \(G\).
    \end{definition}

    \begin{example}
        The standard Borel of \(\GL_n\) is the one given by upper triangular matrices. If \(n\) is even and one intersects this with \(\Sp_{2(\frac{1}{2}n)}\) then we get the standard Borel of \(\Sp_{2(\frac{1}{2}n)}\).
    \end{example}
    
    A Borel can be considered to be a parabolic that is minimal with respect to inclusion. The maximal tori then form the Levis of these parabolics. In particular for a Borel \(B\) we have that 
    \[B = TU\]
    for a maximal torus \(T\) and unipotent \(U\).

    We saw that fixing a minimal parabolic is like fixing a basis in the case of classifying parabolics of classical groups. If a Borel \(B\) is fixed, then a parabolic containing this Borel \(B\subseteq P\) is called standard, the unique Levi of a standard parabolic containing this Borel is called the standard Levi. This gives some intuition as to their important in the classification of reductive groups via root datum\cite[8]{springerLinearAlgebraicGroups1998}, as they correspond to fixing a collection of simple roots.\todo{ref} Above also suggests that Borels are "the same data" as maximal tori from which the root datum is constructed.

    \begin{example}
        \todo[inline]{what is easy to find is definitions, whaat is hard to find is long drawnout matrix algebra that has been latexed.}
    \todo[inline]{maybe put the whole thing in the appendix.}
    \end{example}

    \subsection{Maximal Compact Subgroups}
    Let our group be defined over the global field \(k\)\todo{fix a global convention}. We will often need to fix a maximal compact subgroup \(K\subseteq G(\A)\). These are not unique and as such when fixing one it can be arranged to have many nice properties \cite[I.1.4]{moeglinSpectralDecompositionEisenstein1995}. In particular if we have a group \(G\) and a fixed Borel \(B\):
    \begin{itemize}
        \item First require that 
        \[K = \prod_\nu K_\nu\]
        where the product is over all places of \(k\) and \(K_\nu\subseteq k_\nu\) is maximal compact. \todo{But is the converse true?}
        \item For almost all places \(\nu\) of \(k\) \(G(\mathcal{O}_{k_\nu})\) is defined and is maximal compact in \(G(k_\nu)\) hence we can require \(K_\nu = G(\mathcal{O}_{k_\nu})\) at these places. 
        \item We require 
        \[G(\A) = B(\A)K\]
        \item For every standard parabolic \(P = MU\) we have that 
        \[P(\A)\cap K = \Bigl( M(\A)\cap K \Bigr) \Bigl( U(\A)\cap K \Bigr)\]
        and \(M(\A)\cap K\) is a maximal compact subgroup of \(M(\A)\).
    \end{itemize}
     It is in terms of the third property that we like to think of the maximal compact subgroup, it is the complimentary piece of the Borel. Moreover the fourth property should be thought of as a condition that the maximal compact subgroups are well behaved with the way that we are moving between the bigger an smaller reductive groups.
    
    \begin{example}[\(\GL_n(\A_\Q)\)]
        It is a classical result that the maximal compact subgroup of \(\GL_n(\R) = \GL_n(\Q_\infty)\) is the orthogonal group \(O(\R)\). By \cite[II.IV.A1]{serreLieAlgebrasLie1992} we have that \(\GL_n(\Q_p)\) for \(p<\infty\) is \(\GL_n(\Z_p)\) and hence the maximal compact subgroup of \(\GL_n(\A)\) is the product
        \[K = O(\R) \times \prod_p \GL_n(\Z_p) = O(\R) \times \GL_n(\hat{\Z})\]
    \end{example}

   
 
\section{Coverings}
We are also be interested in certain covering groups of these LAG's. In particular \cite[I.1.1]{moeglinSpectralDecompositionEisenstein1995} we will be intereseted in \(\mathbf{G}\) some topological group given as a finite central cover of \(G(\A)\). If \(\mathrm{pr}: \mathbf{G} \to G(\A)\) is the projection then to the subgroups listed above we can associate their "lifts" (preimages under \(\mathrm{pr}\)). 

\begin{example}[Metaplectic Group]
    If \(G = \Sp_{2n}\) then there is a unique non-trivial double cover 
    \[0\to \mu_2 = \{\pm 1\} \to \Mp_{2n} \to \Sp_{2n} \to 0\]
    such group extensions are classified by their second group cohomology, for this extension the relevant cocycle is called the Rao cocycle \cite{raoExplicitFormulasTheory1993}, \(c\).
    As a set we can think of 
    \[\Mp_{2n} = \Sp_{2n} \times \mu_2\]
    with the group operation 
    \[(a,b)(x,y) = (ax, byc(a,x))\]
    
    There is a rich history and representation theory of this group, which we make no pretense of understanding, however some hints can be found in \cite{kudlaNOTESLOCALTHETA} and the references therein.
\end{example}

There is an analogue of the Langlands program being developed for such groups a nice introduction to which can be found in \cite{ganLgroupsLanglandsProgram2017}. \todo{perhaps a first natural question is whether or not these things can be given the structure of LAG's. That paper mentions representability. Look into it. }

\chapter{Automorphic Forms}
There are different definitions of the words automorphic forms floating around, here we fix a nice one and then explain how they generalize the classical modular forms. We intend to be terse as this material is somewhat standard.

\section{Definition and Role}
The story starts with the classical modular forms, or functions on the upper half plane that satisfy some invariance conditions and differential equations. This evolves into the notions of Maas form on symmetric spaces and eventually reaches its apotheosis in the concept of automorphic form that we will present here. 

We still do not have a good answer as to why the definition below is ``the right'' definition, from a mathematical perspective, as there are many places in which it could be extended or restricted and we are unable to motivate why one shouldn't consider such things. Indeed there are varying notions of automorphic form that appear for this reason and I think it is important to stress that this is ``the right definition'' only in so far as people have been able to prove nice theorems about them, and that when functions appear ``in nature'' this concept has sufficed to encompass and explain their behavior. It is the representation theoretic properties more than anything that suggest the current definition as is mentioned in \cite[1.II.3]{borelAutomorphicFormsRepresentations1979}.

We will present two notions of automorphic form here. In the literature they are both called ``automorphic forms'' however here we will distinguish those that are defined only on the Archimedean points as ``Archimedean automorphic forms'' for clarity.

\subsection{Archimedean Automorphic Form}
Fix a number field \(F\). Let \(\nu\) be an Archimedean place and let \(\infty\) denote the set of Archimedean places. Then \(F_\nu\) is either \R or \C. In particular (the analytification of) \(G(F_\nu)\) is a Lie group and we call a function, \(\phi: G(F_\nu) \to \C\), \textbf{smooth}  if it is smooth in the sense of manifolds.


Now we fix an embedding \(\i : G\to GL_n\) which gives another embedding \(G\to SL_{2n}\) via
	\[g\mapsto \begin{pmatrix}
		\i (g) & \\
		 & (\i (g))^{-t}
	\end{pmatrix}.\]

	A function \(\phi : G(F_\infty) = G(\prod_{v\in\infty} F_\nu)\cong \prod_{v\in \infty}G(F_\nu) \to \C \) is of \textbf{moderate growth} if there are constants \((c,r)\in \R_{>0}\times \R\) such that 
	\[|\phi(g)| \leq c\norm{g}^r = c \left(\prod_{v\in\infty} \sup_{1\leq i, j\leq 2n} |\i (g)_{i,j, \nu}|_\nu\right)^r.\]
	This is taking the maximum of the \(2n\times 2n \times |\infty| \) three dimensional matrix. 

    Because \(G(F_\infty)\) is a Lie group we know how to define its Lie algebra and we now denote \(Z(\mathfrak{g})\) the center of the \textit{universal enveloping algebra} of the \textit{complexification} of \(\mathfrak{g}\), it would be more reasonable to use \(Z(\mathcal{U}(\mathfrak{g}_{\C}))\) but that is too cumbersome so we follow the tradition. 
    A vector in a \(Z(\mathfrak{g})\)-module \(\phi\in V\) is called \(Z(\mathfrak{g})\)-\textbf{finite} if the space \(Z(\mathfrak{g})\phi\) is finite dimensional. 

	Let \(K_\infty\leq G(F_\infty)\) be a maximal compact subgroup. Then again an element of a \(K_\infty\)-module is \(K_\infty\) \textbf{finite} if its orbit is a finite dimensional vector space space (we think here of \(\C[K_\infty]\)-modules).

	To define automorphic forms we look at the representation \(C^\infty(F_\infty)\) with the right regular action, i.e. \(g.f(x) = f(xg)\).  In particular the \(Z(\mathfrak{g})\) module structure is induced from the action of \(\mathfrak{g}\) on \(C^\infty(G(F_\infty))\) by \label{lie_algebra_action}
	\[z.F(g) = \Dif{}{t}F(ge^{tz}).\] 

	\begin{Definition}
		Let \(\Gamma\leq G(F_\infty)\) some (arithmetic) subgroup, an \textbf{automorphic form for \(\Gamma\)} is a smooth function of moderate growth 
		\[\phi: G(F_\infty) \to \C,\]
		that is \(K_\infty\) and \(Z(\mathfrak{g})\) finite with a (left) \(\Gamma\) invariance. We denote the set of these ``Archimedean'' automorphic forms by \(\mathcal{A}(\Gamma \backslash G(F_\infty))\).
	\end{Definition}


\subsection{Adelic Automorphic Form}
Here we follow \cite[I.2.17]{moeglinSpectralDecompositionEisenstein1995}. Let \(G\) be a reductive group over \(F\), we fix a Borel \(B\) and a standard parabolic \(P \) with a standard Levi decomposition \(P = MU\). We let \(K\) be a maximal compact subgroup of \(G(\A)\) satisfying the conditions laid out in the previous section \ref{max_compact_subgroup}.

For \(v\notin \infty\) a non-Archimedean place then we say that a function \(f: G(F_\nu) \to \C\) is smooth if it is locally constant in the induced topology on \(G(F_\nu)\), the details of this topology are spelled out in \cite{conradWeilGrothendieckApproaches2012}. The set of such smooth functions is denoted \(C^\infty(G(F_\nu))\). This suggests the definition of smooth functions on the ``finite adeles'' \(\A_f\) as 
	\[C^\infty(\mathbb{A}_f) \defeq \bigotimes^{}_{\nu \notin \infty} \phantom{}' C^\infty(G(F_\nu)). \]
	Thus for the full adeles we have the notion of smooth as an element of the following,
	\[C^\infty(\mathbb{A}_F) \defeq   C^\infty(G(\mathbb{A}_f))   \otimes   C^\infty(G(F_\infty)).\]
	Notice that a priori the codomain is an infinite tensor product over \C of copies of \C, which is isomorphic to \C. Thus we can conflate a smooth function with its composition along this isomorphism, and think of them as functions into \C.

	We still consider \(Z(\mathfrak{g})\) to be the center of the universal enveloping algebra of the Lie algebra at the infinite places, exactly as before. We define an action by linearly extending
    \[z.(f\tensor g) = f\tensor (z.g),\]
    i.e. it acts on the archimedean places as in the setting of Archimedean automorphic forms. 
	
	The definition of moderate growth carries over verbatim, however we change the set of places multiplied over to be all of them now.
    
    \begin{remark}[\cite{borelAutomorphicFormsRepresentations1979}, 1.II.3]
        The collection of moderate growth functions is independent of the choices of embedding. 
    \end{remark}

\begin{definition}
    A function \(\phi: U(\A)M(F)\backslash G(\A) \to \C\) is an \textbf{automorphic form} if it is smooth, moderate growth, \(Z(\mathfrak{g})\) and \(K\) finite. We will denote the set of these automorphic forms by \(\mathcal{A}(U(\A)M(F)\backslash G(\A))\)
\end{definition}

\begin{remark}
    It is important that \(M(F)\) is treated as a subgroup of \(M(\A)\) via the diagonal embedding.
\end{remark}
	
    

\section{Modular Forms}
One might ask if there is a special case in which automorphic forms yield modular forms. In fact no, the space of automorphic forms is larger than just modular forms, however it gives the space of Maas forms (or modular and Maas forms, depending on convention). This is well covered in the literature \cite{emertonCLASSICALMODULARFORMS}\cite[3.2]{bumpAutomorphicFormsRepresentations1997}\cite{booherVIEWINGMODULARFORMS}\cite{garrettTransitionEisensteinSeries2016}, but so essential to intuiting automorphic forms that we feel it is necessary to present the details here. To be clear we explain modular forms as Archimedean automorphic forms as we think it is where the connection is clearest. 

	Recall the definition of a modular form 
	\begin{Definition}[\cite{diamondFirstCourseModular2005} 1.1.2]
		A function
		\[\phi: \mathcal{H} \to \C,\]
		where \(\mathcal{H}\) is the upper half plane in \C, that is holomorphic, satisfies 
		\[\phi(\gamma.z) = (cz+d)^k\phi(z), \quad \gamma = \begin{pmatrix}
			a &b \\
			c &d
		\end{pmatrix}\in \SL_2(\Z),\]
		and extends holomorphically to \(\infty\) is called a modular form of weight k.
	\end{Definition}
	These are modular forms with trivial character and full level.

	Now give a function on a set \(X\) and an action of a group \(G\) on X, there is a general way of associating to \(\Hom(X, Y)\) a family of maps \(\Hom(G, Y)\) indexed by \(X\). This is a manifestation of the tensor-hom adjunction. Effectively if \(f: X\to Y\) the we get a map for each \(x\in X\) defined on \(f_x : G \to Y\) given by \(g\mapsto f(g.x)\).

	So for our purposes we are trying to take some subset of functions \(\mathcal{H} \to \C\) and shift their domain to the \(\Q_\infty = \R\) points of some reductive group. In particular it would be sufficient to find a reductive group with a well defined action on the upper half plane and in particular we would want the action to be transitive.

	\begin{Theorem}
		\[\mathcal{H} \cong  \SL_2(\R) / SO_2(\R) ,\]
		as topological spaces.
	\end{Theorem}
	\proofbar{
		Consider the action 
		\[\SL_2(\R) \curvearrowright \mathcal{H}: \;\; \begin{pmatrix}
			a & b\\
			c & d
		\end{pmatrix}.z = \frac{az + b}{cz + d}.\]
		Then look at the orbit of \(i\), namely 
		\[\begin{pmatrix}
			a & b\\
			 & d
		\end{pmatrix}.i = \frac{ai + b}{d} = a^2i + ab,\]
		which letting \(a, b\in \R\) vary is clearly surjective onto the whole upper half plane. So there is one orbit, and hence by the orbit stabiliser we know that 
		\[\mathcal{H} \cong \SL_2(\R) /stab(i) ,\]
		so we want to find
		\[stab(i) = \left\{ g = \begin{pmatrix}
			a & b\\
			c & d
		\end{pmatrix}\in \SL_2(\R) : g.i = i   \right\},\]
		in particular we solve 
		\begin{equation*}
			\begin{aligned}
				i &= g.i = \frac{ai + b}{ci + d} = (c^2 + d^2)\inv (ac + bd  + i\det g) .\\
			\end{aligned}
		\end{equation*}
		So equating coefficients we have 
		\[\det g (c^2 + d^2)\inv  = 1 \implies c^2 +  d^2 = \det g = 1,\] 
		on the other hand 
		\[ac + bd = 0.\]
		Now the pairs \(c^2 + d^2 = \det g = 1\) are parameterized by \(\theta\in [0, 2\pi)\) using \(c = \sin \theta, d =  \cos\theta\) hence subbing this into the above equation
		\[\frac{-b}{a} = \tan\theta,\]
		and so \(b = -k\sin\theta, a = k\cos\theta\) for some  \(k\in \R\) but the determinant must be \(1\) so \(k = 1\).
		Hence 
		\[stab(i) = \left\{ \begin{pmatrix}
			\cos\theta & -\sin\theta \\
			\sin\theta & \cos\theta
		\end{pmatrix} : \theta \in [0, 2\pi)\right\} = SO_2(\R).\]
		One then has to check that this is all continuous. 
	}
    \begin{remark}
        Sometimes for to make the action of certian (Hecke) operators more apparent this is exhibited as 
        \[\mathcal{H} \cong \GL_2^+(\R)/ A_{\GL_2}SO_2(\R).\]
        This obscures the connection with the reductive group setting however so we avoid it here. 
    \end{remark}


	
	
	\(\SL_2\) is a reductive group and \(SO_2(\R)\) is its maximal compact subgroup. This decomposition of the upper half plane suggests that function on it might have some invariance along the maximal compact subgroup of the reductive group \(\SL_2\). Indeed if we were to push our modular forms along this isomorphism it would, with the construction that we outlined earlier in terms of a group action on a set, exhibit this invariance. This is merely \textit{evidence} that if we were to change our modular forms to functions on the reductive group \(\SL_2\) they may preserve \textit{some} of that invariance and indeed be K-finite.
    
		% https://q.uiver.app/#q=WzAsNCxbMCwwLCJcXGJlZ2lue3BtYXRyaXh9eV57MS8yfSAmIHggeV57LTEvMn1cXFxcICYgeV57LTEvMn1cXGVuZHtwbWF0cml4fVNPXzIoXFxSKT1cXFNsXzIoXFxSKSJdLFsyLDAsIlxcU2xfMihcXFIpL1NPXzIoXFxSKSJdLFsyLDIsIlxcU2xfMihcXFopXFxzZXRtaW51c1xcU2xfMihcXFIpIl0sWzQsMCwiXFxtYXRoY2Fse0h9Il0sWzEsMywiXFxzaW0iXSxbMCwxLCJwcm9qIl0sWzEsMywieFxcbWFwc3RvIHguaSIsMl0sWzAsMiwiXFx0ZXh0e2Rlc2NlbmQ/Pz99IiwyXV0=
	\[\begin{tikzcd}[cramped]
		{\left\{\begin{pmatrix}y^{1/2} & x y^{-1/2}\\ & y^{-1/2}\end{pmatrix} : x,y\in \R, y\neq 0 \right\} SO_2(\R)=\SL_2(\R)} && {\SL_2(\R)/SO_2(\R)} && {\mathcal{H}} \\
		\\
		&& {\SL_2(\Z)\setminus\SL_2(\R)}
		\arrow["\sim", from=1-3, to=1-5]
		\arrow["\mathrm{project}", from=1-1, to=1-3]
		\arrow["{g\mapsto g.i}"', from=1-3, to=1-5]
		\arrow["{\text{project}}"', from=1-1, to=3-3]
	\end{tikzcd}\]

    Using something like the universal property of the quotient we can lift a function on \(\SL_2(\R) / SO_2(\R)\) to \(\SL_2(\R)\) however this is not \(\SL_2(\Z)\) invariant, thus we need to add a pre-factor to ensure this in our associated automorphic form. The algebro-geometric perspective in \cite{emertonCLASSICALMODULARFORMS} can make this seem slightly less ad hoc.Thus for \(f\) a modular form of weight k the following function on \(\SL_2(\R)\)
	\[F(g) \defeq  (ci + d)^{-k}f(g.i),\]
	we claim is an automorphic form for \(\SL_2(\Z)\). We take for granted its smoothness. The \(\SL_2(\Z)\) invariance is obvious from the modularity condition. It remains to show the three other properties:

	\begin{Lemma}
		\(F(g)\) is of moderate growth.
	\end{Lemma}
	\proofbar{
		Unraveling the definitions we require two constants such that 
		\[|F(g)| = |ci+ d|^{-k}|f(g.i)| \leq c(\sup_{i,j}(g, g\inv))^r,\]
		A direct computation shows that 
		\[Im(g.i) = |ci+ d|^{-2},\]
		hence we require to show
		\[ \mathrm{Im}(g.i)^{k/2}|f(g.i)| \leq c(\sup_{i,j}(g, g\inv))^r.\]
		\textcolor{red}{Somehow invoke polynomial growth...?}
        but the modularity condition has the growth condition that \(\lim_{x\to \infty}f(xi)\) be bounded. 
	}

	\begin{Lemma}
		\(SO_2(\R)\) is a maximal compact subgroup inside \(\SL_2(\R)\). \(F\) is an \(SO_2(\R)\) finite function.
	\end{Lemma}
	\proofbar{
		Using that \(\kappa \in K = SO_2(\R)\) acts trivially on \(i\), an elementary computation shows that for \(g  \in \SL_2(\R)\),
		\begin{equation*}
			\begin{aligned}
				F(g\kappa) = e^{-ik\theta}F(g). \\
			\end{aligned}
		\end{equation*}
		Hence \(F(g)\) is acted on by \(K\) via a one dimensional irreducible representation. In particular it is finite dimensional.
		}

	\begin{Lemma}
		\(F\) is a \(Z(\mathfrak{sl}_2)\) finite function.
	\end{Lemma}
	\proofbar{ Only a sketch. 

		The center of the universal enveloping algebra of the complexified Lie algebra is generated by the Casimir operators. From \cite{garrettInvariantDifferentialOperators2010} we know that the casimir is 
		\[\Omega = \frac{1}{2}H^2 + XY + YX.\]
		We have the coordinates on \(\begin{pmatrix}y^{1/2} & x y^{-1/2}\\ & y^{-1/2}\end{pmatrix}SO_2(\R)=\SL_2(\R)\) from \cite{bumpAutomorphicFormsRepresentations1997}[1.19 pg 139] in which 
		 the casimir acts as the differential operator
		\[\Delta = y^2\left(\left(\Dif{}{x}\right)^2 +\left(\Dif{}{y}\right)^2\right) - y\Dif{^2}{x\partial \theta},\] 
		\cite{bumpAutomorphicFormsRepresentations1997}[1.29 pg 143 ,Prop 2.2.5 pg 155]. Now we claim that F is an eigenfunction for this operator. 
		An element \((x,y,\theta) \defeq \begin{pmatrix}y^{1/2} & x y^{-1/2}\\ & y^{-1/2}\end{pmatrix}\kappa_\theta \in \SL_2(\R)\) acts on \(i\) by sending it to \(x+ iy\) (elementary computation). The bottom row of the product is \(y^{-1/2}\sin\theta ;y^{-1/2}\cos\theta \) which results in 
		\[F(x,y,\theta) = y^{k/2}e^{-ik\theta}f(x + iy).\]
		It is then a calculus exercise to apply \(\Delta\) to this, using the holomorphicity we also get that \(f_{xx} - f_{yy} = 0\) and \(f_y = if_x\) which cancels away terms and we get that 
		\[\Delta F(x,y,\theta) = \frac{k}{2}\left(\frac{k}{2} - 1\right) F(x,y,\theta).\]
		
		Therefore the dimension of \(Z(\mathfrak{g})F\) is simply one.
	}
    This example makes it clear that the two finiteness conditions for automorphic forms are in some sense functional equations that they must satisfy. 
	There is a nice explanation of how to lift this to the adelic setting in several places, however it is stated quite clearly in \cite[2.1]{cogdellLecturesLfunctionsConverse}
\chapter{Automorphic Representations}
The references that will be most helpful are \cite[I.II]{borelAutomorphicFormsRepresentations1979}\cite{getzIntroductionAutomorphicRepresentations2024} for the general theory, we will follow the notation developed in \cite{moeglinSpectralDecompositionEisenstein1995} as it is somewhat standard. For the connection to classical modular forms there is \cite{emertonCLASSICALMODULARFORMS}\cite{bumpAutomorphicFormsRepresentations1997}\cite{booherVIEWINGMODULARFORMS}\cite{garrettTransitionEisensteinSeries2016}. We will discuss some of the details of their representation theory because it is both subtle and needed later. In particular we want to draw attention to what we think of as the "non-algebraic" nature of the representation theory.

\section{\((\mathfrak{g}, K)\)-Modules}

\section{Hecke Algebra}

\section{Automorphic Representations}

\subsection{Cuspidal Representations}
We will recal the definition of a cusp form \todo{reference the definition in the next chapter} in the next chapter.\todo{Chengjing example of isotypic subspaces}

\subsection{Tensor Products of Representations}

\section{Eisenstein Series}
\cite{lapidPerspectivesEisensteinSeries2022}, \cite{arthurEisensteinSeriesTrace1979}

\section{Spectral Decomposition}
\subsection{Definition and Role}
This is another one of the tools that can be used to compartmentalise problems in automorphic forms, by dealing with representations that appear in different parts of the spectrum. 
\todo[inline]{give shahidis conjecture on plancherel measures some time. Make sure to talk about his proof based on a reasonable hypothesis. }

\subsection{The Decomposition of the Spectrum}


\subsection{Residual Representations of \(\GL_n\)}


\section{L-Functions}
\subsection{In General}
\subsection{Standard L-Functions for Classical Groups}
\subsection{L-Functions of Covering Groups}

\chapter{Eisenstein Series}
\section{Eisenstein Series}
As usual we fix a connected reductive group G defined over a number field F, with a Borel B, a standard parabolic with Levi decomposition \(P = MU\). 

Following the setup in \cite[I.1.4]{moeglinSpectralDecompositionEisenstein1995} we consider a \textbf{character} \(\chi\in \mathrm{Rat}(M) \defeq \Hom_{\mathrm{LAG}}(M, \mathbb{G}_m)\), thinking of it below as a natural transformation, and then define 
\[|\chi|: M(\A)\to \C , \;\;\; (m_\nu)\mapsto \prod_\nu|\chi(F_\nu)(m_\nu)|_\nu.\]
The intersection of the kernels of these characters is 
\[M^1 \defeq \bigcap_{\chi\in \mathrm{Rat}(M)}\ker |\chi|.\]
Thus we can define
\[X_M \defeq \Hom_{\textrm{TopGroup}}(M(\A)/M^1, \C^*) .\]
i.e. the collection of characters of \(M(\A)\) that are trivial on \(M^1\).
\begin{remark}
    To make it seem less mysterious this group has some importance in the more general theory, in particular it is one of the pieces in the ``Langlands decomposition'' (\ref{eq:iwasawa_decomposition}) of the Archimedean points of a parabolic and it has the property that \(M(\Q)\backslash M(\A)^1\) has finite measure \cite[4.9]{getzIntroductionAutomorphicRepresentations2024}.
\end{remark}
The set of \textbf{complex characters} of \(M\),
\[\mathfrak{a}_M^* \defeq \mathrm{Rat}(M)\tensor_\Z \C,\]
is isomorphic as \C vector spaces to \(X_M\). If \(Z_{G(\A)}\) is the center of \(G(\A)\) then we also have the space 
\[X_M^G \defeq \Hom_{\textrm{TopGroup}}((M(\A)/M^1)/Z_G, \C^*)\]
which is characters of \(M(\A)/M^1\) which are also trivial on the center of \(G\).

\begin{example}\label{ex:characters}
    For the maximal parabolic \(P_r\) with Levi \(M_r\) of \(\Sp_{2n}\) we have that \( X_{M_r}^{\Sp_{2n}}\) is at most a one dimensional \C vector space. 

    First of all we have that \cite[I.1.4]{moeglinSpectralDecompositionEisenstein1995}
         \[ X_{M_r}^{\Sp_{2n}} \subseteq X_{M_r} \cong \mathfrak{a}_{M_r}^*\defeq Rat(M_r) \tensor_\Z \C.\]
        Thus it is clearly sufficient to bound the dimension of \(\mathfrak{a}_{M_r}^*\) as a \C vector space, moreover this dimension agrees with the dimension of \(Rat(M_r)\) as a free \Z module. 

        Thus we compute \(\dim_\Z(Rat(M_r))\):
        \begin{equation*}
            \begin{aligned}
                Rat(M_r) &= Rat(\GL_r \times \Sp_{2m}) \\
                         &= \Hom(\GL_r \times \Sp_{2m}, \mathbb{G}_m) \\
                         (2)&\cong \Hom(\mathrm{Ab}(\GL_r \times \Sp_{2m}), \mathbb{G}_m) \\
                         (1)&\cong \Hom(\mathrm{Ab}(\GL_r) \times \mathrm{Ab}(\Sp_{2m}), \mathbb{G}_m) \\
                         (3)&\cong \Hom(\mathbb{G}_m \times 1, \mathbb{G}_m) \\
                         &\cong \Z.
            \end{aligned}
        \end{equation*}
        In (2) we have used the universal property of the abelianization \(\mathrm{Ab}(G) = \mathcal{D}(G) \setminus G = [G, G] \setminus G \) because \(\mathbb{G}_m\) is abelian. (1) is that the abelianization commutes with direct products. (3) is because \(\Sp\) is a perfect group.
        %https://groupprops.subwiki.org/wiki/Symplectic_group_is_perfect
        %https://mathoverflow.net/questions/35713/abelianization-of-a-semidirect-product
\end{example}

\begin{remark}\label{Metaplectic_characters}
    This generalises to the metaplectic covers immediately as \( X_{M_r}^{\Mp_{2n}(\A)} \subseteq X_{M_r}\).
\end{remark}
There is the natural map \(m_P: G(\A) \to M^1 \backslash M(\A)\) sending \(umk \mapsto M^1 m\), where \(g = umk\) using the Langlands-Iwasawa decomposition \ref{eq:iwasawa_decomposition}.

Now if we take the collection of irreducible automorphic representations of \(M\),
 \[\hat{\mathcal{A}} \defeq \{(\pi, V) : \pi \text{ is an irreducible automorphic representation of }M\}\]
then we can think of \(X_M^G\) as being one dimensional automorphic representations (with some extra invariance) and so there is a natural action on \(\hat{\mathcal{A}}\) given by tensoring, i.e. if \(\lambda\in X_M^G\) and \((\pi, V)\in \hat{\mathcal{A}}\) then 
\[\lambda.\pi \defeq \lambda\tensor \pi\]
Then \(\hat{\mathcal{A}}\) decomposes as a disjoint union of its orbits. Consider the orbit \(\mathfrak{P}\) of a cuspidal representation \(\pi_0\), then by definition \(X_M^G\) acts transitively but it also acts freely \cite[II.1]{moeglinSpectralDecompositionEisenstein1995}. Thus \(\mathfrak{P}\) is in bijection with \(X_M^G\). Through this bijection we transmit the complex structure on \(\mathfrak{a}_M^*\) to \(X_M\) then to the quotient \(X_M^G\) and finally to \(\mathfrak{P}\).

Now we will define an Eisenstein series: Let \(\mathfrak{P}\) be as above, the orbit of a cuspidal automorphic representation endowed with a complex structure. Let \(\pi\in \mathfrak{P}\) and \(\phi_\pi \in \mathcal{A}(U(\A)M(k)\backslash G(\A))_\pi\), then \(\lambda\in X_M^G\) acts on \(\phi_\pi\) by 
\[\lambda.\phi_\pi(g) = (\lambda \comp m_P)(g) \phi_\pi(g).\]
which is then an element of \(\mathcal{A}(U(\A)M(k)\backslash G(\A))_{\pi\tensor \lambda}\). Finally we have the \textbf{Eisenstein series} which is defined by the following sum
\[E(\phi_\pi, \lambda, g) = \sum_{\gamma \in P(k)\backslash G(k)} \lambda.\phi_\pi(\gamma g)\]
whenever it is convergent. The first thing to note is that for a fixed \(\phi\) there is an open set in \(X_M^G\) and a compact subset of \(G(k)\backslash G(\A)\) such that the Eisenstein series converges (normally) \cite[II.1.5]{moeglinSpectralDecompositionEisenstein1995}.

If \(P = MU, P' = M'U'\) are two standard parabolics of \(G\) that are conjugate, i.e. such that for \(w\in G(k)\) we have \(wMw\inv = M'\)
Then \(w\) maps \(\mathfrak{P}\) to \(w\mathfrak{P}\), an orbit of an irreducible representations of \(M\) to an orbit of irreducible representations of \(M'\).

Then the Eisenstein series is closely related (through its constant terms as discussed in \ref{constant_conjugate_levi}) to the operator
\[M(w, \pi)(\phi_\pi)(g) = \int_{(U'(k)\cap wU(k)w\inv )\backslash U'(\A)} \phi_\pi(w\inv ug) du\]
where \(\pi\in \mathfrak{P}\), \(g\in G(\A)\) and \(\phi_\pi \in \mathcal{A}(U(\A)M(k)\backslash G(\A))_\pi\).

The key properties of both the Eisenstein series and this operator can be found in \cite[IV.1.8, IV.1.9, IV.1.10, IV.1.11]{moeglinSpectralDecompositionEisenstein1995}. Most importantly as a function of \(\mathfrak{P}\) it can be shown that (in the sense of Frechet spaces) they both have a meromorphic continuation to all of \(\mathfrak{P}\). This was also given a second ``soft proof'' more recently in \cite{bernsteinMeromorphicContinuationEisenstein2022}, with the spectral decomposition that follows from it also being worked out in \cite{delormeSpectralTheoremLanglands2021}. Moreover for the Eisenstein series at a point in \(p\in \mathfrak{P}\) at which it is holomorphic then \(E(\phi,p, g)\) is an automorphic form. 

We are not really in a position to convey the true importance of these objects in the theory of automorphic forms, however we will make some comments. First some surveys are \cite{lapidPerspectivesEisensteinSeries2022}, \cite{arthurEisensteinSeriesTrace1979}, \cite{kimEISENSTEINSERIESTHEIR}, \cite{jiangResiduesEisensteinSeries2008a}. To see the relation to the classical Eisenstein series there is \cite{garrettTransitionEisensteinSeries2016}. One thing that Eisenstein series do, as in the theory of modular forms, is that they furnish us with quasi-concrete examples. A we mentioned above \cite[IV.1.9.(b).i]{moeglinSpectralDecompositionEisenstein1995} tells us that at the holomorphic points the Eisenstein series takes an automorphic form and returns an automorphic form, thus we can use them to multiply our examples. Another reason that these functions are important is through their normalisation and constant terms, in which products of L functions appear, we discuss this more in section \todo[inline]{ref later}. This has been a fruitful method for proving theorems about L-functions as in \cite{shahidiEisensteinSeriesAutomorphic2010}\cite{pollackRANKINSELBERGMETHODUSER}\cite{arthurEisensteinSeriesTrace1979}.

\section{Spectral Decomposition}\label{spectral_decomposition}
This is a short explanation of some terms that frequently appear as well as some motivation for the later results. The results contained here-in are proved using the Eisenstein series as an essential component. 

\subsection{The Decomposition of the Spectrum In General}\label{direct_integral}
For this section let \(H\) be a locally compact topological group.
It is a classical theorem that for representations of finite groups over an algebraically closed field the regular representation decomposes into a direct sum, where ever irreducible representation appears \cite[Ch. 2.4 Cor. 2 ]{LinearRepresentationsFinite}. This still holds for compact topological groups, when one considers continuous unitary representations \cite[5.1]{follandCourseAbstractHarmonic2016}.
\begin{remark}
    This is a strict generalisation of the finite groups case, when we give the finite group the discrete topology then all its linear representations are continuous and unitary.
\end{remark}
There is one final more general incarnation of this line of investigation in the Plancherel theorem. A group is \textbf{type I} if for every (continuous unitary) representation \(\pi\) such that the center of \(\Hom_\mathrm{Rep}(\pi, \pi)\) is trivial we have a decomposition as a  direct sum of irreducible representations. 

\begin{example}
    Consider \(G(\A)\) the adelic points of a connected reductive LAG. This is a type one group. 
\end{example}

\begin{example}
    Consider \(G(\A)\) the adelic points of a connected reductive LAG. This is a seecond countable group. 
\end{example}

\begin{example}
    Consider \(G(\A)\) the adelic points of a connected reductive LAG. This is a unimodule group. 
\end{example}
\todo[inline]{fill}

The idea of a direct integral is review in \ref{app:direct_int} to get a quick idea consider the following example:
\begin{example}[Direct Sums]
    Let \(I\) be a countable set with the discrete sigma algebra and counting measure \(\mu\). Let \((\mathcal{H}_i)_{i\in I}\) be a collection of Hilbert spaces then
    \[\bigoplus_{i\in I} \mathcal{H}_i = \left\{ (h_i)_{i\in I}\in \prod_{i\in I} \mathcal{H}_i : \int_I \norm{h_i}_i^2 d\mu <\infty \right\}.\]
    I.e. the Hilbert space direct sum is by definition square summable sequences, but sums are just discrete integrals.
\end{example}

\begin{Theorem}[Plancherel, \cite{follandCourseAbstractHarmonic2016}, 7.44]
    The regular represntation of a type I, second countable and unimodular topological group is a direct integral of the irreducible unitary representations. 
\end{Theorem}
\begin{remark}
    The Plancherel theorem says much more in fact. Like the Peter-Weyl theorem for compact groups it doesnt just give you that some direct integral decomposition exists, it contains many more details about the topology and measure on the set of unitary irreducible representations, and which representations are associated to them in the direct integral. We are being breif as this is motivational.
\end{remark}

Thus what one wants to do is find a decomposition of the regular representation \(G(\A) \curvearrowright \mathrm{L}^2(G(\A))\).
We call such decompositions ``spectral'', alluding to the spectral theorem which provides such a decomposition in terms of the eigenvector of certain operators. Moreover these decompositions are largely proved in terms of the more general spectral theorems. So once accomplished this is another one of the tools that can be used to compartmentalise problems in automorphic forms, by dealing with representations that appear in different parts of the spectrum. 

\subsection{Langlands Decomposition of the Spectrum }
We have the Plancherel theorem but Langlands also provides a fine analysis of the spectrum using automorphic forms. The key result in this theory is the following decomposition,
\begin{Theorem}[\cite{arthurEisensteinSeriesTrace1979}, MAIN THEOREM (b)]
    There is an orthogonal decomposition of the representation of \(G(\A)\) on \(L^2(G(\Q) \backslash G(\A))\) into 
    \[L^2(G(\Q) \backslash G(\A)) = \bigoplus_{\mathscr{P}}L^2_\mathscr{P}(G(\Q) \backslash G(\A)),\]
    where \(\mathscr{P}\) runs over certain ``associate classes'' of parabolics of \(G\) and the summands are the direct integrals of spaces of \(L^2\) automorphic forms.
\end{Theorem}
These direct integrals are in fact constructed out of subspaces generated by Eisenstein series. 

The spectrum of \(L^2(G(\A))\) refers to such a decomposition. In particular we have some important ``pieces'' to such a decomposition. The piece that decomposes into a direct sum of irreducible is called the \textbf{discrete spectrum}. The compliment of the discrete spectrum is called the \textbf{continuous spectrum}. One can define cuspidal \(L^2\) functions in the exact same way as cuspidal automorphic forms \ref{cuspidal_form_definition} and then it has been shown that the \textbf{cuspidal spectrum}, the subspace of \(L^2\) consisting of cusp forms, decomposes as a direct sum \cite[9]{getzIntroductionAutomorphicRepresentations2024}. Thus the cuspidal spectrum is contained in the discrete spectrum in this case. The \textbf{residual spectrum} is defined to be the compliment of the cuspidal spectrum in the discrete spectrum.

\subsection{Residual Spectrum}\label{residual_spec}
Moeglin and Waldspurger also acheived a more fine analysis of the spectrum of \(\GL_n\) in terms of residues of Eisenstein series. 
First consider the group \(\GL_n\). We then let \(n = ab\) for positive integers \(a,b\). If \(\tau\) is an irreducible, cuspidal automorphic rep of \(\GL_a\) then Moeglin and Waldspurger construct a representation of \(\GL_{ab} = \GL_n\) called the ``Speh representation'' and denote it 
\[\Delta(\tau, b).\]
They go on to prove that as \(\tau\) and \(b\) vary these representations span the residual spectrum of \(L^2(\GL_n(F) \backslash \GL_n(\A))\) \cite[Thm. 1.1]{jiangPolesCertainResidual2013}.

This representation is formed by taking iterated residues of Eisenstein series in the sense of \cite[V]{moeglinSpectralDecompositionEisenstein1995}. For a nice survey of problems in this area, of residues of Eisenstein series, there is \cite{jiangResiduesEisensteinSeries2008a}.



\section{Automorphic L-Functions}
We don't intent to define in great detail automorphic L-functions, as there are many other better sources to learn from \cite[Part 2.III.2]{borelAutomorphicFormsRepresentations1979}\cite{shahidiEisensteinSeriesAutomorphic2010}\cite{cogdellLFUNCTIONSFUNCTORIALITY}\cite[9, 10, 11]{bumpIntroductionLanglandsProgram2004}\cite{arthurLfunctionsAutomorphicRepresenta}, we will recall the idea and then discuss some of the properties and relations with Eisenstein series and interwining operators that we will need later.

The first thing is to recall the classification of connected reductive groups defined over an algebraically closed field via root datum. A root datum is a tuple \((X, \Phi, \check{X} , \check{\Phi})\) where \(X\) and \(\check{X}\) are two free abelian groups of finite type, \(\Phi, \check{\Phi}\) are subgroups that are in duality via a perfect pairing on \(X, \check{X}\). Then each reductive group \(G\) over a number field \(F\) has associated the root datum that is associated to its base change to \(\C\). Thus to a connected reductive group over a number field we associate a connected reductive group over \C, given by the dual root datum. We call this the \textbf{dual group} of \(G\) and denote it \(\hat{G}\). The \textbf{Langlands dual group} is then the dual group producted with the \(\mathrm{Gal}(\bar{k}/k)\)
\[^L G \defeq \hat{G} \rtimes \mathrm{Gal}(\bar{k}/k).\]

\begin{example}[Classical Groups, \cite{bumpIntroductionLanglandsProgram2004}, 11.1]
    We have the following table
    \begin{table}[h]
        \centering
        \begin{tabular}{ll}
        \(G\)         & \(\hat{G}\)   \\ \hline
        \(\GL_n\)     & \(\GL_n\)     \\
        \(SO_{2n+1}\) & \(\Sp_{2n}\)  \\
        \(SO_{2n}\)   & \(SO_{2n}\)   \\
        \(\Sp_{2n}\)  & \(SO_{2n+1}\)
        \end{tabular}
        \end{table}
\end{example}

Then, using the Satake isomorphism \cite[2.2]{shahidiEisensteinSeriesAutomorphic2010}, to each unramified representation of \(G(F_\nu)\) we can associate a conjugacy class of \(^LG\), via some map call it \(c\), and hence there is a way to apply a complex representation \(r: ^LG \to \GL_n(\C)\) to representations of \(G(F_\nu)\). Thus the automorphic L-functions are defined as follows: Let \(\rho\) be a representation of \(G(\A)\), let \(r\) be a complex representation of \(^LG\) and \(s\in \C\) then 
\[L(s, \rho, r) \defeq \prod_\nu L_\nu(s, \rho_\nu, r ) = \prod_\nu \frac{1}{\det\bigl( I - r(c(\rho_\nu))q^{-s} \bigr)}  ,\]
where \(\nu\) runs over the unramified places. It is a part of the grand Langlands philosophy that there should be suitable L-functions for the ramified places satisfying very nice properties.

\begin{remark}
    The global L-functions have been defined for many groups at this point and indeed \cite{jiangPolesCertainResidual2013} uses known properties to prove their results. One should note that the questions that we are interested in are still tractable even though the L-functions might not be defined (for instance for the metaplectic group). This is because only finitely many places will ramify, and so as long as those places are neither zero or poles we can transfer questions about zeros and poles from the full global L-functions to L-functions at almost all places. 
\end{remark}

\begin{example}[Standard Representations / Classical Groups]
    In the case of classical groups it is common to see L-functions with only two entries e.g. if \(\rho\) is a representation of \(G = \Sp{2n}\) then you may see 
    \(L(s, \rho).\)
    The reason is that there is a standard representation of the dual groups of classical groups. Namely the standard representation of a matrix group inside \(\GL_n\) is the one that sends \(g\mapsto g\). It is this representation that is to be taken for the dual group in this setting.
\end{example}

\begin{example}[Rankin-Selberg, \cite{cogdellLECTURESINTEGRALREPRESENTATIONS}, 1.2, \cite{arthurLecturesAutomorphicFunctions1991a}, Ch. 2 Example. 2]
    \todo[inline]{fill}
    Let \(\nu\) be a finite place of \(\Q\) and \(\pi, \pi'\) be two unramified generic representations of \(\GL_n(\Q_\nu)\) and \(\GL_m(\Q_\nu)\) respectively. Let \(B_n\) be the standard Borel of upper triangular matricies in \(\GL_n\). Such representations have been classified \todo[inline]{reference}
    in terms of characters of \(\Q^\times_\nu\), in particular for \(\pi\) there are \(\mu_1, ..., \mu_n\) unramified characters such that 
    \[\pi \cong \Ind_{B(\Q_\nu)}^{\GL_n(\Q_\nu)} \big(\mu_1 \tensor \cdots \tensor \mu_n\big).\]
    If we fix a uniformizer \(\varpi\) of \(\Q_\nu\) then we have the so called ``Satake parameters'' \(\mu_i(\varpi)\) which determines \(\pi\) uniquely. Of course the same is true for \(\pi'\), with say characters \(\mu'_1, ..., \mu'_m\). We then define
    \[L(s, \pi\times \pi') \defeq \prod_{i,j} \frac{1}{1-\mu_i(\varpi)\mu'_j(\varpi)q^{-s}}.\]

    Consider the group \(G = \GL_n\times \GL_m\) which has dual \(\GL_n(\C) \times \GL_m(\C)\), then there is a cannonical representation 
    \[r:\GL_n(\C) \times \GL_m(\C) \to \GL_{nm}(\C). \]
    Moreover any automorphic representation \(\pi\) of \(G\) will be a tensor product of a representation of \(\GL_n\) and of \(\GL_m\)
    \[\pi \cong \pi \tensor \pi'.\]
    Then 
    \[L(s, \pi, r) = L(s, \pi\tensor \pi', r) = L(s, \pi \times \tilde{\pi}'),\]
    where the tilde denotes the contragradient.
\end{example}

\begin{example}[Dirichlet L-functions]
    Recall that a Dirichlet character \(\chi\) is a character of the group \((\Z/N\Z)^*\). Through the series of maps 
    \[A^\times \cong \Q^\times \times \R_{>0}^\times \times \hat{\Z}^\times \to (\lim \Z/N\Z)^\times \to (\Z/N\Z)^\times \to \C,\]
    one get a bijection between Dirichlet characters and finite-order Grossencharacters, i.e. characters of \(\A_F^\times/F^\times\).
    Grossencharacters have the associated L-function as they are just automorphic forms of \(\GL_1\), which generate automorphic representations. These give us the classical Dirichlet L-functions.\todo{reference? More details?}
\end{example}
Although this might seem unrelated to the current section on spectral decomposition and Eisenstein series we will see later that the two are inextricably linked
\todo[inline]{ref}

\section{L-Functions and Intertwining}\label{L_inter}
For classical groups it has been known for a while that the intertwining operator has a normalization in terms of ratios of L-functions. The point is then that the normalised operator is holomorphic and so the poles of the constant term depend entirely on the poles of the L-functions. In particular we will look at incarnations of the following statement: There is a holomorphic and non-zero intertwining operator \(N(s, w)\) such that 
    \[M(s, w) = r(s, w)N(s,w),\]
    and \(r(s, w)\) is a ratio of L-functions.

    Note that this is the global statement. There is an analogous set of conjectures for the local pieces, namely \(M = \tensor_\nu A\) the tensor over local intertwiners. Then one wants a normalisation of the local operators \(\mathscr{A}\) satisfying a long list of properties. This is extensively dealt with in \cite{shahidiProofLanglandsConjecture1990}.


    It has been known for a long time that there was some normalisation \(M = rN\) where \(r\) is a ratio of L-functions, for instance Shahidi gives the following \cite{shahidiRamanujanConjectureFiniteness1988}: Let \(\pi\) be an automorphic representation, let \(S\) be a finite set of places such that \(\pi_\nu\) is unramified for \(\nu\notin S\). We have that there are some finite dimensional complex representations \(r_1, ..., r_m\) of \(^LM\) such that 
     \[M(s, \pi)f = \bigotimes_{\nu\in S}A(s, \pi_\nu, w)f_\nu \tensor \bigotimes_{\nu\notin S} \prod_{i=1}^{m}\frac{L_S(is, \pi, \tilde{r_i})}{L_S(1+is, \pi, \tilde{r_i})} \tilde{f}_\nu.\]

    For example for a group over \Q we have the following from \cite{langlandsEulerProductsa} 
    \[M(s) = \left( \prod_\alpha\frac{\pi^{1/2}\Gamma(\frac{1}{2}\mu_\infty(s)(H_\alpha))}{\Gamma(\frac{1}{2}(\mu_\infty(s)(H_\alpha) + 1))} \right)\prod_{p \text{ prime }} \left( \prod_\alpha \frac{\frac{1}{1 - p^{\mu_p(s)(H_\alpha) + 1}}}{1 - \frac{1}{p^{\mu_p(s)(H_\alpha) }}}\right).\]
     However it was not shown until recently, and only for classical groups that this \(N\) indeed has the required properties. In particular the following theorem is sufficient for the cases dealt with in \cite{jiangPolesCertainResidual2013}:
     \begin{Theorem}[\cite{cogdellFUNCTORIALITYCLASSICALGROUPS}, 11.1]
        Suppose that \(\pi_\nu\) is a local component of a globally generic cuspidal representation \(\pi\) of \(G_n(\A)\). Then for any irreducible admissible unitary generic representation \(\pi'_\nu\) of \(\GL_m(k_\nu)\) the normalized intertwining operator \(N'(S, \pi'_\nu\times \pi_\nu, w)\) is holomorphic and non-zero for \(Re(s)\geq 0\)
     \end{Theorem}
\chapter{Constant Terms of Eisenstein Series}
This section is a discussion of the adelic constant term, especially its application to the Eisenstein series. 

Through constant terms we can define cusp forms which play a central role in the theory of automorphic forms. They appear historically as interesting examples such as the Ramanujan tau function, by a theorem of Ribet \cite[T2.3]{serreProceedingsInternationalConference1977} the Galois representation associated to a cusp form is irreducible and they form the ``base case'' for the proof of the spectral decomposition in \cite{moeglinSpectralDecompositionEisenstein1995}.

Constant terms preserve analytic properties whilst sometimes reducing the functions to more tractable forms. This is how they will be used in our calculation of poles of Eisenstein series.\label{constant_terms}


\section{Definition and Role} \label{cuspidal_form_definition}\label{sec:L-functions}
Consider \(P=MU\) a standard parabolic of a classical group \(G\) and \(\phi: U(k)\setminus G(\A) \to \C\) a measurable and locally \(L^1\) function then its \textbf{constant term} along \(P\) is defined to be \cite[I.2.6]{moeglinSpectralDecompositionEisenstein1995},
\[\phi_P :  U(\A)\setminus G(\A) \to \C,\]
\[\phi_P(g) \defeq \int_{U(k)\setminus U(\A)}\phi(ug) \mathrm{d}u.\]

We have dedicated the next chapter (\ref{ch:sigel-phi}) to showing how this is related to classical notions of constant terms. If \(\phi\) is smooth or moderate growth then so is its constant term. Moreover if \(\phi\) is an automorphic form on \(G(\A)\) then its constant term is an automorphic form on \(M(\A)\) \cite[6.5]{getzIntroductionAutomorphicRepresentations2024}.

Let \(\phi\) be an automorphic form on \(U(\A)M(k)\setminus G(\A)\) for \(P = MU\) a standard parabolic. Then \(\phi\) is \textbf{cuspidal} if for all standard parabolics \(P'\subset P\) we have that \(\phi_{P'}\) is identically zero. 

\begin{Theorem}[\cite{moeglinSpectralDecompositionEisenstein1995}, I.4.10]
		Let \(P = MU\) be a standard parabolic of \(G\). If \(\pi\) is a cuspidal automorphic representation induced from \(P\), then for a fixed \(\phi \in\mathcal{A}_0(U(\A)M(k)\setminus G(\A))_\pi \) the Eisenstein series \(E\) can be thought of as a function from some open subset of the cuspidal datum \(\mathfrak{P}\) into \(L^2_{\mathrm{loc}}(G(\A))\) given by 
		\[E(p)(g) = \sum_{\gamma \in P(k)\backslash G(k)} \lambda.\phi(\gamma g), \;\;\; p\in \mathfrak{P},\; g\in G(\A),\]
		where it converges. 
		If \(D\subseteq \mathfrak{P}\), is an open subset minus a finite number of points on which \(E\) is holomorphic then E has a holomorphic continuation to the finite number of points if and only if the constant term of \(E_Q\) has a holomorphic continuation to these finite number of points for all standard parabolics \(Q\).
    \end{Theorem}
    
    \begin{remark}
    	The theorem in Moeglin and Waldspurger is proved in much more generality, however after sufficient symbol pushing this is the essence. 
    \end{remark}
    So one can say that the poles of an Eisenstein series are controlled by its constant terms. We can say more:
    
        
        \begin{theorem}[\cite{moeglinSpectralDecompositionEisenstein1995}, II.1.7]
        	The constant term of an Eisenstein series induced from a standard maximal parabolic \(P\) is zero along any other standard parabolic \(P'\) unless \(P = P'\).
        \end{theorem}
        
        Putting these two theorems together we see that for an Eisenstein series induced from a maximal parabolic \(P\), has a holomorphic continuation to a point if and only if its constant term along \(P\) has a holomorphic continuation. 
      
      

\section{Constant Terms of Eisenstein Series}\label{const_eisenstein}
This computation forms the heart of a well known theorem, \cite[Prop 10.4.2]{getzIntroductionAutomorphicRepresentations2024}\cite[II.1.7]{moeglinSpectralDecompositionEisenstein1995}\cite[6.2]{shahidiEisensteinSeriesAutomorphic2010}, although for an amateur the detail is lacking in other presentations. Notice that the Eisenstein series has a full \(G(k)\) invariance and so we can take its constant terms along \textit{any} standard parabolic.

Also note that we assume the computations are taking place in the domain of \(\mathfrak{P}\) on which the Eisenstein series is given by the sum formula. By the uniqueness of meromorphic continuation taking constant terms commutes with meromorphic continuation.

\subsection{In General}
We will use the following Lemmas to give a simplified expression of the constant term of an Eisenstein series. First fix \(P = MU\) and \(P' = M'U'\) two standard parabolics of a suitable group G over a number field F, with \(E(x, \phi, \lambda)\) defined from P as in section \ref{sec:eisenstein-series}.
\todo[inline, color=blue]{define the Weyl group defined here}
    \begin{Lemma}\label{lem:1}
        \[P(F)\setminus G(F) \cong \coprod_{w\in W_{M'}\setminus W_G / W_{M}} P'(F)\cap wP(F)w\inv \setminus P'(F)\]
    \end{Lemma}
    \proofbar{
        Consider the Bruhat decomposition:
        \[G(F) =\coprod_{w\in W_{M'}\setminus W_G / W_{M}} P(F)w\inv P'(F) \]
        then because the action of \(P(F)\) keeps the disjoint sets disjoint we can move the quotient through and get
        \[P(F)\setminus G(F) = \coprod_w  P(F) \setminus P(F)w\inv P'(F)\]
        so we analyse the summands, by the second isomorphism theorem we have a bijection
        \[P(F)\setminus P(F)w\inv P'(F) \cong P(F)\cap P'(F) \setminus w\inv P'(F) \]
        now if \([w\inv p] \in P(F)\cap P'(F) \setminus w\inv P'(F) \) then its represented by some \(pw\inv p'\) where \(p\in P(F)\cap P'(F)\) and hence multiplying by \(w\), in particular an isomorphism, gives \(wpw\inv p'\in wP(F)w\inv \times P'(F)\) and so 
        \[w(P(F)\cap P'(F) \setminus w\inv P'(F)) \cong wP(F)w\inv \cap P'(F) \setminus P'(F)\]
    }

    \begin{Lemma}\label{lem:2}
        Let \(m'\in M'(F),  u'\in U'(F)\) then 
        \[m'u' \in wP(F)w\inv \iff m'\in wP(F)w\inv \text{ and  }\;\; u'\in (m')\inv wP(F)w\inv m'\]
    \end{Lemma}
    \proofbar{
        The forward implication is stated in \cite{getzIntroductionAutomorphicRepresentations2024}, the converse follows from some algebra:
        First let \(m' = wp_1w\inv\) and \(u' = (m')\inv wp_2w\inv m'\) then 
        \begin{equation*}
            \begin{aligned}
                m'u' &= (wp_1w\inv)\inv wp_2w\inv wp_1w\inv\\
                     &= wp_1\inv w\inv wp_2w\inv wp_1w\inv\\
                     &= wp_1\inv p_2p_1w\inv \in wP(F)w\inv\\
            \end{aligned}
        \end{equation*}
    }
    Taking the contrapositive of this lemma will be used below. This is because our sums will be over quotients like \(A\setminus B\) and therefore summing over the ``elements'' in B that are not in A; by our lemma would be the same as summing over two different such quotients.
Now we will apply our lemmas to simplify and make more explicit the constant term of an Eisenstein series. Denote \([U']\defeq U'(F)\setminus U'(\A)\)
    \begin{equation*}
        \begin{aligned}
            E_{P'}( \phi, \lambda, x) &= \int_{U'(F)\setminus U'(\A)} E( \phi, \lambda, nx) du\\
                                     &= \int_{[U']} \sum_{\delta\in P(F)\setminus G(F)} \lambda.\phi(\delta nx)  du\\
                                    (\text{Lemma }\ref{lem:1}) \;\;\; &= \int_{[U']} \sum_{\delta\in \coprod_{w\in W_{M'}\setminus W_G / W_{M}} P'(F)\cap wP(F)w\inv \setminus P'(F)} \lambda.\phi(\delta ux)  du\\
                                     &= \sum_{ w\in W_{M'}\setminus W_G / W_{M}}\int_{[U']} \sum_{p'\in P'(F)\cap wP(F)w\inv \setminus P'(F)} \lambda.\phi( w\inv p'ux)  du\\
        \end{aligned}
    \end{equation*}
    Now apply Lemma \ref{lem:2} to the above sum and we get the equality 
    \begin{align*}
    	&= \sum_{ w} \sum_{m'\in M'(F)\cap wP(F)w\inv\setminus M'(F)} \int_{[U']} \sum_{u'\in U'(F)\cap (m')\inv wP(F)w\inv m' \setminus U'(F)} \lambda.\phi( w\inv m'u'ux)  du\\
    	&(\text{Change Var}) \;\;\; = \sum_{ w} \sum_{m'} \int_{[U']} \sum_{n'\in U'(F)\cap wP(F)w\inv\setminus U'(F) } \lambda.\phi( w\inv u'um'x)  du\\
    	&(\text{Unfold}) \;\;\;\;\;\;\;\;\;\;\; = \sum_{ w} \sum_{m'} \int_{U'(F)\cap wP(F)w\inv \setminus U'(\A)} \lambda.\phi( w\inv um' x)  du.\\
    \end{align*}

    The change of variables is \((m', u') \mapsto ((m')\inv u' m', (m')\inv u' m')\).

\subsection{Constant Terms of Cuspidal Eisenstein Series}
\begin{Lemma}\label{lem:4}
        For \(w\in W_{M'}\setminus W_G / W_{M} \) we have that \(w\inv P'w\cap M\) is a standard parabolic of \(M\) with Levi \(w\inv M'w\cap M\) and unipotent \(w\inv U'w\cap M\).
    \end{Lemma}
    \proofbar{
        This is \cite[10.4.1]{getzIntroductionAutomorphicRepresentations2024} stated without proof. They give the reference \cite[V.4.6]{renardREPRESENTATIONSGROUPESREDUCTIFS} which is in French..
    }
    \begin{Lemma}\label{lem:5}
        \[w\inv U' w \cap P = (w\inv U' w \cap M)(w\inv U' w \cap U).\]
    \end{Lemma}
    \proofbar{
        \cite[10.4.1]{getzIntroductionAutomorphicRepresentations2024} has some decompositions, as well as the standard decomposition of \(P=MU\) I think I could prove this...
    }
    \begin{Lemma}\label{lem:6}
        \[c\setminus (b\setminus a )= (bc)\setminus a\]
    \end{Lemma}
	\todo[inline,color=blue]{need to fill in these lemmas}
    Continuing the computation of the constant term above, we will focus purely on the inner integral now
    \begin{equation*}
        \begin{aligned}
            \int_{U'(F)\cap wP(F)w\inv \setminus U'(\A)}& \lambda.\phi( w\inv um' x)  du \\&= \int_{w\inv U'(F)w \cap P(F) \setminus w\inv U'(\A)w} \lambda.\phi( uw\inv m' x)  du \\
            (\text{Lemma \ref{lem:5}})&= \int_{(w\inv U' w \cap M)(w\inv U' w \cap U)(F) \setminus w\inv U'(\A)w} \lambda.\phi( uw\inv m' x)  du . \\
        \end{aligned}
    \end{equation*}
    where the first equality is the change of variables \(w\inv u w\mapsto u \). Denote \(A = (w\inv U'(F) w \cap U(F) ) \setminus w\inv U'(\A)w \). If we apply Lemma \ref{lem:6} and unfold we get the equality
    \[= \int_{(w\inv U'(\A)w \cap M(\A)) \setminus A} \int_{w\inv U'(F) w \cap M(F) \setminus w\inv U'(\A)w \cap M(\A)} \lambda.\phi( u_1 u_2 w\inv m' x)  du_1 du_2.\]
     Now look at the inner integral here more closely 
    \[ \int_{w\inv U'(F) w \cap M(F) \setminus w\inv U'(\A)w \cap M(\A)}\lambda. \phi( u_1 u_2 w\inv m' x)  du_1 du_2,\]
    applying Lemma \ref{lem:6} we see that this is a constant term for a parabolic of \(M\), of the function \(m\mapsto \phi(m u_2 w\inv m' x)\). 
    \begin{Lemma}
        \(u_2 w\inv m' x \in K\) with variables as above.
    \end{Lemma}
    This was in complete generality. If we now assume further that the Eisenstein series was induced from a \textit{cuspidal} automorphic representation, then \(m\mapsto \phi(mk)\) is a cusp form and therefore this last integral will vanish whenever \(w\inv U'w \cap M \neq \{1\}\), because in that case the inner integral doesn't exist (its over a point).

    \subsection{Constant Term Of Eisenstein Series for Conjugate Levis}\label{constant_conjugate_levi}
    If we now assume that \(M' = wMw\inv\) and recall the definition of our intertwining operator from section \ref{sec:eisenstein-series} we can use the following 
    \begin{Lemma}[\cite{moeglinSpectralDecompositionEisenstein1995} II.1.7 (6)]
        \[U'(k) \cap wP(k) w\inv = U'(k) \cap wU(k)w\inv,\]
    \end{Lemma}
    to see that 
    \begin{equation*}
        \begin{aligned}
             E_{P'}( \phi, \lambda, x) &= \sum_{ w} \sum_{m'} \int_{U'(F)\cap wP(F)w\inv \setminus U'(\A)} \lambda.\phi( w\inv um' x)  du \\
             &=  \sum_{ w} \sum_{m'} \int_{U'(k) \cap wU(k)w\inv \setminus U'(\A)} \lambda.\phi( w\inv um' x)  du \\
             &= \sum_{ w} \sum_{m'} M(w, \pi)(\lambda.\phi)(x)
        \end{aligned}
    \end{equation*}
    In particular we can combine the conjugate and cuspidal cases to get a much simpler expression for some constant terms of some Eisenstein series, we will go through a detailed example in the final chapter \ref{ch:jiang}.
    


    
\chapter{Siegel Phi Function}


\section{Integration Lemmas}
\begin{Theorem}\label{integrate_unitary_char}
	If \(G\) is a locally compact Hausdorff group with a left Haar measure \(\mu\) and if \(\chi\colon G\to \mathbf C^\times\) is a non-trivial character on \(G\), then
	\[ \intof{G}{\chi(g)}{\mu(g)} = 0. \]
\end{Theorem}
\proofbar{
	Pick an element \(h\) of \(G\) such that \(\chi(h)\neq 1\).
	The equation above then follows from
	\[ \intof{G}{\chi(g)}{\mu(g)} = \intof{G}{\chi(hg)}{\mu(g)} = \intof{G}{\chi(h)\chi(g)}{\mu(g)} = \chi(h) \intof{G}{\chi(g)}{\mu(g)}. \square\]
}
Integrating trivial characters gives the volume of the measure space which we typically normalise to be one.

\begin{Theorem}[\cite{garrettModernAnalysisAutomorphic} 5.2, \cite{follandCourseAbstractHarmonic2016} Thm 2.49]
	Let \(H\leq G\) be a closed subgroup. If \(H\setminus G\) has a right G invariant measure (iff their modular functions agree on H) then the integral is unique up to scalar, namely for a given Haar measures dh on H and dg on G there is a unique invariant measure dq on \(H\setminus G\) such that for all \(f\in C_c^0(G)\)
	\[\int_{H\setminus G}\int_H f(hq)dhdq = \int_G f(g) dg\]
\end{Theorem}
Note that this quotient may not be a group, because H is not required to be normal.


\section{Siegel Phi Operator}
Here we give an example of the constant term which connects it to the classical picture. We thank Chengjing Zhang for showing us this example, and present it here because we cannot find it in the literature. We deal only with the classical Siegel modular forms of full level and moreover are less explicit with the steps as they should be clear after exposure to the previous arguments. 

Because we are trying to connect this to the classical picture it is most convenient to think of things in the Archimedean places, recall the way that modular forms are automorphic forms most naturally in the archimedian sense (\cite[6.2]{getzIntroductionAutomorphicRepresentations2024}) \cite{emertonCLASSICALMODULARFORMS}\cite{bumpAutomorphicFormsRepresentations1997}\cite{booherVIEWINGMODULARFORMS}. So for this section alone, by automorphic form we will mean automorphic forms on the Archimedean places, and the constant term will be taken only on the Archimedean part: i.e. for \(f: G(\R) \to \C\) and automorphic its constant term along a parabolic of G, call it \(P=MN\), is \cite[8.6]{getzIntroductionAutomorphicRepresentations2024}
\[f(x)_P = \int_{N(\Z)\backslash N(\R)}f(xn) \mathrm{d}n.\]


\subsection{Siegel Modular Forms}
We collect some definitions from \cite{bruinier123ModularForms2008} to fix notation. Let the Siegal upper half plane be defined as 
\begin{equation*}
	\begin{aligned}
		\mathcal{H}_g &\defeq \{\tau \in \mathrm{M}_{g\times g}(\C) : \tau \text{ is symmetric and has positive definite imaginary part}\} \\
		& \cong \Sp_{2g}(\R) / U(g)
	\end{aligned}
\end{equation*}
where the isomorphism is as analytic manifolds  and 
\[U(g) \defeq \left\{\begin{pmatrix}
	A & B\\
	-B & D
\end{pmatrix}\in \Sp_{2g}(\R) : AA^t + BB^t = 1\right\}\]

For every \(\gamma= (A \; B; \; C\; D) \in \Sp_{2g}(\Z)\) and \(\tau \in \mathcal{H}_g\) we have the action
\[\gamma.\tau = (A\tau + B)(C\tau + D)\inv \]

We say that a holomorphic function \(f: \mathcal{H}_g \to \C\) is a (classical) Siegel modular form of weight \(k\) if 
\[f(\gamma.\tau) = \det(C\tau + D)^kf(\tau)\]
with the extra condition that if \(g = 1\) it must be holomorphic at \(\infty\). Because \(\Sp_2 = \SL_2\) this is a strict generalisation of an (elliptic) modular form.

The space of Siegel modular forms of weight \(k\) and genus g is denoted \(\mathcal{M}_k(\Sp_{2g}(\Z))\). There is a useful operator know as the Siegel Phi Operator which allows you to lift known modular forms from lower genus to higher genus \cite[5]{bruinier123ModularForms2008}
\[\mathcal{M}_k(\Sp_{2g}(\Z)) \xrightarrow{\Phi} \mathcal{M}_{k}(\Sp_{2(g-1)}(\Z))\]
defined by the limit for \(\tau\in \mathcal{H}_{g-1}\)
\[\Phi(f)(\tau) \defeq \lim_{t\to \infty} f\begin{pmatrix}
	\tau & \\
	& it 
\end{pmatrix}\]
in this context a cusp form is defined to be a Siegel modular form in the kernel of the Siegel \(\Phi\) operator and so it is natural to wonder if there is a constant term that is being taken here. 

\subsection{Automorphising}
Given a Siegel modular form \(f\in \mathcal{M}_k(\Sp_{2g}(\Z))\) we can associate an automorphic form
\[\tilde{f} : \Sp_{2g}(\R)\to\C, \qquad \begin{pmatrix} a & b\\ c & d\end{pmatrix}\mapsto \det(ci+d)^{-k} f\Bigl((ai+b)(ci+d)\inv\Bigr), \]
where \(a,b,c,d\) are \(g\times g\) matrices such that \(\bigl(\begin{smallmatrix} a & b\\ c & d\end{smallmatrix}\bigr)\in\Sp_{2g}(\R)\).Fix the Borel of upper triangular matrices. Now for \(1\leq r\leq g-1\) an integer we have the standard maximal parabolic of \(\Sp_{2g}\), \(P_r = M_rN_r\) such that 
\[M_r \cong \GL_r\times \Sp_{2(g-r)}\]

\begin{Theorem}[Zhang]
	If \(f\) is a classical Siegel modular form of weight \(k\) and degree \(g\), then
	\begin{equation} 
		\tilde f_{P_r}(u\gamma) = \det u^k\cdot (\Phi^{r} f)^\sim(\gamma)
	\end{equation}
	for every element \(\gamma\) of \(\Sp_{2(g-r)}(\R)\) and every element \(u\) of \(\GL_{r}(\R)\).
	
	In particular 
	\[\tilde{f}_{P_{g-1}}\begin{pmatrix} a & 0 & b & 0\\ 0 & 1 & 0 & 0\\ c & 0 & d & 0\\ 0 & 0 & 0 & 1\end{pmatrix} = (\Phi f)^\sim\begin{pmatrix}
		a & b\\
		c& d
	\end{pmatrix}\]
\end{Theorem}

This shows that perhaps the correct generalisation of the Siegel Phi function is just the constant term that we all know and love. We could also attempt to expand this to Siegel modular forms that are vector valued or not of full level. 

The only other work on generalising the Siegel \(\Phi\) operator that we could find appears in \cite{grenierANALOGUESIEGELXOPERATOR2024}. 
Grenier formulates the \(\Phi\) operator in the language of symmetric spaces \cite[Ch. 2]{terrasHarmonicAnalysisSymmetric2016} and then shows that the analogous definition in the case of ``automorphic forms'' in the sense of the symmetric space \(\mathscr{P}_n/\GL_n(\Z)\) of symmetric positive definite real matrices \cite[1.5.1]{terrasHarmonicAnalysisSymmetric2016} behaves in the same way. Namely his \cite[Thm. 2]{grenierAnalogueSiegelFOperator1992} shows that it sends an automorphic form for \(
GL_n(\Z)\) to an automorphic form for \(\GL_{n-1}(\Z)\). The point is that the \(\Phi\) operator can be defined in the generality of symmetric spaces and Grenier shows that at least in one other case it still preserves the relevant notion of automorphic form. This suggests two things that would be interesting to investigate; using the classification of symmetric spaces is it possible to give a uniform definition of the \(\Phi\) operator following Grenier and does this definition agree with the constant term in the way that the Siegel \(\Phi\) operator does. With my limited knowledge of symmetric spaces this seems to be very tractable.

\subsection{Base Case}
The base case is very instructive, it deals with modular forms. So consider \(f\) a (elliptic) modular form of full level and weight k, which has a Fourier expansion given by 
\[f(z) = \sum_{n\geq 0} a_ne^{2\pi i nz }\]
Then one can verify that
\[\tilde f \begin{pmatrix}
	a & b\\
	c & d
\end{pmatrix} = (ci+d)^{-k} f\Bigl(\frac{ai+b}{ci+d}\Bigr)\]
is an automorphic form on \(\Sp_2\). The only non-trivial parabolic P is the one of upper triangular matricies, with Levi and unipotant given respectively 
\[M = \begin{pmatrix} m & 0\\ 0 & m^{-1}\end{pmatrix}\cong \GL_1 , \;\;\; N = \begin{pmatrix} 1 & b\\ 0 & 1\end{pmatrix} \cong \mathbb{G}_a\]
along which we can now compute the constant term 
\begin{equation*}
	\begin{aligned}
		\tilde f_P(m)
		& = \int_{N(\Z)\backslash N(\R)}\tilde f (mb) \mathrm{d}b\\
		& =  \int_{\Z\backslash\R}\tilde f \begin{pmatrix} m & mb\\ 0 & m^{-1}\end{pmatrix}\mathrm{d}b\\
		& = \int_{\Z\backslash\R} m^k f(m^2i+m^2b) \mathrm{d}b\\
		& = m^k a_0 \\
	\end{aligned}
\end{equation*}
We have chosen normalisation to remove the usual factor of \(1/2\pi\) in the constant term of the Fourier series. Moreover we see that
\[\Phi(f)= \lim_{t\to \infty} f(it) =\lim_{t\to \infty} \sum_{n\geq 0} a_ne^{-2\pi nt }  =  a_0\]

\subsection{Simplifying the Constant Term}
As we saw in \ref{maximal_parabolic} for \(1\leq r\leq g-1\) an integer we have the standard maximal parabolic of \(\Sp_{2g}\), \(P_r = M_rN_r\) such that 
\[M_r \cong \GL_r\times \Sp_{2(g-r)}\]
which can be given the explicit matrix representations 
\[m(\gamma, A) \defeq \begin{pmatrix}
	A &&& \\
	&a&&b \\
	&&(A^t)\inv& \\
	&c&&d \\
\end{pmatrix}, \;\;\; A\in \GL_r(F), \; \gamma = \begin{pmatrix}
	a & b\\
	c & d \\
\end{pmatrix} \in \Sp_{2(g-r)}(F) \]

and unipotent 
\[ n(s;h,k) \defeq \begin{pmatrix} 1 & 0 & 0 & h\\ -k^t & 1 & h^t & s+h^t k\\ 0 & 0 & 1 & k\\ 0 & 0 & 0 & 1 \end{pmatrix}, \;\;\; h, k\in\mathrm{Mat}_{(g-r)\times r}(\R),\; s\in\mathrm{Sym}_{r}(\R)\]
We have the following short exact sequence \todo[inline]{prove it}
\[ 1\to \mathrm{Sym}_{r}(\R)\to N_r(\R)\to \mathrm{Mat}_{(g-r)\times r}(\R)\times\mathrm{Mat}_{(g-r)\times r}(\R) \to 1. \]
which we will use to unfold our integral below, for compactness we define \(H_r \defeq \mathrm{Mat}_{(g-r)\times r}\). We will now denote \([G] \defeq G(\Z)\backslash G(\R)\) and compute the constant term
\begin{align}
	\tilde f_{P_r}\bigl(m(\gamma, A)\bigr)
	& = \intof{[N_r]}{\tilde f\bigl(n m(\gamma, A)\bigr)}{n} \notag\\
	& = \intof{[H_r\times H_r]}{\intof{[\mathrm{Sym}_{g-r}]}{\tilde f\bigl(n(s; h, k) m(\gamma, A)\bigr)}{s}}{(h,k)} \notag\\
	& = \intof{[H_r]}{\intof{[H_r]}{\intof{[\mathrm{Sym}_{g-r}]}{\tilde f\bigl(n(s; h, k) m(\gamma, A)\bigr)}{s}}{h}}{k}.
\end{align} 

Now we focus on simplifying the integrand. We want an explicit form of the matrix so we can relate it back to the value of the un-lifted Siegel modular form \(f\); simply multiply the matrices gives, where (all rings are commutative) \(A^{-t} \defeq (A^t)\inv\)
\[
n(s; h, k) m(\gamma, A) =
\begin{pmatrix}
	a & 0 & b & h A^{-t}\\
	-k^t a + h^t c & A & -k^t b + h^t d & s A^{-t} + h^t k A^{-t}\\
	c & 0 & d & k A^{-t}\\
	0 & 0 & 0 & A^{-t}
\end{pmatrix}.
\]
because \(a,b,c,d \in \mathrm{Mat}_{(g-r)\times (g-r)}, A \in \mathrm{Mat}_{r\times r}\) we see that the \(g\times g\) blocks that we now need to take the determinant of are the \(4\times 4\) corners of this picture, hence the matrices below should all be in \(\mathcal{H}_g\subseteq \mathrm{Mat}_{g\times g}\)

\begin{align*}
	&\tilde{f}(n(s; h, k) m(\gamma, A) ) \\
	 &= \det\left(\begin{pmatrix}
		c & 0 \\
		0 & 0
	\end{pmatrix}i+ \begin{pmatrix}
		d & kA^{-t} \\
		0 & A^{-t}
	\end{pmatrix} \right)^{-k} \cdot \\
	&f\left(         \left(\begin{pmatrix}
		a & 0\\
		-k^ta + h^tc & A
	\end{pmatrix}i+\begin{pmatrix}
		b & hA^{-t} \\
		-k^tb + h^td & sA^{-t} + h^tkA^{-t}
	\end{pmatrix}\right)   \left(\begin{pmatrix}
		c & 0 \\
		0 & 0
	\end{pmatrix}i+ \begin{pmatrix}
		d & kA^{-t} \\
		0 & A^{-t}
	\end{pmatrix} \right)\inv      \right) \\
	&= \det\left( \begin{pmatrix}
		ic + d & kA^{-t} \\
		0 & A^{-t}
	\end{pmatrix}\right)^{-k} \cdot \\
	&f\left( \begin{pmatrix}
		ia + b &  hA^{-t}\\
		-k^t(ia +b) + h^t(d + ic) & iA + sA^{-t} + h^tkA^{-t}
	\end{pmatrix}  \begin{pmatrix}
		ic + d & kA^{-t} \\
		0 & A^{-t}
	\end{pmatrix}\inv      \right) \\
	&=\left(\frac{\det(ic + d)}{\det(A)}\right)^{-k} \cdot \\ &f\left( \begin{pmatrix}
		ia + b &  hA^{-t}\\
		-k^t(ia +b) + h^t(d + ic) & iA + sA^{-t} + h^tkA^{-t}
	\end{pmatrix}   \begin{pmatrix}(ci+d)^{-1} & -(ci+d)^{-1} k\\ 0 & A^t \end{pmatrix} \right) \\
	&= \left(\frac{\det(A)}{\det(ic + d)}\right)^{k} f\begin{pmatrix} \tau & -\tau k + h\\ -k^t \tau + h^t & k^t \tau k + A A^t i + s \end{pmatrix}, \;\;\; \tau \defeq (ai+b)(ci + d)\inv \\
	%&= \left(\frac{\det(A)}{\det(ic + d)}\right)^{k} f\bigl(n(s; h, k) m(\gamma,A )(i 1_g)\bigr) I think g here should be 2g but then the multiplication seems wrong IDK what he meant by this..
\end{align*} 
So we have shown that 
\begin{align*}
	&\tilde f_{P_r}\bigl(m(\gamma, A)\bigr) \\
	 &= \intof{[H_r]}{\intof{[H_r]}{\intof{[\mathrm{Sym}_{g-r}]}{  \left(\frac{\det(A)}{\det(ic + d)}\right)^{k} f\begin{pmatrix} \tau & -\tau k + h\\ -k^t \tau + h^t & k^t \tau k + A A^t i + s \end{pmatrix}   }{s}}{h}}{k}\\
	&= \left(\frac{\det(A)}{\det(ic + d)}\right)^{k} \intof{[H_r]}{\intof{[H_r]}{\intof{[\mathrm{Sym}_{g-r}]}{  f\begin{pmatrix} \tau & -\tau k + h\\ -k^t \tau + h^t & k^t \tau k + A A^t i + s \end{pmatrix}   }{s}}{h}}{k}\\
\end{align*}

Again lets focus on this integrand \(f\begin{pmatrix} \tau & -\tau k + h\\ -k^t \tau + h^t & k^t \tau k + A A^t i + s \end{pmatrix}\) and compute its Fourier expansion, see \cite[3.4]{bruinier123ModularForms2008}. Recall that a symmetric matrix \(n\in \GL_g(\Q)\) is called half integral if \(2n\) is integral with even diagonal entries, then a Siegel modular form has a Fourier expansion of the form
\[f(z) = \sum_{n \text{ half integral}}a(n) e^{2\pi i \mathrm{Tr}(nz)} \]
First the space of half integral \(g\times g\) matrices, \(\mathrm{HI}_g\), decomposes as a direct sum via the (additive) group isomorphism \todo[inline]{prove it}
\[ \mathrm{HI}_{g-r} \oplus \tfrac{ 1}{ 2} \mathrm{Mat}_{ r\times (g-r)}(\Z) \oplus \mathrm{HI}_{r}\to\mathrm{HI}_g, \qquad (n, m, l)\mapsto \begin{pmatrix} n & m\\ m^t & l \end{pmatrix}, \]
thus unfolding the (discrete) integral we get 
\begin{align*}
	f\begin{pmatrix} \tau & -\tau k + h\\ -k^t \tau + h^t & k^t \tau k + A A^t i + s \end{pmatrix} &=   \sum_{n\in\mathrm{HI}_{g-r}} \sum_{m\in\frac{1}{2} \mathrm{Mat}_{ r\times (g-r)}(\Z)} \sum_{l\in\mathrm{HI}_{r}} a\begin{pmatrix} n & m\\ m^t & l \end{pmatrix} \\
	&\exp \left(2\pi i \mathrm{Tr} \begin{pmatrix} n & m\\ m^t & l \end{pmatrix}\begin{pmatrix} \tau & -\tau k + h\\ -k^t \tau + h^t & k^t \tau k + A A^t i + s \end{pmatrix} \right)  \\
\end{align*}
because all the block sizes are compatible we can ``block multiply'' the inner matrices and because we are taking the trace we can forget about off diagonal entries
\begin{align*}
	&\begin{pmatrix} n & m\\ m^t & l \end{pmatrix}\begin{pmatrix} \tau & -\tau k + h\\ -k^t \tau + h^t & k^t \tau k + A A^t i + s \end{pmatrix}\\ &= 
	\begin{pmatrix} n\tau + m(-k^t \tau + h^t ) & \ast\\ \ast & m^t(-\tau k + h) + l( k^t \tau k + A A^t i + s) \end{pmatrix}
\end{align*}
putting this into our Fourier expansion
\begin{align*}
	&f\begin{pmatrix} \tau & -\tau k + h\\ -k^t \tau + h^t & k^t \tau k + A A^t i + s \end{pmatrix} \\
	&= \sum_{n} \sum_{m} \sum_{l} a\begin{pmatrix} n & m\\ m^t & l \end{pmatrix} \exp \Big(2\pi i \big( &\mathrm{Tr} (n\tau) +  \mathrm{Tr} (m(-k^t \tau + h^t )) +  \mathrm{Tr} (m^t(-\tau k + h)) \\ & &  +  \mathrm{Tr} (l( k^t \tau k + A A^t i + s))\big)\Big)  \\
\end{align*}

If we denote \(T_l \defeq \mathrm{Tr} (l( k^t \tau k + A A^t i + s))\) and \todo{my Tm differs from Chengjing}
\[T_{m} \defeq \mathrm{Tr} (m(-k^t \tau + h^t )) +  \mathrm{Tr} (m^t(-\tau k + h)) = \mathrm{Tr}(-mk^t\tau - m^t\tau k) + \mathrm{Tr}(mh^t + m^th) \defeq T_{m,k} + T_{m,h}\]
we can substitute this back into our constant term\todo[inline]{Converges uniformly a priori on compact sets, well I don't know if I can swap all these sums haha}
\resizebox{\linewidth}{!}{
	\begin{minipage}{\linewidth}
		\begin{align*}
			&\tilde f_{P_r}\bigl(m(\gamma, A)\bigr)\\
			&= \left(\frac{\det(A)}{\det(ic + d)}\right)^{k} \intof{[H_r]}{\intof{[H_r]}{\intof{[\mathrm{Sym}_{g-r}]}{ \sum_{n} \sum_{m} \sum_{l} a\begin{pmatrix} n & m\\ m^t & l \end{pmatrix}\exp \left(2\pi i (\mathrm{Tr} (n\tau) +  T_m + T_l) \right)
					}{s}}{h}}{k}\\
			&= \left(\frac{\det(A)}{\det(ic + d)}\right)^{k} \sum_{n} \sum_{m} \sum_{l} a\begin{pmatrix} n & m\\ m^t & l \end{pmatrix}e^{2\pi i \mathrm{Tr} (n\tau)}
			\intof{[H_r]}{\intof{[H_r]}{\intof{[\mathrm{Sym}_{g-r}]}{  e^{2\pi i(T_m + T_l)}  }{s}}{h}}{k}\\
			&= \left(\frac{\det(A)}{\det(ic + d)}\right)^{k} \sum_{n} \sum_{m} \sum_{l} a\begin{pmatrix} n & m\\ m^t & l \end{pmatrix}e^{2\pi i \mathrm{Tr} (n\tau)}
			\intof{[H_r]}{e^{2\pi iT_{m,k}}\intof{[H_r]}{e^{2\pi i T_{m,h}} \intof{[\mathrm{Sym}_{g-r}]}{  e^{2\pi iT_l}  }{s}}{h}}{k}\\
		\end{align*}
	\end{minipage}
}

Now we use that the integration of unitary characters is very simple \ref{integrate_unitary_char} and the fact that 
\[s\mapsto  e^{2\pi iT_l} \]
is a non-trivial unitary character of \(\mathrm{Sym}_{g-r}\) whenever \(l\neq 0\) to get that 
\[\intof{[\mathrm{Sym}_{g-r}]}{  e^{2\pi iT_l}  }{s} = \begin{cases}
	1, & l=0 \\
	0, & l\neq 0
\end{cases}\]
we repeat this trick with the second integral, which enforces that \(m = 0\) and end up with 
\begin{align*}
	\tilde f_{P_r}\bigl(m(\gamma, A)\bigr)
	&=\left(\frac{\det(A)}{\det(ic + d)}\right)^{k} \sum_{n\in\mathrm{HI}_{g-r}} a\begin{pmatrix} n & 0\\ 0 & 0 \end{pmatrix}e^{2\pi i \mathrm{Tr} (n\tau)}\\
\end{align*}
but by \cite[3.5]{bruinier123ModularForms2008} we know that the Fourier expansion of the Siegel Phi operator is 
\[(\Phi^{r} f)(\tau) = \sum_{n\in\mathrm{HI}_{g-r}} a\begin{pmatrix} n & 0\\ 0 & 0 \end{pmatrix} e^{2\pi i\mathrm{Tr}(n \tau)}.\]
hence 
\begin{align*}
	\tilde f_{P_r}\bigl(m(\gamma, A)\bigr)
	&=\left(\frac{\det(A)}{\det(ic + d)}\right)^{k} \Phi^r(f)(\tau)\\
	&= \det(A)^k (\Phi^r(f))^{\sim}(\gamma)
\end{align*}
which concludes the proof.
\begin{FlushRight}
	\(\square\)
\end{FlushRight}
\chapter{Poles of Residual Eisenstein Series}
Our goal here is to exposit and survey the work in papers such as \cite{brennerNotesAnalyticProperties2009}\cite{jiangPolesCertainResidual2013}\cite{ginzburgTopFourierCoefficients2021} and perhaps give a trivial extension of them.

\section{Residual Eisenstein Series}
\cite{brennerNotesAnalyticProperties2009} gave an analysis of the residual poles of Eisenstein series attached to \(\Sp_{2n}\), there were some minor errors that were corrected in \cite{jiangPolesCertainResidual2013} where they give esssentially the same proof; theirs however works for the other classical groups. For our purposes, the case of \(\Sp_{2n}\), as a group defined over \(F\) a number field, is most relevant, and we shall therefore focus exclusively on this case, however it should be noted that this limitation in the non-covering case is artificial, although it does simplify things a little by removing some casework, and we hope also in the covering case to be able to remove it in future work. 

We fix an \(n\in \N\) and the Borel of upper triangular matricies in \(\Sp_{2n} \), then we look at partitions of \(n = r + m\), where \todo[inline]{what are the ranges of these}. Then as we saw in \ref{maximal_parabolic} there corresponds a maximal standard parabolic of \(\Sp_{2n}\), which we denote \(P_r = M_rN_r\), such that the Levi component is 
\[\GL_r\times \Sp_{2m} \]
As we saw in \ref{ex:characters} the space of characters \(X^{\Sp_{2n}}_{M_r}\) is one dimensional by the maximality of \(P_r\). If we look at the divisors of \(r = ab\) \todo[inline]{check the exact ranges, will depend on the range of r} and fix a \(\tau\), an irreducible unitary cuspidal automorphic representation of \(\GL_a\), then from \ref{residual_spec} we know that \(\Delta(\tau, b)\) is a residual representation of \(\GL_{ab} = \GL_r\). Now we take an irreducible generic cuspidal automorphic representation \(\sigma\) of \(\Sp_{2m}\), and so their tensor product \(\Delta(\tau, b) \tensor \sigma\) gives a representation of \(\GL_{r}\times \Sp_{2m}\) and hence of the Levi \(M_r\). We now consider the Eisenstein series attached to this representation, namely if 
\[\phi\in \mathcal{A}(N_r(\A)M_r(F)\setminus \Sp_{2n}(\A) )_{\Delta(\tau, b)\tensor \sigma}\] 
then we have the Eisenstein series
\[E(\phi, s)(g) = \sum_{\gamma\in P_r(F) \setminus \Sp_{2n}(F)} s.\phi(\gamma g)\]
for \(g\in \Sp_{2n}(F) \setminus \Sp_{2n}(\A)\). Becuase it is induced from the residual representation \(\Delta(\tau, b)\) we call these residual Eisenstein series.

\todo[inline]{state theorem 4.1 and 4.2}
\begin{Theorem}[]
    
\end{Theorem}

A similar result was given for the Siegel parabolic and the metaplectic cover of \(\Sp\) in \cite{ginzburgTopFourierCoefficients2021}
\begin{Theorem}[]
    
\end{Theorem}

\begin{comment}
    \section{Our Results}
We consider an almost identical setup but we deal with the metaplectic cover of \(\Sp_{2n}\), again over a number field \(F\), \(\Mp_{2n}\)\todo[inline]{reference the section I discuss this in.}. We again fix the Borel of upper triangular matricies, consider partitions of \(n = r+m\) and look at maximal standard parabolics of \(\Sp_{2n}\), \(P_r = M_rN_r\) such that 
\[M_r = \GL_r \times \Sp_{2m}\]
then if \(r = ab\) we still have that \(\Delta(\tau, b)\tensor \sigma\), for \(\tau\) irreducible unitary cuspidal automorphic representation of \(\GL_a\) and \(\sigma\) irreducible generic cuspidal automorphic representation of \(\Sp_{2m}\), is a representation of \(M_r\). The difference is in the parabolic induction as we now consider 
\[\phi\in \mathcal{A}(N_r(\A)M_r(F)\setminus \Mp_{2n}(\A) )_{\Delta(\tau, b)\tensor \sigma}\] 
and then the Eisenstein series is defined in the same way
\[E(\phi, s)(g) = \sum_{\gamma\in P_r(F) \setminus \Sp_{2n}(F)} s.\phi(\gamma g)\]
for \(g\in \Sp_{2n}(F) \setminus \Mp_{2n}(\A)\) and \(s\in \C \cong X^{\Mp_{2n}}_{M_r}\).

\begin{Lemma}
When \(b = 1\) we have the constant term
    \[E(\phi,s)(g)_{P_a} = \phi(g)_{P_a} + M(\omega, \tau\tensor\sigma)(\phi)(g)\]
\end{Lemma}

\todo[inline]{fill in here as theorems or whatever anything that I end up actually checking....}

\end{comment}


\section{Computing the Constant Term}
\todo[inline]{The representation is supposed to be of the covering of the Levi///? need to fix this}
We want to give some of the details of the proof in the \(b=1\) case, the base case for the induction. This will then be mirrored in the metaplectic case, when we extend the result in \cite{ginzburgTopFourierCoefficients2021} to non-siegel parabolics. The first step is to compute a constant term

We here consider the case that \(b=1\), hence \(n = a + m\). Then fixing a standard parabolic of \(\Sp_{2n}\) we have the maximal standard parabolic \(P_a = M_aN_a\) where \(M_a = \GL_a \times \Sp_{2m}\).  Now if \(\tau\) is irreducible unitary cuspidal automorphic representation of \(\GL_a\) then by definition \todo[inline]{Brenner..}
\[\Delta(\tau, 1)(\phi)(g) = E(\phi,s)(g) = s.\phi(g)\]
where the Eisenstein series is defined via the parabolic induction from the Levi \((\GL_a)^{\times b} \) to \(\GL_{ab}\). Thus we have \(\Delta(\tau, 1) = \tau\). So for the appropriate \(\sigma \) a rep of \(\Sp_{2m}\) we get a rep of the Levi of \(\Sp_{2n}\), \(M_r = M_a = \GL_a\times \Sp_{2m}\) given by \(\tau\tensor \sigma\). To this we associate the Eisenstein series for \(\phi\in \mathcal{A}(N_r(\A)M_r(F)\setminus \Mp_{2n}(\A))_{\tau\tensor \sigma}\) \(E(\phi,s)\) as usual. Now we proceed to calculate the constant term of this Eisenstein series along the parabolic \(P_a = MN\). \todo[inline]{M+W II.1.7, all others are zero..?}

By our earlier calculations \ref{const_eisenstein} and the cuspidality of the tensor \ref{cuspidality_tensor} and \cite{jiangPolesCertainResidual2013} we know that 
     \[E(\phi, s)_{P} = \sum_{ w} \sum_{m'} \int_{(w\inv N(\A)w \cap M(\A)) \setminus A} \int_{w\inv N(F) w \cap M(F) \setminus w\inv N(\A)w \cap M(\A)} \phi( n_1 n_2 w\inv m' x)  dn_1 dn_2\] 
     and the inner integral vanishes for all \(w\neq id, \omega\) (\(\omega\) as in \cite{jiangPolesCertainResidual2013}). Hence the first sum becomes over two elements and we have 

     \[E(\phi, s)_{P} = E(\phi, s)_{P, id} + E(\phi, s)_{P, \omega}\]
     where 
     \[E(\phi, s)_{P, w} =  \sum_{m'\in M(F)\cap wP(F)w\inv\setminus M(F)} \int_{N(F)\cap wP(F)w\inv \setminus N(\A)} \phi( w\inv nm' x)  dn\]

First the identity term simplifies

     \begin{equation*}
        \begin{aligned}
            E(\phi, s)(x)_{P', id} &=  \sum_{m'\in M(F)\cap P(F)\setminus M(F)} \int_{N(F)\cap P(F) \setminus N(\A)} \phi( nm' x)  dn\\
            &= \sum_{m'\in M(F)\setminus M(F)} \int_{N(F)\setminus N(\A)} \phi( nm' x)  dn \\
            &=\int_{N(F)\setminus N(\A)} \phi( n x)  dn \\
            &= \phi(x)_P
        \end{aligned}
     \end{equation*}
     \todo[inline]{I really need to fix this s thing that I dropped in the constant term computations.}

     Considering now the \(\omega\) term 
     \[E(\phi, s)_{P, \omega} =  \sum_{m'\in M(F)\cap \omega P(F)\omega\inv\setminus M(F)} \int_{N(F)\cap \omega P(F)\omega\inv \setminus N(\A)} \phi( \omega\inv nm' x)  dn\]

     by \cite[2C]{jiangPolesCertainResidual2013} \(M(F)\cap \omega P(F)\omega\inv \setminus M(F)\) is isomorphic to \(P_0 \setminus \Sp_{2(n-a)}\), but \(P_0\) has Levi \(M_0 = \Sp_{2(n-a)}\) by definition and hence is itself \(\Sp_{2(n-a)}\). Thus the sum is over \(\Sp_{2(n-a)}(F) \setminus \Sp_{2(n-a)}(F)\) and hence is over a point. Therefore we get by definintion of the intertwining operator
     \[E(\phi, s)_{P, \omega} = \int_{N(F)\cap \omega P(F)\omega\inv \setminus N(\A)} \phi( \omega\inv n x)  dn = M(\omega, -)(\phi)(x)\]
     becuase we took the constant term along the same parabolic as the definition of the Eisenstein series we know that the Levis are (the same) conjugate.
    Thus we have shown that 
    \[E(\phi, s)_P(x) = \phi(x)_P + M(\omega, - )(\phi)(x)\]

    Notice that the computation takes place completely at the level of the terminals which are indupendent of the fact that we have taken a covering group, hence we have really only reused work from \cite{jiangPolesCertainResidual2013}.

    \section{Analysing the Intertwining Operator}
    For classical groups it has been known for a while that the intertwining operator has a normalization in terms of ratios of L-functions. The point is then that the normalised operator is holomorphic and so the poles of the constant term depend entirely on the poles of the L-functions. In particular we will look at incarnations of the following statement: There is a holomorphic and non-zero intertwining operator \(N(s, w)\) such that 
    \[M(s, w) = r(s, w)N(s,w),\]
    and \(r(s, w)\) is a ratio of L-functions.

    Note that this is the global statement. There is an analogous set of conjectures for the local pieces, namely \(M = \tensor_\nu A\) the tensor over local intertwiners. Then one wants a normalisation of the local operators \(\mathscr{A}\) satisfying a long list of properties. This is extensively dealt with in \cite{shahidiProofLanglandsConjecture1990}.


    It has been known for a long time that there was some normalisation \(M = rN\) where \(r\) is a ratio of L-functions, for instance Shahidi gives the following \cite{shahidiRamanujanConjectureFiniteness1988}: Let \(\pi\) be an automorphic representation, let \(S\) be a finite set of places such that \(\pi_\nu\) is unramified for \(\nu\notin S\). We have that there are some finite dimensional complex representations \(r_1, ..., r_m\) of \(^LM\) such that 
     \[M(s, \pi)f = \bigotimes_{\nu\in S}A(s, \pi_\nu, w)f_\nu \tensor \bigotimes_{\nu\notin S} \prod_{i=1}^{m}\frac{L_S(is, \pi, \tilde{r_i})}{L_S(1+is, \pi, \tilde{r_i})} \tilde{f}_\nu.\]

    For example for a group over \Q we have the following from \cite{langlandsEulerProductsa} 
    \[M(s) = \left( \prod_\alpha\frac{\pi^{1/2}\Gamma(\frac{1}{2}\mu_\infty(s)(H_\alpha))}{\Gamma(\frac{1}{2}(\mu_\infty(s)(H_\alpha) + 1))} \right)\prod_{p \text{ prime }} \left( \prod_\alpha \frac{\frac{1}{1 - p^{\mu_p(s)(H_\alpha) + 1}}}{1 - \frac{1}{p^{\mu_p(s)(H_\alpha) }}}\right).\]
     However it was not shown until recently, and only for classical groups that this \(N\) indeed has the required properties. In particular the following theorem is sufficient for the cases dealt with in \cite{jiangPolesCertainResidual2013}:
     \begin{Theorem}[\cite{cogdellFUNCTORIALITYCLASSICALGROUPS}, 11.1]
        Suppose that \(\pi_\nu\) is a local component of a globally generic cuspidal representation \(\pi\) of \(G_n(\A)\). Then for any irreducible admissible unitary generic representation \(\pi'_\nu\) of \(\GL_m(k_\nu)\) the normalized intertwining operator \(N'(S, \pi'_\nu\times \pi_\nu, w)\) is holomorphic and non-zero for \(Re(s)\geq 0\)
     \end{Theorem}

     In the case we are considering the normalising factor \(r\) is given by the equation \cite[4A]{jiangPolesCertainResidual2013}
     \[r(w, s) = \frac{L(s, \tau\times \sigma)L(2s, \tau, \rho)}{L(s+1, \tau\times \sigma)L(2s+1, \tau, \rho)}\]
     and this proves that 

     \begin{Lemma}
        The Eisenstein series above has pole at \(s\) if and only if \(r(w,s)\) has a pole. The Eisenstein series has a zero at \(s\) if and only if \(r(w, s)\) has a zero.
     \end{Lemma}
     The final step is then to use the known properties of \(L\)-functions to conclude when our \(r\)-factor will have poles and zeroes. 


    \section{The Metaplectic Generalisation}
    In \cite[]{ginzburgTopFourierCoefficients2021} this setup is also carried out for the Siegel parabolic induced up to the metaplectic group. Here we investigate the literature to hopefully conclude the same result for all maximal parabolics. The steps will be the same we simply need for the Langlands conjectures to be proven in certain cases. 

    Thankfully Kaplan in a series of recent works with collaborators \cite{kaplanDoublingConstructionsComplete2021a}\cite{kaplanDoublingConstructionsTensor2020}\cite{caiDoublingConstructionsGlobal2024} has supplied some of the key peices. 

    First we have that 
    \[M(s, w)f = \frac{}{} f\]
    note that these are metaplectic L-functions as defined in that paper.

    Thus we get 
    \begin{Lemma}
        The Eisenstein series above has pole at \(s\) if and only if \(r(w,s)\) has a pole. The Eisenstein series has a zero at \(s\) if and only if \(r(w, s)\) has a zero.
     \end{Lemma}
     Finally we need to once again see what properties of these L-fuctions have been proven. 


    \appendix
\chapter{L-Functions}
The theory of L-functions is not yet systematic; Langlands has provided a conjectural framework, however it is still under construction.  
In the mean time there are two major ``paradigms'' for constructing and proving theorems about L-functions, those are the Langlands-Shahidi type constructions and the Rankin-Selberg type constructions. General surveys can be found in \cite[Part 2.III.2]{borelAutomorphicFormsRepresentations1979}\cite{shahidiEisensteinSeriesAutomorphic2010}\cite{cogdellLFUNCTIONSFUNCTORIALITY}\cite[9, 10, 11]{bumpIntroductionLanglandsProgram2004}\cite{arthurLfunctionsAutomorphicRepresenta}.

The Rankin-Selberg type functions are surveyed in \cite{bumpRankinSelbergMethodIntroduction2011}. The \(\GL_n \times \GL_m\) case is dealt with in \cite{cogdellLECTURESINTEGRALREPRESENTATIONS}. For Rankin-Selberg L-functions of form \(\Sp_{2n}\times \GL_m \) the theory (for generic cuspidal representations) is worked out in \cite{ginzburg$L$functionsSymplecticGroups1998}.

The Langlands-Shahidi paradigm is explained in \cite{shahidiProofLanglandsConjecture1990, shahidiEisensteinSeriesAutomorphic2010}.

We have the langlands general framework. We have by \cite{cogdellLFUNCTIONSFUNCTORIALITY} some properties uniquenly determininng L functions for tempered things. By \cite{shahidiArthurPacketsRamanujan2011} all generic things are tempered so we can apply the theory of RS and GR to explicitly construct global things and prove theorems. In particular their analytic properties are well understood in these cases from \cite{grbacResidualSpectrumSplit2011, cogdellRemarksRankinSelbergConvolutions}. Note that \cite{grbacResidualSpectrumSplit2011} is conditional on the unfinished work of Arthur \cite{EndoscopicClassificationRepresentations}.

\label{L-functions}

\section{Automorphic L-Functions}
We follow closely Borels exposition in \cite[Part 2. III. 2. ]{borelAutomorphicFormsRepresentations} and \cite{shahidiEisensteinSeriesAutomorphic2010}.
Given a reductive LAG \(G\) defined over \C there is an associated root datum as in \((X, \Phi, \hat{X}, \hat{\Phi})\), where for any choice of maximal torus we have \(X = \Hom(T, \mathbb{G}_m)\), \(\hat{X} = \Hom(\mathbb{G}_m, T)\), and \(\Phi, \hat\Phi\) are the roots and coroots of \(G\) with respect to \(T\) \cite[7.4.3]{springerLinearAlgebraicGroups1998}.  Then each reductive LAG \(G\) over a number field \(F\) has the root datum that is associated to the base change of \(G\) to \(\C\), \((X, \Phi, \hat{X}, \hat{\Phi})\).
By the existence theorem \cite[10]{springerLinearAlgebraicGroups1998} to the dual root datum \(( \hat{X}, \hat{\Phi}, X, \Phi)\) there is a LAG defined over \C that corresponds, we call this the \textbf{dual group} of \(G\) and we denote it \(\hat{G}\). It is possible through the use of the root datum to specify a ``cannonical'' action of \(\mathrm{Gal}(\bar{F}/F)\) on \(\hat{G}\) as in loc. cit.
The \textbf{Langlands dual group} is then the dual group semi-direct producted with the \(\mathrm{Gal}(\bar{k}/k)\) via this action, which we omit
\[^L G \defeq \hat{G} \rtimes \mathrm{Gal}(\bar{k}/k).\]

\begin{example}[Classical Groups, \cite{bumpIntroductionLanglandsProgram2004}, 11.1]
	We have the following table
	\begin{table}[h]
		\centering
		\begin{tabular}{ll}
			\(G\)         & \(\hat{G}\)   \\ \hline
			\(\GL_n\)     & \(\GL_n\)     \\
			\(SO_{2n+1}\) & \(\Sp_{2n}\)  \\
			\(SO_{2n}\)   & \(SO_{2n}\)   \\
			\(\Sp_{2n}\)  & \(SO_{2n+1}\)
		\end{tabular}
	\end{table}
\end{example}

If \(\nu\) is a non-archimedean place of F, then \(\mathcal{O}_\nu\) is a local ring and we denote \(q_\nu\) the cardinality of the residue field i.e. if \(\mathfrak{p}_\nu\) is the unique maximal ideal of \(\mathcal{O}_\nu\) then \(q_\nu \defeq [\mathcal{O}_\nu: \mathfrak{p}_\nu]\). Using the Satake isomorphism, to each unramified representation of \(G(F_\nu)\) we can associate a conjugacy class of \(^LG\), via some map call it \(c\), and hence there is a way to apply a complex representation \(r: \phantom{ }^LG \to \GL_n(\C)\) to unramified representations of \(G(F_\nu)\), details in \cite[2]{shahidiEisensteinSeriesAutomorphic2010}. 
Given such an unramified representation of \(G(F_\nu)\), call it \(\pi_\nu\), the local automorphic L-function is then 
\[ L_\nu(s,\pi_\nu , r ) \defeq  \frac{1}{\det\bigl( I - r(c(\pi_\nu))q_\nu^{-s} \bigr)} ,\;\;\;\; s\in \C.\]
In the global case we consider an irreducible automorphic representation \(\pi = \tensor_\nu \pi_\nu\) of \(G(\A)\), and a finite set of places of \(F\), call it \(S\), such that \(S\) contains all infinite places and for all \(\nu\notin S\) \(\pi_\nu\) is unramified. Recall that we denoted the Langlands dual of \(G\) defined over \(F\) by \(^LG\). We denote the Langlands dual of \(G\) defined over \(F_\nu\) for \(\nu\notin S\) by \(^LG_{F_\nu}\).  If \(r\) is a finite dimensional complex representation of \(^LG\) then the embedding of Galois groups \(\mathrm{Gal}(\bar{F_\nu} / F_\nu) \hookrightarrow \mathrm{Gal}(\bar{F} / F) \) induces a map \(^LG_{F_\nu} \to ^LG\) along which we can pull \(r\) back, giving a representation \(r_\nu\) of \(^LG_{F_\nu}\). Then the partial global L-functions are defined to be 
\[L_S(s, \pi, r) \defeq \prod_{\nu\notin S} L(s, \pi_\nu, r_\nu), \;\;\;\;\; s\in \C.\]


\begin{example}[Standard Representations / Classical Groups]
	In the case of classical groups it is common to see L-functions with only two entries e.g. if \(\rho\) is a representation of \(G = \Sp{2n}\) then you may see 
	\(L(s, \rho).\)
	The reason is that there is a standard representation of the dual groups of classical groups. Namely the standard representation of a matrix group inside \(\GL_n\) is the one that sends \(g\mapsto g\). It is this representation that is to be taken for the dual group in this setting.
\end{example}

\begin{example}[Rankin-Selberg, \cite{cogdellLECTURESINTEGRALREPRESENTATIONS}, 1.2, \cite{arthurLecturesAutomorphicFunctions1991a}, Ch. 2 Example. 2]
	Let \(\nu\) be a finite place of \(\Q\) and \(\pi, \pi'\) be two unramified generic representations of \(\GL_n(\Q_\nu)\) and \(\GL_m(\Q_\nu)\) respectively. Let \(B_n\) be the standard Borel of upper triangular matricies in \(\GL_n\). Such representations have been classified \todo[inline]{reference}
	in terms of characters of \(\Q^\times_\nu\), in particular for \(\pi\) there are \(\mu_1, ..., \mu_n\) unramified characters such that 
	\[\pi \cong \Ind_{B(\Q_\nu)}^{\GL_n(\Q_\nu)} \big(\mu_1 \tensor \cdots \tensor \mu_n\big).\]
	If we fix a uniformizer \(\varpi\) of \(\Q_\nu\) then we have the so called ``Satake parameters'' \(\mu_i(\varpi)\) which determines \(\pi\) uniquely. Of course the same is true for \(\pi'\), with say characters \(\mu'_1, ..., \mu'_m\). We then define
	\[L(s, \pi\times \pi') \defeq \prod_{i,j} \frac{1}{1-\mu_i(\varpi)\mu'_j(\varpi)q^{-s}}.\]
	
	Consider the group \(G = \GL_n\times \GL_m\) which has dual \(\GL_n(\C) \times \GL_m(\C)\), then there is a cannonical representation 
	\[r:\GL_n(\C) \times \GL_m(\C) \to \GL_{nm}(\C). \]
	Then 
	\[L(s, \pi\tensor \pi', r) = L(s, \pi \times \tilde{\pi}'),\]
	where the tilde denotes the contragradient.
\end{example}

\begin{example}[Dirichlet L-functions]
	Recall that a Dirichlet character \(\chi\) is a character of the group \((\Z/N\Z)^*\) and a Grossencharacter (also known as Hecke characters) \(\chi'\) is a character of the group \(\A_\Q^\times/\Q^\times\). Through the series of maps 
	% https://q.uiver.app/#q=WzAsNSxbMiwwLCJcXFJfez4wfV5cXHRpbWVzIFxcdGltZXMgXFxoYXR7XFxafV5cXHRpbWVzIl0sWzQsMCwiXFxiaWcoXFx2YXJwcm9qbGltXFxaL05cXFpcXGJpZyleXFx0aW1lcyAiXSxbNiwwLCIgKFxcWi9OXFxaKV5cXHRpbWVzIl0sWzMsMiwiXFxDIl0sWzAsMCwiXFxBXlxcdGltZXMgLyBcXFFeXFx0aW1lcyJdLFswLDEsIlxcbWF0aHJte3Byb2p9XzMiXSxbMSwyXSxbMiwzLCJcXGNoaSJdLFs0LDMsIlxcY2hpJyIsMl0sWzQsMF1d
	\[\begin{tikzcd}[cramped]
		{\A^\times / \Q^\times} && {\R_{>0}^\times \times \hat{\Z}^\times} && {\big(\varprojlim\Z/N\Z\big)^\times } && { (\Z/N\Z)^\times} \\
		\\
		&&& \C
		\arrow["\sim",from=1-1, to=1-3]
		\arrow["{\chi'}"', from=1-1, to=3-4]
		\arrow["{\mathrm{proj}_3}", from=1-3, to=1-5]
		\arrow[from=1-5, to=1-7]
		\arrow["\chi", from=1-7, to=3-4]
	\end{tikzcd}\]
	we get a bijection between Dirichlet characters and finite-order Grossencharacters, i.e. Grossencharacters \(\chi'\) such that for some \(n\in \N\) we have \((\chi')^n = 1\). These are just grossencharacters trivial in the infinite place, by construction.
	
	Grossencharacters are just automorphic forms on \(\GL_1\) and hence they generate automorphic representations by taking the span of their orbit in the space of automorphic forms. 
	
	The claim is then that the Langlands automorphic L-functions of this representation is, after being pushed forward along this series of maps the regular Dirichlet L-function. \todo{reference? More details?}
\end{example}



\section{L-Functions for \(\GL_n\times \GL_m\)}
\subsection{Generic Representations}
\todo[inline, color=blue]{L-fuunctions chapter}
\subsection{Definition}
\cite{cogdellLECTURESINTEGRALREPRESENTATIONS}

\subsection{Poles and Zeroes}


\section{L-Functions for Classical Groups}

\(L(s, \tau\times \sigma)\)
\cite{remarks on rankin selberg convolutions Cogdel and PT-shapiro}\cite{cogdellFunctorialityClassicalGroups2004} -----> Defined in \cite{shahidiProofLanglandsConjecture1990}

\(L(s, \tau, \sigma)\)
\cite{grbacResidualSpectrumSplit2011}

\begin{example}[]
	The so called ``Rankin-Selberg'' L-function associated to -------, is given by 
	\[L(s, \tau\times \tau) = L(s, \tau, \rho)L(s, \tau, \rho^-)\]	
\end{example}



\listoftodos



\newpage
\bibliographystyle{alpha}
\bibliography{./NumberTheory}
%\end{comment}

\end{document}