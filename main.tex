\documentclass[10pt, openany]{book}

%\documentclass[12pt]{report}
\usepackage[utf8]{inputenc}
\usepackage[a4paper]{geometry}
\usepackage{graphicx}
\usepackage{tikz-cd}
\usepackage{mathrsfs}
\usepackage{amsfonts}
\usepackage{soul}
\usepackage{ stmaryrd }
\usepackage{amsthm}
\usepackage{amsmath}
\usepackage{amssymb}
\usepackage{mathtools}
\usepackage{extarrows}
\usepackage{interval}
\usepackage{bbm}
\usepackage[utf8]{inputenc}
\usepackage[english]{babel}
\usepackage{romannum}
\usepackage{setspace}
\usepackage{listings}
\usepackage{pdfpages}
\usepackage{hyperref}
\usepackage{enumitem}
\usepackage{ragged2e}
\title{A Paraphrase of a Paraphrase}
\author{Riley Moriss}
		\author{Riley Moriss}
	
		%images
		\usepackage{graphicx}
		\usepackage{quiver} %Tikz diagram generator
		\usetikzlibrary{fit}
		
		
		%Bibliography
		\usepackage{hyperref} %Content is clickable
		
		%Math
		\usepackage{amsmath}
		
		%Formatting
		\usepackage{comment} %comment out code blocks
		\usepackage{csquotes} %Quote environment
		\usepackage{amsfonts}
		\usepackage{mathrsfs} %Script fonts

		%Todos	
		\usepackage[colorinlistoftodos,prependcaption,textsize=scriptsize, color=red!20, shadow]{todonotes}
			
		%Math shortcuts
		%Sets
		\newcommand{\R}{\ensuremath{\mathbb{R}}\phantom{ }}
		\newcommand{\C}{\ensuremath{\mathbb{C}}\phantom{ }}
		\newcommand{\Z}{\ensuremath{\mathbb{Z}}\phantom{ }}
		\newcommand{\N}{\ensuremath{\mathbb{N}}\phantom{ }}
		\newcommand{\Q}{\ensuremath{\mathbb{Q}}\phantom{ }}
		\newcommand{\F}{\ensuremath{\mathbb{F}}\phantom{ }}
		\newcommand{\G}{\ensuremath{\mathbb{G}}\phantom{ }}

		%Symbols
		\newcommand{\hilb}{\ensuremath{\mathfrak{h}}\phantom{ }}
		\newcommand{\tensor}{\ensuremath{\otimes}}
		\newcommand{\inner}[1]{\ensuremath{\langle #1\rangle}}
		\newcommand{\norm}[1]{\lVert #1 \rVert}
		\newcommand{\inv}{^{-1}}
		\newcommand{\model}[1]{\llbracket #1\rrbracket}
		\newcommand{\Par}{\rotatebox[origin=c]{180}{\&}}
		\newcommand{\Dif}[2]{\ensuremath{\frac{\partial #1}{\partial #2}}}
		\newcommand{\comp}{\circ}
		\newcommand*{\defeq}{\hspace{2mm}\mathrel{\vcenter{\baselineskip0.5ex \lineskiplimit0pt
					\hbox{\scriptsize.}\hbox{\scriptsize.}}}%
			=\hspace{2mm}}

		%Operators
		\DeclareMathOperator{\Ind}{Ind}
		\DeclareMathOperator{\Spec}{Spec}
		\DeclareMathOperator{\Res}{Res}
		\DeclareMathOperator{\Hom}{Hom}
		\DeclareMathOperator{\Maps}{Maps}
		\DeclareMathOperator{\Aut}{Aut}
		\usepackage{mathtools}
\DeclarePairedDelimiter{\ceil}{\lceil}{\rceil}
\DeclarePairedDelimiter{\floor}{\lfloor}{\rfloor}

		\DeclareMathOperator{\dif}{d}
		\newcommand{\A}{\ensuremath{\mathbb{A}}\phantom{ }}
		\renewcommand{\P}{\ensuremath{\mathbb{P}}\phantom{ }}
		\newcommand{\Struc}{\ensuremath{\mathcal{O}}\phantom{ }}
		\newcommand{\Gcov}{\ensuremath{\mathbf{G}}\phantom{ }}
		
		\DeclareMathOperator{\Sp}{Sp}
		\DeclareMathOperator{\SL}{SL}
		\DeclareMathOperator{\GL}{GL}
		\DeclareMathOperator{\Mp}{Mp}

		%Fonts
		\newcommand{\Cal}[1]{\ensuremath{\mathcal{#1}}}
		\newcommand{\Cat}[1]{\ensuremath{\mathscr{#1}}}
		%Format
		\newcommand{\gap}{\vspace{4mm}}
		\newcommand\scalemath[2]{\scalebox{#1}{\mbox{\ensuremath{\displaystyle #2}}}}
		\usepackage{ragged2e}


		
		%Environments
		\usepackage{amsthm} %No numbering
		\newtheorem{Lemma}{Lemma}
		\newtheorem{Theorem}{Theorem}
		\newtheorem{Remark}{Remark}
		\newtheorem{Exercise}{Exercise}
		\newtheorem{Notation}{Notation}
		\newtheorem{Definition}{Definition}
		\newtheorem{example}{Example}	
		\newtheorem{cor}{Corollary}
		\newtheorem{Prop}{Proposition}
		\newtheorem{Conj}{Conjecture}


		\usepackage{framed}
		%\usepackage{mdframed}
		\newcommand{\proofbar}[1]{
			\begin{leftbar}
				\textit{\textbf{Proof.}} 	#1
			\end{leftbar}
			}

		%Highlighting
		\newcommand{\notice}[1]{\hl{\textbf{\textit{#1}}}}
		\DeclareRobustCommand{\hlcyan}[1]{{\sethlcolor{cyan}\hl{#1}}}
		\DeclareRobustCommand{\hlgreen}[1]{{\sethlcolor{green}\hl{#1}}}
			
			
			
			
			
%No title page.
\makeatletter
\newcommand*{\toccontents}{\@starttoc{toc}}
\makeatother

\let\oldphi\phi 
\let\phi\varphi 
\let\varphi\oldphi

% As an example:
%   \intof[0]{1}{x^2}{x}
\newcommand*\intof[4][]{\int^{#1}_{#2}#3\,d#4}

% As an example:
%   \setof[\Big]{ n\in\N \st n < \binom n 2 }
\newcommand\setof[2][]{\mathord{\mathopen#1\{\,#2\,\mathclose#1\}}}

%\usepackage{multicol}
%\usepackage{geometry}
%\usepackage{soul}
%\usepackage{ stmaryrd }
%\geometry{ margin=3cm}

%%%%%%%%%%%%%%%%%%%%%%%%%%%%%%%%%%%%%%%%%%%%%%%%%%%%%%%%%%%%%%%%%%%%%%
\setcounter{tocdepth}{1}%%%%% Stops subsections in TOC

\newtheorem{theorem}{Theorem}[chapter]
\newtheorem{lemma}[theorem]{Lemma}
\newtheorem{corollary}[theorem]{Corollary}
\newtheorem{proposition}[theorem]{Proposition}
\numberwithin{equation}{section}
\theoremstyle{definition}
\newtheorem{definition}[theorem]{Definition}
%\newtheorem{example}[theorem]{Example}
\newtheorem{problem}[theorem]{Problem}
\theoremstyle{remark}
\newtheorem{remark}[theorem]{Remark}
\numberwithin{equation}{section}
\renewcommand{\baselinestretch}{1.25}
\geometry{
 left=30mm,
 right=30mm,
 top=30mm,
 bottom=30mm,
 includefoot
}
\allowdisplaybreaks
\geometry{ margin=3cm}

%%%%%%%%%%%%%% 
%Type Writer Font
%\usepackage{courier}
%\renewcommand*\familydefault{\ttdefault} %% Only if the base font of the document is to be typewriter style
%\usepackage[T1]{fontenc}

%%%%%%%%%%%%%%%%%%%%%%%%%%%%%%%%%%%%%%%%%%%%%%%%%%%%%%%%% BEGIN
\begin{document}
\begin{comment}%Front page stuff
    \begin{titlepage}
    \begin{center}
    \vspace*{1cm}
    \huge
    \textbf{A Paraphrase of a Paraphrase}\\
    \vspace{2cm}
    \Large
    \text{Riley Moriss}\\
    \vspace{0.5cm}
    \text{Supervisor: Dr. Chenyan Wu }\\    
 
    A thesis submitted in partial fulfillment of the\\
    requirements for the degree of\\
    Master of Science\\
    in the\\
    School of Mathematics and Statistics\\
    at\\
    The University of Melbourne\\
    \vspace{1,5cm}
    October 2024
\end{center}
\end{titlepage}

\pagebreak
\end{comment}
\pagenumbering{roman}

\chapter*{Acknowledgements}
I need to thank
\begin{itemize}
	\item Chenyan Wu,  Alex Ghitza,  Chengjing Zhang	
	\item Bowan Hafey, Oliver and Fei
        \item Miscellaneous lecturers
\end{itemize}
for the math that they taught me. 

Thanks to my bros for being bros. Thanks to Fei Peng for the thesis template. Arun Ram for helping track down the name for parabolic. Thenk you to my proof readers Yuhan, Chengjing, 

\chapter*{Introduction}
\section*{Motivation}
The goal of this thesis is to exposit some of the results in \cite{jiangPolesCertainResidual2013}. We aim our exposition at the other masters students in our cohort.
To explain the results on poles of Eisenstein series to students in other disciplines there is a fair amount of background to be covered.

Here we attempt to put down what we understand of the ``big picture''. It could be argued that we spent too much time trying to understand the motivation for the results in \cite{jiangPolesCertainResidual2013} and not enough time on the results themselves and so this section is to ensure that time was not (very) wasted. 

We should point out that there are many surveys and books on the Langlands program, class field theory and modern topics in number theory that this introduction is indebted to. Some exemplars are \cite{fleigEisensteinSeriesAutomorphic2016, bumpIntroductionLanglandsProgram2004} for longer treatments, in particular the statements of the conjectures are most clearly stated in Cogdell's chapters in \cite{bumpIntroductionLanglandsProgram2004}. Shorter surveys are \cite{gelbartElementaryIntroductionLanglands1984, kimSuperficialIntroductionLanglands, langlandsFunctorialityTheoryAutomorphic, langlandsREPRESENTATIONTHEORYITS, arthurAUTOMORPHICREPRESENTATIONSNUMBER}.

\subsection{From Ancient to Modern}
We follow the wonderful exposition in \cite{weinsteinReciprocityLawsGalois2015}. A problem that Euclid could have understood is ``which positive integers are the sum of two squares''. In 1640 Fermat answered this question, he first reduces the question to when is a prime the sum of two squares. Thus the problem is immediately reformulated as a problem about congruences mod a prime \(p\), ``when does there exist a solution to \(a^2  +b^2 \equiv 0 \;\;(\text{mod }p) \)'', or whats the same, by dividing out \(b^2\), ``when is there a solution to \(x^2 + 1 \equiv 0 \;\;(\text{mod }p)\)''. The famously has the solution 
\begin{Theorem}
	Let \(p\) be an odd prime. Then \(x^2 + 1 \equiv 0 \;\;(\text{mod }p)\) has a solution if and only if \(p\equiv 1 \;\;(\text{mod }4) \).
\end{Theorem}

Recall the Legendre symbol, for \(p, q\) odd and non-equal primes we have 
\[\left(\frac{q}{p}\right) \defeq \begin{cases}
	1 , & \text{there is a solution to }x^2 - q \equiv 0 \;\;(\text{mod }p)\\
	-1 , & \text{else.}
\end{cases}.\]
Then the theorem of Fermat was ``upgraded'' by Gauss to his famous reciprocity law.

\begin{Theorem}
	For \(p, q\) odd and non-equal primes,
	\[\left(\frac{p}{q}\right)\left(\frac{q}{p}\right) = (-1)^{\frac{(p-1)(q-1)}{2}}.\]
\end{Theorem}

Having a solution mod a prime is the same as asking whether the polynomial splits mod that prime. The natural question is then: \textbf{Q1.} Given a monic irreducible polynomial with integral coefficients can we determine by congruences whether it splits mod a prime. Gausses reciprocity is a complete solution to this problem for polynomials of the form \(f(x) = x^2 - q\) for \(q\) odd prime. 
\begin{remark}
	The odd limitation is for brevity here and of course can be lifted. Moreover the solution for primes can be leveraged for a solution for other integers. 
\end{remark}
Recall that if \(f(x)\in \Z[x]\) is monic and irreducible then there is a unique minimal field \(F\) in which it factors as linear polynomials, called the splitting field. The Galois group of \(f(x)\) is then defined to be \(\mathrm{Gal}(F/\Q)\). Class field theory is a solution to problem \textbf{Q1.} when this Galois group is \textit{Abelian}. To explain we need to introduce the standard algebraic number theory setup.

Let \(\Q \subseteq K\) be an extension of number fields, with respective rings of integers \(\Z \subseteq \mathcal{O}_K\) and let \(p\) be a prime in \(\Z\) hence \((p)\) is a prime ideal of \(\Z\) and let 
\[\mathcal{O}_K(p) = \prod_i \mathfrak{P}_i^{e_i},\]
be the prime decomposition in \(\mathcal{O}_K\). Then \((p)\) \textbf{splits} in \(\mathcal{O}_K\) if for every \(i\) we  have \(e_i = 1\) (this is being unramified) and \(\mathcal{O}_K/\mathfrak{P}_i \cong \Z/(p)\). The splitting of primes is related to the splitting of polynomials by the following theorem
\begin{Theorem}[\cite{langAlgebraicNumberTheory1994}, Prop. 26]
	If \(f\in \Z[x]\) monic and irreducible and \(f(\alpha) = 0\) then for \(K = \Q(\alpha)\) we have with finitely many exceptions that \(f\) is split mod \(p\) if and only if \((p)\) splits in \(\mathcal{O}_K\).
\end{Theorem}
\begin{example}
	
\end{example}
So to answer \textbf{Q1.} we now want to solve by congruences when prime ideals split. Every field extension \(K/L\) has a Galois closure, that is an extension \(L'/K\) of minimal degree such that \(L\subseteq L'\) and \(L'\) is Galois over \(K\). 
\begin{Lemma}
	A prime ideal of \(\mathcal{O}_K\) is split in \(\mathcal{O}_L\) if and only if it is split in \(\mathcal{O}_{L'}\).
\end{Lemma}
Thus we lose nothing by considering only Galois extensions of fields. Thus we have ``the main theorem'' of class field theory:
\begin{Theorem}[\cite{weinsteinReciprocityLawsGalois2015}, Thm. 3.2.1]\label{thm:reciprocity}
	Let \(K/\Q\) be an Abelian and Galois extension. There is an ideal \(\mathfrak{f} = (m)\subseteq \mathcal{O}_\Q = \Z\) such that for a prime \(p\in \Z\) the ideal \((p)\) is split in \(\mathcal{O}_K\) if \(p\equiv 1 \;\; (\text{mod }m)\)
\end{Theorem} 
Thus we have a solution to the splitting of primes via congruence relations. 

\begin{example}
	
\end{example}

This we hope motivates class field theory, now we will follow \cite{conradHISTORYCLASSFIELD} for some more detail on class field theory. Class field theory is over a hundred years old with a storied past and many incarnations of the main theorem above. To see the Langlands program as a generalisation of this theory we want to trace the development to where Langlands picked up. 

Class field theory begins with Kronecker in 1853, who constructed an extension of number fields \(K'/K\) whose Galois group is isomorphic to the ideal class group of \(K\), a so called (by Weber) ``class field'' for \(K\). Kronecker would go on to make several conjectures that would form the heart of class field theory, for instance he conjectured that a Galois extension of \Q is determined by the primes of \Z that split over that extension. In fact this was solved by Bauer in 1916
\begin{Theorem}(Bauer)
	Let \(L_1, L_2\) be finite extensions of a number field \(K\), then \(L_1 = L_2\) if and only if the primes of \(\mathcal{O}_K\) that split in \(\mathcal{O}_{L_1}\) is equal to the set of primes that split in \(\mathcal{O}_{L_2}\).
\end{Theorem}
However there was no systematic way of finding \textit{which} primes split over the extension. Takagi was to supply something very close to theorem \ref{thm:reciprocity} in 1920 and it was to be made even more explicit finally by Artin in 1927. Thus \textit{global} class field theory was ``solved'', immediately the natural question was raised, what happens in the \textit{non-Abelian} extensions of number fields. The (global) Langlands conjectures (amongst other things) can be viewed as an attempt to answer this question. 

Another direction that people were interested in was extensions of local fields, as opposed to number fields. It was Hilbert who introduced in 1897 the use of the p-adic numbers, in spirit if not in name, he wrote congruences of arbitrary powers of primes. Let \(\nu\) be a place of \(\Q\), then define the \(\nu\)-adic Hilbert symbol for \(a,b\in \Q^\times\)
\[(a,b)_\nu \defeq \begin{cases}
	1, & a= x^2-by^2 \text{ has a solution in }\Q_\nu\\
	-1, &\text{else}
\end{cases} .\]
\begin{Theorem}[Hilbert's Quadratic Reciprocity]
	For all \(	a,b\in \Q^\times\) 
	\[\prod_\nu (a,b)_\nu = 1.\]
\end{Theorem}
This is equivalent to Gauss's reciprocity law, however much more uniform to state, treating odd and even primes in the same way, and not requiring any co-prime conditions. This moreover treats finite and infinite places uniformly. Building on this work and using Artin reciprocity Hasse, after introducing the p-adic numbers in 1927, proved the first versions of \textit{local} class field theory in 1930, that is reciprocity for extensions of the local fields \(\Q_\nu\). The statements here are too technical for a motivational introduction however replacing all the global fields in the above statements with local fields is not far off. 

Note that the definition and proof of local class field theory \textit{depends logically } on global class field theory. Hasse was able to prove later in 1933 the main results again but without recourse to global class field theory. It lacked the explicit construction of the class fields however which was finally supplied in 1965 by Lubin and Tate.

What remained to do was supply a proof of \textit{global} class field theory from local class field theory. In pursuit of this task the machinery of the ideles and adeles was introduced. In this language (part of) \textit{global} class field theory can be restated as 
\begin{Theorem}[\cite{neukirchGlobalClassField1999}, Prop. 1.3]
	Let the ideal class group of a number field \(K\) be denoted \(\mathrm{Cl}_K\). Then there is a surjection
	\[ \A^*/ K^* \xrightarrow{A} \mathrm{Cl}_K \cong \mathrm{Gal}(K'/K).\]
	Where \(K'\) is the class field of \(K\).
\end{Theorem}
If we think about representations of these groups then this surjection gives a relation between characters \(\chi\) of \(\A^*/K^*\) and characters \(\chi'\) of \(\mathrm{Gal}(K'/K)\) by pulling back along \(A\).

% https://q.uiver.app/#q=WzAsMyxbMCwwLCJcXEFeKi9LXioiXSxbMiwwLCJcXG1hdGhybXtDbH1fSyJdLFsxLDIsIlxcQ14qIl0sWzAsMiwiXFxjaGkiLDFdLFsxLDIsIlxcY2hpJyIsMV0sWzAsMSwiQSIsMV1d
\[\begin{tikzcd}[cramped]
	{\A^*/K^*} && {\mathrm{Gal}(K'/K)} \\
	\\
	& {\C^*}
	\arrow["A"{description}, from=1-1, to=1-3]
	\arrow["\chi"{description}, from=1-1, to=3-2]
	\arrow["{\chi'}"{description}, from=1-3, to=3-2]
\end{tikzcd}\]
Thus \(A\) can be thought of as generating a correspondence
\[\{\text{Maps }\A^*/K^*\to \C^*\} \to\{ \text{Maps }\mathrm{Gal}(K'/K)\to \C^*\}. \]
One then observes that this can be rewritten as 
\[\{\text{Maps }\GL(\A)/\GL_1(K)\to \GL_1(\C)\} \to\{ \text{Maps }\mathrm{Gal}(K'/K)\to \GL_1(\C)\} .\]
This suggests the generalisation to 
\[\{\text{Certain reps of }\GL_n(\A)/\GL_n(K)\} \to\{ \text{Certain reps of  }\mathrm{Gal}(\bar{K}/K)\text{ on } \GL_n\}   .\]
But according to Langlands \cite{langlandsREPRESENTATIONTHEORYITS}, who was inspired by the philosophy of Harish-Chandra, we should treat all reductive groups the same, and so Langlands conjectures that for any reductive linear algebraic group \(G\) there is some correspondence
\[\{\text{Certain reps of }G(\A)/G(K)\} \to\{ \text{Certain reps of  }\mathrm{Gal}(\bar{K}/K)\text{ on } G\} .\]
These two sides of the correspondence are referred to as the ``automorphic'' side and the ``Galois side'' respectively. The content that follows will be almost entirely on the automorphic side. 


\subsection{Harmonic Analysis}
As we mentioned the work of Langlands was inspired by the work of Harish-Chandra in harmonic analysis of Lie groups. Here we want to say something about the precursors to Langlands work in this respect, following \cite{follandCourseAbstractHarmonic2016}.

The story starts with the two Fourier transforms, the one for functions on the circle and functions on the real line. Both appearing in the work of Fourier himself around 1822. 

The circle is a compact topological group and for this we have the generalisation of the Fourier transform given by the Peter-Weyl theorem in 1927.

In 1940 Weil worked out the theory for locally compact \textit{abelian} groups, which of course is a strict generalisation of the case for both \R and \(U(1)\).

In the 1950's Segal and Mautner proved the Plancherel Theorem which is the generalisation of this to type I, second countable and unimodular topological groups. The key is that like the Peter-Weyl theorem we are given more than that \textit{some} decomposition exists, we are given explicit details on how to construct the decomposition, its structure and relation to other parts of the representation.

For this section let \(H\) be a locally compact topological group.
It is a classical theorem that for representations of finite groups over an algebraically closed field the regular representation decomposes into a direct sum, where ever irreducible representation appears \cite[Ch. 2.4 Cor. 2 ]{LinearRepresentationsFinite}. This still holds for compact topological groups, when one considers continuous unitary representations \cite[5.1]{follandCourseAbstractHarmonic2016}.
\begin{remark}
	This is a strict generalisation of the finite groups case, when we give the finite group the discrete topology then all its linear representations are continuous and unitary.
\end{remark}
There is one final more general incarnation of this line of investigation in the Plancherel theorem. A group is \textbf{type I} if for every (continuous unitary) representation \(\pi\) such that the center of \(\Hom_\mathrm{Rep}(\pi, \pi)\) is trivial we have a decomposition as a  direct sum of irreducible representations. 

\begin{example}
	Consider \(G(\A)\) the adelic points of a connected reductive LAG. This is a type one group. This is outside the scope of this thesis but can be found in \cite[Thm. 1.7 + Thm. 2.3]{deitmarTraceClassGroups2017}.
\end{example}

\begin{example}
	Consider \(G(\A)\) the adelic points of a connected reductive LAG. This is a second countable group. First consider the adele ring \(\A_F\) of F . This has the restricted product topology, where if \(\mathcal{O}_\nu\) is the ring of integers of \(F_\nu\), then an arbitrary open subset looks like a union of sets of the form 
	\[U_S \times \prod_{s\notin S} \mathcal{O}_s,\]
	where \(U_S\subseteq \prod_{s\in S}F_s\) is open in the product topology. 
	Because for any place \(F_\nu\) is second countable and the product of second countable spaces is second countable it is clear that \(\prod_{s\in S} F_s\) is second countable. Moreover there is a countable number of finite subsets of \Z, hence there is a bijection between a basis of the restricted product topology and \(\aleph_0\times \aleph_0 \) which is countable hence this topology is second countable.
	
	If \(G \defeq \Spec F[x_1, ..., x_n]/(f_1, ..., f_m)\) is an affine scheme then the topology on \(G(\A)\) is the subspace topology of \(\A^n\) on which all the \(f_1, ..., f_m\) vanish \cite{conradWeilGrothendieckApproaches2012}. In particular the finite product of second countable spaces is second countable and subspaces of second countable spaces are second countable, hence \(G(\A)\) is second countable. 
\end{example}

\begin{example}
	Consider \(G(\A)\) the adelic points of a connected reductive LAG. This is a unimodular group. The proof is outside the scope of this thesis but is stated in \cite[Lem. 2]{conradStanfordNumberTheory}.
\end{example}

The idea of a direct integral is review in \ref{app:direct_int} to get a quick idea consider the following example:
\begin{example}[Direct Sums]
	Let \(I\) be a countable set with the discrete sigma algebra and counting measure \(\mu\). Let \((\mathcal{H}_i)_{i\in I}\) be a collection of Hilbert spaces then
	\[\bigoplus_{i\in I} \mathcal{H}_i = \left\{ (h_i)_{i\in I}\in \prod_{i\in I} \mathcal{H}_i : \int_I \norm{h_i}_i^2 d\mu <\infty \right\}.\]
	I.e. the Hilbert space direct sum is by definition square summable sequences, but sums are just discrete integrals.
\end{example}

\begin{Theorem}[Plancherel, \cite{follandCourseAbstractHarmonic2016}, 7.44]
	The regular represntation of a type I, second countable and unimodular topological group is a direct integral of the irreducible unitary representations. 
\end{Theorem}
\begin{remark}
	The Plancherel theorem says much more in fact. Like the Peter-Weyl theorem for compact groups it doesn't just give you that some direct integral decomposition exists, it contains many more details about the topology and measure on the set of unitary irreducible representations, and which representations are associated to them in the direct integral. We are being brief as this is motivational.
\end{remark}



\subsection{The Work of Langlands}
It is as a continuation or variation of this tradition that we see the work of Langlands in \cite{langlandsFunctionalEquationsSatisfied1976}, in which he provides some decomposition of the spectrum of the adelic points of a connected reductive algebraic group over a number field \(G(\A)\).

\begin{Theorem}[\cite{arthurEisensteinSeriesTrace1979}, MAIN THEOREM (b)]
	There is an orthogonal decomposition of the representation of \(G(\A)\) on \(L^2(G(\Q) \backslash G(\A))\) into 
	\[L^2(G(\Q) \backslash G(\A)) = \bigoplus_{\mathscr{P}}L^2_\mathscr{P}(G(\Q) \backslash G(\A)),\]
	where \(\mathscr{P}\) runs over certain ``associate classes'' of parabolics of \(G\) and the summands are the direct integrals of spaces of \(L^2\) automorphic forms.
\end{Theorem}
These direct integrals are in fact constructed out of subspaces generated by Eisenstein series. 

The spectrum of \(L^2(G(\A))\) refers to such a decomposition. In particular we have some important ``pieces'' to such a decomposition. We call such decompositions ``spectral'', alluding to the spectral theorem which provides such a decomposition in terms of the eigenvector of certain operators. Moreover these decompositions are largely proved in terms of the more general spectral theorems. The piece that decomposes into a direct sum of irreducible is called the \textbf{discrete spectrum}. The compliment of the discrete spectrum is called the \textbf{continuous spectrum}. One can define cuspidal \(L^2\) functions in the exact same way as cuspidal automorphic forms \ref{cuspidal_form_definition} and then it has been shown that the \textbf{cuspidal spectrum}, the subspace of \(L^2\) consisting of cusp forms, decomposes as a direct sum \cite[9]{getzIntroductionAutomorphicRepresentations2024}. Thus the cuspidal spectrum is contained in the discrete spectrum in this case. The \textbf{residual spectrum} is defined to be the compliment of the cuspidal spectrum in the discrete spectrum. Thus what one wants to do is find a decomposition of the regular representation \(G(\A) \curvearrowright \mathrm{L}^2(G(\A))\).

It is during this analysis that the ideas expressed in his famous letter \cite{langlandsLetterweil196701_rpl_0} would begin to form, as he noticed that certain Euler products of analytic functions were appearing in the constant terms of the Eisenstein series. In particular we will see how the intertwining operator \(M(s, w)\) appears in the constant term of Eisenstein series and Langlands observed that \cite{langlandsEulerProductsa} 
\[M(s) = \left( \prod_\alpha\frac{\pi^{1/2}\Gamma(\frac{1}{2}\mu_\infty(s)(H_\alpha))}{\Gamma(\frac{1}{2}(\mu_\infty(s)(H_\alpha) + 1))} \right)\prod_{p \text{ prime }} \left( \prod_\alpha \frac{\frac{1}{1 - p^{\mu_p(s)(H_\alpha) + 1}}}{1 - \frac{1}{p^{\mu_p(s)(H_\alpha) }}}\right).\]
leading to a general conjecture that there is a holomorphic and non-zero intertwining operator \(N(s, w)\) such that 
\[M(s, w) = r(s, w)N(s,w),\]
and \(r(s, w)\) is a ratio of L-functions, as defined by Langlands in for instance \cite{langlandsEulerProductsa}.

Note that this is the global statement. There is an analogous set of conjectures for the local pieces, namely \(M = \tensor_\nu A\) the tensor over local intertwiners. Then one wants a normalisation of the local operators \(\mathscr{A}\) satisfying a long list of properties. This is extensively dealt with in \cite{shahidiProofLanglandsConjecture1990}. Shahidi showed some cases of this conjecture in \cite{shahidiRamanujanConjectureFiniteness1988}: Let \(\pi\) be an automorphic representation, let \(S\) be a finite set of places such that \(\pi_\nu\) is unramified for \(\nu\notin S\). We have that there are some finite dimensional complex representations \(r_1, ..., r_m\) of \(^LM\) such that 
\[M(s, \pi)f = \bigotimes_{\nu\in S}A(s, \pi_\nu, w)f_\nu \tensor \bigotimes_{\nu\notin S} \prod_{i=1}^{m}\frac{L_S(is, \pi, \tilde{r_i})}{L_S(1+is, \pi, \tilde{r_i})} \tilde{f}_\nu.\]

Recently it was shown for classical groups that this \(N\) indeed has the required properties. In particular the following theorem is sufficient for the cases dealt with in \cite{jiangPolesCertainResidual2013}:
\begin{Theorem}[\cite{cogdellFUNCTORIALITYCLASSICALGROUPS}, 11.1]
	Suppose that \(\pi_\nu\) is a local component of a globally generic cuspidal representation \(\pi\) of \(G_n(\A)\). Then for any irreducible admissible unitary generic representation \(\pi'_\nu\) of \(\GL_m(k_\nu)\) the normalized intertwining operator \(N'(S, \pi'_\nu\times \pi_\nu, w)\) is holomorphic and non-zero for \(Re(s)\geq 0\)
\end{Theorem}

\subsection{Poles of Residual Eisenstein Series}
Moeglin and Waldspurger also achieved a more fine analysis of the spectrum of \(\GL_n\) in terms of residues of Eisenstein series. 
First consider the group \(\GL_n\). We then let \(n = ab\) for positive integers \(a,b\). If \(\tau\) is an irreducible, cuspidal automorphic rep of \(\GL_a\) then Moeglin and Waldspurger construct a representation of \(\GL_{ab} = \GL_n\) called the ``Speh representation'' and denote it 
\[\Delta(\tau, b).\]
They go on to prove that as \(\tau\) and \(b\) vary these representations span the residual spectrum of \(L^2(\GL_n(F) \backslash \GL_n(\A))\) \cite[Thm. 1.1]{jiangPolesCertainResidual2013}.

This representation is formed by taking iterated residues of Eisenstein series in the sense of \cite[V]{moeglinSpectralDecompositionEisenstein1995}. For a nice survey of problems in this area, of residues of Eisenstein series, there is \cite{jiangResiduesEisensteinSeries2008a}.

\todo[inline, color=blue]{Introduction}

\section*{Outline of Content}
Chapter one deals with the generalities of linear algebraic groups, the objects whose representation theory is the subject of discussion. First we define them and then look at the important subgroups that are used in the study of automorphic forms arising on the adelic points of these groups. We focus on the classical groups.

Chapter two deals with automorphic forms. We define automorphic forms in both the Archimedean and adelic places. Finally we give the details of how to view modular forms as automorphic forms. 

Chapter three is a discussion of the concept of the constant term in the Archimedean place. First we define the constant term of an automorphic form (Archimedean) and then we show how it is related to the constant term of the Fourier series of a modular form. Finally we show how the classical Siegel Phi operator can be realised as a constant term.

Chapter four is dedicated to automorphic representations. We define them and specify some important constructions that are needed in the final section.

In chapter five we define adelic Eisenstein series and show how they generalise the classical modular forms also known as Eisenstein series.

Chapter six is dedicated to the constant term in the adelic setting. We first define them and then go through the process of computing them in great detail for Eisenstein series. 

Chapter seven is for defining L-functions, the analytic invariants that are central to the Langlands program. We will give several of the special cases that appear through out history and the literature. 

Finally chapter eight contains some exposition of recent work on the poles of residual Eisenstein series.




\tableofcontents
\pagenumbering{arabic}

\chapter*{Notation}
\todo[inline]{Delete this section later perhaps: Use for reference while writing.}

\begin{itemize}
    \item \(F\) is a number field
    \item \(\nu\) is a place of \(F\)
    \item \(F_\nu\) is the completion of \(F\) at \(\nu\)
    \item \(\A\) of \(\A_F\) is the Adele ring, \(\A_f\) is the finite adeles, and \(\A_\infty\) the infinite adeles.
    \item \(B, P, M, U\): Borel, parabolic, Levi, unipotent
\end{itemize}

\chapter{Classical Groups}
We will recall a small amount of the theory of linear algebraic groups to fix conventions, for a more detailed treatment one should consult the litany of sources on this matter: For a full treatment see \cite{milneAlgebraicGroupsTheory2017}\cite{milneLieAlgebrasAlgebraic}\cite{milneBasicTheoryAffine}\cite{springerLinearAlgebraicGroups1998}. Excellent example computations can also be found in \cite{BuildingsClassicalGroups}\cite{makisumiStructureTheoryReductive}\cite{malleLinearAlgebraicGroups}\cite{NotesClassAlgebraic}. Or for a brief brush up on the main facts consult \cite[I.I.1]{borelAutomorphicFormsRepresentations1979}. The purpose of this section is to treat the classical groups and more specifically \(\Sp_{2n}\) as an example and work out some of the details of the general theory, in order to ``get our hands dirty'' and have some familiarity with this object, to become fundamental in what follows. Because this theory is made up of simple ideas that can often be obscured by the generality we will make several restrictive assumptions for ease of exposition. 

\section{Definition}
An \textbf{algebraic group} is for us a group scheme that is reduced, of finite type and defined over a field. \textbf{A linear algebraic group} (LAG) is simply an affine algebraic group.

\begin{proposition}
    An algebraic group is affine if and only if it is isomorphic to a Zariski closed subgroup of \(\GL_n\).
\end{proposition}
\proofbar{
    The forward implication is \cite[2.3.7(i)]{springerLinearAlgebraicGroups1998}. The converse is the basic fact that closed sub-schemes of affine schemes are affine \cite[II.5.T3]{mumfordRedBookVarieties1999}.
}

As Milne points out in \cite{milneAlgebraicGroupsTheory2017} these are matrix groups defined by polynomials, which happen to be the natural combinations of symbols that matrix multiplication will lead to. This means that they come with the powerful but cumbersome (for the beginner) technology of algebraic geometry. In particular one must be adept at moving between the following equivalences:
\begin{Theorem}[\cite{milneBasicTheoryAffine}, II.6, III.4]
    For \(F\) a field then the following categories are equivalent
    \begin{itemize}
        \item Group objects in \(\mathrm{Alg}_F^{op}\)
        \item Representable (in the category of groups) functors \(\mathrm{Alg}_F \to \mathrm{Group}\)
        \item Group object in the category of affine schemes over \(F\)
        \item Commutative \(F\)-Hopf algebras.
    \end{itemize}
\end{Theorem}

\begin{example}[\(\mathbb{G}_m\)]
    The first example that we will see again and again is the ``multiplicative group'' or \(\GL_1\) defined over the field \(K\). This is 
    \[\mathbb{G}_m \defeq \Spec\Bigl( K[x,y]/(xy - 1)\Bigr).\]

    As a representable functor this sends a \(K\)-algebra R to \(\Hom_K(K[x,y]/(xy - 1), R)\). These are ring maps that are K-linear, and  because \(y = x\inv\) we know that \(f(y) = f(x\inv) = f(x)\inv\) for \(f\in \mathbb{G}_m(R)\). Thus the maps are determined by where they send \(x\), moreover they always send it to a unit, i.e. \(Im f \subseteq R^\times\). For each element \(r\in R^\times\) we also have a map sending \(x\to r\) hence there is an isomorphism (of sets) that allows us to induce a group structure. 
\end{example}

The other important examples of such groups are the ``classical groups''. The exact groups that an author might mean by classical may vary, so we define them explicitly here. First let V be a finite dimensional F-vector space with a bilinear form \(\inner{,}\). An automorphism of this form is a map \(\alpha\in \Aut(V)\) such that 
\[\inner{\alpha(x), \alpha(y)} = \inner{x,y}.\]
Therefore we can consider the space of automorphisms of this form \(\Aut(V, \inner{,})\). This space, depending on the properties of the bilinear form, will define our classical groups. 

If the form is trivial, by which we mean, \(\forall x,y \;\; \inner{x, y} = 0\) then we define 
\[\GL(V) \defeq \Aut(V, \inner{,}) = \Aut(V).\]
If the form is non-degenerate and symmetric \(\forall x,y \;\; \inner{x, y} = \inner{y,x}\) then we define
\[\mathrm{O}(V) \defeq \Aut(V, \inner{,}).\]
Finally if the form is non-degenerate and skew symmetric \(\forall x,y \;\; \inner{x, y} = -\inner{y,x}\) then 
\[\Sp(V) \defeq \Aut(V, \inner{,}).\]
There are the further classical groups given by the determinant one subgroups, \(\SL(V), \mathrm{SO}(V)\) respectively (\(\Sp(V)\) one can show already implies that the determinant is one). We can make this into a functor from \(F\)-algebras to groups, by sending an \(F\)-algebra \(R\) to \(G(V)\tensor_F R\). Thus these define LAG's. 

\begin{remark}
    Often the unitary groups are considered classical, however we don't want to deal with field extensions and so omit them here. 
\end{remark}

\section{Subgroups}
From now on let \(G\) be a (split reductive) LAG defined over a number field \(F\).
\begin{remark}[For the experts]
    We restrict to split reductive LAG in what follows. This is justified by the fact that the classical groups are all split reductive over number fields.  
\end{remark}

Subgroups with special properties allow us to reduce and break up problems into smaller ones. Here we will briefly review and compute some examples of special subgroups. The point of these subgroups is two fold. Some of them will help us perform ``induction'' from smaller simpler groups to larger ones. Others are there essentially as a part of the combinatorial data that classifies the groups we are working with. In particular we need to understand all the pieces of the so called \textbf{Langlands-Iwasawa decomposition},
\begin{equation}\label{eq:iwasawa_decomposition}
    G(\A) = M(\A)U(\A) K = T(\A)U(\A)K.
\end{equation}

\subsection{Parabolics, Levis and Unipotents}
Parabolic subgroups have two equivalent formulations, both useful.
\begin{definition}
    A subgroup \(P\subseteq G\) is called \textbf{parabolic} if \(G/P\) is a complete variety. Eqquivilently we can ask for \(P\) to contain a Borel (see \ref{borel_torus}).
\end{definition}

Completeness is the algebro-geometric analogue of compact, which is always a desirable property. The fact that they contain a Borel gives us an algebraic ``parametrisation'' of these subgroups, in the case of the classical groups through the use of flags or roots. It is very important to have a parametrisation of the parabolic subgroups when it comes to taking constant terms of Eisenstein series which we will discuss in the later section \ref{constant_terms}.

Parabolics also have the nice property that they split into a semi-direct product 
where one of the factors is a reductive group \(M\). For this recall the definition
\begin{definition}\label{unipotent_radical_definition}
    A matrix \(m\) is \textbf{unipotent} if for some \(n\geq 0\) we have that \((m-1)^n = 0\). A subgroup is \textbf{unipotent} if all its elements are unipotent. The \textbf{unipotent radical} of \(G\)is the maximal closed, connected, unipotent subgroup. A linear algebraic group is \textbf{reductive} if its unipotent radical is trivial.
\end{definition}
Then we have the following fact / definition:

\begin{Lemma}[\cite{borelLinearAlgebraicGroups1991} 11.22]
    There is a split exact sequence (of algebraic groups)
    \[0 \to U \to P \to M \to 0,\]
    where \(U\) is the unipotent radical of P, and \(M\) is a reductive group known as a \textbf{Levi} (unique up to conjugacy).
\end{Lemma}

Thus doing things on a parabolic allows us to induce said actions up to the whole group, whist maintaining the nice property of being reductive. This is the technique of ``parabolic induction'' \cite[Thm. 10]{bernsteinREPRESENTATIONSPADICGROUPS1992} that we wont explicitly talk about here but which is happening secretly in the background in \ref{automorphic_isotypic_subspaces}.

\begin{Remark}[Bad Etymology]
    The origin of the name parabolic is a mystery. Borel in his history \cite[VI.\S 2]{EssaysHistoryLie} attributes it to R. Godement in \cite{godementGroupesLineairesAlgebriques}. Godement conjectures that the quotient \(G(\A) / G(\Q)\) is compact if and only if every element of \(G(\Q)\) is semi-simple, as is the case in classical groups. \todo[inline]{this is probably known by now.} He says that 
    \begin{quote}
        Lorsque n'est pas compact, il est non moins facile de conjecturer qu’on doit pouvoir définir quelque chose d’analogue aux classiques ``pointes paraboliques'', lesquelles doivent correspondre à des  sous-groupes unipotents non triviaux de \(G_\Q\)
    \end{quote}
    which roughly (google) translates to that one can also conjecture that non-trivial unipotent elements should correspond to ``parabolic points'' in a fundamental domain.

    In the case of modular forms the fundamental domain is \(\mathcal{H} = \SL_2(\R)/SO_2(\R)\) (using orbit stabiliser theorem). We have the classification of elements of  \(\SL_2(\R) -\{\pm 1\}\) as in \cite[3.5]{borelAutomorphicFormsSL21997} via their trace
    \[g\text{ is of type } \;\;\; 
    \begin{cases}
        \text{Elliptic if} & \frac{1}{2}|tr(g)| < 1 \\
        \text{Parabolic if} & \frac{1}{2}|tr(g)| = 1 \\
        \text{Hyperbolic if} & \frac{1}{2}|tr(g)| > 1 \\
    \end{cases}
    .\]
    Being parabolic is equivalent to having eigenvalue 1 hence by the Jordan decomposition we know that parabolics in \(\SL_2\) are conjugate (over \(\C\)) to 
    \[\begin{pmatrix}
        1 & 1\\
        0 & 1
    \end{pmatrix},\;\;\; \pm\begin{pmatrix}
        1 & 0\\
        0 & 1
    \end{pmatrix}.\]
    Clearly the standard parabolic 
    \[\begin{pmatrix}
        a & b \\
         & a\inv
    \end{pmatrix} \subseteq \SL_2(\R),\]
    contains these matrices, and moreover all parabolics are \textit{conjugate} to this parabolic. Hence all parabolic elements are contained in a parabolic subgroup. This classification, it seems, relies entirely on the \textit{aesthetic} connection with the classification of the sections of conics via eccentricity.

    To connect this to Godement's concept we have two facts from classical geometry. Proper parabolic subgroups of \(\SL_2(\R)\) can be realised as the stabilisers of lines in \(\R^2\) under the standard action of \(\SL_2\) on \(\R^2\) \cite[2.6]{borelAutomorphicFormsSL21997} and moreover some an element of \(\SL_2(\R)\) is parabolic if and only if it has one fixed point on \(\partial\bar{\mathcal{H}}\) and none on \(\mathcal{H}\) \cite[3.5]{borelAutomorphicFormsSL21997}. 

    The take away is that perhaps the folklore of the name being for ``para-Borelic'', as in kind of a Borel, is probably a better way of thinking of them.
\end{Remark}

\subsubsection{The Example of \(\Sp_{2n}\)}
We collect the following facts as they will be useful in what is to come. Good references are the notes \cite{conradStandardParabolicSubgroups} and the book \cite[\S 8]{BuildingsClassicalGroups}. 

Let \((V, \inner{,})\) be a symplectic space as above and \(Sp(V)\) is the automorphisms preserving the form. A \textbf{flag} of \(V\) is a sequence of strict inclusions of vector subspaces 
\[\{0\}\subset V_1 \subset \cdots \subset V_{n-1} \subset V. \]
A subspace of \(V\) is said to be \textbf{isotropic} if the form is constantly zero on it (in both variables). A flag is \textbf{isotropic} if the proper subspaces in it are isotropic subspaces. A \textbf{maximal isotropic} flag is one with exactly \(n\) components. \(\Sp_{2n}\) acts on a flag by acting on each of the subspaces. This action preserves isotropic flags i.e. it sends an isotropic flag to an isotropic flag. Stabilisers of isotropic flags give parabolics of \(\Sp\) and moreover all parabolics arise in this way \cite[Exercise 3.2.16, 6.2.11]{springerLinearAlgebraicGroups1998}.

\begin{example}
        Consider a four dimensional vector space \(V\) with a form given by the matrix
        \[\begin{pmatrix}
            & I_2 \\
            -I_2 & 
        \end{pmatrix},\]
        then a maximal isotropic flag is 
        \[0 \subset Fe_1 \subset Fe_1 \oplus Fe_2 \subset F^4,\]
        where \(e_i = (\delta_i^j)_j\). This has stabiliser consisting of matrices in \(\Sp\) of the form
        \[\begin{pmatrix}
            *&*&*&* \\
             &*&*&* \\
             && *& \\
             && *& *
        \end{pmatrix}.\]
    \end{example}

    \label{maximal_parabolic}
    In particular maximal parabolics of \(\Sp\) are stabilizers of \textit{minimal} (non-trivial flags), i.e. stabilisers of non-zero isotropic subspaces,
    \[0 \subset V_\ell \subset V,\]
    where \(V_\ell = span_F(e_1, ..., e_\ell)\). Then the stabilizer is 
    \[\begin{pmatrix}
        * &*&*&* \\
        0 &*&*&* \\
        0 &*&*&* \\
        0 &*&*&* \\
    \end{pmatrix},\]
    with the sizes of the diagonal blocks being (these numbers square)
    \[\begin{pmatrix}
        \ell &*&*&* \\
        0 &n-\ell&*&* \\
        0 &*&\ell&* \\
        0 &*&*&n-\ell \\
    \end{pmatrix}.\]
    This has Levi
    \[\begin{pmatrix}
        A &&& \\
         &a&&b \\
         &&(A^T)\inv& \\
         &c&&d \\
    \end{pmatrix}, \;\;\; A\in \GL_\ell(F), \;\;\; \begin{pmatrix}
        a & b\\
        c & d \\
    \end{pmatrix} \in \Sp_{2(n-\ell)}(F),\]
    and unipotent 
    \[\begin{pmatrix}
        1 &*&*&* \\
        & 1&*& \\
        && 1& \\
        &&*&1
    \end{pmatrix},\]
    with relations among the entries.

    \subsection{Borel and Torus}\label{borel_torus}    
    A \textbf{split torus} is an algebraic group that is isomorphic to \(\GL_1^b\) for some \(b\in\N\).

    \begin{example}[Bad Etymology]
        \(\GL_1 /\C\) is a split torus. Consider the field extension \(\C/\R\). Then \C has the inner product given by 
        \[\inner{z, z'} \defeq \bar{z}z'.\]
        We can look at the elements of \(\C\) that preserve this inner product, 
        \begin{align*}
            U(1)&\defeq \{c\in \GL_1(\C) : \forall z,z'\in \C , \quad \inner{cz, cz'} = \overline{cz}cz' = \bar{z}z'\} \\
             &= \{c\in \GL_1(\C) : |c| = 1\}.
        \end{align*}
        Note that this is a (real) line topologically so we dont expect it to be a complex varity. Indeed this defines a \textbf{real} algebraic group given by the zero locus in \(\R^2\) of the two variable polynomial \(x^2 + y^2 - 1\). In other words
        \[ U(1) \cong \mathrm{MaxSpec}\big(\R[x,y]/(x^2 + y^2 - 1)\big) .\]
        Now if we base change to \C we have 
        \begin{align*}
            \R[x,y]/(x^2 + y^2 - 1) \tensor_\R \C &\cong \C[x,y]/\big((x+iy)(x-iy) -1 \big)\\
             &\cong \C[s,t]/(st -1) \\
             &\cong \C^\ast.
        \end{align*}
        Thus \(\GL_1 / \C\) is the complexification of the torus \(U(1)\).
    \end{example}
    
    \begin{remark}
        These tori also play the same role in the classification of reductive LAG as the real Lie groups called tori play in the classification of Lie groups \cite{hallLieGroupsLie2015}.
    \end{remark}
    
    A subgroup that is isomorphic to a split torus and is maximal in this respect is called a maximal split torus. 
     \begin{example}
        The classic example of a maximal split torus is the group of diagonal matrices in \(\GL_n\).
     \end{example}

    A \textbf{Borel} is a maximal closed solvable connected subgroup of \(G\).

    \begin{example}
        The standard Borel of \(\GL_n\) is the group of upper triangular matrices. If \(n\) is even and one intersects this Borel with \(\Sp_{2(\frac{1}{2}n)}\) then we get the standard Borel of \(\Sp_{2(\frac{1}{2}n)}\).

        Lets prove this in \(\GL_2\) and then believe that the only complication to going to larger \(n\) is keeping track of indices. So let 
        \[B = \begin{pmatrix}
            \ast & \ast \\
             & \ast
        \end{pmatrix},\]
        we need to show that the derived series terminates for it to be solvable. So let 
        \[g = \begin{pmatrix}
            x & y\\
             & z
        \end{pmatrix}, \;\;\; h = \begin{pmatrix}
            a & b \\
            & c
        \end{pmatrix},\]
        be arbitrary in \(\GL_2\), their commutator is then 
        \[g\inv h\inv gh =  \begin{pmatrix}
            1 & \frac{bx - ay}{ax} \\ & 1
        \end{pmatrix} .\]
        Hence
        \[[B, B] = \begin{pmatrix}
            1 & \ast \\ & 1
        \end{pmatrix}.\]
        Commutate two arbitrary elements again shows that  
        \[[[B, B], [B, B]] = 1.\]
    It is clear that this is a closed subgroup because it is itself a linear algebraic group, moreover for LAG's we have the algebraic criterion of connectedness given by having the only idempotents in the representing algebra being \(0, 1\) \cite[1.5]{getzIntroductionAutomorphicRepresentations2024}. Because \(B = \Spec \Z[x_{i,j}: 1\leq i,j\leq 2][y] / (\det(x_{ij})y - 1, x_{2,1})\) it is clear that this group is connected.     
    Finally it is clear that if a subgroup strictly contains this one then it is in fact all of \(\GL_2\) and hence this is maximal. Therefore this is a Borel.
    \end{example}
        
    A Borel can be considered to be a parabolic that is minimal with respect to inclusion. The maximal tori then form the Levis of these parabolics. In particular for a Borel \(B\) we have that 
    \[B = TU,\]
    for a maximal torus \(T\) and unipotent \(U\).

    If a Borel \(B\) is fixed, then a parabolic containing this Borel \(B\subseteq P\) is called standard, the unique Levi of a standard parabolic containing this Borel is called the \textbf{standard Levi}.

    \subsection{Maximal Compact Subgroups}\label{max_compact_subgroup}
    We will often need to fix a maximal compact subgroup \(K\subseteq G(\A)\), note that the topology is not the Zariski topology but the one specified in \cite{conradWeilGrothendieckApproaches2012}, this is sometimes known as the ``Hausdorff'' topology. These maximal compact subgroups are not unique and as such when fixing one it can be arranged to have many nice properties \cite[I.1.4]{moeglinSpectralDecompositionEisenstein1995}. In particular if we have a group \(G\) and a fixed Borel \(B\):
    \begin{itemize}
        \item First require that 
        \[K = \prod_\nu K_\nu,\]
        where the product is over all places of \(F\) and \(K_\nu\subseteq F_\nu\) is maximal compact.
        \item For almost all places \(\nu\), \(G(\mathcal{O}_{F_\nu})\) is defined and is maximal compact in \(G(F_\nu)\) hence we can require \(K_\nu = G(\mathcal{O}_{F_\nu})\) at these places. 
        \item We require 
        \[G(\A) = B(\A)K.\]
        \item For every standard parabolic \(P = MU\) we have that 
        \[P(\A)\cap K = \Bigl( M(\A)\cap K \Bigr) \Bigl( U(\A)\cap K \Bigr),\]
        and \(M(\A)\cap K\) is a maximal compact subgroup of \(M(\A)\).
    \end{itemize}
     It is in terms of the third property that we like to think of the maximal compact subgroup, it is the complimentary piece of the Borel. Moreover the fourth property should be thought of as a condition that the maximal compact subgroups are well behaved with the way that we are moving between the bigger and smaller reductive groups.
    Maximal compact groups with all these properties are said to be in \textbf{good position}.

\chapter{Automorphic Forms}

The story starts with the classical modular forms, or functions on the upper half plane that satisfy some invariance conditions and differential equations. This evolves into the notions of Maass forms on symmetric spaces and eventually reaches its apotheosis in the concept of automorphic form that we will present here. 

We will present two notions of automorphic forms here. In the literature they are both called ``automorphic forms'' however here we will distinguish those that are defined only on the Archimedean points as ``Archimedean automorphic forms'' for clarity.

The first natural question is if there is a special case of automorphic forms which yield modular forms. The space of automorphic forms is larger than just modular forms, it gives the space of Maass forms (or modular and Maas forms, depending on convention). This is well covered in the literature \cite{emertonCLASSICALMODULARFORMS}\cite[3.2]{bumpAutomorphicFormsRepresentations1997}\cite{booherVIEWINGMODULARFORMS}\cite{garrettTransitionEisensteinSeries2016}. We explain modular forms as Archimedean automorphic forms as we think it is where the connection is clearest. We will give an example of modular forms as adelic automorphic forms when we come to the Eisenstein series in section \ref{sec:classic-eisenstein}.




\section{Archimedean Automorphic Form}
Fix a number field \(F\) and a classical group \(G\) defined over \(F\). Let \(\infty\) denote the set of Archimedean places. We denote \(\A_\infty = F_\infty \defeq \prod_{\nu\in\infty} F_\nu\) and note that \(G(F_\infty) \cong \prod_{\nu\in\infty}G(F_\nu)\). We denote \(\A_f \defeq \sideset{}{'}\prod_{\nu\notin \infty} F_\nu\) the finite adeles. Consider \(\nu\in \infty\) one such Archimedean place, then \(F_\nu\) is either \R or \C. In particular (the analytification of) \(G(F_\nu)\) is a Lie group and we call a function, \(\phi: G(F_\nu) \to \C\), \textbf{smooth}  if it is smooth in the sense of functions on manifolds. The collection of such smooth functions on \(G(F_\infty)\) will be denoted \(C^\infty(G(F_\infty))\).

    Because \(G(F_\infty)\) is a Lie group we know how to define its Lie algebra \(\mathfrak{g}\), as the tangent space at the identity, and we now denote \(Z(\mathfrak{g})\) the centre of the \textit{universal enveloping algebra} of the \textit{complexification} of \(\mathfrak{g}\), it would be more reasonable to use \(Z(\mathcal{U}(\mathfrak{g}_{\C}))\) but that is too cumbersome so we follow the tradition. 
    A vector in a \(Z(\mathfrak{g})\)-module \(\phi\in V\) is called \(Z(\mathfrak{g})\)-\textbf{finite} if the space \(\mathrm{span} \big(Z(\mathfrak{g})\phi\big)\) is finite dimensional. 

	Let \(K_\infty\subseteq  G(F_\infty)\) be a maximal compact subgroup. Then again an element of a \(K_\infty\)-module is \(K_\infty\) \textbf{finite} if the span of its orbit is a finite dimensional vector space (we think here of \(\C[K_\infty]\)-modules).

	To define automorphic forms we look at the representation \(C^\infty(G(F_\infty))\) with the right regular action of \(K_\infty\), i.e. \(g.f(x) = f(xg)\).  In particular the \(Z(\mathfrak{g})\) module structure is induced from the action of \(\mathfrak{g}\) on \(C^\infty(G(F_\infty))\) by \label{lie_algebra_action}
	\[z.F(g) = \Dif{}{t}F(ge^{tz})|_{t=0}, \quad x\in \mathfrak{g}.\] 
	
	Finally we want a growth condition. Fix an embedding \(\iota : G\to GL_n\) which gives another embedding \(G\to SL_{2n}\) via
	\[\iota': g\mapsto \begin{pmatrix}
		\iota (g) & \\
		& (\iota (g))^{-t}
	\end{pmatrix}.\]
	We have denoted the inverse of the transpose by \(-t\). A function \(\phi : G(F_\infty) \to \C \) is of \textbf{moderate growth} if there are constants \((c,r)\in \R_{>0}\times \R\) such that 
	\[|\phi(g)| \leq c\norm{g}^r \defeq c \left(\prod_{\nu\in\infty} \sup_{1\leq i, j\leq 2n} |\iota' (g)_{i,j, \nu}|_\nu\right)^r.\]

	\begin{remark}
		One can define norms on \(G(\A)\) via the linearisation of such groups, i.e. their representations. Concretely if \(\sigma\) is a finite dimensional complex representation of \(G(\A)\) on some Hilbert space with a \(K_\infty\) invariant inner product and \(\ast\) is the adjoint matrix with respect to this Hilbert space structure then a \textbf{norm} on \(G(\A)\) is a function of the form
		\[g\mapsto \big(\mathrm{tr }\; \sigma(g)^*\sigma(g)\big)^{\frac{1}{2}}.\]
		This moderate growth condition is then equivalent to some norm \(\norm{ - }\) existing on \(G(F_\infty)\) such that 
		\[|\phi(x)| \leq C\norm{x}^n,\]
		for some \(C>0, n\in \N\) and all \(x\in G(F_\infty)\). This is also equivalent to all such norms satisfying this condition \cite[Part 1, ``Automorphic Forms and Automorphic Representations'', 1.2]{borelAutomorphicFormsRepresentations1979}.
	\end{remark}
	
	Finally a subgroup \(\Gamma \subseteq G(F_\infty)\subseteq \GL_n(F_\infty)\) is called \textbf{arithmetic} if \(\Gamma \cap \GL_n(\Struc_\infty)\) is a finite index subgroup in both \(\Gamma\) and \(\GL_n(\Struc_\infty)\).

	\begin{Definition}
		Let \(\Gamma\leq G(F_\infty)\) some (arithmetic) subgroup, an \textbf{automorphic form for \(\Gamma\)} is a smooth function of moderate growth 
		\[\phi: G(F_\infty) \to \C,\]
		that is \(K_\infty\) and \(Z(\mathfrak{g})\) finite with  (left) \(\Gamma\) invariance. We denote the set of these ``Archimedean'' automorphic forms by \(\mathcal{A}(\Gamma \backslash G(F_\infty))\).
	\end{Definition}


\section{Adelic Automorphic Form}
Here we follow \cite[I.2.17]{moeglinSpectralDecompositionEisenstein1995} and \cite[Part 1, ``Automorphic Forms and Automorphic Representations'', 1.2]{borelAutomorphicFormsRepresentations1979}. Fix a Borel \(B\subseteq G\) and a standard parabolic \(B\subseteq P \subseteq G\) with a standard Levi decomposition \(P = MU\). We let \(K\) be a maximal compact subgroup of \(G(\A)\) that is in good position as in section \ref{max_compact_subgroup}.

We say that \(f: G(\A_f) \to \C\) is smooth if it is locally constant in the Hausdorff topology and we denote the set of such smooth functions \(C^\infty(G(\A_f))\).

	Thus for the full adeles we have the notion of smooth as an element of the tensor product,
	\[C^\infty(\mathbb{A}_F) \defeq   C^\infty(G(\mathbb{A}_f))   \otimes   C^\infty(G(F_\infty)).\]
	Notice that a priori the codomain is an infinite tensor product over \C of copies of \C, this is \textit{canonically} isomorphic to \C, thus we can conflate a smooth function with its composition along this isomorphism and think of them as functions into \C.

	We still consider \(Z(\mathfrak{g})\) to be the center of the universal enveloping algebra of the complexified Lie algebra at the infinite places, exactly as before. We define an action by linearly extending
    \[z.(f\tensor g) = f\tensor (z.g),\]
    i.e. it acts on the Archimedean places as in the setting of Archimedean automorphic forms. 
	
	The definition of moderate growth carries over verbatim, however we change the set of places multiplied over to be all of them now. Specifically fix an embedding \(\iota : G\to GL_n\) which gives another embedding \(G\to SL_{2n}\) via
	\[\iota': g\mapsto \begin{pmatrix}
		\iota (g) & \\
		& (\iota (g))^{-t}
	\end{pmatrix}.\]
	We have denoted the inverse of the transpose by \(-t\). A function \(\phi : G(\A) \to \C \) is of \textbf{moderate growth} if there are constants \((c,r)\in \R_{>0}\times \R\) such that 
	\[|\phi(g)| \leq c\norm{g}^r \defeq c \left(\prod_{\nu} \sup_{1\leq i, j\leq 2n} |\iota' (g)_{i,j, \nu}|_\nu\right)^r.\]
    
    \begin{remark}[\cite{borelAutomorphicFormsRepresentations1979}, Part 1, ``Automorphic Forms and Automorphic Representations'']
        The collection of moderate growth functions is independent of the choices of embedding. 
    \end{remark}

\begin{definition}
    A function \(\phi: U(\A)M(F)\backslash G(\A) \to \C\) is an \textbf{automorphic form} if it is smooth, moderate growth, \(Z(\mathfrak{g})\) and \(K\) finite. We will denote the set of these automorphic forms by \(\mathcal{A}(U(\A)M(F)\backslash G(\A))\).
\end{definition}

\begin{remark}
    It is important that \(M(F)\) is treated as a subgroup of \(M(\A)\) via the diagonal embedding.
\end{remark}
\begin{remark}
	What we have called automorphic forms are sometimes referred to as ``smooth K-finite automorphic forms'' \cite[2.2]{cogdellLecturesLfunctionsConverse}.
\end{remark}
\begin{remark}
	This is a more general setup than in the Archimedean case as we only require \(U(\A)M(F)\) invariance. By choosing the parabolic to be \(G\) itself we get full \(G(F)\) invariance as in the Archimedean case. 
\end{remark}
	
    

\section{Modular Forms} \label{sec:modular-forms}

	Recall the definition of a \textbf{modular form of weight k} (of full level and trivial character) \cite[1.1.2]{diamondFirstCourseModular2005} as a function
		\[\phi: \mathcal{H} \to \C,\]
		where \(\mathcal{H}\) is the upper half plane in \C, that is holomorphic, satisfies 
		\[\phi(\gamma.z) = (cz+d)^k\phi(z), \quad \gamma = \begin{pmatrix}
			a &b \\
			c &d
		\end{pmatrix}\in \SL_2(\Z),\]
		and is of moderate growth, that is sub-exponential growth.

	We want to think of the upper half plane as a quotient of the \(\Q_\infty = \R\) points of some reductive group. If we have a transitive action of some reductive group then by the orbit stabiliser theorem we would have a bijection of sets.

	\begin{Theorem}\label{thm:upper_half}
		\[\mathcal{H} \cong  \SL_2(\R) / \SO_2(\R) ,\]
		as sets.
	\end{Theorem}
	\proofbar{
		Consider the action 
		\[\SL_2(\R) \curvearrowright \mathcal{H}: \;\; \begin{pmatrix}
			a & b\\
			c & d
		\end{pmatrix}.z = \frac{az + b}{cz + d}.\]
		Then look at the orbit of \(i\), namely 
		\[\begin{pmatrix}
			a & b\\
			 & d
		\end{pmatrix}.i = \frac{ai + b}{d} = a^2i + ab,\]
		which letting \(a, b\in \R\) vary is clearly surjective onto the whole upper half plane. So there is one orbit, and hence by the orbit stabiliser we know that 
		\[\mathcal{H} \cong \SL_2(\R) /\mathrm{stab}(i) ,\]
		so we want to find
		\[\mathrm{stab}(i) = \left\{ g = \begin{pmatrix}
			a & b\\
			c & d
		\end{pmatrix}\in \SL_2(\R) : g.i = i   \right\},\]
		in particular we solve 
		\begin{equation*}
			\begin{aligned}
				i &= g.i = \frac{ai + b}{ci + d} = (c^2 + d^2)\inv (ac + bd  + i\det g) .\\
			\end{aligned}
		\end{equation*}
		So equating coefficients we have 
		\[\det g (c^2 + d^2)\inv  = 1 \implies c^2 +  d^2 = \det g = 1,\] 
		on the other hand 
		\[ac + bd = 0.\]
		Now the pairs \(c^2 + d^2 = \det g = 1\) are parameterized by \(\theta\in [0, 2\pi)\) using \(c = \sin \theta, d =  \cos\theta\) hence subbing this into the above equation
		\[\frac{-b}{a} = \tan\theta,\]
		and so \(b = -k\sin\theta, a = k\cos\theta\) for some  \(k\in \R\) but the determinant must be \(1\) so \(k = 1\).
		Hence 
		\[\mathrm{stab}(i) = \left\{ \begin{pmatrix}
			\cos\theta & -\sin\theta \\
			\sin\theta & \cos\theta
		\end{pmatrix} : \theta \in [0, 2\pi)\right\} = \SO_2(\R).\]

	}
    \begin{remark}
    	This can be beefed up to an isomorphism of complex analytic spaces. 
        Sometimes to make the action of certain (Hecke) operators more apparent this is exhibited as 
        \[\mathcal{H} \cong \GL_2^+(\R)/ A_{\GL_2}\SO_2(\R).\]
        This obscures the connection with the reductive group setting however so we avoid it here. 
    \end{remark}

	\(\SL_2\) is a reductive group and \(\SO_2(\R)\) is its maximal compact subgroup of \(\SL_2(\R)\). The decomposition of the upper half plane in \ref{thm:upper_half} suggests that function on the upper half plane might have some invariance along the maximal compact subgroup of the reductive group \(\SL_2\). If we define 
	\[B \defeq \left\{\begin{pmatrix}y^{1/2} & x y^{-1/2}\\ & y^{-1/2}\end{pmatrix} : x,y\in \R, y\neq 0 \right\}\]
	which happens to be the real points of a Borel subgroup of \(\SL_2\) we have the picture;
    
		% https://q.uiver.app/#q=WzAsNCxbMCwwLCJcXGJlZ2lue3BtYXRyaXh9eV57MS8yfSAmIHggeV57LTEvMn1cXFxcICYgeV57LTEvMn1cXGVuZHtwbWF0cml4fVNPXzIoXFxSKT1cXFNsXzIoXFxSKSJdLFsyLDAsIlxcU2xfMihcXFIpL1NPXzIoXFxSKSJdLFsyLDIsIlxcU2xfMihcXFopXFxzZXRtaW51c1xcU2xfMihcXFIpIl0sWzQsMCwiXFxtYXRoY2Fse0h9Il0sWzEsMywiXFxzaW0iXSxbMCwxLCJwcm9qIl0sWzEsMywieFxcbWFwc3RvIHguaSIsMl0sWzAsMiwiXFx0ZXh0e2Rlc2NlbmQ/Pz99IiwyXV0=
		
		\vspace{6mm}
		
	\[\begin{tikzcd}[cramped, transform canvas={scale=1}]
		{B \SO_2(\R)=\SL_2(\R)} && {\SL_2(\R)/\SO_2(\R)} && {\mathcal{H}} \\
		\\
		&& {\SL_2(\Z)\setminus\SL_2(\R)}
		\arrow["\sim", from=1-3, to=1-5]
		\arrow["\mathrm{project}", from=1-1, to=1-3]
		\arrow["{g\mapsto g.i}"', from=1-3, to=1-5]
		\arrow["{\text{project}}"', from=1-1, to=3-3]
	\end{tikzcd}\]
	
		\vspace{12mm}
		
   	We can lift a function on \(\SL_2(\R) / \SO_2(\R)\) to \(\SL_2(\R)\) by composing with the projection, however this is not \(\SL_2(\Z)\) invariant, thus we need to add a pre-factor to ensure this in our associated automorphic form. The algebro-geometric perspective in \cite{emertonCLASSICALMODULARFORMS} can make this seem slightly less ad hoc. 

   	Thus for \(f\) a modular form of weight \(k\) the following function on \(\SL_2(\R)\)
	\[F(g) \defeq  (ci + d)^{-k}f(g.i),\]
	we claim is an Archimedean automorphic form for \(\SL_2(\Z)\). We take for granted its smoothness. The \(\SL_2(\Z)\) invariance is obvious from the modularity condition and we consider the moderate growth condition to be tautological. It remains to show the last two properties:

	\begin{Lemma}
		\(\SO_2(\R)\) is a maximal compact subgroup inside \(\SL_2(\R)\). \(F\) is an \(\SO_2(\R)\)-finite function.
	\end{Lemma}
	\proofbar{
		Using that \(\kappa = \begin{pmatrix}
			\cos\theta & -\sin\theta \\
			\sin\theta & \cos\theta
		\end{pmatrix} \in K = \SO_2(\R)\) acts trivially on \(i\), an elementary computation shows that for \(g  \in \SL_2(\R)\),
		\begin{equation*}
			\begin{aligned}
				F(g\kappa) = e^{-ik\theta}F(g). \\
			\end{aligned}
		\end{equation*}
		Hence \(F(g)\) is acted on by \(K\) via a one dimensional irreducible representation. In particular it is finite dimensional.
		}

	\begin{Lemma}
		\(F\) is a \(Z(\mathfrak{sl}_2)\) finite function.
	\end{Lemma}
	\proofbar{ Only a sketch. 

		The center of the universal enveloping algebra of the complexified Lie algebra is generated by the Casimir operator. We have the coordinates on \(\SL_2(\R) \) from \cite[1.19]{bumpAutomorphicFormsRepresentations1997} 
		\[\left\{\begin{pmatrix}y^{1/2} & x y^{-1/2}\\ & y^{-1/2}\end{pmatrix} : x,y\in \R, y\neq 0\right\}\SO_2(\R),  \]  in which 
		the Casimir acts as the differential operator
		\[\Delta = y^2\left(\left(\Dif{}{x}\right)^2 +\left(\Dif{}{y}\right)^2\right) - y\Dif{^2}{x\partial \theta},\] 
		\cite[1.29,Prop 2.2.5]{bumpAutomorphicFormsRepresentations1997}. Now we claim that F is an eigenfunction for this operator. 
		An element \((x,y,\theta) \defeq \begin{pmatrix}y^{1/2} & x y^{-1/2}\\ & y^{-1/2}\end{pmatrix}\kappa_\theta \in \SL_2(\R)\) acts on \(i\) by sending it to \(x+ iy\) (elementary computation). The bottom row of the product is \(\big(y^{-1/2}\sin\theta ;y^{-1/2}\cos\theta\big) \) which results in 
		\[F(x,y,\theta) = y^{k/2}e^{-ik\theta}f(x + iy).\]
		It is then a calculus exercise to apply \(\Delta\) to this, using the holomorphicity we also get that \(f_{xx} - f_{yy} = 0\) and \(f_y = if_x\) which cancels away terms and we get that 
		\[\Delta F(x,y,\theta) = \frac{k}{2}\left(\frac{k}{2} - 1\right) F(x,y,\theta).\]
		
		Therefore the dimension of \(Z(\mathfrak{g})F\) is simply one.
	}
    This example makes it clear that the two finiteness conditions for automorphic forms are in some sense functional equations that they must satisfy. 
	There is a nice explanation of how to lift this to the adelic setting in several places, the key is essentially the isomorphism 
	\[\Z \backslash \R \cong \Q \backslash \A_\Q / \hat{\Z} \]
	The details are quite clear in \cite[2.1]{cogdellLecturesLfunctionsConverse} or \cite{booherVIEWINGMODULARFORMS}. We will revisit this perspective through the example of the Eisenstein series in section \ref{sec:classic-eisenstein}.

\chapter{Automorphic Representations}
The references that will be most helpful are \cite[I.II]{borelAutomorphicFormsRepresentations1979}\cite{getzIntroductionAutomorphicRepresentations2024} for the general theory, we will follow the notation developed in \cite{moeglinSpectralDecompositionEisenstein1995} as it is somewhat standard. For the connection to classical modular forms there is \cite{emertonCLASSICALMODULARFORMS}\cite{bumpAutomorphicFormsRepresentations1997}\cite{booherVIEWINGMODULARFORMS}\cite{garrettTransitionEisensteinSeries2016}. We will discuss some of the details of their representation theory because it is both subtle and needed later. In particular we want to draw attention to what we think of as the "non-algebraic" nature of the representation theory.

\section{\((\mathfrak{g}, K)\)-Modules}

\section{Hecke Algebra}

\section{Automorphic Representations}

\subsection{Cuspidal Representations}
We will recal the definition of a cusp form \todo{reference the definition in the next chapter} in the next chapter.\todo{Chengjing example of isotypic subspaces}

\subsection{Tensor Products of Representations}

\section{Eisenstein Series}
\cite{lapidPerspectivesEisensteinSeries2022}, \cite{arthurEisensteinSeriesTrace1979}

\section{Spectral Decomposition}
\subsection{Definition and Role}
This is another one of the tools that can be used to compartmentalise problems in automorphic forms, by dealing with representations that appear in different parts of the spectrum. 

\subsection{The Decomposition of the Spectrum}


\subsection{Residual Representations of \(\GL_n\)}


\section{L-Functions}
\subsection{In General}
\subsection{Standard L-Functions for Classical Groups}
\subsection{L-Functions of Covering Groups}

\chapter{Spectral Decomposition and Eisenstein Series}
\section{Eisenstein Series}
As usual we fix a connected reductive group G defined over a number field F, with a Borel B, a standard parabolic with Levi decomposition \(P = MU\). 

Following the setup in \cite[I.1.4]{moeglinSpectralDecompositionEisenstein1995} we consider a \textbf{character} \(\chi\in \mathrm{Rat}(M) \defeq \Hom_{\mathrm{LAG}}(M, \mathbb{G}_m)\), thinking of it below as a natural transformation, and then define 
\[|\chi|: M(\A)\to \C , \;\;\; (m_\nu)\mapsto \prod_\nu|\chi(F_\nu)(m_\nu)|_\nu.\]
The intersection of the kernels of these characters is 
\[M^1 \defeq \bigcap_{\chi\in \mathrm{Rat}(M)}\ker |\chi|.\]
Thus we can define
\[X_M \defeq \Hom_{\textrm{TopGroup}}(M(\A)/M^1, \C^*) .\]
i.e. the collection of characters of \(M(\A)\) that are trivial on \(M^1\).
\begin{remark}
    To make it seem less mysterious this group has some importance in the more general theory, in particular it is one of the pieces in the ``Langlands decomposition'' (\ref{eq:iwasawa_decomposition}) of the Archimedean points of a parabolic and it has the property that \(M(\Q)\backslash M(\A)^1\) has finite measure \cite[4.9]{getzIntroductionAutomorphicRepresentations2024}.
\end{remark}
The set of \textbf{complex characters} of \(M\),
\[\mathfrak{a}_M^* \defeq \mathrm{Rat}(M)\tensor_\Z \C,\]
is isomorphic as \C vector spaces to \(X_M\). If \(Z_{G(\A)}\) is the center of \(G(\A)\) then we also have the space 
\[X_M^G \defeq \Hom_{\textrm{TopGroup}}((M(\A)/M^1)/Z_G, \C^*)\]
which is characters of \(M(\A)/M^1\) which are also trivial on the center of \(G\).

\begin{example}\label{ex:characters}
    For the maximal parabolic \(P_r\) with Levi \(M_r\) of \(\Sp_{2n}\) we have that \( X_{M_r}^{\Sp_{2n}}\) is at most a one dimensional \C vector space. 

    First of all we have that \cite[I.1.4]{moeglinSpectralDecompositionEisenstein1995}
         \[ X_{M_r}^{\Sp_{2n}} \subseteq X_{M_r} \cong \mathfrak{a}_{M_r}^*\defeq Rat(M_r) \tensor_\Z \C.\]
        Thus it is clearly sufficient to bound the dimension of \(\mathfrak{a}_{M_r}^*\) as a \C vector space, moreover this dimension agrees with the dimension of \(Rat(M_r)\) as a free \Z module. 

        Thus we compute \(\dim_\Z(Rat(M_r))\):
        \begin{equation*}
            \begin{aligned}
                Rat(M_r) &= Rat(\GL_r \times \Sp_{2m}) \\
                         &= \Hom(\GL_r \times \Sp_{2m}, \mathbb{G}_m) \\
                         (2)&\cong \Hom(\mathrm{Ab}(\GL_r \times \Sp_{2m}), \mathbb{G}_m) \\
                         (1)&\cong \Hom(\mathrm{Ab}(\GL_r) \times \mathrm{Ab}(\Sp_{2m}), \mathbb{G}_m) \\
                         (3)&\cong \Hom(\mathbb{G}_m \times 1, \mathbb{G}_m) \\
                         &\cong \Z.
            \end{aligned}
        \end{equation*}
        In (2) we have used the universal property of the abelianization \(\mathrm{Ab}(G) = \mathcal{D}(G) \setminus G = [G, G] \setminus G \) because \(\mathbb{G}_m\) is abelian. (1) is that the abelianization commutes with direct products. (3) is because \(\Sp\) is a perfect group.
        %https://groupprops.subwiki.org/wiki/Symplectic_group_is_perfect
        %https://mathoverflow.net/questions/35713/abelianization-of-a-semidirect-product
\end{example}

\begin{remark}
    This generalises to the metaplectic covers immediately as \( X_{M_r}^{\Mp_{2n}(\A)} \subseteq X_{M_r}\).
\end{remark}
There is the natural map \(m_P: G(\A) \to M^1 \backslash M(\A)\) sending \(umk \mapsto M^1 m\), where \(g = umk\) using the Langlands-Iwasawa decomposition \ref{eq:iwasawa_decomposition}.

Now if we take the collection of irreducible automorphic representations of \(M\),
 \[\hat{\mathcal{A}} \defeq \{(\pi, V) : \pi \text{ is an irreducible automorphic representation of }M\}\]
then we can think of \(X_M^G\) as being one dimensional automorphic representations (with some extra invariance) and so there is a natural action on \(\hat{\mathcal{A}}\) given by tensoring, i.e. if \(\lambda\in X_M^G\) and \((\pi, V)\in \hat{\mathcal{A}}\) then 
\[\lambda.\pi \defeq \lambda\tensor \pi\]
Then \(\hat{\mathcal{A}}\) decomposes as a disjoint union of its orbits. Consider the orbit \(\mathfrak{P}\) of a cuspidal representation \(\pi_0\), then by definition \(X_M^G\) acts transitively but it also acts freely \cite[II.1]{moeglinSpectralDecompositionEisenstein1995}. Thus \(\mathfrak{P}\) is in bijection with \(X_M^G\). Through this bijection we transmit the complex structure on \(\mathfrak{a}_M^*\) to \(X_M\) then to the quotient \(X_M^G\) and finally to \(\mathfrak{P}\).

Now we will define an Eisenstein series: Let \(\mathfrak{P}\) be as above, the orbit of a cuspidal automorphic representation endowed with a complex structure. Let \(\pi\in \mathfrak{P}\) and \(\phi_\pi \in \mathcal{A}(U(\A)M(k)\backslash G(\A))_\pi\), then \(\lambda\in X_M^G\) acts on \(\phi_\pi\) by 
\[\lambda.\phi_\pi(g) = (\lambda \comp m_P)(g) \phi_\pi(g).\]
which is then an element of \(\mathcal{A}(U(\A)M(k)\backslash G(\A))_{\pi\tensor \lambda}\). Finally we have the \textbf{Eisenstein series} which is defined by the following sum
\[E(\phi_\pi, \lambda, g) = \sum_{\gamma \in P(k)\backslash G(k)} \lambda.\phi_\pi(\gamma g)\]
whenever it is convergent. The first thing to note is that for a fixed \(\phi\) there is an open set in \(X_M^G\) and a compact subset of \(G(k)\backslash G(\A)\) such that the Eisenstein series converges (normally) \cite[II.1.5]{moeglinSpectralDecompositionEisenstein1995}.

If \(P = MU, P' = M'U'\) are two standard parabolics of \(G\) that are conjugate, i.e. such that for \(w\in G(k)\) we have \(wMw\inv = M'\)
Then \(w\) maps \(\mathfrak{P}\) to \(w\mathfrak{P}\), an orbit of an irreducible representations of \(M\) to an orbit of irreducible representations of \(M'\).

Then the Eisenstein series is closely related (through its constant terms as discussed in \ref{constant_conjugate_levi}) to the operator
\[M(w, \pi)(\phi_\pi)(g) = \int_{(U'(k)\cap wU(k)w\inv )\backslash U'(\A)} \phi_\pi(w\inv ug) du\]
where \(\pi\in \mathfrak{P}\), \(g\in G(\A)\) and \(\phi_\pi \in \mathcal{A}(U(\A)M(k)\backslash G(\A))_\pi\).

The key properties of both the Eisenstein series and this operator can be found in \cite[IV.1.8, IV.1.9, IV.1.10, IV.1.11]{moeglinSpectralDecompositionEisenstein1995}. Most importantly as a function of \(\mathfrak{P}\) it can be shown that (in the sense of Frechet spaces) they both have a meromorphic continuation to all of \(\mathfrak{P}\). This was also given a second ``soft proof'' more recently in \cite{bernsteinMeromorphicContinuationEisenstein2022}, with the spectral decomposition that follows from it also being worked out in \cite{delormeSpectralTheoremLanglands2021}. Moreover for the Eisenstein series at a point in \(p\in \mathfrak{P}\) at which it is holomorphic then \(E(\phi,p, g)\) is an automorphic form. 

We are not really in a position to convey the true importance of these objects in the theory of automorphic forms, however we will make some comments. First some surveys are \cite{lapidPerspectivesEisensteinSeries2022}, \cite{arthurEisensteinSeriesTrace1979}, \cite{kimEISENSTEINSERIESTHEIR}, \cite{jiangResiduesEisensteinSeries2008a}. To see the relation to the classical Eisenstein series there is \cite{garrettTransitionEisensteinSeries2016}. One thing that Eisenstein series do, as in the theory of modular forms, is that they furnish us with quasi-concrete examples. A we mentioned above \cite[IV.1.9.(b).i]{moeglinSpectralDecompositionEisenstein1995} tells us that at the holomorphic points the Eisenstein series takes an automorphic form and returns an automorphic form, thus we can use them to multiply our examples. Another reason that these functions are important is through their normalisation and constant terms, in which products of L functions appear, we discuss this more in section \todo[inline]{ref later}. This has been a fruitful method for proving theorems about L-functions as in \cite{shahidiEisensteinSeriesAutomorphic2010}\cite{pollackRANKINSELBERGMETHODUSER}\cite{arthurEisensteinSeriesTrace1979}.

\section{Spectral Decomposition}\label{spectral_decomposition}
This is a short explanation of some terms that frequently appear as well as some motivation for the later results. The results contained here-in are proved using the Eisenstein series as an essential component. 

\subsection{The Decomposition of the Spectrum In General}\label{direct_integral}
For this section let \(H\) be a locally compact topological group.
It is a classical theorem that for representations of finite groups over an algebraically closed field the regular representation decomposes into a direct sum, where ever irreducible representation appears \cite[Ch. 2.4 Cor. 2 ]{LinearRepresentationsFinite}. This still holds for compact topological groups, when one considers continuous unitary representations \cite[5.1]{follandCourseAbstractHarmonic2016}.
\begin{remark}
    This is a strict generalisation of the finite groups case, when we give the finite group the discrete topology then all its linear representations are continuous and unitary.
\end{remark}
There is one final more general incarnation of this line of investigation in the Plancherel theorem. A group is \textbf{type I} if for every (continuous unitary) representation \(\pi\) such that the center of \(\Hom_\mathrm{Rep}(\pi, \pi)\) is trivial we have a decomposition as a  direct sum of irreducible representations. 

\begin{example}
    Consider \(G(\A)\) the adelic points of a connected reductive LAG. This is a type one group. 
\end{example}

\begin{example}
    Consider \(G(\A)\) the adelic points of a connected reductive LAG. This is a seecond countable group. 
\end{example}

\begin{example}
    Consider \(G(\A)\) the adelic points of a connected reductive LAG. This is a unimodule group. 
\end{example}
\todo[inline]{fill}

The idea of a direct integral is review in \ref{app:direct_int} to get a quick idea consider the following example:
\begin{example}[Direct Sums]
    Let \(I\) be a countable set with the discrete sigma algebra and counting measure \(\mu\). Let \((\mathcal{H}_i)_{i\in I}\) be a collection of Hilbert spaces then
    \[\bigoplus_{i\in I} \mathcal{H}_i = \left\{ (h_i)_{i\in I}\in \prod_{i\in I} \mathcal{H}_i : \int_I \norm{h_i}_i^2 d\mu <\infty \right\}.\]
    I.e. the Hilbert space direct sum is by definition square summable sequences, but sums are just discrete integrals.
\end{example}

\begin{Theorem}[Plancherel, \cite{follandCourseAbstractHarmonic2016}, 7.44]
    The regular represntation of a type I, second countable and unimodular topological group is a direct integral of the irreducible unitary representations. 
\end{Theorem}
\begin{remark}
    The Plancherel theorem says much more in fact. Like the Peter-Weyl theorem for compact groups it doesnt just give you that some direct integral decomposition exists, it contains many more details about the topology and measure on the set of unitary irreducible representations, and which representations are associated to them in the direct integral. We are being breif as this is motivational.
\end{remark}

Thus what one wants to do is find a decomposition of the regular representation \(G(\A) \curvearrowright \mathrm{L}^2(G(\A))\).
We call such decompositions ``spectral'', alluding to the spectral theorem which provides such a decomposition in terms of the eigenvector of certain operators. Moreover these decompositions are largely proved in terms of the more general spectral theorems. So once accomplished this is another one of the tools that can be used to compartmentalise problems in automorphic forms, by dealing with representations that appear in different parts of the spectrum. 

\subsection{Langlands Decomposition of the Spectrum }
We have the Plancherel theorem but Langlands also provides a fine analysis of the spectrum using automorphic forms. The key result in this theory is the following decomposition,
\begin{Theorem}[\cite{arthurEisensteinSeriesTrace1979}, MAIN THEOREM (b)]
    There is an orthogonal decomposition of the representation of \(G(\A)\) on \(L^2(G(\Q) \backslash G(\A))\) into 
    \[L^2(G(\Q) \backslash G(\A)) = \bigoplus_{\mathscr{P}}L^2_\mathscr{P}(G(\Q) \backslash G(\A)),\]
    where \(\mathscr{P}\) runs over certain ``associate classes'' of parabolics of \(G\) and the summands are the direct integrals of spaces of \(L^2\) automorphic forms.
\end{Theorem}
These direct integrals are in fact constructed out of subspaces generated by Eisenstein series. 

The spectrum of \(L^2(G(\A))\) refers to such a decomposition. In particular we have some important ``pieces'' to such a decomposition. The piece that decomposes into a direct sum of irreducible is called the \textbf{discrete spectrum}. The compliment of the discrete spectrum is called the \textbf{continuous spectrum}. One can define cuspidal \(L^2\) functions in the exact same way as cuspidal automorphic forms \ref{cuspidal_form_definition} and then it has been shown that the \textbf{cuspidal spectrum}, the subspace of \(L^2\) consisting of cusp forms, decomposes as a direct sum \cite[9]{getzIntroductionAutomorphicRepresentations2024}. Thus the cuspidal spectrum is contained in the discrete spectrum in this case. The \textbf{residual spectrum} is defined to be the compliment of the cuspidal spectrum in the discrete spectrum.

\subsection{Residual Spectrum}\label{residual_spec}
Moeglin and Waldspurger also acheived a more fine analysis of the spectrum of \(\GL_n\) in terms of residues of Eisenstein series. 
First consider the group \(\GL_n\). We then let \(n = ab\) for positive integers \(a,b\). If \(\tau\) is an irreducible, cuspidal automorphic rep of \(\GL_a\) then Moeglin and Waldspurger construct a representation of \(\GL_{ab} = \GL_n\) called the ``Speh representation'' and denote it 
\[\Delta(\tau, b).\]
They go on to prove that as \(\tau\) and \(b\) vary these representations span the residual spectrum of \(L^2(\GL_n(F) \backslash \GL_n(\A))\) \cite[Thm. 1.1]{jiangPolesCertainResidual2013}.

This representation is formed by taking iterated residues of Eisenstein series in the sense of \cite[V]{moeglinSpectralDecompositionEisenstein1995}. For a nice survey of problems in this area, of residues of Eisenstein series, there is \cite{jiangResiduesEisensteinSeries2008a}.



\section{Automorphic L-Functions}
We don't intent to define in great detail automorphic L-functions, as there are many other better sources to learn from \cite[Part 2.III.2]{borelAutomorphicFormsRepresentations1979}\cite{shahidiEisensteinSeriesAutomorphic2010}\cite{cogdellLFUNCTIONSFUNCTORIALITY}\cite[9, 10, 11]{bumpIntroductionLanglandsProgram2004}\cite{arthurLfunctionsAutomorphicRepresenta}, we will recall the idea and then discuss some of the properties and relations with Eisenstein series and interwining operators that we will need later.

The first thing is to recall the classification of connected reductive groups defined over an algebraically closed field via root datum. A root datum is a tuple \((X, \Phi, \check{X} , \check{\Phi})\) where \(X\) and \(\check{X}\) are two free abelian groups of finite type, \(\Phi, \check{\Phi}\) are subgroups that are in duality via a perfect pairing on \(X, \check{X}\). Then each reductive group \(G\) over a number field \(F\) has associated the root datum that is associated to its base change to \(\C\). Thus to a connected reductive group over a number field we associate a connected reductive group over \C, given by the dual root datum. We call this the \textbf{dual group} of \(G\) and denote it \(\hat{G}\). The \textbf{Langlands dual group} is then the dual group producted with the \(\mathrm{Gal}(\bar{k}/k)\)
\[^L G \defeq \hat{G} \rtimes \mathrm{Gal}(\bar{k}/k).\]

\begin{example}[Classical Groups, \cite{bumpIntroductionLanglandsProgram2004}, 11.1]
    We have the following table
    \begin{table}[h]
        \centering
        \begin{tabular}{ll}
        \(G\)         & \(\hat{G}\)   \\ \hline
        \(\GL_n\)     & \(\GL_n\)     \\
        \(SO_{2n+1}\) & \(\Sp_{2n}\)  \\
        \(SO_{2n}\)   & \(SO_{2n}\)   \\
        \(\Sp_{2n}\)  & \(SO_{2n+1}\)
        \end{tabular}
        \end{table}
\end{example}

Then, using the Satake isomorphism \cite[2.2]{shahidiEisensteinSeriesAutomorphic2010}, to each unramified representation of \(G(F_\nu)\) we can associate a conjugacy class of \(^LG\), via some map call it \(c\), and hence there is a way to apply a complex representation \(r: ^LG \to \GL_n(\C)\) to representations of \(G(F_\nu)\). Thus the automorphic L-functions are defined as follows: Let \(\rho\) be a representation of \(G(\A)\), let \(r\) be a complex representation of \(^LG\) and \(s\in \C\) then 
\[L(s, \rho, r) \defeq \prod_\nu L_\nu(s, \rho_\nu, r ) = \prod_\nu \frac{1}{\det\bigl( I - r(c(\rho_\nu))q^{-s} \bigr)}  ,\]
where \(\nu\) runs over the unramified places. It is a part of the grand Langlands philosophy that there should be suitable L-functions for the ramified places satisfying very nice properties.

\begin{remark}
    The global L-functions have been defined for many groups at this point and indeed \cite{jiangPolesCertainResidual2013} uses known properties to prove their results. One should note that the questions that we are interested in are still tractable even though the L-functions might not be defined (for instance for the metaplectic group). This is because only finitely many places will ramify, and so as long as those places are neither zero or poles we can transfer questions about zeros and poles from the full global L-functions to L-functions at almost all places. 
\end{remark}

\begin{example}[Standard Representations / Classical Groups]
    In the case of classical groups it is common to see L-functions with only two entries e.g. if \(\rho\) is a representation of \(G = \Sp{2n}\) then you may see 
    \(L(s, \rho).\)
    The reason is that there is a standard representation of the dual groups of classical groups. Namely the standard representation of a matrix group inside \(\GL_n\) is the one that sends \(g\mapsto g\). It is this representation that is to be taken for the dual group in this setting.
\end{example}

\begin{example}[Rankin-Selberg]
    \todo[inline]{fill}
\end{example}


\begin{example}[Dirichlet L-functions]
    Recall that a Dirichlet character \(\chi\) is a character of the group \((\Z/N\Z)^*\). Through the series of maps 
    \[A^\times \cong \Q^\times \times \R_{>0}^\times \times \hat{\Z}^\times \to (\lim \Z/N\Z)^\times \to (\Z/N\Z)^\times \to \C,\]
    one get a bijection between Dirichlet characters and finite-order Grossencharacters, i.e. characters of \(\A_F^\times/F^\times\).
    Grossencharacters have the associated L-function as they are just automorphic forms of \(\GL_1\), which generate automorphic representations. These give us the classical Dirichlet L-functions.\todo{reference? More details?}
\end{example}
Although this might seem unrelated to the current section on spectral decomposition and Eisenstein series we will see later that the two are inextricably linked
\todo[inline]{ref}


\chapter{Constant Terms}
Here we will explain the role of the constant term in our calculation of poles.\label{constant_terms}

\section{Definition and Role} \label{cuspidal_form_definition}\label{sec:L-functions}
The constant term is an operation defined on a large class of functions and is supposed to generalise the constant term of a Fourier expansion, we will see this later, although one may consult \cite[1.6]{bumpAutomorphicFormsRepresentations1997} for some examples as well. In particular \cite[I.2.6]{moeglinSpectralDecompositionEisenstein1995} give the definition as follows: We consider \(P=MU\) a standard parabolic of \(G\) and \(\phi: U(k)\setminus \mathbf{G} \to \C\) a measurable and locally \(L^1\) function then its \textbf{constant term} along \(P\) is 
\[\phi_P :  U(\A)\setminus \mathbf{G} \to \C\]
\[\phi_P(g) \defeq \int_{U(k)\setminus U(\A)}\phi(ug) \mathrm{d}u\]
which inherits many of the properties of \(\phi\) such as smoothness and moderate growth. Moreover if \(\phi\) is an automorphic form on \(\mathbf{G}\) then its constant term is an automorphic form on \(M\) \cite[6.5]{getzIntroductionAutomorphicRepresentations2024}.

\todo{change this from }
Recall the definition of \textbf{cuspidal automorphic forms} or ``cusp forms''. Let \(\phi\) be an automorphic form on \(U(\A)M(k)\setminus \mathbf{G}\) for \(P = MU\) a standard parabolic. Then \(\phi\) is cuspidal if for all standard parabolics \(P'\subset P\) we have that \(\phi_{P'}\) is identically zero. 

\begin{Theorem}[\cite{moeglinSpectralDecompositionEisenstein1995}, I.4.10]\todo{fix}
		Let \(P = MU\) be a standard parabolic of \(G\). If \(\pi\) is a cuspidal automorphic representation induced from \(P\), then for a fixed \(\phi \in\mathcal{A}_0(U(\A)M(k)\setminus G(\A))_\pi \) the Eisenstein series \(E\) can be thought of as a function from some open subset of the cuspidal datum \(\mathfrak{P}\) into \(L^2_{\mathrm{loc}}(G)\) given by 
		\[E(p)(g) = \sum_{\gamma \in P(k)\backslash G(k)} \lambda.\phi(\gamma g), \;\;\; p\in \mathfrak{P},\; g\in G(\A),\]
		where it converges. 
		If \(D\subseteq \mathfrak{P}\), is an open subset minus a finite number of points on which \(E\) is holomorphic then E has a holomorphic continuation to the finite number of points if and only if the constant term of \(E_Q\) has a holomorphic continuation to these finite number of points for all standard parabolics \(Q\).
    \end{Theorem}
    
    \begin{remark}
    	The theorem in Moeglin and Waldspurger is proved in much more generality, however after sufficient symbol pushing this is the essence. 
    \end{remark}
    So one can say that the poles of an Eisenstein series are controlled by its constant terms. We can say more:
    
        
        \begin{theorem}[\cite{moeglinSpectralDecompositionEisenstein1995}, II.1.7]
        	The constant term of an Eisenstein series induced from a standard maximal parabolic \(P\) is zero along any other standard parabolic \(P'\) unless \(P = P'\).
        \end{theorem}
      
      Finally we remark on the central importance of cusp forms in the theory of automorphic forms, this is for several reasons. They appear historically as interesting examples such as the Ramanujan tau function, by a theorem of Ribet \cite[T2.3]{serreProceedingsInternationalConference1977} the Galois representation associated to a cusp form is irreducible and they form the ``base case'' for the proof of the spectral decomposition in \cite{moeglinSpectralDecompositionEisenstein1995}.

\section{Constant Terms of Eisenstein Series}\label{const_eisenstein}
This computation forms the heart of a well known theorem, \cite[Prop 10.4.2]{getzIntroductionAutomorphicRepresentations2024}\cite[II.1.7]{moeglinSpectralDecompositionEisenstein1995}\cite[6.2]{shahidiEisensteinSeriesAutomorphic2010}, although for an amateur the detail is lacking in other presentations.

Notice that the Eisenstein series has a full \(G(k)\) invariance and so we can take its constant terms along \textit{any} standard parabolic.

\subsection{In General}
We will use the following Lemmas to give a simplified expression of the constant term of an Eisenstein series. First fix \(P = MN\) and \(P' = M'N'\) two standard parabolics of a suitable group G over F, with \(E(x, \phi, \lambda)\) defined via  parabolic induction from P.

    \begin{Lemma}\label{lem:1}
        \[P(F)\setminus G(F) \cong \coprod_{w\in W_{M'}\setminus W_G / W_{M}} P'(F)\cap wP(F)w\inv \setminus P'(F)\]
    \end{Lemma}
    \proofbar{
        Consider the Bruhat decomposition:
        \[G(F) =\coprod_{w\in W_{M'}\setminus W_G / W_{M}} P(F)w\inv P'(F) \]
        then because the action of \(P(F)\) keeps the disjoint sets disjoint we can move the quotient through and get
        \[P(F)\setminus G(F) = \coprod_w  P(F) \setminus P(F)w\inv P'(F)\]
        so we analyse the summands, by the second isomorphism theorem we have a bijection
        \[P(F)\setminus P(F)w\inv P'(F) \cong P(F)\cap P'(F) \setminus w\inv P'(F) \]
        now if \([w\inv p] \in P(F)\cap P'(F) \setminus w\inv P'(F) \) then its represented by some \(pw\inv p'\) where \(p\in P(F)\cap P'(F)\) and hence multiplying by \(w\), in particular an isomorphism, gives \(wpw\inv p'\in wP(F)w\inv \times P'(F)\) and so 
        \[w(P(F)\cap P'(F) \setminus w\inv P'(F)) \cong wP(F)w\inv \cap P'(F) \setminus P'(F)\]
    }

    \begin{Lemma}\label{lem:2}
        Let \(m', n'\in M'(F)\times N'(F)\) then 
        \[m'n' \in wP(F)w\inv \iff m'\in wP(F)w\inv \text{ and  }\;\; n'\in (m')\inv wP(F)w\inv m'\]
    \end{Lemma}
    \proofbar{
        The forward implication is stated in \cite{getzIntroductionAutomorphicRepresentations2024}, the converse follows from some algebra:
        First let \(m' = wp_1w\inv\) and \(n' = (m')\inv wp_2w\inv m'\) then 
        \begin{equation*}
            \begin{aligned}
                m'n' &= (wp_1w\inv)\inv wp_2w\inv wp_1w\inv\\
                     &= wp_1\inv w\inv wp_2w\inv wp_1w\inv\\
                     &= wp_1\inv p_2p_1w\inv \in wP(F)w\inv\\
            \end{aligned}
        \end{equation*}
    }
    Taking the contrapositive of this lemma will be used below. This is because our sums will be over quotients like \(A\setminus B\) and therefore summing over the ``elements'' in B that are not in A; by our lemma would be the same as summing over two different such quotients.
Now we will apply our lemmas to simplify and make more explicit the constant term of an Eisenstein series. Denote \(([N']\defeq N'(F)\setminus N'(\A))\)
    \begin{equation*}
        \begin{aligned}
            E_{P'}( \phi, \lambda, x) &= \int_{N'(F)\setminus N'(\A)} E( \phi, \lambda, nx) dn\\
                                     &= \int_{[N']} \sum_{\delta\in P(F)\setminus G(F)} \lambda.\phi(\delta nx)  dn\\
                                    (\text{Lemma 1}) \;\;\; &= \int_{[N']} \sum_{\delta\in \coprod_{w\in W_{M'}\setminus W_G / W_{M}} P'(F)\cap wP(F)w\inv \setminus P'(F)} \lambda.\phi(\delta nx)  dn\\
                                     &= \sum_{ w\in W_{M'}\setminus W_G / W_{M}}\int_{[N']} \sum_{p'\in P'(F)\cap wP(F)w\inv \setminus P'(F)} \lambda.\phi( w\inv p'nx)  dn\\
        \end{aligned}
    \end{equation*}
    Now apply lemma 2 to the above sum and we get the equality 
    \begin{align*}
    	&= \sum_{ w} \sum_{m'\in M'(F)\cap wP(F)w\inv\setminus M'(F)} \int_{[N']} \sum_{n'\in N'(F)\cap (m')\inv wP(F)w\inv m' \setminus N'(F)} \lambda.\phi( w\inv m'n'nx)  dn\\
    	&(\text{Change Var}) \;\;\; = \sum_{ w} \sum_{m'} \int_{[N']} \sum_{n'\in N'(F)\cap wP(F)w\inv\setminus N'(F) } \lambda.\phi( w\inv n'nm'x)  dn\\
    	&(\text{Unfold}) \;\;\; = \sum_{ w} \sum_{m'} \int_{N'(F)\cap wP(F)w\inv \setminus N'(\A)} \lambda.\phi( w\inv nm' x)  dn.\\
    \end{align*}
    \todo[inline]{I need to fix all the lemma labeling }
    The change of variables is \((m', n') \mapsto ((m')\inv n' m', (m')\inv n' m')\).
    Again we assume that our $x$ is sufficiently large so all the integrals converge.\todo[inline]{Maybe appologise for doing integrals naivelly lol.. clarify this.... is it even the x that I need to worry about here?}

\subsection{Constant Terms of Cuspidal Eisenstein Series}
\begin{Lemma}[4]
        For \(w\in W_{M'}\setminus W_G / W_{M} \) we have that \(w\inv P'w\cap M\) is a standard parabolic of \(M\) with Levi \(w\inv M'w\cap M\) and unipotent \(w\inv N'w\cap M\).
    \end{Lemma}
    \proofbar{
        This is \cite[10.4.1]{getzIntroductionAutomorphicRepresentations2024} stated without proof. They give the reference \cite[V.4.6]{renardREPRESENTATIONSGROUPESREDUCTIFS} which is in French..
    }
    \begin{Lemma}[5]
        \[w\inv U' w \cap P = (w\inv U' w \cap M)(w\inv U' w \cap U).\]
    \end{Lemma}
    \proofbar{
        \cite[10.4.1]{getzIntroductionAutomorphicRepresentations2024} has some decompositions, as well as the standard decomposition of \(P=MU\) I think I could prove this...
    }
    \begin{Lemma}[6]
        \[c\setminus (b\setminus a )= (bc)\setminus a\]
    \end{Lemma}

    Continuing the computation of the constant term above, we will focus purely on the inner integral now
    \begin{equation*}
        \begin{aligned}
            \int_{N'(F)\cap wP(F)w\inv \setminus N'(\A)}& \lambda.\phi( w\inv nm' x)  dn \\&= \int_{w\inv N'(F)w \cap P(F) \setminus w\inv N'(\A)w} \lambda.\phi( nw\inv m' x)  dn \\
            (\text{Lemma 5})&= \int_{(w\inv U' w \cap M)(w\inv U' w \cap U)(F) \setminus w\inv N'(\A)w} \lambda.\phi( nw\inv m' x)  dn . \\
        \end{aligned}
    \end{equation*}
    where the first equality is the change of variables \(w\inv n w\mapsto n \). Denote \(A = (w\inv U'(F) w \cap U(F) ) \setminus w\inv N'(\A)w \). If we apply lemma 6 and unfold we get the equality
    \[= \int_{(w\inv U'(\A)w \cap M(\A)) \setminus A} \int_{w\inv U'(F) w \cap M(F) \setminus w\inv U'(\A)w \cap M(\A)} \lambda.\phi( n_1 n_2 w\inv m' x)  dn_1 dn_2.\]
     Now look at the inner integral here more closely 
    \[ \int_{w\inv U'(F) w \cap M(F) \setminus w\inv U'(\A)w \cap M(\A)}\lambda. \phi( n_1 n_2 w\inv m' x)  dn_1 dn_2,\]
    applying Lemma (6) we see that this is a constant term for a parabolic of \(M\), of the function \(m\mapsto \phi(m n_2 w\inv m' x)\). 
    \begin{Lemma}
        \(n_2 w\inv m' x \in K\) with variables as above.
    \end{Lemma}
    This was all in complete generality as well. If we now assume further that the Eisenstein series was induced from a \textit{cuspidal} automorphic representation, then \(m\mapsto \phi(mk)\) is a cusp form and therefore this last integral will vanish whenever \(w\inv U'w \cap M \neq \{1\}\), because in that case the inner integral doesn't exist (its over a point).

    \subsection{Constant Term Of Eisenstein Series for Conjugate Levis}\label{constant_conjugate_levi}
    If we now assume that \(M' = wMw\inv\) and recall the definition of our intertwining operator \ref{intertwining_operator} we can use the following 
    \begin{Lemma}[\cite{moeglinSpectralDecompositionEisenstein1995} II.1.7 (6)]
        \[U'(k) \cap wP(k) w\inv = U'(k) \cap wU(k)w\inv,\]
    \end{Lemma}
    to see that 
    \begin{equation*}
        \begin{aligned}
             E_{P'}( \phi, \lambda, x) &= \sum_{ w} \sum_{m'} \int_{N'(F)\cap wP(F)w\inv \setminus N'(\A)} \lambda.\phi( w\inv nm' x)  dn \\
             &=  \sum_{ w} \sum_{m'} \int_{N'(k) \cap wN(k)w\inv \setminus N'(\A)} \lambda.\phi( w\inv nm' x)  dn \\
             &= \sum_{ w} \sum_{m'} M(w, \pi)(\lambda.\phi)(x)
        \end{aligned}
    \end{equation*}\todo[inline]{I have mixed up my N's and U's too much...}
    In particular we can combine the conjugate and cuspidal cases to get a much simpler expression for some constant terms of some Eisenstein series, we will go through a detailed example in the final section \todo{}.
    


    
\chapter{Poles of Residual Eisenstein Series}
Our goal here is to exposit and survey the work in papers such as \cite{brennerNotesAnalyticProperties2009}\cite{jiangPolesCertainResidual2013}\cite{ginzburgTopFourierCoefficients2021} and perhaps give a trivial extension of them.

\todo[inline]{rearange: Have the setup and exposition. Then my stuff at the very end.}

\section{Their Results}
\cite{brennerNotesAnalyticProperties2009} gave an analysis of the residual poles of Eisenstein series attached to \(\Sp_{2n}\), there were some minor errors that were corrected in \cite{jiangPolesCertainResidual2013} where they give esssentially the same proof; theirs however works for the other classical groups. For our purposes, the case of \(\Sp_{2n}\), as a group defined over \(F\) a number field, is most relevant, and we shall therefore focus exclusively on this case, however it should be noted that this limitation in the non-covering case is artificial, although it does simplify things a little by removing some casework, and we hope also in the covering case to be able to remove it in future work. 

We fix an \(n\in \N\) and the Borel of upper triangular matricies in \(\Sp_{2n} \), then we look at partitions of \(n = r + m\), where \todo[inline]{what are the ranges of these}. Then as we saw in \todo[inline]{write this up in the classical group section} ---- there corresponds a maximal standard parabolic of \(\Sp_{2n}\), which we denote \(P_r = M_rN_r\), such that the Levi component is 
\[\GL_r\times \Sp_{2m} \]
As we saw in \todo[inline]{explain this in the Eisenstein series section}---- the space of characters \(X^{\Sp_{2n}}_{M_r}\) is one dimensional by the maximality of \(P_r\). If we look at the divisors of \(r = ab\) \todo[inline]{check the exact ranges, will depend on the range of r} and fix a \(\tau\), an irreducible unitary cuspidal automorphic representation of \(\GL_a\), then from \todo[inline]{link to the residual representation section} ---- we know that \(\Delta(\tau, b)\) is a residual representation of \(\GL_{ab} = \GL_r\). Now we take an irreducible generic cuspidal automorphic representation \(\sigma\) of \(\Sp_{2m}\), and so their tensor product \(\Delta(\tau, b) \tensor \sigma\) gives a representation of \(\GL_{r}\times \Sp_{2m}\) and hence of the Levi \(M_r\). We now consider the Eisenstein series attached to this representation, namely if 
\[\phi\in \mathcal{A}(N_r(\A)M_r(F)\setminus \Sp_{2n}(\A) )_{\Delta(\tau, b)\tensor \sigma}\] 
then we have the Eisenstein series
\[E(\phi, s)(g) = \sum_{\gamma\in P_r(F) \setminus \Sp_{2n}(F)} s.\phi(\gamma g)\]
\todo[inline]{explain the s action in Eisenstein series section.}
for \(g\in \Sp_{2n}(F) \setminus \Sp_{2n}(\A)\). Becuase it is induced from the residual representation \(\Delta(\tau, b)\) we call these residual Eisenstein series. The main theorem is then a statement that after some normalisation the poles of this Eisenstein series are limited to a particular set and that they are simple. Details will be given later.


\section{Our Results}
We consider an almost identical setup but we deal with the metaplectic cover of \(\Sp_{2n}\), again over a number field \(F\), \(\Mp_{2n}\)\todo[inline]{reference the section I discuss this in.}. We again fix the Borel of upper triangular matricies, consider partitions of \(n = r+m\) and look at maximal standard parabolics of \(\Sp_{2n}\), \(P_r = M_rN_r\) such that 
\[M_r = \GL_r \times \Sp_{2m}\]
then if \(r = ab\) we still have that \(\Delta(\tau, b)\tensor \sigma\), for \(\tau\) irreducible unitary cuspidal automorphic representation of \(\GL_a\) and \(\sigma\) irreducible generic cuspidal automorphic representation of \(\Sp_{2m}\), is a representation of \(M_r\). The difference is in the parabolic induction as we now consider 
\[\phi\in \mathcal{A}(N_r(\A)M_r(F)\setminus \Mp_{2n}(\A) )_{\Delta(\tau, b)\tensor \sigma}\] 
and then the Eisenstein series is defined in the same way
\[E(\phi, s)(g) = \sum_{\gamma\in P_r(F) \setminus \Sp_{2n}(F)} s.\phi(\gamma g)\]
for \(g\in \Sp_{2n}(F) \setminus \Mp_{2n}(\A)\) and \(s\in \C \cong X^{\Mp_{2n}}_{M_r}\).

\begin{Lemma}
When \(b = 1\) we have the constant term
    \[E(\phi,s)(g)_{P_a} = \phi(g)_{P_a} + M(\omega, \tau\tensor\sigma)(\phi)(g)\]
\end{Lemma}

\todo[inline]{fill in here as theorems or whatever anything that I end up actually checking....}

\section{Setup}
In the setup we used that \(s\in \C \cong X^{\Mp_{2n}}_{M_r}\) the first step is to make sure that this is actually true
\begin{Lemma}
        \( X_{M_r}^{\Mp_{2n}(\A)}\) is at most a one dimensional \C vector space. 
    \end{Lemma}
    \proofbar{
         First of all we have that \cite[I.1.4]{moeglinSpectralDecompositionEisenstein1995}
         \[ X_{M_r}^{\Mp_{2n}(\A)} \subseteq X_{M_r} \cong \mathfrak{a}_{M_r}^*\defeq Rat(M_r) \tensor_\Z \C\]
        thus it is clearly sufficent to bound the dimension of \(\mathfrak{a}_{M_r}^*\) as a \C vector space, moreover this dimension agrees with the dimension of \(Rat(M_r)\) as a free \Z module. 

        Thus we compute \(\dim_\Z(Rat(M_r))\):
        \begin{equation*}
            \begin{aligned}
                Rat(M_r) &= Rat(\GL_r \times \Sp_{2m}) \\
                         &= \Hom(\GL_r \times \Sp_{2m}, \mathbb{G}_m) \\
                         (2)&\cong \Hom(\mathrm{Ab}(\GL_r \times \Sp_{2m}), \mathbb{G}_m) \\
                         (1)&\cong \Hom(\mathrm{Ab}(\GL_r) \times \mathrm{Ab}(\Sp_{2m}), \mathbb{G}_m) \\
                         (3)&\cong \Hom(\mathbb{G}_m \times 1, \mathbb{G}_m) \\
                         &\cong \Z
            \end{aligned}
        \end{equation*}
        in (2) we have used the universal property of the abelianization \(\mathrm{Ab}(G) = \mathcal{D}(G) \setminus G = [G, G] \setminus G \) because \(\mathbb{G}_m\) is abelian. (1) is that the abelianization commutes with direct products (citation as comment in Tex). (3) is because \(\Sp\) is a perfect group.
        %https://groupprops.subwiki.org/wiki/Symplectic_group_is_perfect
        %https://mathoverflow.net/questions/35713/abelianization-of-a-semidirect-product
}\todo[inline]{I havent shown that it is not trivial...}


\section{Lemma 1}
\todo[inline]{The representation is supposed to be of the covering of the Levi///? need to fix this}
We here consider the case that \(b=1\), hence \(n = a + m\). Then fixing a standard parabolic of \(\Sp_{2n}\) we have the maximal standard parabolic \(P_a = M_aN_a\) where \(M_a = \GL_a \times \Sp_{2m}\).  Now if \(\tau\) is irreducible unitary cuspidal automorphic representation of \(\GL_a\) then by definition \todo[inline]{Brenner..}
\[\Delta(\tau, 1)(\phi)(g) = E(\phi,s)(g) = s.\phi(g)\]
where the Eisenstein series is defined via the parabolic induction from the Levi \((\GL_a)^{\times b} \) to \(\GL_{ab}\). Thus we have \(\Delta(\tau, 1) = \tau\). So for the appropriate \(\sigma \) a rep of \(\Sp_{2m}\) we get a rep of the Levi of \(\Sp_{2n}\), \(M_r = M_a = \GL_a\times \Sp_{2m}\) given by \(\tau\tensor \sigma\). To this we associate the Eisenstein series for \(\phi\in \mathcal{A}(N_r(\A)M_r(F)\setminus \Mp_{2n}(\A))_{\tau\tensor \sigma}\) \(E(\phi,s)\) as usual. Now we proceed to calculate the constant term of this Eisenstein series along the parabolic \(P_a = MN\). \todo[inline]{M+W II.1.7, all others are zero..?}

By our calculations\todo[inline]{red}, the cuspidality of the tensor\todo[inline]{ref} and \cite{jiangPolesCertainResidual2013} we know that 
     \[E(\phi, s)_{P} = \sum_{ w} \sum_{m'} \int_{(w\inv N(\A)w \cap M(\A)) \setminus A} \int_{w\inv N(F) w \cap M(F) \setminus w\inv N(\A)w \cap M(\A)} \phi( n_1 n_2 w\inv m' x)  dn_1 dn_2\] 
     and the inner integral vanishes for all \(w\neq id, \omega\) (\(\omega\) as in \cite{jiangPolesCertainResidual2013}). Hence the first sum becomes over two elements and we have 

     \[E(\phi, s)_{P} = E(\phi, s)_{P, id} + E(\phi, s)_{P, \omega}\]
     where 
     \[E(\phi, s)_{P, w} =  \sum_{m'\in M(F)\cap wP(F)w\inv\setminus M(F)} \int_{N(F)\cap wP(F)w\inv \setminus N(\A)} \phi( w\inv nm' x)  dn\]

First the identity term simplifies

     \begin{equation*}
        \begin{aligned}
            E(\phi, s)(x)_{P', id} &=  \sum_{m'\in M(F)\cap P(F)\setminus M(F)} \int_{N(F)\cap P(F) \setminus N(\A)} \phi( nm' x)  dn\\
            &= \sum_{m'\in M(F)\setminus M(F)} \int_{N(F)\setminus N(\A)} \phi( nm' x)  dn \\
            &=\int_{N(F)\setminus N(\A)} \phi( n x)  dn \\
            &= \phi(x)_P
        \end{aligned}
     \end{equation*}
     \todo[inline]{I really need to fix this s thing that I dropped in the constant term computations.}

     Considering now the \(\omega\) term 
     \[E(\phi, s)_{P, \omega} =  \sum_{m'\in M(F)\cap \omega P(F)\omega\inv\setminus M(F)} \int_{N(F)\cap \omega P(F)\omega\inv \setminus N(\A)} \phi( \omega\inv nm' x)  dn\]

     by \cite[2C]{jiangPolesCertainResidual2013} \(M(F)\cap \omega P(F)\omega\inv \setminus M(F)\) is isomorphic to\todo[inline]{this is not clear in their paper im just guessing this is what they mean} \(P_0 \setminus \Sp_{2(n-a)}\) where \(P_0 \setminus \Sp_{2(n-a)}\) by defintion, but \(P_0\) has Levi \(M_0 = \Sp_{2(n-a)}\) by definition and hence is itself \(\Sp_{2(n-a)}\). Thus the sum is over \(\Sp_{2(n-a)}(F) \setminus \Sp_{2(n-a)}(F)\) and hence is over a point. Therefore we get by definintion of the intertwining operator
     \[E(\phi, s)_{P, \omega} = \int_{N(F)\cap \omega P(F)\omega\inv \setminus N(\A)} \phi( \omega\inv n x)  dn = M(\omega, -)(\phi)(x)\]
     becuase we took the constant term along the same parabolic as the definition of the Eisenstein series we know that the Levis are (the same) conjugate.
    Thus we have shown that 
    \[E(\phi, s)_P(x) = \phi(x)_P + M(\omega, - )(\phi)(x)\]

    Notice that the computation takes place completely at the level of the terminals which are indupendent of the fact that we have taken a covering group, hence we have really only reused work from \cite{jiangPolesCertainResidual2013}.


\appendix
\chapter{Direct Integrals}\label{app:direct_int}
\subsection{Of Spaces}
    Consider a countable collection of Hilbert spaces \((\mathcal{H}_\alpha)_{\alpha\in A}\) then their direct sum is defined to be 
    \[\bigoplus_{\alpha} \mathcal{H}_\alpha \defeq \left\{(h_\alpha)\in \prod_\alpha \mathcal{H}_\alpha : \sum_\alpha \norm{h_\alpha}_\alpha^2 < \infty\right\}\]
    i.e. square summable sequences from the product. This is to ensure that the resulting space is still complete. If we recall that summing over a countable set is the same as \textit{integrating} over that countable set when we equip it with the counting measure and discrete sigma algebra then this can be re-written as 
    \[\bigoplus_{\alpha} \mathcal{H}_\alpha = \left\{(h_\alpha)\in \prod_\alpha \mathcal{H}_\alpha : \int_A \norm{h_\alpha}^2_\alpha d\alpha < \infty\right\}\]

    This definition can be obviously generalised to an indexing set that is now an arbitrary measure space, \((A, \mathcal{M}, \mu)\). We need to make some technical arrangment to accompany this change, namely ensureing everything agrees with the measure structure, if we're to integrate we better only integrate \textit{measurable} things. So now a collection \((\mathcal{H}_\alpha)_{\alpha\in A}\) along with a countable set of elements \(e_j \in \prod_\alpha \mathcal{H}_\alpha, j\geq 1\) is called a measurable field over A if 
    \[\forall j,k\geq 1 \;\; \alpha \mapsto \inner{e_j(\alpha), e_k(\alpha)}\]
    is measurable and for each \(\alpha \in A\) 
    \[span\{e_j(\alpha)\}_{j=1}^\infty \subseteq \mathcal{H}_\alpha\]
    is dense; fixing an \(\alpha\) and varying the j form a basis of each of the hilbert spaces, fixing the indecies and variying the \(\alpha\) is measurable. An element \(f \in \prod_\alpha \mathcal{H}_\alpha\) is called a measurable vector field if 
    \[\forall j \;\; \alpha \mapsto \inner{f(\alpha), e_j(\alpha)}_\alpha\]
    is a measurable function. Note that we consider elements of the potentially uncountable product as functions from the indexing set into the relevant space (functions into the union of the hilbert spaces satisfying the property that \(f(\alpha)\in \mathcal{H}_\alpha\)). Now we define 
    \[\int^\oplus \mathcal{H}_\alpha d\mu(\alpha) \defeq \left\{f\in \prod_\alpha \mathcal{H}_\alpha : f \text{ is measurable and } \int_A \norm{f(\alpha)}_\alpha^2 < \infty \right\}\] 
    Indeed this forms a Hilbert space. Note that a priori this construction depended on the basis \((e_j)\) that we picked but up to isomorphism the basis doesnt matter. 

    \subsection{Of Operators}
    We want to decompose representations and so we should look at how operators fit into this picture. We call an element 
    \[T\in \prod_\alpha \mathcal{L(H_\alpha)}\]
    a field of operators on A. It defines a linear map from \(\prod_\alpha \mathcal{H}_\alpha\) to itself via 
    \[ \left(\int^\oplus T\right)(f)(\alpha) \defeq T(\alpha)(f(\alpha)) \]
    We say that it is measurable if for all measurable vector fields \(f\) the function 
    \[\alpha \mapsto \left(\int^\oplus T\right)(f)(\alpha)\]
    is measurable. If moreover \(\mathrm{ess}\sup_\alpha \norm{T(\alpha)} <\infty\) then \(\int^\oplus T\) defines a bounded operator on \(\int^\oplus \mathcal{H}_\alpha\).

    \subsection{Of Representations}
    Now we consider a group G and a collection of unitary represntations \(\pi_\alpha\) on \(\mathcal{H}_\alpha\) such that for every \(\alpha\)  and every \(x\in G\)
    \[\alpha \mapsto \pi_\alpha(x)\]
    is a measurable field of operators. We call such a collection a measurable field of representations; a G indexed collection of measurable fields of operators. From a measurable field of representations we get a unitary representation 
    \[\pi(x) \defeq \int^\oplus \pi_\alpha(x)\]
    of \(G\) on \(\int^\oplus \mathcal{H}_\alpha\) which we call the direct integral of representations. 




\listoftodos
\todo[inline]{Disambiguate all teh L2 automorphic form stuff. What isi teh action of the Lie algebra then....?}


\newpage
\bibliographystyle{alpha}
\bibliography{./NumberTheory}
\end{document}