		\author{Riley Moriss}
	
		%images
		\usepackage{graphicx}
		\usepackage{quiver} %Tikz diagram generator
		\usetikzlibrary{fit}
		
		
		%Bibliography
		\usepackage{hyperref} %Content is clickable
		
		%Math
		\usepackage{amsmath}
		
		%Formatting
		\usepackage{comment} %comment out code blocks
		\usepackage{csquotes} %Quote environment
		\usepackage{amsfonts}
		\usepackage{mathrsfs} %Script fonts

		%Todos	
		\usepackage[colorinlistoftodos,prependcaption,textsize=scriptsize, color=red!20, shadow]{todonotes}
			
		%Math shortcuts
		%Sets
		\newcommand{\R}{\ensuremath{\mathbb{R}}\phantom{ }}
		\newcommand{\C}{\ensuremath{\mathbb{C}}\phantom{ }}
		\newcommand{\Z}{\ensuremath{\mathbb{Z}}\phantom{ }}
		\newcommand{\N}{\ensuremath{\mathbb{N}}\phantom{ }}
		\newcommand{\Q}{\ensuremath{\mathbb{Q}}\phantom{ }}
		\newcommand{\F}{\ensuremath{\mathbb{F}}\phantom{ }}
		\newcommand{\G}{\ensuremath{\mathbb{G}}\phantom{ }}

		%Symbols
		\newcommand{\hilb}{\ensuremath{\mathfrak{h}}\phantom{ }}
		\newcommand{\tensor}{\ensuremath{\otimes}}
		\newcommand{\inner}[1]{\ensuremath{\langle #1\rangle}}
		\newcommand{\norm}[1]{\lVert #1 \rVert}
		\newcommand{\inv}{^{-1}}
		\newcommand{\model}[1]{\llbracket #1\rrbracket}
		\newcommand{\Par}{\rotatebox[origin=c]{180}{\&}}
		\newcommand{\Dif}[2]{\ensuremath{\frac{\partial #1}{\partial #2}}}
		\newcommand{\comp}{\circ}
		\newcommand*{\defeq}{\hspace{2mm}\mathrel{\vcenter{\baselineskip0.5ex \lineskiplimit0pt
					\hbox{\scriptsize.}\hbox{\scriptsize.}}}%
			=\hspace{2mm}}

		%Operators
		\DeclareMathOperator{\Ind}{Ind}
		\DeclareMathOperator{\Spec}{Spec}
		\DeclareMathOperator{\Res}{Res}
		\DeclareMathOperator{\Hom}{Hom}
		\DeclareMathOperator{\Maps}{Maps}
		\DeclareMathOperator{\Aut}{Aut}
		\usepackage{mathtools}
\DeclarePairedDelimiter{\ceil}{\lceil}{\rceil}
\DeclarePairedDelimiter{\floor}{\lfloor}{\rfloor}

		\DeclareMathOperator{\dif}{d}
		\newcommand{\A}{\ensuremath{\mathbb{A}}\phantom{ }}
		\renewcommand{\P}{\ensuremath{\mathbb{P}}\phantom{ }}
		\newcommand{\Struc}{\ensuremath{\mathcal{O}}\phantom{ }}
		\newcommand{\Gcov}{\ensuremath{\mathbf{G}}\phantom{ }}
		
		\DeclareMathOperator{\Sp}{Sp}
		\DeclareMathOperator{\SL}{SL}
		\DeclareMathOperator{\GL}{GL}
		\DeclareMathOperator{\Mp}{Mp}

		%Fonts
		\newcommand{\Cal}[1]{\ensuremath{\mathcal{#1}}}
		\newcommand{\Cat}[1]{\ensuremath{\mathscr{#1}}}
		%Format
		\newcommand{\gap}{\vspace{4mm}}
		\newcommand\scalemath[2]{\scalebox{#1}{\mbox{\ensuremath{\displaystyle #2}}}}
		\usepackage{ragged2e}


		
		%Environments
		\usepackage{amsthm} %No numbering
		\newtheorem*{Lemma}{Lemma}
		\newtheorem*{Theorem}{Theorem}
		\newtheorem*{Remark}{Remark}
		\newtheorem*{Exercise}{Exercise}
		\newtheorem*{Notation}{Notation}
		\newtheorem*{Definition}{Definition}
		\newtheorem*{example}{Example}	
		\newtheorem*{cor}{Corollary}
		\newtheorem*{Prop}{Proposition}

		\usepackage{framed}
		%\usepackage{mdframed}
		\newcommand{\proofbar}[1]{
			\begin{leftbar}
				\textit{\textbf{Proof.}} 	#1
			\end{leftbar}
			}

		%Highlighting
		\newcommand{\notice}[1]{\hl{\textbf{\textit{#1}}}}
		\DeclareRobustCommand{\hlcyan}[1]{{\sethlcolor{cyan}\hl{#1}}}
		\DeclareRobustCommand{\hlgreen}[1]{{\sethlcolor{green}\hl{#1}}}
			
			
			
			
			