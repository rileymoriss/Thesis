The theory of L-functions is not yet systematic; Langlands has provided a conjectural framework, however it is still under construction.  
In the mean time there are two major ``paradigms'' for constructing and proving theorems about L-functions, those are the Langlands-Shahidi type constructions and the Rankin-Selberg type constructions. General surveys can be found in \cite[Part 2.III.2]{borelAutomorphicFormsRepresentations1979}\cite{shahidiEisensteinSeriesAutomorphic2010}\cite{cogdellLFUNCTIONSFUNCTORIALITY}\cite[9, 10, 11]{bumpIntroductionLanglandsProgram2004}\cite{arthurLfunctionsAutomorphicRepresentations}.

The Rankin-Selberg type functions are surveyed in \cite{bumpRankinSelbergMethodIntroduction2011}. The \(\GL_n \times \GL_m\) case is dealt with in \cite{cogdellLECTURESINTEGRALREPRESENTATIONS}. For Rankin-Selberg L-functions of the form \(\Sp_{2n}\times \GL_m \) the theory (for generic cuspidal representations) is worked out in \cite{ginzburg$L$functionsSymplecticGroups1998}.

The Langlands-Shahidi paradigm is explained in \cite{shahidiProofLanglandsConjecture1990, shahidiEisensteinSeriesAutomorphic2010}.

We have by \cite{cogdellLFUNCTIONSFUNCTORIALITY} some properties uniquely determining L-functions for \textit{tempered} representations. It is a conjecture that all generic representations are tempered, some work in this direction is in \cite{shahidiArthurPacketsRamanujan2011}, under this hypothesis we can apply the theory of Rankin-Selberg and Ginzburg-Ralis to explicitly construct global L-functions and prove theorems about them. In particular their analytic properties are well understood in these cases from \cite{grbacResidualSpectrumSplit2011, cogdellRemarksRankinSelbergConvolutions}. Note that \cite{grbacResidualSpectrumSplit2011} is conditional on the unfinished work of Arthur \cite{arthurEndoscopicClassificationRepresentations2013b}.

\label{L-functions}

\section{The Langlands Framework}
We follow closely Borels exposition in \cite[Part 2, ``Automorphic L-functions'']{borelAutomorphicFormsRepresentations1979} and \cite{shahidiEisensteinSeriesAutomorphic2010}.
Given a reductive LAG \(G\) defined over \C there is an associated root datum \((X, \Phi, \hat{X}, \hat{\Phi})\), where for any choice of maximal torus we have \(X = \Hom(T, \mathbb{G}_m)\), \(\hat{X} = \Hom(\mathbb{G}_m, T)\), and \(\Phi, \hat\Phi\) are the roots and coroots of \(G\) with respect to \(T\) \cite[7.4.3]{springerLinearAlgebraicGroups1998}.  Then each reductive LAG \(G\) over a number field \(F\) has the root datum that is associated to the base change of \(G\) to \(\C\), \((X, \Phi, \hat{X}, \hat{\Phi})\).
By the existence theorem \cite[10]{springerLinearAlgebraicGroups1998} to the dual root datum \(( \hat{X}, \hat{\Phi}, X, \Phi)\) there is a LAG defined over \C that corresponds, we call this the \textbf{dual group} of \(G\) and we denote it \(\hat{G}\). It is possible through the use of the root datum to specify a ``cannonical'' action of \(\mathrm{Gal}(\bar{F}/F)\) on \(\hat{G}\) as in loc. cit.
The \textbf{Langlands dual group} is then the dual group semi-direct producted with the \(\mathrm{Gal}(\bar{F}/F)\) via this action, which we omit
\[^L G \defeq \hat{G} \rtimes \mathrm{Gal}(\bar{F}/F).\]

\begin{example}[Classical Groups, \cite{bumpIntroductionLanglandsProgram2004}, 11.1]
	We have the following table
	\begin{table}[h]
		\centering
		\begin{tabular}{ll}
			\(G\)         & \(\hat{G}\)   \\ \hline
			\(\GL_n\)     & \(\GL_n\)     \\
			\(\SO_{2n+1}\) & \(\Sp_{2n}\)  \\
			\(\SO_{2n}\)   & \(\SO_{2n}\)   \\
			\(\Sp_{2n}\)  & \(\SO_{2n+1}\)
		\end{tabular}
	\end{table}
\end{example}

If \(\nu\) is a non-archimedean place of F, then \(\mathcal{O}_\nu\) is a local ring and we denote \(q_\nu\) the cardinality of the residue field i.e. if \(\mathfrak{p}_\nu\) is the unique maximal ideal of \(\mathcal{O}_\nu\) then \(q_\nu \defeq [\mathcal{O}_\nu: \mathfrak{p}_\nu]\). Using the Satake isomorphism, to each unramified representation of \(G(F_\nu)\) we can associate a conjugacy class of \(^LG\), via some map call it \(c\), and hence there is a way to apply a complex representation \(r: \phantom{ }^LG \to \GL_n(\C)\) to unramified representations of \(G(F_\nu)\), details in \cite[2]{shahidiEisensteinSeriesAutomorphic2010}. 
Given such an unramified representation of \(G(F_\nu)\), call it \(\pi_\nu\), the local automorphic L-function is then 
\[ L_\nu(s,\pi_\nu , r ) \defeq  \frac{1}{\det\bigl( I - r(c(\pi_\nu))q_\nu^{-s} \bigr)} ,\;\;\;\; s\in \C.\]
In the global case we consider an irreducible automorphic representation \(\pi = \tensor_\nu \pi_\nu\) of \(G(\A)\), and a finite set of places of \(F\), call it \(S\), such that \(S\) contains all infinite places and for all \(\nu\notin S\) \(\pi_\nu\) is unramified. Recall that we denoted the Langlands dual of \(G\) defined over \(F\) by \(^LG\). We denote the Langlands dual of \(G\) defined over \(F_\nu\) for \(\nu\notin S\) by \(^LG_{F_\nu}\).  If \(r\) is a finite dimensional complex representation of \(^LG\) then the embedding of Galois groups \(\mathrm{Gal}(\bar{F_\nu} / F_\nu) \hookrightarrow \mathrm{Gal}(\bar{F} / F) \) induces a map \(^LG_{F_\nu} \to ^LG\) along which we can pull \(r\) back, giving a representation \(r_\nu\) of \(^LG_{F_\nu}\). Then the partial global L-functions are defined to be 
\[L_S(s, \pi, r) \defeq \prod_{\nu\notin S} L(s, \pi_\nu, r_\nu), \;\;\;\;\; s\in \C.\]


\begin{example}[Standard Representations / Classical Groups]
	In the case of classical groups it is common to see L-functions with only two entries e.g. if \(\rho\) is a representation of \(G = \Sp_{2n}\) then you may see 
	\(L(s, \rho).\)
	The reason is that there is a standard representation of the dual groups of classical groups. Namely the standard representation of a matrix group inside \(\GL_m\) is the one that sends \(g\mapsto g\). It is this representation that is to be taken for the dual group in this setting.
\end{example}

\begin{example}[Rankin-Selberg, \cite{cogdellLECTURESINTEGRALREPRESENTATIONS}, 1.2, \cite{arthurLecturesAutomorphicFunctions1991a}, Ch. 2 Example. 2]
	Let \(\nu\) be a finite place of \(\Q\) and \(\pi, \pi'\) be two unramified generic representations of \(\GL_n(\Q_\nu)\) and \(\GL_m(\Q_\nu)\) respectively. Let \(B_n\) be the standard Borel of upper triangular matricies in \(\GL_n\). Such representations have been classified
	in terms of characters of \(\Q^\times_\nu\), in particular for \(\pi\) there are \(\mu_1, ..., \mu_n\) unramified characters such that 
	\[\pi \cong \Ind_{B(\Q_\nu)}^{\GL_n(\Q_\nu)} \big(\mu_1 \tensor \cdots \tensor \mu_n\big).\]
	If we fix a uniformizer \(\varpi\) of \(\Q_\nu\) then we have the so called ``Satake parameters'' \(\mu_i(\varpi)\) which determines \(\pi\) uniquely. Of course the same is true for \(\pi'\), with say characters \(\mu'_1, ..., \mu'_m\). We then define
	\[L(s, \pi\times \pi') \defeq \prod_{i,j} \frac{1}{1-\mu_i(\varpi)\mu'_j(\varpi)q^{-s}}.\]
	
	Consider the group \(G = \GL_n\times \GL_m\) which has dual \(\GL_n(\C) \times \GL_m(\C)\), then there is a cannonical representation 
	\[\tensor:\GL_n(\C) \times \GL_m(\C) \to \GL_{nm}(\C). \]
	Then 
	\[L(s, \pi\tensor \pi', \tensor) = L(s, \pi \times \tilde{\pi}'),\]
	where the tilde denotes the contragradient.
\end{example}

