The theory of L-functions is not yet systematic; Langlands has provided a conjectural framework in which one hopes all L-functions will live, however as far as we can see this is very much still under construction.  
In the mean time there are two major ``paradigms'' for constructing and proving theorems about L-functions, those are the Langlands-Shahidi type constructions and the Rankin-Selberg type constructions.

We have neither the time, space or knowledge to survey all of these things in any meaningful way, however we should say something, as these are key to the final section. 

The Rankin-Selberg type functions are surveyed in \cite{bumpRankinSelbergMethodIntroduction2011}. The \(\GL_n \times \GL_m\) case is dealt with in \cite{cogdellLECTURESINTEGRALREPRESENTATIONS}. For Rankin-Selberg L-functions of form \(\Sp_{2n}\times \GL_m \) the theory (for generic cuspidal representations) is worked out in \cite{ginzburg$L$functionsSymplecticGroups1998}.

The Langlands-Shahidi paradigm is explained in \cite{shahidiProofLanglandsConjecture1990, shahidiEisensteinSeriesAutomorphic2010}.

We have the langlands general framework. We have by \cite{cogdellLFUNCTIONSFUNCTORIALITY} some properties uniquenly determininng L functions for tempered things. By \cite{shahidiArthurPacketsRamanujan2011} all generic things are tempered so we can apply the theory of RS and GR to explicitly construct global things and prove theorems. In particular their analytic properties are well understood in these cases from \cite{grbacResidualSpectrumSplit2011, cogdellRemarksRankinSelbergConvolutions}. Note that \cite{grbacResidualSpectrumSplit2011} is conditional on the controversial work of Arthur \cite{EndoscopicClassificationRepresentations} (apparently his proof refers to unfinished pre-prints).

\label{L-functions}
\section{L-Functions for \(\GL_n\times \GL_m\)}
\subsection{Generic Representations}
\todo[inline, color=blue]{L-fuunctions chapter}
\subsection{Definition}
\cite{cogdellLECTURESINTEGRALREPRESENTATIONS}

\subsection{Poles and Zeroes}


\section{L-Functions for Classical Groups}

\(L(s, \tau\times \sigma)\)
\cite{remarks on rankin selberg convolutions Cogdel and PT-shapiro}\cite{cogdellFunctorialityClassicalGroups2004} -----> Defined in \cite{shahidiProofLanglandsConjecture1990}

\(L(s, \tau, \sigma)\)
\cite{grbacResidualSpectrumSplit2011}

\begin{example}[]
	The so called ``Rankin-Selberg'' L-function associated to -------, is given by 
	\[L(s, \tau\times \tau) = L(s, \tau, \rho)L(s, \tau, \rho^-)\]	
\end{example}

\section{Automorphic L-Functions}
We don't intent to define in great detail automorphic L-functions, as there are many other better sources to learn from \cite[Part 2.III.2]{borelAutomorphicFormsRepresentations1979}\cite{shahidiEisensteinSeriesAutomorphic2010}\cite{cogdellLFUNCTIONSFUNCTORIALITY}\cite[9, 10, 11]{bumpIntroductionLanglandsProgram2004}\cite{arthurLfunctionsAutomorphicRepresenta}, we will recall the idea and then discuss some of the properties and relations with Eisenstein series and interwining operators that we will need later.

The first thing is to recall the classification of connected reductive groups defined over an algebraically closed field via root datum. A root datum is a tuple \((X, \Phi, \check{X} , \check{\Phi})\) where \(X\) and \(\check{X}\) are two free abelian groups of finite type, \(\Phi, \check{\Phi}\) are subgroups that are in duality via a perfect pairing on \(X, \check{X}\). Then each reductive group \(G\) over a number field \(F\) has associated the root datum that is associated to its base change to \(\C\). Thus to a connected reductive group over a number field we associate a connected reductive group over \C, given by the dual root datum. We call this the \textbf{dual group} of \(G\) and denote it \(\hat{G}\). The \textbf{Langlands dual group} is then the dual group producted with the \(\mathrm{Gal}(\bar{k}/k)\)
\[^L G \defeq \hat{G} \rtimes \mathrm{Gal}(\bar{k}/k).\]

\begin{example}[Classical Groups, \cite{bumpIntroductionLanglandsProgram2004}, 11.1]
	We have the following table
	\begin{table}[h]
		\centering
		\begin{tabular}{ll}
			\(G\)         & \(\hat{G}\)   \\ \hline
			\(\GL_n\)     & \(\GL_n\)     \\
			\(SO_{2n+1}\) & \(\Sp_{2n}\)  \\
			\(SO_{2n}\)   & \(SO_{2n}\)   \\
			\(\Sp_{2n}\)  & \(SO_{2n+1}\)
		\end{tabular}
	\end{table}
\end{example}

Then, using the Satake isomorphism \cite[2.2]{shahidiEisensteinSeriesAutomorphic2010}, to each unramified representation of \(G(F_\nu)\) we can associate a conjugacy class of \(^LG\), via some map call it \(c\), and hence there is a way to apply a complex representation \(r: ^LG \to \GL_n(\C)\) to representations of \(G(F_\nu)\). Thus the automorphic L-functions are defined as follows: Let \(\rho\) be a representation of \(G(\A)\), let \(r\) be a complex representation of \(^LG\) and \(s\in \C\) then 
\[L(s, \rho, r) \defeq \prod_\nu L_\nu(s, \rho_\nu, r ) = \prod_\nu \frac{1}{\det\bigl( I - r(c(\rho_\nu))q^{-s} \bigr)}  ,\]
where \(\nu\) runs over the unramified places. It is a part of the grand Langlands philosophy that there should be suitable L-functions for the ramified places satisfying very nice properties.

\begin{remark}
	The global L-functions have been defined for many groups at this point and indeed \cite{jiangPolesCertainResidual2013} uses known properties to prove their results. One should note that the questions that we are interested in are still tractable even though the L-functions might not be defined (for instance for the metaplectic group). This is because only finitely many places will ramify, and so as long as those places are neither zero or poles we can transfer questions about zeros and poles from the full global L-functions to L-functions at almost all places. 
\end{remark}

\begin{example}[Standard Representations / Classical Groups]
	In the case of classical groups it is common to see L-functions with only two entries e.g. if \(\rho\) is a representation of \(G = \Sp{2n}\) then you may see 
	\(L(s, \rho).\)
	The reason is that there is a standard representation of the dual groups of classical groups. Namely the standard representation of a matrix group inside \(\GL_n\) is the one that sends \(g\mapsto g\). It is this representation that is to be taken for the dual group in this setting.
\end{example}

\begin{example}[Rankin-Selberg, \cite{cogdellLECTURESINTEGRALREPRESENTATIONS}, 1.2, \cite{arthurLecturesAutomorphicFunctions1991a}, Ch. 2 Example. 2]
	Let \(\nu\) be a finite place of \(\Q\) and \(\pi, \pi'\) be two unramified generic representations of \(\GL_n(\Q_\nu)\) and \(\GL_m(\Q_\nu)\) respectively. Let \(B_n\) be the standard Borel of upper triangular matricies in \(\GL_n\). Such representations have been classified \todo[inline]{reference}
	in terms of characters of \(\Q^\times_\nu\), in particular for \(\pi\) there are \(\mu_1, ..., \mu_n\) unramified characters such that 
	\[\pi \cong \Ind_{B(\Q_\nu)}^{\GL_n(\Q_\nu)} \big(\mu_1 \tensor \cdots \tensor \mu_n\big).\]
	If we fix a uniformizer \(\varpi\) of \(\Q_\nu\) then we have the so called ``Satake parameters'' \(\mu_i(\varpi)\) which determines \(\pi\) uniquely. Of course the same is true for \(\pi'\), with say characters \(\mu'_1, ..., \mu'_m\). We then define
	\[L(s, \pi\times \pi') \defeq \prod_{i,j} \frac{1}{1-\mu_i(\varpi)\mu'_j(\varpi)q^{-s}}.\]
	
	Consider the group \(G = \GL_n\times \GL_m\) which has dual \(\GL_n(\C) \times \GL_m(\C)\), then there is a cannonical representation 
	\[r:\GL_n(\C) \times \GL_m(\C) \to \GL_{nm}(\C). \]
	Then 
	\[L(s, \pi\tensor \pi', r) = L(s, \pi \times \tilde{\pi}'),\]
	where the tilde denotes the contragradient.
\end{example}

\begin{example}[Dirichlet L-functions]
	Recall that a Dirichlet character \(\chi\) is a character of the group \((\Z/N\Z)^*\) and a Grossencharacter (also known as Hecke characters) \(\chi'\) is a character of the group \(\A_\Q^\times/\Q^\times\). Through the series of maps 
	% https://q.uiver.app/#q=WzAsNSxbMiwwLCJcXFJfez4wfV5cXHRpbWVzIFxcdGltZXMgXFxoYXR7XFxafV5cXHRpbWVzIl0sWzQsMCwiXFxiaWcoXFx2YXJwcm9qbGltXFxaL05cXFpcXGJpZyleXFx0aW1lcyAiXSxbNiwwLCIgKFxcWi9OXFxaKV5cXHRpbWVzIl0sWzMsMiwiXFxDIl0sWzAsMCwiXFxBXlxcdGltZXMgLyBcXFFeXFx0aW1lcyJdLFswLDEsIlxcbWF0aHJte3Byb2p9XzMiXSxbMSwyXSxbMiwzLCJcXGNoaSJdLFs0LDMsIlxcY2hpJyIsMl0sWzQsMF1d
	\[\begin{tikzcd}[cramped]
		{\A^\times / \Q^\times} && {\R_{>0}^\times \times \hat{\Z}^\times} && {\big(\varprojlim\Z/N\Z\big)^\times } && { (\Z/N\Z)^\times} \\
		\\
		&&& \C
		\arrow["\sim",from=1-1, to=1-3]
		\arrow["{\chi'}"', from=1-1, to=3-4]
		\arrow["{\mathrm{proj}_3}", from=1-3, to=1-5]
		\arrow[from=1-5, to=1-7]
		\arrow["\chi", from=1-7, to=3-4]
	\end{tikzcd}\]
	we get a bijection between Dirichlet characters and finite-order Grossencharacters, i.e. Grossencharacters \(\chi'\) such that for some \(n\in \N\) we have \((\chi')^n = 1\). These are just grossencharacters trivial in the infinite place, by construction.
	
	Grossencharacters are just automorphic forms on \(\GL_1\) and hence they generate automorphic representations by taking the span of their orbit in the space of automorphic forms. 
	
	The claim is then that the Langlands automorphic L-functions of this representation is, after being pushed forward along this series of maps the regular Dirichlet L-function. \todo{reference? More details?}
\end{example}