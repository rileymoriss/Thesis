\section*{Outline of Content}
The goal of this thesis is to exposit and hopefully provide some small generalisation of some of the results in \cite{jiangPolesCertainResidual2013}. We aim our exposition at the other masters students in our cohort.
To explain the results on poles of Eisenstein series to students in other disciplines there is a fair amount of background to be covered. This is the content of the first 5 chapters. Our aim is to aquire similar results as \cite{jiangPolesCertainResidual2013} but for metaplectic coverings. Despite this we largely avoid talking about the metaplectic technicallities in the background sections. We feel that if one is learning from this work those technicallities will only obscure the ideas, moreover we have nothing to add to the exposition that isn't covered in \cite{moeglinSpectralDecompositionEisenstein1995}. We will however include some details in the final section when they become more integral. 

Chapter one deals with the generalities of linear algebraic groups, the objects whose representation theory is the subject of discussion. First we define them and then look at the important subgroups that are used in the study of automorphic forms arising on the adelic points of these groups. We pay specific attention to the classical groups and for this reason, and to maintain the brevity and accessability of this work, assume that all our groups are split. Finally we mention the metaplectic covers of these groups and the appropriate associated subgroups. 

Chapter two deals with automorphic forms. We define automorphic forms in both the archemedean and adelic places. Finally we give the details of how to view modular forms as automorphic forms. We feel this is essential intuition for understanding the highly abstract definition of automorphic form. 

Chapter three is dedicated to automorphic representations. We define them and specify some important constructions that are needed in the final section.

Chapter four serves two purposes, we introduce the Eisenstein series and then as a motivation for later results we talk about its use in decomposing the regular representation of \(G(\A)\) on \(L^2(G(\A))\). Finally in this section we summarise the theory of automorphic L-functions. We do not go into detail as this theory is vast, confusing and well presented elsewhere. In particular we try to convey briefly how these functions are constructed, give some examples and collect some results that will be needed in the sequel. 

Chapter five is a grand look into the constant terms of automorphic forms and Eisenstein series in particular. We present a proof of a well known theorem in great detail that should be helpful to any new comers trying to understand the theory. We also describe some things in the classical setting, including the constant term (in the Fourier sense) of a modular form as well as the Siegel phi operator, as constant terms of certain automorphic forms. We should also make several appologies about this section as we have failed at every step to do our analytic due diligence, naively assuming that we can interchange sums and integrals and apply unfolding techniques without checking the hypothesis. 

Finally chapter six contains some exposition of recent work on the poles of residual Eisenstein series. We also retread some ground describing results on L-functions and automorphic representations, now in the case of metaplectic groups.

\section*{Motivation}
