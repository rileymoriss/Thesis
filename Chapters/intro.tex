\section*{Motivation}
The goal of this thesis is to exposit some of the results in \cite{jiangPolesCertainResidual2013}. We aim our exposition at the other masters students in our cohort.
To explain the results on poles of Eisenstein series to students in other disciplines there is a fair amount of background to be covered.

\section*{Outline of Content}
Chapter one deals with the generalities of linear algebraic groups, the objects whose representation theory is the subject of discussion. First we define them and then look at the important subgroups that are used in the study of automorphic forms arising on the adelic points of these groups. We focus on the classical groups.

Chapter two deals with automorphic forms. We define automorphic forms in both the Archimedean and adelic places. Finally we give the details of how to view modular forms as automorphic forms. 

Chapter three is a discussion of the concept of the constant term in the Archimedean place. First we define the constant term of an automorphic form (Archimedean) and then we show how it is related to the constant term of the Fourier series of a modular form. Finally we show how the classical Siegel Phi operator can be realised as a constant term.

Chapter four is dedicated to automorphic representations. We define them and specify some important constructions that are needed in the final section.

In chapter five we define adelic Eisenstein series and show how they generalise the classical modular forms also known as Eisenstein series.

Chapter six is dedicated to the constant term in the adelic setting. We first define them and then go through the process of computing them in great detail for Eisenstein series. 

Chapter seven is for defining L-functions, the analytic invariants that are central to the Langlands program. We will give several of the special cases that appear through out history and the literature. 

Finally chapter eight contains some exposition of recent work on the poles of residual Eisenstein series.


