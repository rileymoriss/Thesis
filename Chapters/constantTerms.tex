Here we will explain the role of the constant term in our calculation of poles.\label{constant_terms}

\section{Definition and Role} \label{cuspidal_form_definition}
The constant term is an operation defined on a large class of functions and is supposed to generalise the constant term of a Fourier expansion, we will see this later, although one may consult \cite[1.6]{bumpAutomorphicFormsRepresentations1997} for some examples as well. In particular \cite[I.2.6]{moeglinSpectralDecompositionEisenstein1995} give the definition as follows: We consider \(P=MU\) a standard parabolic of \(G\) and \(\phi: U(k)\setminus \mathbf{G} \to \C\) a measurable and locally \(L^1\) function then its \textbf{constant term} along \(P\) is 
\[\phi_P :  U(\A)\setminus \mathbf{G} \to \C\]
\[\phi_P(g) \defeq \int_{U(k)\setminus U(\A)}\phi(ug) \mathrm{d}u\]
which inherits many of the properties of \(\phi\) such as smoothness and moderate growth. Moreover if \(\phi\) is an automorphic form on \(\mathbf{G}\) then its constant term is an automorphic form on \(M\) \cite[6.5]{getzIntroductionAutomorphicRepresentations2024}.

This allows us to recall the definition of cuspidal automorphic forms or ``cusp forms''. Let \(\phi\) be an automorphic form on \(U(\A)M(k)\setminus \mathbf{G}\) for \(P = MU\) a standard parabolic. Then \(\phi\) is cuspidal if for all standard parabolics \(P'\subset P\) we have that 
\[\phi_{P'} = 0\]
Cusp forms have a central role in the theory of automorphic forms, this is for several reasons. They appear historically as interesting examples such as the Ramanujan tau function, by a theorem of Ribet \cite[T2.3]{serreProceedingsInternationalConference1977} the Galois representation associated to a cusp form is irreducible, they just make formulas easier to work with as things will become zero (as we see in section \todo{}) and they form the ``base case'' for the proof of the spectral decomposition \ref{spectral_decomposition}.

The constant term itself is of central importance. One reason for this is that it controls the growth of the automorphic forms. More precisely we have the following theorem \cite[I.4.10]{moeglinSpectralDecompositionEisenstein1995}
\todo[color=blue]{M+W}
\begin{Theorem}
        Let \(P = MU\) be a standard parabolic of \(G\), \(V_P\subseteq \mathcal{A}_0(U(\A)M(k)\setminus G)\) a finite dimensional subspace, \(\Gamma_P\subseteq X_M\) a compact subset and \(N_P\) an integer. Let \(n\in \Z\), \(D\subseteq \C^n\) open and connected, \(f: D\to \C\) holomorphic and not identically zero. Let \(D' = \{x\in D : f(x)\neq 0\}\).

        If \(\phi: D' \to L^2_{loc}(G)\) is a function such that for all \(z\in D', \phi(z)\in \mathcal{A}((V_P, \Gamma_P, N_P)_{P_0\subset P \subset G})\) and for all P 
        \[\psi_P: D' \to L^2_{loc}(G)\]
        \[z \mapsto \psi_P(z) = \phi_P^{cusp}(z)\]
        is a holomorphic function. 

        Then \(\phi\) can be analytically continued to a holomorphic function on \(D\) iff \(\psi_P\) can be continued to a holomorphic function on \(D\).
    \end{Theorem}
    
    Unpacking this a bit we see that an Eisenstein series satisfies the hypothesis on \(\phi\) by \cite[IV.1.9]{moeglinSpectralDecompositionEisenstein1995} \todo[inline]{make this more precise}, we think of \(D'\) as the positive open cone on which the Eisenstein series converges, and then \(D = \C^n \setminus S \) as the rest of the cuspidal datum \(\mathfrak{P}\) on which the Eisenstein series is holomorphic (\(S\) is its set of singularities). Therefore the domain on which the Eisenstein series is holomorphic is the same as the domain on which (roughly) its constant term is holomorphic.
    
        \todo[color=blue]{Constant terms vanish along other parabolics for E-series}

\section{Constant Terms of Eisenstein Series}\label{const_eisenstein}
This computation forms the heart of a well known theorem, \cite[Prop 10.4.2]{getzIntroductionAutomorphicRepresentations2024}\cite[II.1.7]{moeglinSpectralDecompositionEisenstein1995}\cite[6.2]{shahidiEisensteinSeriesAutomorphic2010}, although for an amateur the detail is lacking in other presentations.

Notice that the Eisenstein series has a full \(G(k)\) invariance and so we can take its constant terms along \textit{any} standard parabolic.

\subsection{In General}
We will use the following Lemmas to give a simplified expression of the constant term of an Eisenstein series. First fix \(P = MN\) and \(P' = M'N'\) two standard parabolics of a suitable group G over F, with \(E(x, \phi, \lambda)\) defined via  parabolic induction from P.

    \begin{Lemma}\label{lem:1}
        \[P(F)\setminus G(F) \cong \coprod_{w\in W_{M'}\setminus W_G / W_{M}} P'(F)\cap wP(F)w\inv \setminus P'(F)\]
    \end{Lemma}
    \proofbar{
        Consider the Bruhat decomposition:
        \[G(F) =\coprod_{w\in W_{M'}\setminus W_G / W_{M}} P(F)w\inv P'(F) \]
        then because the action of \(P(F)\) keeps the disjoint sets disjoint we can move the quotient through and get
        \[P(F)\setminus G(F) = \coprod_w  P(F) \setminus P(F)w\inv P'(F)\]
        so we analyse the summands, by the second isomorphism theorem we have a bijection
        \[P(F)\setminus P(F)w\inv P'(F) \cong P(F)\cap P'(F) \setminus w\inv P'(F) \]
        now if \([w\inv p] \in P(F)\cap P'(F) \setminus w\inv P'(F) \) then its represented by some \(pw\inv p'\) where \(p\in P(F)\cap P'(F)\) and hence multiplying by \(w\), in particular an isomorphism, gives \(wpw\inv p'\in wP(F)w\inv \times P'(F)\) and so 
        \[w(P(F)\cap P'(F) \setminus w\inv P'(F)) \cong wP(F)w\inv \cap P'(F) \setminus P'(F)\]
    }

    \begin{Lemma}\label{lem:2}
        Let \(m', n'\in M'(F)\times N'(F)\) then 
        \[m'n' \in wP(F)w\inv \iff m'\in wP(F)w\inv \text{ and  }\;\; n'\in (m')\inv wP(F)w\inv m'\]
    \end{Lemma}
    \proofbar{
        The forward implication is stated in \cite{getzIntroductionAutomorphicRepresentations2024}, the converse follows from some algebra:
        First let \(m' = wp_1w\inv\) and \(n' = (m')\inv wp_2w\inv m'\) then 
        \begin{equation*}
            \begin{aligned}
                m'n' &= (wp_1w\inv)\inv wp_2w\inv wp_1w\inv\\
                     &= wp_1\inv w\inv wp_2w\inv wp_1w\inv\\
                     &= wp_1\inv p_2p_1w\inv \in wP(F)w\inv\\
            \end{aligned}
        \end{equation*}
    }
    Taking the contrapositive of this lemma will be used below. This is because our sums will be over quotients like \(A\setminus B\) and therefore summing over the ``elements'' in B that are not in A; by our lemma would be the same as summing over two different such quotients.
Now we will apply our lemmas to simplify and make more explicit the constant term of an Eisenstein series
    \begin{equation*}
        \begin{aligned}
            E_{P'}( \phi, \lambda, x) &= \int_{N'(F)\setminus N'(\A)} E( \phi, \lambda, nx) dn\\
                                    ([N']\defeq N'(F)\setminus N'(\A)) \;\;\; &= \int_{[N']} \sum_{\delta\in P(F)\setminus G(F)} \lambda.\phi(\delta nx)  dn\\
                                    (\text{Lemma 1}) \;\;\; &= \int_{[N']} \sum_{\delta\in \coprod_{w\in W_{M'}\setminus W_G / W_{M}} P'(F)\cap wP(F)w\inv \setminus P'(F)} \lambda.\phi(\delta nx)  dn\\
                                     &= \sum_{ w\in W_{M'}\setminus W_G / W_{M}}\int_{[N']} \sum_{p'\in P'(F)\cap wP(F)w\inv \setminus P'(F)} \lambda.\phi( w\inv p'nx)  dn\\
                                    (\text{Lemma 2}) \;\;\; &= \sum_{ w} \sum_{m'\in M'(F)\cap wP(F)w\inv\setminus M'(F)} \int_{[N']} \sum_{n'\in N'(F)\cap (m')\inv wP(F)w\inv m' \setminus N'(F)} \lambda.\phi( w\inv m'n'nx)  dn\\
                                    (\text{Change Var}) \;\;\; &= \sum_{ w} \sum_{m'} \int_{[N']} \sum_{n'\in N'(F)\cap wP(F)w\inv\setminus N'(F) } \lambda.\phi( w\inv n'nm'x)  dn\\
                                    (\text{Unfold}) \;\;\; &= \sum_{ w} \sum_{m'} \int_{N'(F)\cap wP(F)w\inv \setminus N'(\A)} \lambda.\phi( w\inv nm' x)  dn.\\
        \end{aligned}
    \end{equation*}
    \todo[inline]{I need to fix all the lemma labeling }
    The change of variables is \((m', n') \mapsto ((m')\inv n' m', (m')\inv n' m')\).
    Again we assume that our $x$ is sufficiently large so all the integrals converge.\todo[inline]{Maybe appologise for doing integrals naivelly lol.. clarify this.... is it even the x that I need to worry about here?}

\subsection{Constant Terms of Cuspidal Eisenstein Series}
\begin{Lemma}[4]
        For \(w\in W_{M'}\setminus W_G / W_{M} \) we have that \(w\inv P'w\cap M\) is a standard parabolic of \(M\) with Levi \(w\inv M'w\cap M\) and unipotent \(w\inv N'w\cap M\).
    \end{Lemma}
    \proofbar{
        This is \cite[10.4.1]{getzIntroductionAutomorphicRepresentations2024} stated without proof. They give the reference \cite[V.4.6]{renardREPRESENTATIONSGROUPESREDUCTIFS} which is in French..
    }
    \begin{Lemma}[5]
        \[w\inv U' w \cap P = (w\inv U' w \cap M)(w\inv U' w \cap U).\]
    \end{Lemma}
    \proofbar{
        \cite[10.4.1]{getzIntroductionAutomorphicRepresentations2024} has some decompositions, as well as the standard decomposition of \(P=MU\) I think I could prove this...
    }
    \begin{Lemma}[6]
        \[c\setminus (b\setminus a )= (bc)\setminus a\]
    \end{Lemma}

    Continuing the computation of the constant term above, we will focus purely on the inner integral now
    \begin{equation*}
        \begin{aligned}
            \int_{N'(F)\cap wP(F)w\inv \setminus N'(\A)} \lambda.\phi( w\inv nm' x)  dn &= \int_{w\inv N'(F)w \cap P(F) \setminus w\inv N'(\A)w} \lambda.\phi( nw\inv m' x)  dn \\
            (\text{Lemma 5})&= \int_{(w\inv U' w \cap M)(w\inv U' w \cap U)(F) \setminus w\inv N'(\A)w} \lambda.\phi( nw\inv m' x)  dn \\
            (\text{Unfold + Lemma 6})  &= \int_{(w\inv U'(\A)w \cap M(\A)) \setminus A} \int_{w\inv U'(F) w \cap M(F) \setminus w\inv U'(\A)w \cap M(\A)} \lambda.\phi( n_1 n_2 w\inv m' x)  dn_1 dn_2. \\
        \end{aligned}
    \end{equation*}
    the first equality is the change of variables \(w\inv n w\mapsto n \) and \(A = (w\inv U'(F) w \cap U(F) ) \setminus w\inv N'(\A)w \). Now look at the inner integral here more closely 
    \[ \int_{w\inv U'(F) w \cap M(F) \setminus w\inv U'(\A)w \cap M(\A)}\lambda. \phi( n_1 n_2 w\inv m' x)  dn_1 dn_2,\]
    applying Lemma (6) we see that this is a constant term for a parabolic of \(M\), of the function \(m\mapsto \phi(m n_2 w\inv m' x)\). 
    \begin{Lemma}
        \(n_2 w\inv m' x \in K\) with variables as above.
    \end{Lemma}
    This was all in complete generality as well. If we now assume further that the Eisenstein series was induced from a \textit{cuspidal} automorphic representation, then \(m\mapsto \phi(mk)\) is a cusp form and therefore this last integral will vanish whenever \(w\inv U'w \cap M \neq \{1\}\), because in that case the inner integral doesn't exist (its over a point).

    \subsection{Constant Term Of Eisenstein Series for Conjugate Levis}\label{constant_conjugate_levi}
    If we now assume that \(M' = wMw\inv\) and recall the definition of our intertwining operator \ref{intertwining_operator} we can use the following 
    \begin{Lemma}[\cite{moeglinSpectralDecompositionEisenstein1995} II.1.7 (6)]
        \[U'(k) \cap wP(k) w\inv = U'(k) \cap wU(k)w\inv,\]
    \end{Lemma}
    to see that 
    \begin{equation*}
        \begin{aligned}
             E_{P'}( \phi, \lambda, x) &= \sum_{ w} \sum_{m'} \int_{N'(F)\cap wP(F)w\inv \setminus N'(\A)} \lambda.\phi( w\inv nm' x)  dn \\
             &=  \sum_{ w} \sum_{m'} \int_{N'(k) \cap wN(k)w\inv \setminus N'(\A)} \lambda.\phi( w\inv nm' x)  dn \\
             &= \sum_{ w} \sum_{m'} M(w, \pi)(\lambda.\phi)(x)
        \end{aligned}
    \end{equation*}\todo[inline]{I have mixed up my N's and U's too much...}
    In particular we can combine the conjugate and cuspidal cases to get a much simpler expression for some constant terms of some Eisenstein series, we will go through a detailed example in the final section \todo{}.
    


    