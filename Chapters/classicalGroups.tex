We will recall a small amount of the theory of linear algebraic groups to fix conventions, for a more detailed treatment one should consult the litany of sources on this matter: For a full treatment see \cite{milneAlgebraicGroupsTheory2017}\cite{milneLieAlgebrasAlgebraic}\cite{milneBasicTheoryAffine}\cite{springerLinearAlgebraicGroups1998}. Excellent example computations can also be found in \cite{BuildingsClassicalGroups}\cite{makisumiStructureTheoryReductive}\cite{malleLinearAlgebraicGroups}\cite{NotesClassAlgebraic}. Or for a brief brush up on the main facts consult \cite[I.I.1]{borelAutomorphicFormsRepresentations1979}. The purpose of this section is to treat the classical groups and more specifically \(\Sp_{2n}\) as an example and work out some of the details of the general theory, in order to ``get our hands dirty'' and have some familiarity with this object, to become fundamental in what follows. Because this theory is made up of simple ideas that can often be obscured by the generality we will make several restrictive assumptions for ease of exposition. 

\section{Definition}
An \textbf{algebraic group} is for us a group scheme that is reduced, of finite type and defined over a field. \textbf{A linear algebraic group} (LAG) is simply an affine algebraic group.

\begin{proposition}
    An algebraic group is affine if and only if it is isomorphic to a Zariski closed subgroup of \(\GL_n\).
\end{proposition}
\proofbar{
    The forward implication is \cite[2.3.7(i)]{springerLinearAlgebraicGroups1998}. The converse is the basic fact that closed sub-schemes of affine schemes are affine \cite[II.5.T3]{mumfordRedBookVarieties1999}.
}

As Milne points out in \cite{milneAlgebraicGroupsTheory2017} these are matrix groups defined by polynomials, which happen to be the natural combinations of symbols that matrix multiplication will lead to. This means that they come with the powerful but cumbersome (for the beginner) technology of algebraic geometry. In particular one must be adept at moving between the following equivalences:
\begin{Theorem}[\cite{milneBasicTheoryAffine}, II.6, III.4]
    For \(F\) a field then the following categories are equivalent
    \begin{itemize}
        \item Group objects in \(\mathrm{Alg}_F^{op}\)
        \item Representable (in the category of groups) functors \(\mathrm{Alg}_F \to \mathrm{Group}\)
        \item Group object in the category of affine schemes over \(F\)
        \item Commutative \(F\)-Hopf algebras.
    \end{itemize}
\end{Theorem}

\begin{example}[\(\mathbb{G}_m\)]
    The first example that we will see again and again is the ``multiplicative group'' or \(\GL_1\) defined over the field \(K\). This is 
    \[\mathbb{G}_m \defeq \Spec\Bigl( K[x,y]/(xy - 1)\Bigr).\]

    As a representable functor this sends a \(K\)-algebra R to \(\Hom_K(K[x,y]/(xy - 1), R)\). These are ring maps that are K-linear, and  because \(y = x\inv\) we know that \(f(y) = f(x\inv) = f(x)\inv\) for \(f\in \mathbb{G}_m(R)\). Thus the maps are determined by where they send \(x\), moreover they always send it to a unit, i.e. \(Im f \subseteq R^\times\). For each element \(r\in R^\times\) we also have a map sending \(x\to r\) hence there is an isomorphism (of sets) that allows us to induce a group structure. 
\end{example}

The other important examples of such groups are the ``classical groups''. The exact groups that an author might mean by classical may vary, so we define them explicitly here. First let V be a finite dimensional F-vector space with a bilinear form \(\inner{,}\). An automorphism of this form is a map \(\alpha\in \Aut(V)\) such that 
\[\inner{\alpha(x), \alpha(y)} = \inner{x,y}.\]
Therefore we can consider the space of automorphisms of this form \(\Aut(V, \inner{,})\). This space, depending on the properties of the bilinear form, will define our classical groups. 

If the form is trivial, by which we mean, \(\forall x,y \;\; \inner{x, y} = 0\) then we define 
\[\GL(V) \defeq \Aut(V, \inner{,}) = \Aut(V).\]
If the form is non-degenerate and symmetric \(\forall x,y \;\; \inner{x, y} = \inner{y,x}\) then we define
\[\mathrm{O}(V) \defeq \Aut(V, \inner{,}).\]
Finally if the form is non-degenerate and skew symmetric \(\forall x,y \;\; \inner{x, y} = -\inner{y,x}\) then 
\[\Sp(V) \defeq \Aut(V, \inner{,}).\]
There are the further classical groups given by the determinant one subgroups, \(\SL(V), \mathrm{SO}(V)\) respectively (\(\Sp(V)\) one can show already implies that the determinant is one). We can make this into a functor from \(F\)-algebras to groups, by sending an \(F\)-algebra \(R\) to \(G(V)\tensor_F R\). Thus these define LAG's. 

\begin{remark}
    Often the unitary groups are considered classical, however we don't want to deal with field extensions and so omit them here. 
\end{remark}

\section{Subgroups}
From now on let \(G\) be a (split reductive) LAG defined over a number field \(F\).
\begin{remark}[For the experts]
    We restrict to split reductive LAG in what follows. This is justified by the fact that the classical groups are all split reductive over number fields.  
\end{remark}

Subgroups with special properties allow us to reduce and break up problems into smaller ones. Here we will briefly review and compute some examples of special subgroups. The point of these subgroups is two fold. Some of them will help us perform ``induction'' from smaller simpler groups to larger ones. Others are there essentially as a part of the combinatorial data that classifies the groups we are working with. In particular we need to understand all the pieces of the so called \textbf{Langlands-Iwasawa decomposition},
\begin{equation}\label{eq:iwasawa_decomposition}
    G(\A) = M(\A)U(\A) K = T(\A)U(\A)K.
\end{equation}

\subsection{Parabolics, Levis and Unipotents}
Parabolic subgroups have two equivalent formulations, both useful.
\begin{definition}
    A subgroup \(P\subseteq G\) is called \textbf{parabolic} if \(G/P\) is a complete variety. Eqquivilently we can ask for \(P\) to contain a Borel (see \ref{borel_torus}).
\end{definition}

Completeness is the algebro-geometric analogue of compact, which is always a desirable property. The fact that they contain a Borel gives us an algebraic ``parametrisation'' of these subgroups, in the case of the classical groups through the use of flags or roots. It is very important to have a parametrisation of the parabolic subgroups when it comes to taking constant terms of Eisenstein series which we will discuss in the later section \ref{constant_terms}.

Parabolics also have the nice property that they split into a semi-direct product 
where one of the factors is a reductive group \(M\). For this recall the definition
\begin{definition}\label{unipotent_radical_definition}
    A matrix \(m\) is \textbf{unipotent} if for some \(n\geq 0\) we have that \((m-1)^n = 0\). A subgroup is \textbf{unipotent} if all its elements are unipotent. The \textbf{unipotent radical} of \(G\)is the maximal closed, connected, unipotent subgroup. A linear algebraic group is \textbf{reductive} if its unipotent radical is trivial.
\end{definition}
Then we have the following fact / definition:

\begin{Lemma}[\cite{borelLinearAlgebraicGroups1991} 11.22]
    There is a split exact sequence (of algebraic groups)
    \[0 \to U \to P \to M \to 0,\]
    where \(U\) is the unipotent radical of P, and \(M\) is a reductive group known as a \textbf{Levi} (unique up to conjugacy).
\end{Lemma}

Thus doing things on a parabolic allows us to induce said actions up to the whole group, whist maintaining the nice property of being reductive. This is the technique of ``parabolic induction'' \cite[Thm. 10]{bernsteinREPRESENTATIONSPADICGROUPS1992} that we wont explicitly talk about here but which is happening secretly in the background in \ref{automorphic_isotypic_subspaces}.

\begin{Remark}[Bad Etymology]
    The origin of the name parabolic is a mystery. Borel in his history \cite[VI.\S 2]{EssaysHistoryLie} attributes it to R. Godement in \cite{godementGroupesLineairesAlgebriques}. Godement conjectures that the quotient \(G(\A) / G(\Q)\) is compact if and only if every element of \(G(\Q)\) is semi-simple, as is the case in classical groups. \todo[inline]{this is probably known by now.} He says that 
    \begin{quote}
        Lorsque n'est pas compact, il est non moins facile de conjecturer qu’on doit pouvoir définir quelque chose d’analogue aux classiques ``pointes paraboliques'', lesquelles doivent correspondre à des  sous-groupes unipotents non triviaux de \(G_\Q\)
    \end{quote}
    which roughly (google) translates to that one can also conjecture that non-trivial unipotent elements should correspond to ``parabolic points'' in a fundamental domain.

    In the case of modular forms the fundamental domain is \(\mathcal{H} = \SL_2(\R)/SO_2(\R)\) (using orbit stabiliser theorem). We have the classification of elements of  \(\SL_2(\R) -\{\pm 1\}\) as in \cite[3.5]{borelAutomorphicFormsSL21997} via their trace
    \[g\text{ is of type } \;\;\; 
    \begin{cases}
        \text{Elliptic if} & \frac{1}{2}|tr(g)| < 1 \\
        \text{Parabolic if} & \frac{1}{2}|tr(g)| = 1 \\
        \text{Hyperbolic if} & \frac{1}{2}|tr(g)| > 1 \\
    \end{cases}
    .\]
    Being parabolic is equivalent to having eigenvalue 1 hence by the Jordan decomposition we know that parabolics in \(\SL_2\) are conjugate (over \(\C\)) to 
    \[\begin{pmatrix}
        1 & 1\\
        0 & 1
    \end{pmatrix},\;\;\; \pm\begin{pmatrix}
        1 & 0\\
        0 & 1
    \end{pmatrix}.\]
    Clearly the standard parabolic 
    \[\begin{pmatrix}
        a & b \\
         & a\inv
    \end{pmatrix} \subseteq \SL_2(\R),\]
    contains these matrices, and moreover all parabolics are \textit{conjugate} to this parabolic. Hence all parabolic elements are contained in a parabolic subgroup. This classification, it seems, relies entirely on the \textit{aesthetic} connection with the classification of the sections of conics via eccentricity.

    To connect this to Godement's concept we have two facts from classical geometry. Proper parabolic subgroups of \(\SL_2(\R)\) can be realised as the stabilisers of lines in \(\R^2\) under the standard action of \(\SL_2\) on \(\R^2\) \cite[2.6]{borelAutomorphicFormsSL21997} and moreover some an element of \(\SL_2(\R)\) is parabolic if and only if it has one fixed point on \(\partial\bar{\mathcal{H}}\) and none on \(\mathcal{H}\) \cite[3.5]{borelAutomorphicFormsSL21997}. 

    The take away is that perhaps the folklore of the name being for ``para-Borelic'', as in kind of a Borel, is probably a better way of thinking of them.
\end{Remark}

\subsubsection{The Example of \(\Sp_{2n}\)}
We collect the following facts as they will be useful in what is to come. Good references are the notes \cite{conradStandardParabolicSubgroups} and the book \cite[\S 8]{BuildingsClassicalGroups}. 

Let \((V, \inner{,})\) be a symplectic space as above and \(Sp(V)\) is the automorphisms preserving the form. A \textbf{flag} of \(V\) is a sequence of strict inclusions of vector subspaces 
\[\{0\}\subset V_1 \subset \cdots \subset V_{n-1} \subset V. \]
A subspace of \(V\) is said to be \textbf{isotropic} if the form is constantly zero on it (in both variables). A flag is \textbf{isotropic} if the proper subspaces in it are isotropic subspaces. A \textbf{maximal isotropic} flag is one with exactly \(n\) components. \(\Sp_{2n}\) acts on a flag by acting on each of the subspaces. This action preserves isotropic flags i.e. it sends an isotropic flag to an isotropic flag. Stabilisers of isotropic flags give parabolics of \(\Sp\) and moreover all parabolics arise in this way \cite[Exercise 3.2.16, 6.2.11]{springerLinearAlgebraicGroups1998}.

\begin{example}
        Consider a four dimensional vector space \(V\) with a form given by the matrix
        \[\begin{pmatrix}
            & I_2 \\
            -I_2 & 
        \end{pmatrix},\]
        then a maximal isotropic flag is 
        \[0 \subset Fe_1 \subset Fe_1 \oplus Fe_2 \subset F^4,\]
        where \(e_i = (\delta_i^j)_j\). This has stabiliser consisting of matrices in \(\Sp\) of the form
        \[\begin{pmatrix}
            *&*&*&* \\
             &*&*&* \\
             && *& \\
             && *& *
        \end{pmatrix}.\]
    \end{example}

    \label{maximal_parabolic}
    In particular maximal parabolics of \(\Sp\) are stabilizers of \textit{minimal} (non-trivial flags), i.e. stabilisers of non-zero isotropic subspaces,
    \[0 \subset V_\ell \subset V,\]
    where \(V_\ell = span_F(e_1, ..., e_\ell)\). Then the stabilizer is 
    \[\begin{pmatrix}
        * &*&*&* \\
        0 &*&*&* \\
        0 &*&*&* \\
        0 &*&*&* \\
    \end{pmatrix},\]
    with the sizes of the diagonal blocks being (these numbers square)
    \[\begin{pmatrix}
        \ell &*&*&* \\
        0 &n-\ell&*&* \\
        0 &*&\ell&* \\
        0 &*&*&n-\ell \\
    \end{pmatrix}.\]
    This has Levi
    \[\begin{pmatrix}
        A &&& \\
         &a&&b \\
         &&(A^T)\inv& \\
         &c&&d \\
    \end{pmatrix}, \;\;\; A\in \GL_\ell(F), \;\;\; \begin{pmatrix}
        a & b\\
        c & d \\
    \end{pmatrix} \in \Sp_{2(n-\ell)}(F),\]
    and unipotent 
    \[\begin{pmatrix}
        1 &*&*&* \\
        & 1&*& \\
        && 1& \\
        &&*&1
    \end{pmatrix},\]
    with relations among the entries.

    \subsection{Borel and Torus}\label{borel_torus}    
    A \textbf{split torus} is an algebraic group that is isomorphic to \(\GL_1^b\) for some \(b\in\N\).

    \begin{example}[Bad Etymology]
        \(\GL_1 /\C\) is a split torus. Consider the field extension \(\C/\R\). Then \C has the inner product given by 
        \[\inner{z, z'} \defeq \bar{z}z'.\]
        We can look at the elements of \(\C\) that preserve this inner product, 
        \begin{align*}
            U(1)&\defeq \{c\in \GL_1(\C) : \forall z,z'\in \C , \quad \inner{cz, cz'} = \overline{cz}cz' = \bar{z}z'\} \\
             &= \{c\in \GL_1(\C) : |c| = 1\}.
        \end{align*}
        Note that this is a (real) line topologically so we dont expect it to be a complex varity. Indeed this defines a \textbf{real} algebraic group given by the zero locus in \(\R^2\) of the two variable polynomial \(x^2 + y^2 - 1\). In other words
        \[ U(1) \cong \mathrm{MaxSpec}\big(\R[x,y]/(x^2 + y^2 - 1)\big) .\]
        Now if we base change to \C we have 
        \begin{align*}
            \R[x,y]/(x^2 + y^2 - 1) \tensor_\R \C &\cong \C[x,y]/\big((x+iy)(x-iy) -1 \big)\\
             &\cong \C[s,t]/(st -1) \\
             &\cong \C^\ast.
        \end{align*}
        Thus \(\GL_1 / \C\) is the complexification of the torus \(U(1)\).
    \end{example}
    
    \begin{remark}
        These tori also play the same role in the classification of reductive LAG as the real Lie groups called tori play in the classification of Lie groups \cite{hallLieGroupsLie2015}.
    \end{remark}
    
    A subgroup that is isomorphic to a split torus and is maximal in this respect is called a maximal split torus. 
     \begin{example}
        The classic example of a maximal split torus is the group of diagonal matrices in \(\GL_n\).
     \end{example}

    A \textbf{Borel} is a maximal closed solvable connected subgroup of \(G\).

    \begin{example}
        The standard Borel of \(\GL_n\) is the group of upper triangular matrices. If \(n\) is even and one intersects this Borel with \(\Sp_{2(\frac{1}{2}n)}\) then we get the standard Borel of \(\Sp_{2(\frac{1}{2}n)}\).

        Lets prove this in \(\GL_2\) and then believe that the only complication to going to larger \(n\) is keeping track of indices. So let 
        \[B = \begin{pmatrix}
            \ast & \ast \\
             & \ast
        \end{pmatrix},\]
        we need to show that the derived series terminates for it to be solvable. So let 
        \[g = \begin{pmatrix}
            x & y\\
             & z
        \end{pmatrix}, \;\;\; h = \begin{pmatrix}
            a & b \\
            & c
        \end{pmatrix},\]
        be arbitrary in \(\GL_2\), their commutator is then 
        \[g\inv h\inv gh =  \begin{pmatrix}
            1 & \frac{bx - ay}{ax} \\ & 1
        \end{pmatrix} .\]
        Hence
        \[[B, B] = \begin{pmatrix}
            1 & \ast \\ & 1
        \end{pmatrix}.\]
        Commutate two arbitrary elements again shows that  
        \[[[B, B], [B, B]] = 1.\]
    It is clear that this is a closed subgroup because it is itself a linear algebraic group, moreover for LAG's we have the algebraic criterion of connectedness given by having the only idempotents in the representing algebra being \(0, 1\) \cite[1.5]{getzIntroductionAutomorphicRepresentations2024}. Because \(B = \Spec \Z[x_{i,j}: 1\leq i,j\leq 2][y] / (\det(x_{ij})y - 1, x_{2,1})\) it is clear that this group is connected.     
    Finally it is clear that if a subgroup strictly contains this one then it is in fact all of \(\GL_2\) and hence this is maximal. Therefore this is a Borel.
    \end{example}
        
    A Borel can be considered to be a parabolic that is minimal with respect to inclusion. The maximal tori then form the Levis of these parabolics. In particular for a Borel \(B\) we have that 
    \[B = TU,\]
    for a maximal torus \(T\) and unipotent \(U\).

    If a Borel \(B\) is fixed, then a parabolic containing this Borel \(B\subseteq P\) is called standard, the unique Levi of a standard parabolic containing this Borel is called the \textbf{standard Levi}.

    \subsection{Maximal Compact Subgroups}\label{max_compact_subgroup}
    We will often need to fix a maximal compact subgroup \(K\subseteq G(\A)\), note that the topology is not the Zariski topology but the one specified in \cite{conradWeilGrothendieckApproaches2012}, this is sometimes known as the ``Hausdorff'' topology. These maximal compact subgroups are not unique and as such when fixing one it can be arranged to have many nice properties \cite[I.1.4]{moeglinSpectralDecompositionEisenstein1995}. In particular if we have a group \(G\) and a fixed Borel \(B\):
    \begin{itemize}
        \item First require that 
        \[K = \prod_\nu K_\nu,\]
        where the product is over all places of \(F\) and \(K_\nu\subseteq F_\nu\) is maximal compact.
        \item For almost all places \(\nu\), \(G(\mathcal{O}_{F_\nu})\) is defined and is maximal compact in \(G(F_\nu)\) hence we can require \(K_\nu = G(\mathcal{O}_{F_\nu})\) at these places. 
        \item We require 
        \[G(\A) = B(\A)K.\]
        \item For every standard parabolic \(P = MU\) we have that 
        \[P(\A)\cap K = \Bigl( M(\A)\cap K \Bigr) \Bigl( U(\A)\cap K \Bigr),\]
        and \(M(\A)\cap K\) is a maximal compact subgroup of \(M(\A)\).
    \end{itemize}
     It is in terms of the third property that we like to think of the maximal compact subgroup, it is the complimentary piece of the Borel. Moreover the fourth property should be thought of as a condition that the maximal compact subgroups are well behaved with the way that we are moving between the bigger and smaller reductive groups.
    Maximal compact groups with all these properties are said to be in \textbf{good position}.
