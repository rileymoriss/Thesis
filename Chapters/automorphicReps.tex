The references that will be most helpful are \cite[I.II]{borelAutomorphicFormsRepresentations1979}\cite{getzIntroductionAutomorphicRepresentations2024} for the general theory, we will follow the notation developed in \cite{moeglinSpectralDecompositionEisenstein1995} as it is somewhat standard. We will discuss some of the details of their representation theory because it is both subtle and needed later. In particular we want to draw attention to the failure of this theory to be ``categorical'' or ``algebraic'' in any nice sense. 
\section{Local Representation Theory}
\subsection{At the Archimedean Places}
What we have called automorphic forms are sometimes referred to as ``smooth K-finite automorphic forms'' \cite[2.2]{cogdellLecturesLfunctionsConverse}.
We want to consider the right regular action of \(G(\A)\) on \(\mathcal{A}(U(\A)M(k)\backslash G(\A))\), unfortunately this space is not stable under this action. The problem is in the \(K\)-finiteness in the infinite places. 

\begin{example}[\cite{cogdellLecturesLfunctionsConverse}, 2.3]
    If \(\phi\in \mathcal{A}(\Gamma \backslash G(F_\infty))\) is \(K_\infty\)-finite, then \(g.\phi\) is \(gK_\infty g\inv\)-finite. This is still a maximal compact subgroup, however in the infinite place it will a priori have only the identity in common with the original \(K\).

    For example csonsider \(\SL_2\) where the maximal compact is \(SO_2\), if we conjugate we get \(gSO_2g\inv\)
    \begin{align*}
        \begin{pmatrix}
            a & b\\
            c & d
        \end{pmatrix}
        \begin{pmatrix}
            \cos\theta & -\sin\theta \\
            \sin\theta & \cos\theta
        \end{pmatrix}
        \begin{pmatrix}
            d & -b\\
            -c & a
        \end{pmatrix} &=  \begin{pmatrix}
            (da - cb)\cos\theta + (db  + ca)\sin\theta & -\sin\theta( a^2+b^2)\\
            \sin\theta(d^2 + c^2) & ( ad  -bc)\cos\theta - (b d  + ac)\sin\theta
        \end{pmatrix}\\
        &=  \begin{pmatrix}
            \cos\theta + (db  + ca)\sin\theta & -\sin\theta( a^2+b^2)\\
            \sin\theta(d^2 + c^2) & \cos\theta - (b d  + ac)\sin\theta
        \end{pmatrix}.\\
    \end{align*}
    This intersects with the \(SO_2\) if and only if \(bd+ac = 0\) and \(a^2 + b^2 = d^2 + c^2 = 1\) i.e. when \(g\in SO_2\).

    Finally it is worth noting that this is not an issue at the finite places, namely if \(K = K_fK_\infty\) is our maximal compact subgroup of \(G(\A)\) then \(K_f\) is also open and hence \(K_f \cap gK_f g\inv\) is of finnite index in both \(K_f\) and \(gK_f g\inv\) and so their notions of \(K\)-finiteness will agree. 
\end{example}

For this reason we will need to talk about \((\mathfrak{g}, K)\)-modules. This is the solution of Harish-Chandra that we do not yet understand the full significance of. 

\begin{definition}[\cite{getzIntroductionAutomorphicRepresentations2024}, 4.4.6]
    Let \(G\) be a Lie group (for example the analytification of the real or complex points of our favourite reductive LAG) and \(K\) be a maximal compact subgroup of \(G\). Let \(\mathfrak{g}\) be the complexified Lie algebra of \(G\) and \(\mathfrak{k}\) the real Lie algebra of \(K\). 
    
    A \((\mathfrak{g}, K)\)-module is a complex vector space \(V\) with two representations 
    \[\tilde{\pi}: \mathfrak{g} \to End(V), \quad \pi: K\to \GL(V),\]
    satisfying the following axioms
    \begin{itemize}
        \item \(V\) decomposes into a countable direct sum of finite dimensional \(K\) representations.
        \item The representations should be compatible: For all \(X \in \mathfrak{k}\) and \(v\in V\)
        \[\tilde{\pi}(X)(v) = \frac{\mathrm{d}}{\mathrm{d}t}\pi(e^{tX})(v)|_{t=0} = \lim_{t\to 0}\frac{\pi(e^{tX})(v) - v}{t}.\]
        In particular the right hand limit exists
        \item And compatible with the adjoint representation: For \(k\in K\) and \(X\in \mathfrak{g}\) 
         \[\pi(k)\tilde{\pi}(X) \pi(k\inv )(v) = \tilde{\pi}(Ad(k)(X))(v).\]
    \end{itemize}
\end{definition}

\begin{remark}
    It is common to use the same symbol for both of these representations in the \((\mathfrak{g}, K)\)-module.
\end{remark}
Now one can check that the space of Archimedean automorphic forms is in fact an (admissible) \((\mathfrak{g}, K_\infty)\)-module, under the representations that we have already specified; namely the regular action given by \(K\) and the representation that we defined \ref{lie_algebra_action} when talking about the center of the enveloping algebra \cite[Thm. 6.2.6]{getzIntroductionAutomorphicRepresentations2024}.

\subsection{At the Non-Archimedean Places}
As we noted above the right regular representation on the space of automorphic forms is well defined for the finite places i.e. 
if \(\phi(x)\in \mathcal{A}(U(\A)M(F)\backslash G(\A))\) and \(g\in G(\A_f)\) then \(\phi(xg) \in \mathcal{A}(U(\A)M(F)\backslash G(\A)) \). Hence \(\mathcal{A}(U(\A)M(F)\backslash G(\A))\) is a \(G(\A_f)\)-module. In particular it is a module for \(G(F_\nu)\) for all \(\nu\) non-Archimedean.

\section{Automorphic Representations}
Recall that if \(A\) is an \(R\) module and we have the inclusions of \(R\) modules \(C \subseteq B \subseteq A\) then we call \(B/C\) a subquotient of \(A\). We now think of \(\mathcal{A}(U(\A)M(F)\backslash G(\A))\) as being a \(G(\A_f)\times (\mathfrak{g}, K)\) module. An automorphic representation is then an irreducible subquotient of this representation.\todo[inline]{by definition its irreducible??}
\begin{remark}
    This is the first place that I believe we must remark that we really need a set theoretic definition here. The quotient of these modules cannot be considered up to isomorphism but must be the classical set theoretic realisation of this object, defined as equivalence classes of elements of the module.
\end{remark}

\begin{remark}
    Automorphic representations can also be defined as representations of a commutative algebra \(\mathcal{H}\), the global Hecke algebra. This is the approach in \cite[I.II(4.6)]{borelAutomorphicFormsRepresentations1979}, and can be a helpful perspective to simplify definitions. 
\end{remark}

\begin{example}
    It is very hard to really write down something explicit. One thing that we can do is take a modular form \(f\). Then we know how to associate a concrete automorphic form \(\tilde{f}\). To this (or any fixed automorphic form) we have an automorphic representation given by acting on this vector:
    \[(G(\A_f)\times (\mathfrak{g}, K)).\tilde{f} \subseteq \mathcal{A}(U(\A)M(F)\backslash G(\A))\]
\end{example}

\subsection{Cuspidal Representations}
Recall that an automorphic form \(\phi\in \mathcal{A}(U(\A)M(F) \backslash G(\A))\) is called cuspidal  if all its constant terms vanish, see the next section for more detail \ref{cuspidal_form_definition}.
The space of such automorphic forms is denoted \(\mathcal{A}_0(U(\A)M(F) \backslash G(\A))\). An automorphic representation is called cuspidal if it is an irreducible subquotient of \(\mathcal{A}_0(U(\A)M(F) \backslash G(\A))\).

\begin{Remark}
    Here we really mean contained as sets, not isomorphic as / to a subrepresentation.
\end{Remark}

\subsection{Isotypic Components}\label{automorphic_isotypic_subspaces}
Following the convention of \cite[II.1]{moeglinSpectralDecompositionEisenstein1995} we make two cases:
Let \(\pi\) be an irreducible subquotient of the space \(\mathcal{A}(M(k) \backslash M(\A))\), that is \textit{not cuspidal}. Then we denote the \(\pi\) isotypic component of \(\mathcal{A}(M(k) \backslash M(\A))\) by \(\mathcal{A}(M(k) \backslash M(\A))_\pi\).

If we will also need the space 
\[\mathcal{A}(U(\A)M(F) \backslash G(\A))_\pi \defeq \{\phi \in \mathcal{A}(U(\A)M(F) \backslash G(\A)) : \forall k\in K, \; \phi_k\in\mathcal{A}(M(k) \backslash M(\A))_\pi \}\]
where \(\phi_k: M(\A) \to \C\) is given by \(\phi_k(x) = \phi(xk)\).

Now if \(\pi\) is cuspidal, we define \(\mathcal{A}(M(k) \backslash M(\A))_\pi\) to be the isotypic component of \(\pi\) in \(\mathcal{A}_0(M(k) \backslash M(\A))\) and similarly we have 
\[\mathcal{A}(U(\A)M(F) \backslash G(\A))_\pi \defeq \{\phi \in \mathcal{A}_0(U(\A)M(F) \backslash G(\A)) : \forall k\in K, \; \phi_k\in\mathcal{A}(M(k) \backslash M(\A))_{\textcolor{red}{\pi}} \}\]
\todo[inline]{AHHHHHHHHH BUT DO I TAKE THE CUSPIDAL PART OF INSIDE HERE OR NOT AHHHHHH DEFINE YOU TERMSSSSSS AHHHH }
\begin{Remark}
    We cannot simply take the isotypic components we need to take the isotypic components after explicitly restricting the spaces. This is a knock on effect from our last remark.
\end{Remark}

The point is that we want the isotypic component corresponding to a cuspidal representation to be cuspidal, however this just might not be the case. 
Yamana and Teck Gan have a counter example when one allows unitary groups over division algebras (non-commutative fields). It could be interesting to investigate this example more closely to see if the example can be pulled back to a unitary group over a field. In \cite{yamanaSiegelWeilFormulaQuaternionic2013} there is an automorphic representation of the quarternionic unitary group constructed, \(\Pi(V)\), that according to \cite[Rm. 7.12]{yamanaSiegelWeilFormulaQuaternionic2013} appears in both the cuspidal and residual spectrum (more details late \ref{spectral_decomposition}). By that Yamana means that up to isomorphism the representation can been seen in both residual and cuspidal spectrum. In particular if one were to take the component that is in the cuspidal spectrum and look at its isotypic component then the versions in the residual spectrum would also occur and hence by definition of residual spectrum would not be cuspidal.

If we restrict to the cases dealt with in for instance \cite{moeglinSpectralDecompositionEisenstein1995}, namely not dealing with quarternions, then we have been told that this is an open problem whether or not this restriction is superflous. 

\subsection{Tensor Products}
This is also true for things like taking tensor products, direct sums etc of representations. If one wants to say that the tensor of cuspidal representations is cuspidal then you really need to fix what you mean by the tensor, set theoretically. There is a usual construction of a tensor product that is indeed what is meant, but it is important because this thing is only unique \textit{up to isomorphism} which doesn't not preserve cuspidality.

Lets make this explicit in the case of tensor products because we will use it later. First fix \((\pi, V), (\pi', V')\) to automorphic representations of \(G(\A)\). Then, if we think of these representations as \(\mathcal{H}(G(\A))\) modules, the space on which the tensor \(\pi\tensor \pi'\) would act is usually constructed as a set as 
\[V\tensor V' \defeq V\times V' / \sim\]
where the relation \(\sim\) is generated, \(\forall m,n\in V, x,y\in V', r\in \mathcal{H}(G(\A)) \), by
\begin{itemize}
    \item \((m,x) + (m,y) = (m,x+y)\)
    \item \((m,x) + (n,x) = (m+n,x)\)
    \item \((r.m, x) = (m,r.x)\).
\end{itemize}
This has the natural action from \(\mathcal{H}(G(\A))\) still by acting on the components. But now we need to somehow think of this as a subquotient of \(\mathcal{A}(U(\A)M(F) \backslash G(\A))\). From the above definition it would seem that \(f\tensor g\) is a function on \(G(\A)^2\) a priori. 
\todo[inline]{yeah wtf}