The references that will be most helpful are \cite[I.II]{borelAutomorphicFormsRepresentations1979}\cite{getzIntroductionAutomorphicRepresentations2024} for the general theory, we will follow the notation developed in \cite{moeglinSpectralDecompositionEisenstein1995} as it is somewhat standard. We will discuss some of the details of their representation theory because it is both subtle and needed later. In particular we want to draw attention to what we think of as the "non-algebraic" nature of the representation theory.

\section{\((\mathfrak{g}, K)\)-Modules}

\section{Hecke Algebra}

\section{Automorphic Representations}

\subsection{Cuspidal Representations}
We will recal the definition of a cusp form \todo{reference the definition in the next chapter} in the next chapter.\todo{Chengjing example of isotypic subspaces}

\subsection{Tensor Products of Representations}

\subsubsection{Boxed Tensors}

\section{Eisenstein Series}
\cite{lapidPerspectivesEisensteinSeries2022}, \cite{arthurEisensteinSeriesTrace1979}

\section{Spectral Decomposition}
\subsection{Definition and Role}
This is another one of the tools that can be used to compartmentalise problems in automorphic forms, by dealing with representations that appear in different parts of the spectrum. 
\todo[inline]{give shahidis conjecture on plancherel measures some time. Make sure to talk about his proof based on a reasonable hypothesis. }

\subsection{The Decomposition of the Spectrum}


\subsection{Residual Representations of \(\GL_n\)}


\section{L-Functions}
\subsection{In General}
\subsection{Standard L-Functions for Classical Groups}
\subsection{L-Functions of Covering Groups}
