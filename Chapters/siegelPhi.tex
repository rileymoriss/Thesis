\label{ch:sigel-phi}
In the last chapter we saw some general computations around constant terms of automorphic forms. This chapter we continue with more computations, however we attempt to relate the constant term to the classical setting. This should be understood as a continuation of the example of modular forms as archimedean automorphic forms, as here we investigate the constant term in this setting in order to gain similar intuition. Specifically we will relate the constant term to the Fourier series constant term and the Siegel Phi operator. 

We thank Chengjing Zhang for showing us this example, and present it here in detail because we cannot find it in the literature.



\section{Constant Terms}
Let \(G\) be a classical group over a number field \(\Q\). For an archimedean automorphic form \(f: G(\R) \to \C\) its \textbf{constant term } along a parabolic of G, \(P=MN\subseteq G\), is defined to be \cite[8.6]{getzIntroductionAutomorphicRepresentations2024}
\[f(x)_P = \int_{N(\Z)\backslash N(\R)}f(xn) \mathrm{d}n.\]
To effectively compute constant terms we will routinely use the following two lemmas.

\begin{Theorem}\label{integrate_unitary_char}
	If \(G\) is a locally compact Hausdorff group with a left Haar measure \(\mu\) and if \(\chi\colon G\to \mathbf C^\times\) is a non-trivial character on \(G\), then
	\[ \intof{G}{\chi(g)}{\mu(g)} = 0. \]
\end{Theorem}
\proofbar{
	Pick an element \(h\) of \(G\) such that \(\chi(h)\neq 1\).
	The equation above then follows from
	\[ \intof{G}{\chi(g)}{\mu(g)} = \intof{G}{\chi(hg)}{\mu(g)} = \intof{G}{\chi(h)\chi(g)}{\mu(g)} = \chi(h) \intof{G}{\chi(g)}{\mu(g)}. \square\]
}
Integrating trivial characters gives the volume of the measure space which we typically normalise to be one.

\begin{Theorem}[Unfolding, \cite{garrettModernAnalysisAutomorphic2018} 5.2, \cite{follandCourseAbstractHarmonic2016a} Thm 2.49]
	Let \(H\leq G\) be a closed subgroup. If \(H\setminus G\) has a right G invariant measure (iff their modular functions agree on H) then the integral is unique up to scalar, namely for a given Haar measures dh on H and dg on G there is a unique invariant measure dq on \(H\setminus G\) such that for all \(f\in C_c^0(G)\)
	\[\int_{H\setminus G}\int_H f(hq)dhdq = \int_G f(g) dg.\]
\end{Theorem}
Note that this quotient may not be a group, because H is not required to be normal. The use of this lemma is called \textbf{unfolding} the integral.

\section{Siegel Modular Forms}
Following \cite{bruinier123ModularForms2008}. Recall the \textbf{Siegal upper half plane} of ``genus" \(g\in \N\) is
\begin{equation*}
	\begin{aligned}
		\mathcal{H}_g &\defeq \{\tau \in \mathrm{M}_{g\times g}(\C) : \tau \text{ is symmetric and has positive definite imaginary part}\} \\
		&\;\; \cong \Sp_{2g}(\R) / U(g).
	\end{aligned}
\end{equation*}
where the isomorphism is as analytic manifolds  and 
\[U(g) \defeq \left\{\begin{pmatrix}
	A & B\\
	-B & D
\end{pmatrix}\in \Sp_{2g}(\R) : AA^t + BB^t = 1\right\}.\]

For every \(\gamma= (A \; B; \; C\; D) \in \Sp_{2g}(\Z)\) and \(\tau \in \mathcal{H}_g\) we have the action
\[\gamma.\tau = (A\tau + B)(C\tau + D)\inv .\]

We say that a holomorphic function \(f: \mathcal{H}_g \to \C\) is a (classical)\textbf{ Siegel modular form} of weight \(k\) if 
\[f(\gamma.\tau) = \det(C\tau + D)^kf(\tau),\]
with the extra condition that if \(g = 1\) it must be holomorphic at \(\infty\). Because \(\Sp_2 = \SL_2\) this is a strict generalisation of an (elliptic) modular form.

The space of Siegel modular forms of weight \(k\) and genus g is denoted \(\mathcal{M}_k(\Sp_{2g}(\Z))\). There is a useful operator know as the \textbf{Siegel Phi Operator} which allows you to lift known modular forms from lower genus to higher genus \cite[5]{bruinier123ModularForms2008}
\[\mathcal{M}_k(\Sp_{2g}(\Z)) \xrightarrow{\Phi} \mathcal{M}_{k}(\Sp_{2(g-1)}(\Z)),\]
defined by the limit for \(\tau\in \mathcal{H}_{g-1}\)
\[\Phi(f)(\tau) \defeq \lim_{t\to \infty} f\begin{pmatrix}
	\tau & \\
	& it 
\end{pmatrix}.\]
In this context a cusp form is defined to be a Siegel modular form in the kernel of the Siegel \(\Phi\) operator and so it is natural to wonder if there is a constant term that is being taken here. 

\subsection{Automorphising}
Just as in the case of modular forms, given a Siegel modular form \(f\in \mathcal{M}_k(\Sp_{2g}(\Z))\) we can associate an automorphic form
\[\tilde{f} : \Sp_{2g}(\R)\to\C, \qquad \begin{pmatrix} a & b\\ c & d\end{pmatrix}\mapsto \det(ci+d)^{-k} f\Bigl((ai+b)(ci+d)\inv\Bigr), \]
where \(a,b,c,d\) are \(g\times g\) matrices such that \(\bigl(\begin{smallmatrix} a & b\\ c & d\end{smallmatrix}\bigr)\in\Sp_{2g}(\R)\). Fix the Borel of upper triangular matrices. Now for \(1\leq r\leq g-1\) an integer we have the standard maximal parabolic of \(\Sp_{2g}\), \(P_r = M_rN_r\) such that 
\[M_r \cong \GL_r\times \Sp_{2(g-r)}\]

\begin{Theorem}[Zhang]
	If \(f\) is a classical Siegel modular form of weight \(k\) and degree \(g\), then
	\begin{equation} 
		\tilde f_{P_r}(u\gamma) = \det u^k\cdot (\Phi^{r} f)^\sim(\gamma)
	\end{equation}
	for every element \(\gamma\) of \(\Sp_{2(g-r)}(\R)\) and every element \(u\) of \(\GL_{r}(\R)\).
	
	In particular 
	\[\tilde{f}_{P_{g-1}}\begin{pmatrix} a & 0 & b & 0\\ 0 & 1 & 0 & 0\\ c & 0 & d & 0\\ 0 & 0 & 0 & 1\end{pmatrix} = (\Phi f)^\sim\begin{pmatrix}
		a & b\\
		c& d
	\end{pmatrix}\]
\end{Theorem}

This shows that perhaps the correct generalisation of the Siegel Phi function is just the constant term that we all know and love. We could also attempt to expand this to Siegel modular forms that are vector valued or not of full level. 

The only other work on generalising the Siegel \(\Phi\) operator that we could find appears in \cite{grenierAnalogueSiegelsFOperator1992}. 
Grenier formulates the \(\Phi\) operator in the language of symmetric spaces \cite[Ch. 2]{terrasHarmonicAnalysisSymmetric2016} and then shows that the analogous definition in the case of ``automorphic forms'' in the sense of the symmetric space \(\mathscr{P}_n/\GL_n(\Z)\) of symmetric positive definite real matrices \cite[1.5.1]{terrasHarmonicAnalysisSymmetric2016} behaves in the same way. Namely his \cite[Thm. 2]{grenierAnalogueSiegelsFOperator1992} shows that it sends an automorphic form for \(
GL_n(\Z)\) to an automorphic form for \(\GL_{n-1}(\Z)\). The point is that the \(\Phi\) operator can be defined in the generality of symmetric spaces and Grenier shows that at least in one other case it still preserves the relevant notion of automorphic form. This suggests two things that would be interesting to investigate; using the classification of symmetric spaces is it possible to give a uniform definition of the \(\Phi\) operator following Grenier and does this definition agree with the constant term in the way that the Siegel \(\Phi\) operator does. With my limited knowledge of symmetric spaces this seems to be very tractable.

\subsection{Modular Form Case}
The base case is very instructive, it deals with modular forms. So consider \(f\) a (elliptic) modular form of full level and weight k, which has a Fourier expansion given by 
\[f(z) = \sum_{n\geq 0} a_ne^{2\pi i nz }\]
Then one can verify that
\[\tilde f \begin{pmatrix}
	a & b\\
	c & d
\end{pmatrix} = (ci+d)^{-k} f\Bigl(\frac{ai+b}{ci+d}\Bigr)\]
is an automorphic form on \(\Sp_2\). The only non-trivial parabolic P is the one of upper triangular matricies, with Levi and unipotant given respectively 
\[M = \begin{pmatrix} m & 0\\ 0 & m^{-1}\end{pmatrix}\cong \GL_1 , \;\;\; N = \begin{pmatrix} 1 & b\\ 0 & 1\end{pmatrix} \cong \mathbb{G}_a\]
along which we can now compute the constant term 
\begin{equation*}
	\begin{aligned}
		\tilde f_P(m)
		& = \int_{N(\Z)\backslash N(\R)}\tilde f (mb) \mathrm{d}b\\
		& =  \int_{\Z\backslash\R}\tilde f \begin{pmatrix} m & mb\\ 0 & m^{-1}\end{pmatrix}\mathrm{d}b\\
		& = \int_{\Z\backslash\R} m^k f(m^2i+m^2b) \mathrm{d}b\\
		& = m^k a_0 \\
	\end{aligned}
\end{equation*}
We have chosen normalisation to remove the usual factor of \(1/2\pi\) in the constant term of the Fourier series. Moreover we see that
\[\Phi(f)= \lim_{t\to \infty} f(it) =\lim_{t\to \infty} \sum_{n\geq 0} a_ne^{-2\pi nt }  =  a_0\]

\subsection{Simplifying the Constant Term}
As we saw in \ref{maximal_parabolic} for \(1\leq r\leq g-1\) an integer we have the standard maximal parabolic of \(\Sp_{2g}\), \(P_r = M_rN_r\) such that 
\[M_r \cong \GL_r\times \Sp_{2(g-r)}\]
which can be given the explicit matrix representations 
\[m(\gamma, A) \defeq \begin{pmatrix}
	A &&& \\
	&a&&b \\
	&&(A^t)\inv& \\
	&c&&d \\
\end{pmatrix}, \;\;\; A\in \GL_r(F), \; \gamma = \begin{pmatrix}
	a & b\\
	c & d \\
\end{pmatrix} \in \Sp_{2(g-r)}(F) \]

and unipotent 
\[ n(s;h,k) \defeq \begin{pmatrix} 1 & 0 & 0 & h\\ -k^t & 1 & h^t & s+h^t k\\ 0 & 0 & 1 & k\\ 0 & 0 & 0 & 1 \end{pmatrix}, \;\;\; h, k\in\mathrm{Mat}_{(g-r)\times r}(\R),\; s\in\mathrm{Sym}_{r}(\R)\]
We have the following short exact sequence 
\[ 1\to \mathrm{Sym}_{r}(\R)\to N_r(\R)\to \mathrm{Mat}_{(g-r)\times r}(\R)\times\mathrm{Mat}_{(g-r)\times r}(\R) \to 1. \]
which we will use to unfold our integral below, for compactness we define \(H_r \defeq \mathrm{Mat}_{(g-r)\times r}\). We will now denote \([G] \defeq G(\Z)\backslash G(\R)\) and compute the constant term
\begin{align}
	\tilde f_{P_r}\bigl(m(\gamma, A)\bigr)
	& = \intof{[N_r]}{\tilde f\bigl(n m(\gamma, A)\bigr)}{n} \notag\\
	& = \intof{[H_r\times H_r]}{\intof{[\mathrm{Sym}_{g-r}]}{\tilde f\bigl(n(s; h, k) m(\gamma, A)\bigr)}{s}}{(h,k)} \notag\\
	& = \intof{[H_r]}{\intof{[H_r]}{\intof{[\mathrm{Sym}_{g-r}]}{\tilde f\bigl(n(s; h, k) m(\gamma, A)\bigr)}{s}}{h}}{k}.
\end{align} 

Now we focus on simplifying the integrand. We want an explicit form of the matrix so we can relate it back to the value of the un-lifted Siegel modular form \(f\); simply multiply the matrices gives, where (all rings are commutative) \(A^{-t} \defeq (A^t)\inv\)
\[
n(s; h, k) m(\gamma, A) =
\begin{pmatrix}
	a & 0 & b & h A^{-t}\\
	-k^t a + h^t c & A & -k^t b + h^t d & s A^{-t} + h^t k A^{-t}\\
	c & 0 & d & k A^{-t}\\
	0 & 0 & 0 & A^{-t}
\end{pmatrix}.
\]
because \(a,b,c,d \in \mathrm{Mat}_{(g-r)\times (g-r)}, A \in \mathrm{Mat}_{r\times r}\) we see that the \(g\times g\) blocks that we now need to take the determinant of are the \(4\times 4\) corners of this picture, hence the matrices below should all be in \(\mathcal{H}_g\subseteq \mathrm{Mat}_{g\times g}\)

\begin{align*}
	&\tilde{f}(n(s; h, k) m(\gamma, A) ) \\
	 &= \det\left(\begin{pmatrix}
		c & 0 \\
		0 & 0
	\end{pmatrix}i+ \begin{pmatrix}
		d & kA^{-t} \\
		0 & A^{-t}
	\end{pmatrix} \right)^{-k} \cdot \\
	&f\left(         \left(\begin{pmatrix}
		a & 0\\
		-k^ta + h^tc & A
	\end{pmatrix}i+\begin{pmatrix}
		b & hA^{-t} \\
		-k^tb + h^td & sA^{-t} + h^tkA^{-t}
	\end{pmatrix}\right)   \left(\begin{pmatrix}
		c & 0 \\
		0 & 0
	\end{pmatrix}i+ \begin{pmatrix}
		d & kA^{-t} \\
		0 & A^{-t}
	\end{pmatrix} \right)\inv      \right) \\
	&= \det\left( \begin{pmatrix}
		ic + d & kA^{-t} \\
		0 & A^{-t}
	\end{pmatrix}\right)^{-k} \cdot \\
	&f\left( \begin{pmatrix}
		ia + b &  hA^{-t}\\
		-k^t(ia +b) + h^t(d + ic) & iA + sA^{-t} + h^tkA^{-t}
	\end{pmatrix}  \begin{pmatrix}
		ic + d & kA^{-t} \\
		0 & A^{-t}
	\end{pmatrix}\inv      \right) \\
	&=\left(\frac{\det(ic + d)}{\det(A)}\right)^{-k} \cdot \\ &f\left( \begin{pmatrix}
		ia + b &  hA^{-t}\\
		-k^t(ia +b) + h^t(d + ic) & iA + sA^{-t} + h^tkA^{-t}
	\end{pmatrix}   \begin{pmatrix}(ci+d)^{-1} & -(ci+d)^{-1} k\\ 0 & A^t \end{pmatrix} \right) \\
	&= \left(\frac{\det(A)}{\det(ic + d)}\right)^{k} f\begin{pmatrix} \tau & -\tau k + h\\ -k^t \tau + h^t & k^t \tau k + A A^t i + s \end{pmatrix}, \;\;\; \tau \defeq (ai+b)(ci + d)\inv \\
	%&= \left(\frac{\det(A)}{\det(ic + d)}\right)^{k} f\bigl(n(s; h, k) m(\gamma,A )(i 1_g)\bigr) I think g here should be 2g but then the multiplication seems wrong IDK what he meant by this..
\end{align*} 
So we have shown that 
\begin{align*}
	&\tilde f_{P_r}\bigl(m(\gamma, A)\bigr) \\
	 &= \intof{[H_r]}{\intof{[H_r]}{\intof{[\mathrm{Sym}_{g-r}]}{  \left(\frac{\det(A)}{\det(ic + d)}\right)^{k} f\begin{pmatrix} \tau & -\tau k + h\\ -k^t \tau + h^t & k^t \tau k + A A^t i + s \end{pmatrix}   }{s}}{h}}{k}\\
	&= \left(\frac{\det(A)}{\det(ic + d)}\right)^{k} \intof{[H_r]}{\intof{[H_r]}{\intof{[\mathrm{Sym}_{g-r}]}{  f\begin{pmatrix} \tau & -\tau k + h\\ -k^t \tau + h^t & k^t \tau k + A A^t i + s \end{pmatrix}   }{s}}{h}}{k}\\
\end{align*}

Again lets focus on this integrand \(f\begin{pmatrix} \tau & -\tau k + h\\ -k^t \tau + h^t & k^t \tau k + A A^t i + s \end{pmatrix}\) and compute its Fourier expansion, see \cite[3.4]{bruinier123ModularForms2008}. Recall that a symmetric matrix \(n\in \GL_g(\Q)\) is called half integral if \(2n\) is integral with even diagonal entries, then a Siegel modular form has a Fourier expansion of the form
\[f(z) = \sum_{n \text{ half integral}}a(n) e^{2\pi i \mathrm{Tr}(nz)} \]
First the space of half integral \(g\times g\) matrices, \(\mathrm{HI}_g\), decomposes as a direct sum via the (additive) group isomorphism
\[ \mathrm{HI}_{g-r} \oplus \tfrac{ 1}{ 2} \mathrm{Mat}_{ r\times (g-r)}(\Z) \oplus \mathrm{HI}_{r}\to\mathrm{HI}_g, \qquad (n, m, l)\mapsto \begin{pmatrix} n & m\\ m^t & l \end{pmatrix}, \]
thus unfolding the (discrete) integral we get 
\begin{align*}
	f\begin{pmatrix} \tau & -\tau k + h\\ -k^t \tau + h^t & k^t \tau k + A A^t i + s \end{pmatrix} &=   \sum_{n\in\mathrm{HI}_{g-r}} \sum_{m\in\frac{1}{2} \mathrm{Mat}_{ r\times (g-r)}(\Z)} \sum_{l\in\mathrm{HI}_{r}} a\begin{pmatrix} n & m\\ m^t & l \end{pmatrix} \\
	&\exp \left(2\pi i \mathrm{Tr} \begin{pmatrix} n & m\\ m^t & l \end{pmatrix}\begin{pmatrix} \tau & -\tau k + h\\ -k^t \tau + h^t & k^t \tau k + A A^t i + s \end{pmatrix} \right)  \\
\end{align*}
because all the block sizes are compatible we can ``block multiply'' the inner matrices and because we are taking the trace we can forget about off diagonal entries
\begin{align*}
	&\begin{pmatrix} n & m\\ m^t & l \end{pmatrix}\begin{pmatrix} \tau & -\tau k + h\\ -k^t \tau + h^t & k^t \tau k + A A^t i + s \end{pmatrix}\\ &= 
	\begin{pmatrix} n\tau + m(-k^t \tau + h^t ) & \ast\\ \ast & m^t(-\tau k + h) + l( k^t \tau k + A A^t i + s) \end{pmatrix}
\end{align*}
putting this into our Fourier expansion
\begin{align*}
	&f\begin{pmatrix} \tau & -\tau k + h\\ -k^t \tau + h^t & k^t \tau k + A A^t i + s \end{pmatrix} \\
	&= \sum_{n} \sum_{m} \sum_{l} a\begin{pmatrix} n & m\\ m^t & l \end{pmatrix} \exp \Big(2\pi i \big( &\mathrm{Tr} (n\tau) +  \mathrm{Tr} (m(-k^t \tau + h^t )) +  \mathrm{Tr} (m^t(-\tau k + h)) \\ & &  +  \mathrm{Tr} (l( k^t \tau k + A A^t i + s))\big)\Big)  \\
\end{align*}

If we denote \(T_l \defeq \mathrm{Tr} (l( k^t \tau k + A A^t i + s))\),
\[T_{m,h}\defeq \mathrm{Tr}(mh^t + m^th), \;\; T_{m,k} \defeq \mathrm{Tr}(-mk^t\tau - m^t\tau k) , \]
and \(T_m \defeq T_{m,h} + T_{m,k}\) then we can substitute this back into our constant term
\resizebox{\linewidth}{!}{
	\begin{minipage}{\linewidth}
		\begin{align*}
			&\tilde f_{P_r}\bigl(m(\gamma, A)\bigr)\\
			&= \left(\frac{\det(A)}{\det(ic + d)}\right)^{k} \intof{[H_r]}{\intof{[H_r]}{\intof{[\mathrm{Sym}_{g-r}]}{ \sum_{n} \sum_{m} \sum_{l} a\begin{pmatrix} n & m\\ m^t & l \end{pmatrix}\exp \left(2\pi i (\mathrm{Tr} (n\tau) +  T_m + T_l) \right)
					}{s}}{h}}{k}\\
			&= \left(\frac{\det(A)}{\det(ic + d)}\right)^{k} \sum_{n} \sum_{m} \sum_{l} a\begin{pmatrix} n & m\\ m^t & l \end{pmatrix}e^{2\pi i \mathrm{Tr} (n\tau)}
			\intof{[H_r]}{\intof{[H_r]}{\intof{[\mathrm{Sym}_{g-r}]}{  e^{2\pi i(T_m + T_l)}  }{s}}{h}}{k}\\
			&= \left(\frac{\det(A)}{\det(ic + d)}\right)^{k} \sum_{n} \sum_{m} \sum_{l} a\begin{pmatrix} n & m\\ m^t & l \end{pmatrix}e^{2\pi i \mathrm{Tr} (n\tau)}
			\intof{[H_r]}{e^{2\pi iT_{m,k}}\intof{[H_r]}{e^{2\pi i T_{m,h}} \intof{[\mathrm{Sym}_{g-r}]}{  e^{2\pi iT_l}  }{s}}{h}}{k}\\
		\end{align*}
	\end{minipage}
}
\begin{remark}
	We have a priori uniform convergence on compact subsets of these integrals, however we leave it for future work to check the details of interchanging these sums and integrals.
\end{remark}

Now we use lemma \ref{integrate_unitary_char} and the fact that 
\[s\mapsto  e^{2\pi iT_l} \]
is a non-trivial unitary character of \(\mathrm{Sym}_{g-r}\) whenever \(l\neq 0\) to get that 
\[\intof{[\mathrm{Sym}_{g-r}]}{  e^{2\pi iT_l}  }{s} = \begin{cases}
	1, & l=0 \\
	0, & l\neq 0
\end{cases}\]
we repeat this trick with the second integral, which enforces that \(m = 0\) and end up with 
\begin{align*}
	\tilde f_{P_r}\bigl(m(\gamma, A)\bigr)
	&=\left(\frac{\det(A)}{\det(ic + d)}\right)^{k} \sum_{n\in\mathrm{HI}_{g-r}} a\begin{pmatrix} n & 0\\ 0 & 0 \end{pmatrix}e^{2\pi i \mathrm{Tr} (n\tau)}\\
\end{align*}
but by \cite[3.5]{bruinier123ModularForms2008} we know that the Fourier expansion of the Siegel Phi operator is 
\[(\Phi^{r} f)(\tau) = \sum_{n\in\mathrm{HI}_{g-r}} a\begin{pmatrix} n & 0\\ 0 & 0 \end{pmatrix} e^{2\pi i\mathrm{Tr}(n \tau)}.\]
hence 
\begin{align*}
	\tilde f_{P_r}\bigl(m(\gamma, A)\bigr)
	&=\left(\frac{\det(A)}{\det(ic + d)}\right)^{k} \Phi^r(f)(\tau)\\
	&= \det(A)^k (\Phi^r(f))^{\sim}(\gamma)
\end{align*}
which concludes the proof.
\begin{FlushRight}
	\(\square\)
\end{FlushRight}