\section{Eisenstein Series}
As usual we fix a connected reductive group G defined over a number field F, with a Borel B, a standard parabolic with Levi decomposition \(P = MU\). 

Following the setup in \cite[I.1.4]{moeglinSpectralDecompositionEisenstein1995} we consider a \textbf{character} \(\chi\in \mathrm{Rat}(M) \defeq \Hom_{\mathrm{LAG}}(M, \mathbb{G}_m)\), thinking of it below as a natural transformation, and then define 
\[|\chi|: M(\A)\to \C , \;\;\; (m_\nu)\mapsto \prod_\nu|\chi(F_\nu)(m_\nu)|_\nu.\]
The intersection of the kernels of these characters is 
\[M^1 \defeq \bigcap_{\chi\in \mathrm{Rat}(M)}\ker |\chi|.\]
Thus we can define
\[X_M \defeq \Hom_{\textrm{TopGroup}}(M(\A)/M^1, \C^*) .\]
i.e. the collection of characters of \(M(\A)\) that are trivial on \(M^1\).
\begin{remark}
    To make it seem less mysterious this group has some importance in the more general theory, in particular it is one of the pieces in the ``Langlands decomposition'' (\ref{eq:iwasawa_decomposition}) of the Archimedean points of a parabolic and it has the property that \(M(\Q)\backslash M(\A)^1\) has finite measure \cite[4.9]{getzIntroductionAutomorphicRepresentations2024}.
\end{remark}
The set of \textbf{complex characters} of \(M\),
\[\mathfrak{a}_M^* \defeq \mathrm{Rat}(M)\tensor_\Z \C,\]
is isomorphic as \C vector spaces to \(X_M\). If \(Z_{G(\A)}\) is the center of \(G(\A)\) then we also have the space 
\[X_M^G \defeq \Hom_{\textrm{TopGroup}}((M(\A)/M^1)/Z_G, \C^*)\]
which is characters of \(M(\A)/M^1\) which are also trivial on the center of \(G\).

\begin{example}\label{ex:characters}
    For the maximal parabolic \(P_r\) with Levi \(M_r\) of \(\Sp_{2n}\) we have that \( X_{M_r}^{\Sp_{2n}}\) is at most a one dimensional \C vector space. 

    First of all we have that \cite[I.1.4]{moeglinSpectralDecompositionEisenstein1995}
         \[ X_{M_r}^{\Sp_{2n}} \subseteq X_{M_r} \cong \mathfrak{a}_{M_r}^*\defeq Rat(M_r) \tensor_\Z \C.\]
        Thus it is clearly sufficient to bound the dimension of \(\mathfrak{a}_{M_r}^*\) as a \C vector space, moreover this dimension agrees with the dimension of \(Rat(M_r)\) as a free \Z module. 

        Thus we compute \(\dim_\Z(Rat(M_r))\):
        \begin{equation*}
            \begin{aligned}
                Rat(M_r) &= Rat(\GL_r \times \Sp_{2m}) \\
                         &= \Hom(\GL_r \times \Sp_{2m}, \mathbb{G}_m) \\
                         (2)&\cong \Hom(\mathrm{Ab}(\GL_r \times \Sp_{2m}), \mathbb{G}_m) \\
                         (1)&\cong \Hom(\mathrm{Ab}(\GL_r) \times \mathrm{Ab}(\Sp_{2m}), \mathbb{G}_m) \\
                         (3)&\cong \Hom(\mathbb{G}_m \times 1, \mathbb{G}_m) \\
                         &\cong \Z.
            \end{aligned}
        \end{equation*}
        In (2) we have used the universal property of the abelianization \(\mathrm{Ab}(G) = \mathcal{D}(G) \setminus G = [G, G] \setminus G \) because \(\mathbb{G}_m\) is abelian. (1) is that the abelianization commutes with direct products. (3) is because \(\Sp\) is a perfect group.
        %https://groupprops.subwiki.org/wiki/Symplectic_group_is_perfect
        %https://mathoverflow.net/questions/35713/abelianization-of-a-semidirect-product
\end{example}

\begin{remark}
    This generalises to the metaplectic covers immediately as \( X_{M_r}^{\Mp_{2n}(\A)} \subseteq X_{M_r}\).
\end{remark}
There is the natural map \(m_P: G(\A) \to M^1 \backslash M(\A)\) sending \(umk \mapsto M^1 m\), where \(g = umk\) using the Langlands-Iwasawa decomposition \ref{eq:iwasawa_decomposition}.

Now if we take the collection of irreducible automorphic representations of \(M\),
 \[\hat{\mathcal{A}} \defeq \{(\pi, V) : \pi \text{ is an irreducible automorphic representation of }M\}\]
then we can think of \(X_M^G\) as being one dimensional automorphic representations (with some extra invariance) and so there is a natural action on \(\hat{\mathcal{A}}\) given by tensoring, i.e. if \(\lambda\in X_M^G\) and \((\pi, V)\in \hat{\mathcal{A}}\) then 
\[\lambda.\pi \defeq \lambda\tensor \pi\]
Then \(\hat{\mathcal{A}}\) decomposes as a disjoint union of its orbits. Consider the orbit \(\mathfrak{P}\) of a cuspidal representation \(\pi_0\), then by definition \(X_M^G\) acts transitively but it also acts freely \cite[II.1]{moeglinSpectralDecompositionEisenstein1995}. Thus \(\mathfrak{P}\) is in bijection with \(X_M^G\). Through this bijection we transmit the complex structure on \(\mathfrak{a}_M^*\) to \(X_M\) then to the quotient \(X_M^G\) and finally to \(\mathfrak{P}\).

Now we will define an Eisenstein series: Let \(\mathfrak{P}\) be as above, the orbit of a cuspidal automorphic representation endowed with a complex structure. Let \(\pi\in \mathfrak{P}\) and \(\phi_\pi \in \mathcal{A}(U(\A)M(k)\backslash G(\A))_\pi\), then \(\lambda\in X_M^G\) acts on \(\phi_\pi\) by 
\[\lambda.\phi_\pi(g) = (\lambda \comp m_P)(g) \phi_\pi(g).\]
which is then an element of \(\mathcal{A}(U(\A)M(k)\backslash G(\A))_{\pi\tensor \lambda}\). Finally we have the \textbf{Eisenstein series} which is defined by the following sum
\[E(\phi_\pi, \lambda, g) = \sum_{\gamma \in P(k)\backslash G(k)} \lambda.\phi_\pi(\gamma g)\]
whenever it is convergent. The first thing to note is that for a fixed \(\phi\) there is an open set in \(X_M^G\) and a compact subset of \(G(k)\backslash G(\A)\) such that the Eisenstein series converges (normally) \cite[II.1.5]{moeglinSpectralDecompositionEisenstein1995}.

If \(P = MU, P' = M'U'\) are two standard parabolics of \(G\) that are conjugate, i.e. such that for \(w\in G(k)\) we have \(wMw\inv = M'\)
Then \(w\) maps \(\mathfrak{P}\) to \(w\mathfrak{P}\), an orbit of an irreducible representations of \(M\) to an orbit of irreducible representations of \(M'\).

Then the Eisenstein series is closely related (through its constant terms as discussed in \ref{constant_conjugate_levi}) to the operator
\[M(w, \pi)(\phi_\pi)(g) = \int_{(U'(k)\cap wU(k)w\inv )\backslash U'(\A)} \phi_\pi(w\inv ug) du\]
where \(\pi\in \mathfrak{P}\), \(g\in G(\A)\) and \(\phi_\pi \in \mathcal{A}(U(\A)M(k)\backslash G(\A))_\pi\).

The key properties of both the Eisenstein series and this operator can be found in \cite[IV.1.8, IV.1.9, IV.1.10, IV.1.11]{moeglinSpectralDecompositionEisenstein1995}. Most importantly as a function of \(\mathfrak{P}\) it can be shown that (in the sense of Frechet spaces) they both have a meromorphic continuation to all of \(\mathfrak{P}\). This was also given a second ``soft proof'' more recently in \cite{bernsteinMeromorphicContinuationEisenstein2022}, with the spectral decomposition that follows from it also being worked out in \cite{delormeSpectralTheoremLanglands2021}. Moreover for the Eisenstein series at a point in \(p\in \mathfrak{P}\) at which it is holomorphic then \(E(\phi,p, g)\) is an automorphic form. 

We are not really in a position to convey the true importance of these objects in the theory of automorphic forms, however we will make some comments. First some surveys are \cite{lapidPerspectivesEisensteinSeries2022}, \cite{arthurEisensteinSeriesTrace1979}, \cite{kimEISENSTEINSERIESTHEIR}, \cite{jiangResiduesEisensteinSeries2008a}. To see the relation to the classical Eisenstein series there is \cite{garrettTransitionEisensteinSeries2016}. One thing that Eisenstein series do, as in the theory of modular forms, is that they furnish us with quasi-concrete examples. A we mentioned above \cite[IV.1.9.(b).i]{moeglinSpectralDecompositionEisenstein1995} tells us that at the holomorphic points the Eisenstein series takes an automorphic form and returns an automorphic form, thus we can use them to multiply our examples. Another reason that these functions are important is through their normalisation and constant terms, in which products of L functions appear, we discuss this more in section \todo[inline]{ref later}. This has been a fruitful method for proving theorems about L-functions as in \cite{shahidiEisensteinSeriesAutomorphic2010}\cite{pollackRANKINSELBERGMETHODUSER}\cite{arthurEisensteinSeriesTrace1979}.

\section{Spectral Decomposition}\label{spectral_decomposition}
This is a short explanation of some terms that frequently appear as well as some motivation for the later results. The results contained here-in are proved using the Eisenstein series as an essential component. 

\subsection{The Decomposition of the Spectrum In General}\label{direct_integral}
For this section let \(H\) be a locally compact topological group.
It is a classical theorem that for representations of finite groups over an algebraically closed field the regular representation decomposes into a direct sum, where ever irreducible representation appears \cite[Ch. 2.4 Cor. 2 ]{LinearRepresentationsFinite}. This still holds for compact topological groups, when one considers continuous unitary representations \cite[5.1]{follandCourseAbstractHarmonic2016}.
\begin{remark}
    This is a strict generalisation of the finite groups case, when we give the finite group the discrete topology then all its linear representations are continuous and unitary.
\end{remark}
There is one final more general incarnation of this line of investigation in the Plancherel theorem. A group is \textbf{type I} if for every (continuous unitary) representation \(\pi\) such that the center of \(\Hom_\mathrm{Rep}(\pi, \pi)\) is trivial we have a decomposition as a  direct sum of irreducible representations. 

\begin{example}
    Consider \(G(\A)\) the adelic points of a connected reductive LAG. This is a type one group. 
\end{example}

\begin{example}
    Consider \(G(\A)\) the adelic points of a connected reductive LAG. This is a seecond countable group. 
\end{example}

\begin{example}
    Consider \(G(\A)\) the adelic points of a connected reductive LAG. This is a unimodule group. 
\end{example}
\todo[inline]{fill}

The idea of a direct integral is review in \ref{app:direct_int} to get a quick idea consider the following example:
\begin{example}[Direct Sums]
    Let \(I\) be a countable set with the discrete sigma algebra and counting measure \(\mu\). Let \((\mathcal{H}_i)_{i\in I}\) be a collection of Hilbert spaces then
    \[\bigoplus_{i\in I} \mathcal{H}_i = \left\{ (h_i)_{i\in I}\in \prod_{i\in I} \mathcal{H}_i : \int_I \norm{h_i}_i^2 d\mu <\infty \right\}.\]
    I.e. the Hilbert space direct sum is by definition square summable sequences, but sums are just discrete integrals.
\end{example}

\begin{Theorem}[Plancherel, \cite{follandCourseAbstractHarmonic2016}, 7.44]
    The regular represntation of a type I, second countable and unimodular topological group is a direct integral of the irreducible unitary representations. 
\end{Theorem}
\begin{remark}
    The Plancherel theorem says much more in fact. Like the Peter-Weyl theorem for compact groups it doesnt just give you that some direct integral decomposition exists, it contains many more details about the topology and measure on the set of unitary irreducible representations, and which representations are associated to them in the direct integral. We are being breif as this is motivational.
\end{remark}

Thus what one wants to do is find a decomposition of the regular representation \(G(\A) \curvearrowright \mathrm{L}^2(G(\A))\).
We call such decompositions ``spectral'', alluding to the spectral theorem which provides such a decomposition in terms of the eigenvector of certain operators. Moreover these decompositions are largely proved in terms of the more general spectral theorems. So once accomplished this is another one of the tools that can be used to compartmentalise problems in automorphic forms, by dealing with representations that appear in different parts of the spectrum. 

\subsection{Langlands Decomposition of the Spectrum }
We have the Plancherel theorem but Langlands also provides a fine analysis of the spectrum using automorphic forms. The key result in this theory is the following decomposition,
\begin{Theorem}[\cite{arthurEisensteinSeriesTrace1979}, MAIN THEOREM (b)]
    There is an orthogonal decomposition of the representation of \(G(\A)\) on \(L^2(G(\Q) \backslash G(\A))\) into 
    \[L^2(G(\Q) \backslash G(\A)) = \bigoplus_{\mathscr{P}}L^2_\mathscr{P}(G(\Q) \backslash G(\A)),\]
    where \(\mathscr{P}\) runs over certain ``associate classes'' of parabolics of \(G\) and the summands are the direct integrals of spaces of \(L^2\) automorphic forms.
\end{Theorem}
These direct integrals are in fact constructed out of subspaces generated by Eisenstein series. 

The spectrum of \(L^2(G(\A))\) refers to such a decomposition. In particular we have some important ``pieces'' to such a decomposition. The piece that decomposes into a direct sum of irreducible is called the \textbf{discrete spectrum}. The compliment of the discrete spectrum is called the \textbf{continuous spectrum}. One can define cuspidal \(L^2\) functions in the exact same way as cuspidal automorphic forms \ref{cuspidal_form_definition} and then it has been shown that the \textbf{cuspidal spectrum}, the subspace of \(L^2\) consisting of cusp forms, decomposes as a direct sum \cite[9]{getzIntroductionAutomorphicRepresentations2024}. Thus the cuspidal spectrum is contained in the discrete spectrum in this case. The \textbf{residual spectrum} is defined to be the compliment of the cuspidal spectrum in the discrete spectrum.

\subsection{Residual Spectrum}\label{residual_spec}
Moeglin and Waldspurger also acheived a more fine analysis of the spectrum of \(\GL_n\) in terms of residues of Eisenstein series. 
First consider the group \(\GL_n\). We then let \(n = ab\) for positive integers \(a,b\). If \(\tau\) is an irreducible, cuspidal automorphic rep of \(\GL_a\) then Moeglin and Waldspurger construct a representation of \(\GL_{ab} = \GL_n\) called the ``Speh representation'' and denote it 
\[\Delta(\tau, b).\]
They go on to prove that as \(\tau\) and \(b\) vary these representations span the residual spectrum of \(L^2(\GL_n(F) \backslash \GL_n(\A))\) \cite[Thm. 1.1]{jiangPolesCertainResidual2013}.

This representation is formed by taking iterated residues of Eisenstein series in the sense of \cite[V]{moeglinSpectralDecompositionEisenstein1995}. For a nice survey of problems in this area, of residues of Eisenstein series, there is \cite{jiangResiduesEisensteinSeries2008a}.



\section{Automorphic L-Functions}
We don't intent to define in great detail automorphic L-functions, as there are many other better sources to learn from \cite[Part 2.III.2]{borelAutomorphicFormsRepresentations1979}\cite{shahidiEisensteinSeriesAutomorphic2010}\cite{cogdellLFUNCTIONSFUNCTORIALITY}\cite[9, 10, 11]{bumpIntroductionLanglandsProgram2004}\cite{arthurLfunctionsAutomorphicRepresenta}, we will recall the idea and then discuss some of the properties and relations with Eisenstein series and interwining operators that we will need later.

The first thing is to recall the classification of connected reductive groups defined over an algebraically closed field via root datum. A root datum is a tuple \((X, \Phi, \check{X} , \check{\Phi})\) where \(X\) and \(\check{X}\) are two free abelian groups of finite type, \(\Phi, \check{\Phi}\) are subgroups that are in duality via a perfect pairing on \(X, \check{X}\). Then each reductive group \(G\) over a number field \(F\) has associated the root datum that is associated to its base change to \(\C\). Thus to a connected reductive group over a number field we associate a connected reductive group over \C, given by the dual root datum. We call this the \textbf{dual group} of \(G\) and denote it \(\hat{G}\). The \textbf{Langlands dual group} is then the dual group producted with the \(\mathrm{Gal}(\bar{k}/k)\)
\[^L G \defeq \hat{G} \rtimes \mathrm{Gal}(\bar{k}/k).\]

\begin{example}[Classical Groups, \cite{bumpIntroductionLanglandsProgram2004}, 11.1]
    We have the following table
    \begin{table}[h]
        \centering
        \begin{tabular}{ll}
        \(G\)         & \(\hat{G}\)   \\ \hline
        \(\GL_n\)     & \(\GL_n\)     \\
        \(SO_{2n+1}\) & \(\Sp_{2n}\)  \\
        \(SO_{2n}\)   & \(SO_{2n}\)   \\
        \(\Sp_{2n}\)  & \(SO_{2n+1}\)
        \end{tabular}
        \end{table}
\end{example}

Then, using the Satake isomorphism \cite[2.2]{shahidiEisensteinSeriesAutomorphic2010}, to each unramified representation of \(G(F_\nu)\) we can associate a conjugacy class of \(^LG\), via some map call it \(c\), and hence there is a way to apply a complex representation \(r: ^LG \to \GL_n(\C)\) to representations of \(G(F_\nu)\). Thus the automorphic L-functions are defined as follows: Let \(\rho\) be a representation of \(G(\A)\), let \(r\) be a complex representation of \(^LG\) and \(s\in \C\) then 
\[L(s, \rho, r) \defeq \prod_\nu L_\nu(s, \rho_\nu, r ) = \prod_\nu \frac{1}{\det\bigl( I - r(c(\rho_\nu))q^{-s} \bigr)}  ,\]
where \(\nu\) runs over the unramified places. It is a part of the grand Langlands philosophy that there should be suitable L-functions for the ramified places satisfying very nice properties.

\begin{remark}
    The global L-functions have been defined for many groups at this point and indeed \cite{jiangPolesCertainResidual2013} uses known properties to prove their results. One should note that the questions that we are interested in are still tractable even though the L-functions might not be defined (for instance for the metaplectic group). This is because only finitely many places will ramify, and so as long as those places are neither zero or poles we can transfer questions about zeros and poles from the full global L-functions to L-functions at almost all places. 
\end{remark}

\begin{example}[Standard Representations / Classical Groups]
    In the case of classical groups it is common to see L-functions with only two entries e.g. if \(\rho\) is a representation of \(G = \Sp{2n}\) then you may see 
    \(L(s, \rho).\)
    The reason is that there is a standard representation of the dual groups of classical groups. Namely the standard representation of a matrix group inside \(\GL_n\) is the one that sends \(g\mapsto g\). It is this representation that is to be taken for the dual group in this setting.
\end{example}

\begin{example}[Rankin-Selberg]
    \todo[inline]{fill}
\end{example}


\begin{example}[Dirichlet L-functions]
    Recall that a Dirichlet character \(\chi\) is a character of the group \((\Z/N\Z)^*\). Through the series of maps 
    \[A^\times \cong \Q^\times \times \R_{>0}^\times \times \hat{\Z}^\times \to (\lim \Z/N\Z)^\times \to (\Z/N\Z)^\times \to \C,\]
    one get a bijection between Dirichlet characters and finite-order Grossencharacters, i.e. characters of \(\A_F^\times/F^\times\).
    Grossencharacters have the associated L-function as they are just automorphic forms of \(\GL_1\), which generate automorphic representations. These give us the classical Dirichlet L-functions.\todo{reference? More details?}
\end{example}
Although this might seem unrelated to the current section on spectral decomposition and Eisenstein series we will see later that the two are inextricably linked
\todo[inline]{ref}

