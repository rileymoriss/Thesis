\section{Eisenstein Series}
As usual we fix a connected reductive group G defined over a number field F, with a Borel B, a standard parabolic with Levi decomposition \(P = MU\). 

Following the setup in \cite[I.1.4]{moeglinSpectralDecompositionEisenstein1995} we consider a \textbf{character} \(\chi\in \mathrm{Rat}(M) \defeq \Hom_{\mathrm{LAG}}(M, \mathbb{G}_m)\), thinking of it below as a natural transformation, and then define 
\[|\chi|: M(\A)\to \C , \;\;\; (m_\nu)\mapsto \prod_\nu|\chi(F_\nu)(m_\nu)|_\nu.\]
The intersection of the kernels of these characters is 
\[M^1 \defeq \bigcap_{\chi\in \mathrm{Rat}(M)}\ker |\chi|.\]
Thus we can define
\[X_M \defeq \Hom_{\textrm{TopGroup}}(M(\A)/M^1, \C^*) .\]
i.e. the collection of characters of \(M(\A)\) that are trivial on \(M^1\).
\begin{remark}
    To make it seem less mysterious this group has some importance in the more general theory, in particular it is one of the pieces in the ``Langlands decomposition'' (\ref{eq:iwasawa_decomposition}) of the Archimedean points of a parabolic and it has the property that \(M(\Q)\backslash M(\A)^1\) has finite measure \cite[4.9]{getzIntroductionAutomorphicRepresentations2024}.
\end{remark}
The set of \textbf{complex characters} of \(M\),
\[\mathfrak{a}_M^* \defeq \mathrm{Rat}(M)\tensor_\Z \C,\]
is isomorphic as \C vector spaces to \(X_M\). If \(Z_{G(\A)}\) is the center of \(G(\A)\) then we also have the space 
\[X_M^G \defeq \Hom_{\textrm{TopGroup}}((M(\A)/M^1)/Z_G, \C^*)\]
which is characters of \(M(\A)/M^1\) which are also trivial on the center of \(G\).

\begin{example}\label{ex:characters}
    For the maximal parabolic \(P_r\) with Levi \(M_r\) of \(\Sp_{2n}\) we have that \( X_{M_r}^{\Sp_{2n}}\) is at most a one dimensional \C vector space. 

    First of all we have that \cite[I.1.4]{moeglinSpectralDecompositionEisenstein1995}
         \[ X_{M_r}^{\Sp_{2n}} \subseteq X_{M_r} \cong \mathfrak{a}_{M_r}^*\defeq Rat(M_r) \tensor_\Z \C.\]
        Thus it is clearly sufficient to bound the dimension of \(\mathfrak{a}_{M_r}^*\) as a \C vector space, moreover this dimension agrees with the dimension of \(Rat(M_r)\) as a free \Z module. 

        Thus we compute \(\dim_\Z(Rat(M_r))\):
        \begin{equation*}
            \begin{aligned}
                Rat(M_r) &= Rat(\GL_r \times \Sp_{2m}) \\
                         &= \Hom(\GL_r \times \Sp_{2m}, \mathbb{G}_m) \\
                         (2)&\cong \Hom(\mathrm{Ab}(\GL_r \times \Sp_{2m}), \mathbb{G}_m) \\
                         (1)&\cong \Hom(\mathrm{Ab}(\GL_r) \times \mathrm{Ab}(\Sp_{2m}), \mathbb{G}_m) \\
                         (3)&\cong \Hom(\mathbb{G}_m \times 1, \mathbb{G}_m) \\
                         &\cong \Z.
            \end{aligned}
        \end{equation*}
        In (2) we have used the universal property of the abelianization \(\mathrm{Ab}(G) = \mathcal{D}(G) \setminus G = [G, G] \setminus G \) because \(\mathbb{G}_m\) is abelian. (1) is that the abelianization commutes with direct products. (3) is because \(\Sp\) is a perfect group.
        %https://groupprops.subwiki.org/wiki/Symplectic_group_is_perfect
        %https://mathoverflow.net/questions/35713/abelianization-of-a-semidirect-product
\end{example}

\begin{remark}\label{Metaplectic_characters}
    This generalises to the metaplectic covers immediately as \( X_{M_r}^{\Mp_{2n}(\A)} \subseteq X_{M_r}\).
\end{remark}
There is the natural map \(m_P: G(\A) \to M^1 \backslash M(\A)\) sending \(umk \mapsto M^1 m\), where \(g = umk\) using the Langlands-Iwasawa decomposition \ref{eq:iwasawa_decomposition}.

Now if we take the collection of irreducible automorphic representations of \(M\),
 \[\hat{\mathcal{A}} \defeq \{(\pi, V) : \pi \text{ is an irreducible automorphic representation of }M\}\]
then we can think of \(X_M^G\) as being one dimensional automorphic representations (with some extra invariance) and so there is a natural action on \(\hat{\mathcal{A}}\) given by tensoring, i.e. if \(\lambda\in X_M^G\) and \((\pi, V)\in \hat{\mathcal{A}}\) then 
\[\lambda.\pi \defeq \lambda\tensor \pi\]
Then \(\hat{\mathcal{A}}\) decomposes as a disjoint union of its orbits. Consider the orbit \(\mathfrak{P}\) of a cuspidal representation \(\pi_0\), then by definition \(X_M^G\) acts transitively but it also acts freely \cite[II.1]{moeglinSpectralDecompositionEisenstein1995}. Thus \(\mathfrak{P}\) is in bijection with \(X_M^G\). Through this bijection we transmit the complex structure on \(\mathfrak{a}_M^*\) to \(X_M\) then to the quotient \(X_M^G\) and finally to \(\mathfrak{P}\).

Now we will define an Eisenstein series: Let \(\mathfrak{P}\) be as above, the orbit of a cuspidal automorphic representation endowed with a complex structure. Let \(\pi\in \mathfrak{P}\) and \(\phi_\pi \in \mathcal{A}(U(\A)M(k)\backslash G(\A))_\pi\), then \(\lambda\in X_M^G\) acts on \(\phi_\pi\) by 
\[\lambda.\phi_\pi(g) = (\lambda \comp m_P)(g) \phi_\pi(g).\]
which is then an element of \(\mathcal{A}(U(\A)M(k)\backslash G(\A))_{\pi\tensor \lambda}\). Finally we have the \textbf{Eisenstein series} which is defined by the following sum
\[E(\phi_\pi, \lambda, g) = \sum_{\gamma \in P(k)\backslash G(k)} \lambda.\phi_\pi(\gamma g)\]
whenever it is convergent. The first thing to note is that for a fixed \(\phi\) there is an open set in \(X_M^G\) and a compact subset of \(G(k)\backslash G(\A)\) such that the Eisenstein series converges (normally) \cite[II.1.5]{moeglinSpectralDecompositionEisenstein1995}.

If \(P = MU, P' = M'U'\) are two standard parabolics of \(G\) that are conjugate, i.e. such that for \(w\in G(k)\) we have \(wMw\inv = M'\)
Then \(w\) maps \(\mathfrak{P}\) to \(w\mathfrak{P}\), an orbit of an irreducible representations of \(M\) to an orbit of irreducible representations of \(M'\).

Then the Eisenstein series is closely related (through its constant terms as discussed in \ref{constant_conjugate_levi}) to the operator
\[M(w, \pi)(\phi_\pi)(g) = \int_{(U'(k)\cap wU(k)w\inv )\backslash U'(\A)} \phi_\pi(w\inv ug) du\]
where \(\pi\in \mathfrak{P}\), \(g\in G(\A)\) and \(\phi_\pi \in \mathcal{A}(U(\A)M(k)\backslash G(\A))_\pi\).

The key properties of both the Eisenstein series and this operator can be found in \cite[IV.1.8, IV.1.9, IV.1.10, IV.1.11]{moeglinSpectralDecompositionEisenstein1995}. Most importantly as a function of \(\mathfrak{P}\) it can be shown that (in the sense of Frechet spaces) they both have a meromorphic continuation to all of \(\mathfrak{P}\). This was also given a second ``soft proof'' more recently in \cite{bernsteinMeromorphicContinuationEisenstein2022}, with the spectral decomposition that follows from it also being worked out in \cite{delormeSpectralTheoremLanglands2021}. Moreover for the Eisenstein series at a point in \(p\in \mathfrak{P}\) at which it is holomorphic then \(E(\phi,p, g)\) is an automorphic form. 

We are not really in a position to convey the true importance of these objects in the theory of automorphic forms, however we will make some comments. First some surveys are \cite{lapidPerspectivesEisensteinSeries2022}, \cite{arthurEisensteinSeriesTrace1979}, \cite{kimEISENSTEINSERIESTHEIR}, \cite{jiangResiduesEisensteinSeries2008a}. To see the relation to the classical Eisenstein series there is \cite{garrettTransitionEisensteinSeries2016}. One thing that Eisenstein series do, as in the theory of modular forms, is that they furnish us with quasi-concrete examples. A we mentioned above \cite[IV.1.9.(b).i]{moeglinSpectralDecompositionEisenstein1995} tells us that at the holomorphic points the Eisenstein series takes an automorphic form and returns an automorphic form, thus we can use them to multiply our examples. Another reason that these functions are important is through their normalisation and constant terms, in which products of L functions appear, we discuss this more in section \todo[inline]{ref later}. This has been a fruitful method for proving theorems about L-functions as in \cite{shahidiEisensteinSeriesAutomorphic2010}\cite{pollackRANKINSELBERGMETHODUSER}\cite{arthurEisensteinSeriesTrace1979}.

