The Eisenstein series is from our perspective the most important tool in the theory of automorphic forms. Some surveys on its role, properties and open problems are \cite{lapidPerspectivesEisensteinSeries2022}, \cite{arthurEisensteinSeriesTrace1979}, \cite{kimEISENSTEINSERIESTHEIR} and \cite{jiangResiduesEisensteinSeries2008a}. To see the relation to the classical Eisenstein series there is \cite{garrettTransitionEisensteinSeries2016} which we will also go through in section \ref{sec:classic-eisenstein}. One thing that Eisenstein series do, as in the theory of modular forms, is that they furnish us with quasi-concrete examples of automorphic forms. Another reason that these functions are important is through their normalisation and constant terms, in which products of L functions appear, we discuss this more in section \ref{sec:L-functions}. This has been a fruitful method for proving theorems about L-functions as in \cite{shahidiEisensteinSeriesAutomorphic2010}\cite{pollackRANKINSELBERGMETHODUSER}\cite{arthurEisensteinSeriesTrace1979}, or conversely proving theorems about Eisenstein series \cite{jiangPolesCertainResidual2013}.

\section{Eisenstein Series}\label{sec:eisenstein-series}
As usual we fix a classical group G defined over a number field F, with a Borel B and a standard parabolic with Levi decomposition \(P = MU\). 

Following the setup in \cite[I.1.4]{moeglinSpectralDecompositionEisenstein1995} we consider a \textbf{character} \(\chi\in \mathrm{Rat}(M) \defeq \Hom_{\mathrm{LAG}}(M, \mathbb{G}_m)\), thinking of it below as a natural transformation, and then define 
\[|\chi|: M(\A)\to \C , \;\;\; (m_\nu)\mapsto \prod_\nu|\chi(F_\nu)(m_\nu)|_\nu.\]
The intersection of the kernels of these characters is 
\[M^1 \defeq \bigcap_{\chi\in \mathrm{Rat}(M)}\ker |\chi|.\]
Thus we can define
\[X_M \defeq \Hom_{\textrm{TopGroup}}(M(\A)/M^1, \C^*) .\]
i.e. the collection of characters of \(M(\A)\) that are trivial on \(M^1\).
\begin{remark}
    To make it seem less mysterious we comment that this group has some importance in the more general theory. It is one of the pieces in the ``Langlands decomposition'' of the Archimedean points of a parabolic \(P = MU\), if \(\nu\) is an archimedean place then,
    \[P(F_\nu) = A_M M(F_\nu)U(F_\nu).\]
    We will not define \(A_M\).
    It also has the property that \(M(\Q)\backslash M(\A)^1\) has finite measure \cite[4.9]{getzIntroductionAutomorphicRepresentations2024}.
\end{remark}
The set of \textbf{complex characters} of \(M\),
\[\mathfrak{a}_M^* \defeq \mathrm{Rat}(M)\tensor_\Z \C,\]
is isomorphic as \C vector spaces to \(X_M\). If \(Z_{G(\A)}\) is the center of \(G(\A)\) then we also have the space 
\[X_M^G \defeq \Hom_{\textrm{TopGroup}}((M(\A)/M^1)/Z_G, \C^*)\]
which is characters of \(M(\A)/M^1\) which are also trivial on the center of \(G\).

\begin{example}\label{ex:characters}
    For the maximal parabolic \(P_r\) with Levi \(M_r\) of \(\Sp_{2n}\) we have that \( X_{M_r}^{\Sp_{2n}}\) is at most a one dimensional \C vector space. 

    First of all we have that \cite[I.1.4]{moeglinSpectralDecompositionEisenstein1995}
         \[ X_{M_r}^{\Sp_{2n}} \subseteq X_{M_r} \cong \mathfrak{a}_{M_r}^*\defeq Rat(M_r) \tensor_\Z \C.\]
        Thus it is clearly sufficient to bound the dimension of \(\mathfrak{a}_{M_r}^*\) as a \C vector space, moreover this dimension agrees with the dimension of \(Rat(M_r)\) as a free \Z module. 

        Thus we compute \(\dim_\Z(Rat(M_r))\):
        \begin{equation*}
            \begin{aligned}
                Rat(M_r) &= Rat(\GL_r \times \Sp_{2m}) \\
                         &= \Hom(\GL_r \times \Sp_{2m}, \mathbb{G}_m) \\
                         (1)&\cong \Hom(\mathrm{Ab}(\GL_r \times \Sp_{2m}), \mathbb{G}_m) \\
                         (2)&\cong \Hom(\mathrm{Ab}(\GL_r) \times \mathrm{Ab}(\Sp_{2m}), \mathbb{G}_m) \\
                         (3)&\cong \Hom(\mathbb{G}_m \times 1, \mathbb{G}_m) \\
                         &\cong \Z.
            \end{aligned}
        \end{equation*}
        In (1) we have used the universal property of the abelianization \(\mathrm{Ab}(G) = \mathcal{D}(G) \setminus G = [G, G] \setminus G \) because \(\mathbb{G}_m\) is abelian. (2) is that the abelianization commutes with direct products. (3) is because \(\Sp\) is a perfect group.
        %https://groupprops.subwiki.org/wiki/Symplectic_group_is_perfect
        %https://mathoverflow.net/questions/35713/abelianization-of-a-semidirect-product
\end{example}


There is the natural map \(m_P: G(\A) \to M^1 \backslash M(\A)\) sending \(umk \mapsto M^1 m\), where \(g = umk\) using the Langlands-Iwasawa decomposition of equation \ref{eq:iwasawa_decomposition}.

Now if we take the collection of irreducible automorphic representations of \(M\),
 \[\hat{\mathcal{M}} \defeq \{(\pi, V) : \pi \text{ is an irreducible automorphic representation of }M\}\]
then we can think of \(X_M^G\) as being one dimensional automorphic representations (with some extra invariance) and so there is a natural action on \(\hat{\mathcal{M}}\) given by tensoring, i.e. if \(\lambda\in X_M^G\) and \((\pi, V)\in \hat{\mathcal{M}}\) then 
\[\lambda.\pi \defeq \lambda\tensor \pi\]
Then \(\hat{\mathcal{M}}\) decomposes as a disjoint union of its orbits. The orbit \(\mathfrak{P}\) of a cuspidal representation \(\pi_0\) is called a \textbf{cuspidal datum}. By definition \(X_M^G\) acts transitively on any cuspidal datum \(\mathfrak{P}\) but by \cite[II.1]{moeglinSpectralDecompositionEisenstein1995} it also acts freely. Thus \(\mathfrak{P}\) is in bijection with \(X_M^G\). Through this bijection we transmit the complex structure on \(\mathfrak{a}_M^*\) to \(X_M\) then to the quotient \(X_M^G\) and finally to \(\mathfrak{P}\).

Let \(\mathfrak{P}\) be a cuspidal datum with a complex structure as above. Let \(\pi\in \mathfrak{P}\) and \(\phi_\pi \in \mathcal{A}(U(\A)M(k)\backslash G(\A))_\pi\), then \(\lambda\in X_M^G\) acts on \(\phi_\pi\) by 
\[\lambda.\phi_\pi(g) = (\lambda \comp m_P)(g) \phi_\pi(g).\]
which is then an element of \(\mathcal{A}(U(\A)M(k)\backslash G(\A))_{\pi\tensor \lambda}\). Finally we have the \textbf{Eisenstein series} which is defined by the following sum
\[E(\phi_\pi, \lambda, g) = \sum_{\gamma \in P(k)\backslash G(k)} \lambda.\phi_\pi(\gamma g)\]
whenever it is convergent. The first thing to note is that for a fixed \(\phi\) there is an open set in \(X_M^G\) and a compact subset of \(G(k)\backslash G(\A)\) such that the Eisenstein series converges (normally) \cite[II.1.5]{moeglinSpectralDecompositionEisenstein1995}.

If \(P = MU, P' = M'U'\) are two standard parabolics of \(G\) that are conjugate, i.e. such that for \(w\in G(k)\) we have \(wMw\inv = M'\).
Then \(w\) maps \(\mathfrak{P}\) to \(w\mathfrak{P}\), an orbit of an irreducible representations of \(M\) to an orbit of irreducible representations of \(M'\).
The Eisenstein series is closely related through its constant terms (as discussed in section \ref{constant_conjugate_levi}) to the operator
\[M(w, \pi)(\phi_\pi)(g) = \int_{(U'(k)\cap wU(k)w\inv )\backslash U'(\A)} \phi_\pi(w\inv ug) du\]
where \(\pi\in \mathfrak{P}\), \(g\in G(\A)\) and \(\phi_\pi \in \mathcal{A}(U(\A)M(k)\backslash G(\A))_\pi\).

The Eisenstein series has three inputs and can be considered as a function in different variables, it can be a tedious task to specify the correct domain and codomains for these maps however. If \(\pi\) is a cuspidal automorphic representation induced from \(P\), then for a fixed \(\phi \in\mathcal{A}_0(U(\A)M(k)\setminus G(\A))_\pi \) the Eisenstein series \(E(\phi)\) can be thought of as a function from some open subset of the cuspidal datum \(\mathfrak{P}\) into \(L^2_{\mathrm{loc}}(G)\), the set of locally square integrable complex valued functions on \(G(\A)\), given by 
\[E(\phi)(\lambda)(g) = \sum_{\gamma \in P(k)\backslash G(k)} \lambda.\phi(\gamma g), \;\;\; \lambda\in \mathfrak{P},\; g\in G(\A),\]
where it converges. The space \(L^2_{\mathrm{loc}}(G(\A))\) can be endowed with a Frechet space structure coming from the semi-norms associated to compact sets \(C\subseteq G(\A)\) given by 
\[\phi \mapsto \norm{\phi|_C}_{L^2} = \int_C |\phi(x)|^2 \mathrm{d}x.\] 
Then it makes sense to talk about the holomorphicity of \(E(\phi)\) in this sense \cite[I.4.9]{moeglinSpectralDecompositionEisenstein1995}. The key properties of both the Eisenstein series and this operator can be found in \cite[IV.1.8, IV.1.9, IV.1.10, IV.1.11]{moeglinSpectralDecompositionEisenstein1995}. Most importantly as a function of \(\mathfrak{P}\) it can be shown that they both have a meromorphic continuation to all of \(\mathfrak{P}\). This was also given a second ``soft proof'' more recently in \cite{bernsteinMeromorphicContinuationEisenstein2022}, with the spectral decomposition that follows from it also being worked out in \cite{delormeSpectralTheoremLanglands2021}. Moreover an Eisenstein series attached to an automorphic form, at a point \(p\in \mathfrak{P}\) at which it is holomorphic, is also an automorphic form. 

\section{Classical Eisenstein Series}\label{sec:classic-eisenstein}

We will follow the excellent exposition in \cite{garrettTransitionEisensteinSeries2016}, the section \cite[1.2]{bruinier123ModularForms2008} on classical Eisenstein series. The typical example of a classical Eisenstein series is that introduced by Maas in 1949 \cite{lapidPerspectivesEisensteinSeries2022}  given by the sum 
\[\mathbf{E}(z, s) \defeq \frac{1}{2} \sum_{(m,n)\in \Z^2\setminus \{(0,0)\}, \text{ coprime }} \frac{\mathrm{Im}(z)^{s}}{|mz+n|^{2s}}, \quad z\in \mathcal{H},\;\; \mathrm{Re}(s)>\frac{1}{2},\]
which converges absolutely. Consider the algebraic group \(\SL_2\) with the parabolic of upper triangular matrices \(P\).

First we want to look at the index of the sum, we aim to define a map
	\[\omega: P(\Z) \backslash \SL_2(\Z) \to  \{(m,n)\in \Z^2\backslash\{(0,0)\} : m,n \text{ are co-prime}\}\]

	The cosets of \(P(\Z) \backslash \SL_2(\Z)\) look like
	\[P(\Z)\begin{pmatrix}
		a & b \\ c& d
	\end{pmatrix} = \left\{\pm\begin{pmatrix}
		1 & n \\&1
	\end{pmatrix}\begin{pmatrix}
	a & b \\ c& d
	\end{pmatrix} : n\in \Z\right\} = \left\{\pm\begin{pmatrix}
	a+nc & b+nd \\ c & d
	\end{pmatrix} : n\in \Z\right\}.\]
	Moreover because \(\begin{pmatrix}
		a & b \\ c& d
	\end{pmatrix}\in \SL_2(\Z)\) we have by Bezout's lemma (applied to the determinant expression) that \(c\) and \(d\) are co-prime. Therefore there is a well defined map 
	\[P(\Z) \backslash \SL_2(\Z) \to  \{(m,n)\in \Z^2\backslash\{(0,0)\} : m,n \text{ are co-prime}\},\]
	if we denote \(\mathrm{ind}\{c<0\} = \begin{cases}
		0, & c\geq 0\\
		1, & c<0
	\end{cases}\) then it is given by 
	\[\left\{\pm\begin{pmatrix}
		a+nc & b+nd \\ c & d
	\end{pmatrix} : n\in \Z\right\}\mapsto (|c|,(-1)^{\mathrm{ind}\{c<0\}}d).\]
	The point is that \(|mz + n| = |(-m)z + (-n)|\) and so the sum in the Eisenstein series, having a prefactor of a half is really just the sum over \( \{(m,n)\in \Z^2\backslash\{(0,0)\} : m,n \text{ are co-prime \textbf{and} }m\geq 0\}\), which by our argument is in bijection with \(P(\Z) \backslash \SL_2(\Z)\) via \(\omega\).
	

	

\begin{Lemma}[\cite{garrettTransitionEisensteinSeries2016}, 3.5]
	\[P(\Z) \backslash \SL_2(\Z) \cong P(\Q) \backslash \SL_2(\Q). \]
\end{Lemma}
\proofbar{
	The bijection is explicitly
	\[P(\Z)g\mapsto P(\Q)g.\]

	
	Injectivity is clear because if \(g, g'\in \SL_2(\Z)\) are in the same \(P(\Q)\) orbit then we can cancel the denominators of the \(P(\Q)\) matrix and hence \(g,g'\) are in the same \(P(\Z)\) orbit.
	
	Surjectivity follows from a repeated application of the orbit stabilizer theorem, as in \cite[3.5]{garrettTransitionEisensteinSeries2016}.
}
Recall that \(\SL_2(\Z)\) acts via Mobius transformations on the upper half plane. If \(z= x+ iy \in \mathcal{H}\), \(s\in \C\) and \(\gamma = \begin{pmatrix}
	a & b\\ c& d \end{pmatrix}\in \SL_2(\Z)\) then an elementary computation shows that,
\[\mathrm{Im}(\gamma.z) = \frac{\mathrm{Im}(z)}{|cz + d|^{2}}.\]
Hence the classical Eisenstein series is 
\[\mathbf{E}(z, s) \defeq \frac{1}{2} \sum_{m,n\in \Z, \text{ coprime }} \frac{\mathrm{Im}(z)^{s}}{|mz+n|^{2s}} = \sum_{\gamma\in P(\Q)\backslash \SL_2(\Q)} \mathrm{Im}(\gamma.z)^{s}.\]

We want to realize this as the Eisenstein series associated to an automorphic form so first we must design a function on \(\SL_2(\A)\). For any place \(\nu\) of \Q we have the local Iwasawa decomposition \(\SL_2(\Q_\nu) = P(\Q_\nu) K_\nu\) where 
\[K_\nu \defeq \begin{cases}
	\SL_2(\Z_\nu), & \nu \text{ non-Archimedean}\\
	SO_2(\R), & \nu \text{ Archimedean}
\end{cases}\]
 are the local maximal compact subgroups. We define a function on the adeles by defining it on the local pieces,
 \[\phi_{\nu, s}\left(\begin{pmatrix}
 	a & b\\ & d
 \end{pmatrix}k\right) \defeq \left| \frac{a}{d} \right|_\nu^s .\]
 Finally we define \(\phi_s\) as the map 
 \[(g_\nu)_{\nu} \mapsto \prod_\nu \phi_{\nu,s}(g_\nu.)\]
 
 \begin{Lemma}
 	\(\phi_s\) is an automorphic form on \(\SL_2(\A)\).
 \end{Lemma}
 \proofbar{
 	Smooth, moderate growth and \(K\)-finiteness are obvious from the definition.	
 	Using the product formula, i.e. for all \(x\in \Q^\times\) we have that \(\prod_\nu |x|_\nu = 1\), we get that \(\phi_s\) is left \(\SL_2(\Q)\) invariant.
 	
 \todo[inline]{	Z(g) finite}
 }
 
 To this we have an Eisenstein series associate as in the adelic setting by 
 \[E(\phi, g) \defeq \sum_{\gamma\in P(\Q)\backslash \SL_2(\Q)} \phi_s(\gamma g) \]

\begin{Lemma}
	Let \(g\in \SL_2(\R)\) then we consider it as an element of \(\SL_2(\A)\), denoted by \(\iota(g)\), by setting all other entries to 1. Then 
	\[E(\phi_s, \iota(g)) = \mathbf{E}(g.i,s)\]
\end{Lemma}
\proofbar{
	First the left hand side,
	\begin{align*}
		E(\phi_s, \iota(g)) &= \sum_{\gamma\in P(\Q)\backslash \SL_2(\Q)} \phi_s(\gamma \iota(g)) \\
		&= \sum_{\gamma\in P(\Z)\backslash \SL_2(\Z)} \prod_\nu\phi_{\nu,s}(\gamma g_\nu) \\
		&= \sum_{\gamma\in P(\Z)\backslash \SL_2(\Z)} \phi_{\infty,s}(\gamma g) \prod_{\nu<\infty} \phi_{s, \nu}(\gamma)\\
		&= \sum_{\gamma\in P(\Z)\backslash \SL_2(\Z)} \phi_{\infty,s}(\gamma g).\\
	\end{align*}
	Because \(\gamma\in \SL_2(\Z)\subseteq \SL_2(\Z_\nu)\) for each place \(\nu\) and so \(\phi_{s, \nu}\) is by definition trivial on these.  The final step is then to show that 
	\[\phi_{\infty, s} (g) = |\mathrm{Im}(g.i)|^s.\]
	
	If \(\begin{pmatrix}
		a& b\\ c&d
	\end{pmatrix}\in \SL_2(\R)\) then for some \(k\in SO_2(\R)\) we have that  \cite{conradDecomposingSL2R},
	\[\begin{pmatrix}
		a& b\\ c&d
	\end{pmatrix} = \begin{pmatrix}
		(a^2 + c^2)^{\frac{1}{2}} & \ast \\ & (a^2 + c^2)^{-\frac{1}{2}}
	\end{pmatrix} .\]
	With this explicit Iwasawa decomposition the proof is finished with some elementary matrix manipulation, this is done very explicitly in \cite[3.3]{garrettTransitionEisensteinSeries2016}.
}





