There are different definitions of the words automorphic forms floating around, here we fix a nice one and then explain how they generalise the classical modular forms. Although this has been done in several places, none suit my taste to a tee, and so we briefly present our own take on this. We intend to be terse as this material is somewhat standard.
\todo[inline]{references}

\section{Definition and Role}
The story starts with the classical modular forms, or functions on the upper half plane that satisfy some invariance conditions and differential equations. This evolves into the notions of Maas form on symmetric spaces and eventually reaches its apotheosis in the concept of automorphic form that we will present here. 
\todo[inline]{reference or more detail for the history}

We still do not have a good answer as to why the definition below is "the right" definition, from a mathematical perspective, as there are many places in which it could be extended or restricted and we are unable to motivate why one shouldnt consider such things. Indeed there are varying notions of automorphic form that appear for this reason and I think it is important to stress that this is "the right definition" only in so far as people have been able to prove nice theorems about them, and that when functions appear "in nature" this concept has sufficed to encompass and explain their behaviour. It is the represntation theoretic properties more than anything that suggest the current definition as is mentioned in \cite[1.II.3]{borelAutomorphicFormsRepresentations1979}.

For concreteness we want to provide one, certainly a-historical, reason to be interested in them. It was proved by Jacquet-Langlands that all representations of reductive groups appear as subquotients of the regular represtntation on a space of automorphic forms:

\begin{Theorem}[\cite{getzIntroductionAutomorphicRepresentations2024} 4.9, 8.3.5, 10.6.1]
    Let \(G\) be a connected reductive LAG over a field \(F\).
    \todo[inline]{actually check hypothesis on G here}
    All smooth irreducible representations of \(G(F)\) are a subquotient of a parabolically induced representation from a Levi. Any irreducible subquotient of such a parabolically induced representation is an automorphic representation and moreover all automorphic representations appear in this way. 
\end{Theorem}
Thus if one is interested in reductive group representations then automorphic representations and hence automorphic forms are unavoidable.

We will present three notions of automorphic form here. In the literature they are all called ``automorphic forms'' however here we will distinguish them with our own terminology for clarity.

\subsection{Archimedian Automorphic Form}
Fix a global (number) field \(F\). Let \(\nu\) be an archimedian place and let \(\infty\) denote the set of archimedian places. Then \(F_\nu\) is either \R or \C. In particular (the analytification of) \(G(F_\nu)\) is a Lie group and we call a function, \(\phi: G(F_\nu) \to \C\), \textit{smooth}  if it is smooth in the sense of manifolds.


Now we fix an embedding \(\i : G\to GL_n\) which gives another embedding \(G\to SL_{2n}\) via
	\[g\mapsto \begin{pmatrix}
		\i (g) & \\
		 & (\i (g))^{-t}
	\end{pmatrix}\]

	A function \(\phi : G(F_\infty) = G(\prod_{v|\infty} F_\nu) \to \C \) is of \textit{moderate growth} if there are constants \((c,r)\in \R_{>0}\times \R\) such that 
	\[|\phi(g)| \leq c\norm{g}^r = c \left(\prod_{v|\infty} \sup_{1\leq i, j\leq 2n} |\i (g)_{i,j, \nu}|_\nu\right)^r\]
	this is taking the maximum of the \(2n\times 2n \times |\infty| \) three dimensional matrix. 

    \todo[inline]{there is a nice characterisation over Q for representations and norms as traces of blah in Borel. I like it a lot just need to check its fine for other number fields.  }

    From \todo[inline]{reference the Lie algebra section} we know how to define the Lie algebra of \(G\) and we now denote \(Z(\mathfrak{g})\) the center of the \textit{universal enveloping algebra} of the \textit{complexification} of \(\mathfrak{g}\), it would be more reasonable to use \(Z(\mathcal{U}(\mathfrak{g}_{\C}))\) but that is too cumbersome so we follow the tradition. 
    A vector in a \(Z(\mathfrak{g})\) module \(\phi\in V\) is called \(Z(\mathfrak{g})\)-\textit{finite} if the space \(Z(\mathfrak{g})\phi\) is finite dimensional. 

	Let \(K_\infty\leq G(F_\infty)\) be a maximal compact subgroup. Then again an element of a \(K_\infty\) module is \(K_\infty\) finite if its orbit is a finite dimensional \todo[inline]{vector space over what??} space.

	To define automorphic forms we look at the representation \(C^\infty(F_\infty)\) with the left regular action.  In particular the \(Z(\mathfrak{g})\) module structure is induced from the action of \(\mathfrak{g}\) on \(C^\infty(G(F_\infty))\) by 
	\[z.F(g) = \Dif{}{t}F(ge^{tz})\] 

	\begin{Definition}
		Let \(\Gamma\leq G(F_\infty)\) some arithmetic subgroup, an automorphic form for \(\Gamma\) is a smooth function of moderate growth 
		\[\phi: G(F_\infty) \to \C\]
		that is \(K_\infty\) and \(Z(\mathfrak{g})\) finite with a (left) \(\Gamma\) invariance. 

	\end{Definition}

\subsection{Adelic Automorphic Form}
Here we follow \cite[I.2.17]{moeglinSpectralDecompositionEisenstein1995}. Let \(G\) be a reductive group over \(F\), we fix a Borel \(B\) and a standard parabolic \(P \) with a standard Levi decomposition \(P = MU\). We let \(K\) be a maximal compact subgroup satisfying the conditions laid out in \todo[inline]{reference the previous section}
\begin{definition}
    A function \(\phi: U(\A)M(F)\backslash G(\A) \to \C\) is an automorphic form if it is smooth, moderate growth, \(Z(\mathfrak{g})\) and \(K\) finite. 
\end{definition}

\begin{remark}
    It is important that \(M(F)\) is treated as a subgroup of \(M(\A)\) via the diagonal embedding.
\end{remark}

For \(v\notin \infty\) a non-archimedian place then we say that a function \(f: G(F_\nu) \to \C\) is smooth if it is locally constant in the induced topology on \(G(F_\nu)\), the details of this topology are spelled out in \cite{conradWeilGrothendieckApproaches2012}. The set of such smooth functions is denoted \(C^\infty(G(F_\nu))\).
	
	For the non-archimedian places we define smooth functions on the ``finite adeles'' \(\A_f\) as 
	\[C^\infty(\mathbb{A}_f) \defeq \bigotimes^{}_{\nu \notin \infty} \phantom{}' C^\infty(G(F_\nu)) \]
	
	And for the archimedian places we define
	\[C^\infty(G(F_\infty)) \defeq C^\infty\left(\prod_{\nu|\infty} G(F_\nu)\right)\]
	
	For the full Adele we define 
	\[C^\infty(\mathbb{A}_F) \defeq C^\infty(G(F_\infty)) \otimes C^\infty(G(\mathbb{A}_F^\infty))\]
	
	A function on the adeles is smooth if it is in this set. Notice that a priori the codomain is an infinite tensor product over \C of copies of \C, which is isomorphic to \C. Thus we can conflate a smooth function with its composition along this isomorphism, and think of them as functions into \C.
	
	The definition of moderate growth carries over verbatim, however we change the set of places multiplied over to be all of them now.
    
    \begin{remark}[\cite{borelAutomorphicFormsRepresentations1979}, 1.II.3]
        The collection of moderate growth functions is independent of the choices of embedding. 
    \end{remark}
    
    We know what it means for a general module to be \(Z(\mathfrak{g})\) and \(K\) finite, so to complete the definition we need only specify the module structure for these algebras on the set of smooth functions. 

\section{Modular Forms}