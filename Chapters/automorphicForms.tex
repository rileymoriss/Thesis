There are different definitions of the words automorphic forms floating around, here we fix a nice one and then explain how they generalize the classical modular forms.

\section{Definition and Role}
The story starts with the classical modular forms, or functions on the upper half plane that satisfy some invariance conditions and differential equations. This evolves into the notions of Maas form on symmetric spaces and eventually reaches its apotheosis in the concept of automorphic form that we will present here. 

We still do not have a good answer as to why the definition below is ``the right'' definition, from a mathematical perspective, as there are many places in which it could be extended or restricted and we are unable to motivate why one shouldn't consider such things. Indeed there are varying notions of automorphic form that appear for this reason and I think it is important to stress that this is ``the right definition'' only in so far as people have been able to prove nice theorems about them, and that when functions appear ``in nature'' this concept has sufficed to encompass and explain their behavior. It is the representation theoretic properties more than anything that suggest the current definition as is mentioned in \cite[1.II.3]{borelAutomorphicFormsRepresentations1979}.

We will present two notions of automorphic form here. In the literature they are both called ``automorphic forms'' however here we will distinguish those that are defined only on the Archimedean points as ``Archimedean automorphic forms'' for clarity.

\subsection{Archimedean Automorphic Form}
Fix a number field \(F\). Let \(\nu\) be an Archimedean place and let \(\infty\) denote the set of Archimedean places. Then \(F_\nu\) is either \R or \C. In particular (the analytification of) \(G(F_\nu)\) is a Lie group and we call a function, \(\phi: G(F_\nu) \to \C\), \textbf{smooth}  if it is smooth in the sense of manifolds.
\todo[color=orange]{First of all is the analytification the same as the Hausdorff topology? Secondly what is the reference for these facts..}

Now we fix an embedding \(\i : G\to GL_n\) which gives another embedding \(G\to SL_{2n}\) via
	\[g\mapsto \begin{pmatrix}
		\i (g) & \\
		 & (\i (g))^{-t}
	\end{pmatrix}.\]

	A function \(\phi : G(F_\infty) = G(\prod_{v\in\infty} F_\nu)\cong \prod_{v\in \infty}G(F_\nu) \to \C \) is of \textbf{moderate growth} if there are constants \((c,r)\in \R_{>0}\times \R\) such that 
	\[|\phi(g)| \leq c\norm{g}^r = c \left(\prod_{v\in\infty} \sup_{1\leq i, j\leq 2n} |\i (g)_{i,j, \nu}|_\nu\right)^r.\]
	This is taking the maximum of the \(2n\times 2n \times |\infty| \) three dimensional matrix. 
	\todo[color=orange]{Look at the equivilent condition frmo Borel}

    Because \(G(F_\infty)\) is a Lie group we know how to define its Lie algebra and we now denote \(Z(\mathfrak{g})\) the center of the \textit{universal enveloping algebra} of the \textit{complexification} of \(\mathfrak{g}\), it would be more reasonable to use \(Z(\mathcal{U}(\mathfrak{g}_{\C}))\) but that is too cumbersome so we follow the tradition. 
    A vector in a \(Z(\mathfrak{g})\)-module \(\phi\in V\) is called \(Z(\mathfrak{g})\)-\textbf{finite} if the space \(Z(\mathfrak{g})\phi\) is finite dimensional. 

	Let \(K_\infty\leq G(F_\infty)\) be a maximal compact subgroup. Then again an element of a \(K_\infty\)-module is \(K_\infty\) \textbf{finite} if its orbit is a finite dimensional vector space space (we think here of \(\C[K_\infty]\)-modules).

	To define automorphic forms we look at the representation \(C^\infty(F_\infty)\) with the right regular action, i.e. \(g.f(x) = f(xg)\).  In particular the \(Z(\mathfrak{g})\) module structure is induced from the action of \(\mathfrak{g}\) on \(C^\infty(G(F_\infty))\) by \label{lie_algebra_action}
	\[z.F(g) = \Dif{}{t}F(ge^{tz}).\] 

	\begin{Definition}
		Let \(\Gamma\leq G(F_\infty)\) some (arithmetic) subgroup, an \textbf{automorphic form for \(\Gamma\)} is a smooth function of moderate growth 
		\[\phi: G(F_\infty) \to \C,\]
		that is \(K_\infty\) and \(Z(\mathfrak{g})\) finite with a (left) \(\Gamma\) invariance. We denote the set of these ``Archimedean'' automorphic forms by \(\mathcal{A}(\Gamma \backslash G(F_\infty))\).
	\end{Definition}


\subsection{Adelic Automorphic Form}
Here we follow \cite[I.2.17]{moeglinSpectralDecompositionEisenstein1995}. Let \(G\) be a reductive group over \(F\), we fix a Borel \(B\) and a standard parabolic \(P \) with a standard Levi decomposition \(P = MU\). We let \(K\) be a maximal compact subgroup of \(G(\A)\) that is in good position (\ref{max_compact_subgroup}).

For \(v\notin \infty\) a non-Archimedean place then we say that a function \(f: G(F_\nu) \to \C\) is smooth if it is locally constant in the induced topology on \(G(F_\nu)\), the details of this topology are spelled out in \cite{conradWeilGrothendieckApproaches2012}. The set of such smooth functions is denoted \(C^\infty(G(F_\nu))\). This suggests the definition of smooth functions on the ``finite adeles'' \(\A_f\) as 
	\[C^\infty(\mathbb{A}_f) \defeq \bigotimes^{}_{\nu \notin \infty} \phantom{}' C^\infty(G(F_\nu)). \]
	Thus for the full adeles we have the notion of smooth as an element of the following,
	\[C^\infty(\mathbb{A}_F) \defeq   C^\infty(G(\mathbb{A}_f))   \otimes   C^\infty(G(F_\infty)).\]
	Notice that a priori the codomain is an infinite tensor product over \C of copies of \C, which is isomorphic to \C. Thus we can conflate a smooth function with its composition along this isomorphism, and think of them as functions into \C.

	We still consider \(Z(\mathfrak{g})\) to be the center of the universal enveloping algebra of the Lie algebra at the infinite places, exactly as before. We define an action by linearly extending
    \[z.(f\tensor g) = f\tensor (z.g),\]
    i.e. it acts on the archimedean places as in the setting of Archimedean automorphic forms. 
	
	The definition of moderate growth carries over verbatim, however we change the set of places multiplied over to be all of them now.
    
    \begin{remark}[\cite{borelAutomorphicFormsRepresentations1979}, 1.II.3]
        The collection of moderate growth functions is independent of the choices of embedding. 
    \end{remark}

\begin{definition}
    A function \(\phi: U(\A)M(F)\backslash G(\A) \to \C\) is an \textbf{automorphic form} if it is smooth, moderate growth, \(Z(\mathfrak{g})\) and \(K\) finite. We will denote the set of these automorphic forms by \(\mathcal{A}(U(\A)M(F)\backslash G(\A))\).
\end{definition}

\begin{remark}
    It is important that \(M(F)\) is treated as a subgroup of \(M(\A)\) via the diagonal embedding.
\end{remark}
	
    

\section{Modular Forms}
One might ask if there is a special case in which automorphic forms yield modular forms. In fact no, the space of automorphic forms is larger than just modular forms, however it gives the space of Maas forms (or modular and Maas forms, depending on convention). This is well covered in the literature \cite{emertonCLASSICALMODULARFORMS}\cite[3.2]{bumpAutomorphicFormsRepresentations1997}\cite{booherVIEWINGMODULARFORMS}\cite{garrettTransitionEisensteinSeries2016}, but so essential to intuiting automorphic forms that we feel it is necessary to present the details here. To be clear we explain modular forms as Archimedean automorphic forms as we think it is where the connection is clearest. 

	Recall the definition of a modular form 
	\begin{Definition}[\cite{diamondFirstCourseModular2005} 1.1.2]
		A function
		\[\phi: \mathcal{H} \to \C,\]
		where \(\mathcal{H}\) is the upper half plane in \C, that is holomorphic, satisfies 
		\[\phi(\gamma.z) = (cz+d)^k\phi(z), \quad \gamma = \begin{pmatrix}
			a &b \\
			c &d
		\end{pmatrix}\in \SL_2(\Z),\]
		and extends holomorphically to \(\infty\) is called a modular form of weight k.
	\end{Definition}
	These are modular forms with trivial character and full level.

	Now give a function on a set \(X\) and an action of a group \(G\) on X, there is a general way of associating to \(\Hom(X, Y)\) a family of maps \(\Hom(G, Y)\) indexed by \(X\). This is a manifestation of the tensor-hom adjunction. Effectively if \(f: X\to Y\) the we get a map for each \(x\in X\) defined on \(f_x : G \to Y\) given by \(g\mapsto f(g.x)\).

	So for our purposes we are trying to take some subset of functions \(\mathcal{H} \to \C\) and shift their domain to the \(\Q_\infty = \R\) points of some reductive group. In particular it would be sufficient to find a reductive group with a well defined action on the upper half plane and in particular we would want the action to be transitive.

	\begin{Theorem}
		\[\mathcal{H} \cong  \SL_2(\R) / SO_2(\R) ,\]
		as topological spaces.
	\end{Theorem}
	\proofbar{
		Consider the action 
		\[\SL_2(\R) \curvearrowright \mathcal{H}: \;\; \begin{pmatrix}
			a & b\\
			c & d
		\end{pmatrix}.z = \frac{az + b}{cz + d}.\]
		Then look at the orbit of \(i\), namely 
		\[\begin{pmatrix}
			a & b\\
			 & d
		\end{pmatrix}.i = \frac{ai + b}{d} = a^2i + ab,\]
		which letting \(a, b\in \R\) vary is clearly surjective onto the whole upper half plane. So there is one orbit, and hence by the orbit stabiliser we know that 
		\[\mathcal{H} \cong \SL_2(\R) /stab(i) ,\]
		so we want to find
		\[stab(i) = \left\{ g = \begin{pmatrix}
			a & b\\
			c & d
		\end{pmatrix}\in \SL_2(\R) : g.i = i   \right\},\]
		in particular we solve 
		\begin{equation*}
			\begin{aligned}
				i &= g.i = \frac{ai + b}{ci + d} = (c^2 + d^2)\inv (ac + bd  + i\det g) .\\
			\end{aligned}
		\end{equation*}
		So equating coefficients we have 
		\[\det g (c^2 + d^2)\inv  = 1 \implies c^2 +  d^2 = \det g = 1,\] 
		on the other hand 
		\[ac + bd = 0.\]
		Now the pairs \(c^2 + d^2 = \det g = 1\) are parameterized by \(\theta\in [0, 2\pi)\) using \(c = \sin \theta, d =  \cos\theta\) hence subbing this into the above equation
		\[\frac{-b}{a} = \tan\theta,\]
		and so \(b = -k\sin\theta, a = k\cos\theta\) for some  \(k\in \R\) but the determinant must be \(1\) so \(k = 1\).
		Hence 
		\[stab(i) = \left\{ \begin{pmatrix}
			\cos\theta & -\sin\theta \\
			\sin\theta & \cos\theta
		\end{pmatrix} : \theta \in [0, 2\pi)\right\} = SO_2(\R).\]
		One then has to check that this is all continuous. 
	}
    \begin{remark}
        Sometimes for to make the action of certian (Hecke) operators more apparent this is exhibited as 
        \[\mathcal{H} \cong \GL_2^+(\R)/ A_{\GL_2}SO_2(\R).\]
        This obscures the connection with the reductive group setting however so we avoid it here. 
    \end{remark}


	
	
	\(\SL_2\) is a reductive group and \(SO_2(\R)\) is its maximal compact subgroup. This decomposition of the upper half plane suggests that function on it might have some invariance along the maximal compact subgroup of the reductive group \(\SL_2\). Indeed if we were to push our modular forms along this isomorphism it would, with the construction that we outlined earlier in terms of a group action on a set, exhibit this invariance. This is merely \textit{evidence} that if we were to change our modular forms to functions on the reductive group \(\SL_2\) they may preserve \textit{some} of that invariance and indeed be K-finite.
    
		% https://q.uiver.app/#q=WzAsNCxbMCwwLCJcXGJlZ2lue3BtYXRyaXh9eV57MS8yfSAmIHggeV57LTEvMn1cXFxcICYgeV57LTEvMn1cXGVuZHtwbWF0cml4fVNPXzIoXFxSKT1cXFNsXzIoXFxSKSJdLFsyLDAsIlxcU2xfMihcXFIpL1NPXzIoXFxSKSJdLFsyLDIsIlxcU2xfMihcXFopXFxzZXRtaW51c1xcU2xfMihcXFIpIl0sWzQsMCwiXFxtYXRoY2Fse0h9Il0sWzEsMywiXFxzaW0iXSxbMCwxLCJwcm9qIl0sWzEsMywieFxcbWFwc3RvIHguaSIsMl0sWzAsMiwiXFx0ZXh0e2Rlc2NlbmQ/Pz99IiwyXV0=
	\[\begin{tikzcd}[cramped]
		{\left\{\begin{pmatrix}y^{1/2} & x y^{-1/2}\\ & y^{-1/2}\end{pmatrix} : x,y\in \R, y\neq 0 \right\} SO_2(\R)=\SL_2(\R)} && {\SL_2(\R)/SO_2(\R)} && {\mathcal{H}} \\
		\\
		&& {\SL_2(\Z)\setminus\SL_2(\R)}
		\arrow["\sim", from=1-3, to=1-5]
		\arrow["\mathrm{project}", from=1-1, to=1-3]
		\arrow["{g\mapsto g.i}"', from=1-3, to=1-5]
		\arrow["{\text{project}}"', from=1-1, to=3-3]
	\end{tikzcd}\]

    Using something like the universal property of the quotient we can lift a function on \(\SL_2(\R) / SO_2(\R)\) to \(\SL_2(\R)\) however this is not \(\SL_2(\Z)\) invariant, thus we need to add a pre-factor to ensure this in our associated automorphic form. The algebro-geometric perspective in \cite{emertonCLASSICALMODULARFORMS} can make this seem slightly less ad hoc.Thus for \(f\) a modular form of weight k the following function on \(\SL_2(\R)\)
	\[F(g) \defeq  (ci + d)^{-k}f(g.i),\]
	we claim is an automorphic form for \(\SL_2(\Z)\). We take for granted its smoothness. The \(\SL_2(\Z)\) invariance is obvious from the modularity condition. It remains to show the three other properties:

	\begin{Lemma}
		\(F(g)\) is of moderate growth.
	\end{Lemma}
	\proofbar{
		Unraveling the definitions we require two constants such that 
		\[|F(g)| = |ci+ d|^{-k}|f(g.i)| \leq c(\sup_{i,j}(g, g\inv))^r,\]
		A direct computation shows that 
		\[Im(g.i) = |ci+ d|^{-2},\]
		hence we require to show
		\[ \mathrm{Im}(g.i)^{k/2}|f(g.i)| \leq c(\sup_{i,j}(g, g\inv))^r.\]
		\textcolor{red}{Somehow invoke polynomial growth...?}
        but the modularity condition has the growth condition that \(\lim_{x\to \infty}f(xi)\) be bounded. 
	}

	\begin{Lemma}
		\(SO_2(\R)\) is a maximal compact subgroup inside \(\SL_2(\R)\). \(F\) is an \(SO_2(\R)\) finite function.
	\end{Lemma}
	\proofbar{
		Using that \(\kappa \in K = SO_2(\R)\) acts trivially on \(i\), an elementary computation shows that for \(g  \in \SL_2(\R)\),
		\begin{equation*}
			\begin{aligned}
				F(g\kappa) = e^{-ik\theta}F(g). \\
			\end{aligned}
		\end{equation*}
		Hence \(F(g)\) is acted on by \(K\) via a one dimensional irreducible representation. In particular it is finite dimensional.
		}

	\begin{Lemma}
		\(F\) is a \(Z(\mathfrak{sl}_2)\) finite function.
	\end{Lemma}
	\proofbar{ Only a sketch. 

		The center of the universal enveloping algebra of the complexified Lie algebra is generated by the Casimir operators. From \cite{garrettInvariantDifferentialOperators2010} we know that the casimir is 
		\[\Omega = \frac{1}{2}H^2 + XY + YX.\]
		We have the coordinates on \(\begin{pmatrix}y^{1/2} & x y^{-1/2}\\ & y^{-1/2}\end{pmatrix}SO_2(\R)=\SL_2(\R)\) from \cite{bumpAutomorphicFormsRepresentations1997}[1.19 pg 139] in which 
		 the casimir acts as the differential operator
		\[\Delta = y^2\left(\left(\Dif{}{x}\right)^2 +\left(\Dif{}{y}\right)^2\right) - y\Dif{^2}{x\partial \theta},\] 
		\cite{bumpAutomorphicFormsRepresentations1997}[1.29 pg 143 ,Prop 2.2.5 pg 155]. Now we claim that F is an eigenfunction for this operator. 
		An element \((x,y,\theta) \defeq \begin{pmatrix}y^{1/2} & x y^{-1/2}\\ & y^{-1/2}\end{pmatrix}\kappa_\theta \in \SL_2(\R)\) acts on \(i\) by sending it to \(x+ iy\) (elementary computation). The bottom row of the product is \(y^{-1/2}\sin\theta ;y^{-1/2}\cos\theta \) which results in 
		\[F(x,y,\theta) = y^{k/2}e^{-ik\theta}f(x + iy).\]
		It is then a calculus exercise to apply \(\Delta\) to this, using the holomorphicity we also get that \(f_{xx} - f_{yy} = 0\) and \(f_y = if_x\) which cancels away terms and we get that 
		\[\Delta F(x,y,\theta) = \frac{k}{2}\left(\frac{k}{2} - 1\right) F(x,y,\theta).\]
		
		Therefore the dimension of \(Z(\mathfrak{g})F\) is simply one.
	}
    This example makes it clear that the two finiteness conditions for automorphic forms are in some sense functional equations that they must satisfy. 
	There is a nice explanation of how to lift this to the adelic setting in several places, however it is stated quite clearly in \cite[2.1]{cogdellLecturesLfunctionsConverse}

	\section{Constant Terms}
\subsection{Integration Lemmas}
\begin{Theorem}\label{integrate_unitary_char}
	If \(G\) is a locally compact Hausdorff group with a left Haar measure \(\mu\) and if \(\chi\colon G\to \mathbf C^\times\) is a non-trivial character on \(G\), then
	\[ \intof{G}{\chi(g)}{\mu(g)} = 0. \]
\end{Theorem}
\proofbar{
	Pick an element \(h\) of \(G\) such that \(\chi(h)\neq 1\).
	The equation above then follows from
	\[ \intof{G}{\chi(g)}{\mu(g)} = \intof{G}{\chi(hg)}{\mu(g)} = \intof{G}{\chi(h)\chi(g)}{\mu(g)} = \chi(h) \intof{G}{\chi(g)}{\mu(g)}. \square\]
}
    Integrating trivial characters gives the volume of the measure space which we typically normalise to be one.

\begin{Theorem}[\cite{garrettModernAnalysisAutomorphic} 5.2, \cite{follandCourseAbstractHarmonic2016} Thm 2.49]
        Let \(H\leq G\) be a closed subgroup. If \(H\setminus G\) has a right G invariant measure (iff their modular functions agree on H) then the integral is unique up to scalar, namely for a given Haar measures dh on H and dg on G there is a unique invariant measure dq on \(H\setminus G\) such that for all \(f\in C_c^0(G)\)
        \[\int_{H\setminus G}\int_H f(hq)dhdq = \int_G f(g) dg\]
    \end{Theorem}
    Note that this quotient may not be a group, because H is not required to be normal.
	
	
\section{Siegel Phi Operator}
Here we give an example of the constant term which connects it to the classical picture. We thank Chengjing Zhang for showing us this example, and present it here because we cannot find it in the literature. We deal only with the classical Siegel modular forms of full level and moreover are less explicit with the steps as they should be clear after exposure to the previous arguments. 

Because we are trying to connect this to the classical picture it is most convenient to think of things in the Archimedean places, recall the way that modular forms are automorphic forms most naturally in the archimedian sense (\cite[6.2]{getzIntroductionAutomorphicRepresentations2024}) \cite{emertonCLASSICALMODULARFORMS}\cite{bumpAutomorphicFormsRepresentations1997}\cite{booherVIEWINGMODULARFORMS}. So for this section alone, by automorphic form we will mean automorphic forms on the Archimedean places, and the constant term will be taken only on the Archimedean part: i.e. for \(f: G(\R) \to \C\) and automorphic its constant term along a parabolic of G, call it \(P=MN\), is \cite[8.6]{getzIntroductionAutomorphicRepresentations2024}
\[f(x)_P = \int_{N(\Z)\backslash N(\R)}f(xn) \mathrm{d}n.\]


\subsection{Siegel Modular Forms}
We collect some definitions from \cite{bruinier123ModularForms2008} to fix notation. Let the Siegal upper half plane be defined as 
\begin{equation*}
    \begin{aligned}
        \mathcal{H}_g &\defeq \{\tau \in \mathrm{M}_{g\times g}(\C) : \tau \text{ is symmetric and has positive definite imaginary part}\} \\
        & \cong \Sp_{2g}(\R) / U(g)
    \end{aligned}
\end{equation*}
where the isomorphism is as analytic manifolds  and 
\[U(g) \defeq \left\{\begin{pmatrix}
    A & B\\
    -B & D
\end{pmatrix}\in \Sp_{2g}(\R) : AA^t + BB^t = 1\right\}\]

For every \(\gamma= (A \; B; \; C\; D) \in \Sp_{2g}(\Z)\) and \(\tau \in \mathcal{H}_g\) we have the action
\[\gamma.\tau = (A\tau + B)(C\tau + D)\inv \]

We say that a holomorphic function \(f: \mathcal{H}_g \to \C\) is a (classical) Siegel modular form of weight \(k\) if 
\[f(\gamma.\tau) = \det(C\tau + D)^kf(\tau)\]
with the extra condition that if \(g = 1\) it must be holomorphic at \(\infty\). Because \(\Sp_2 = \SL_2\) this is a strict generalisation of an (elliptic) modular form.

The space of Siegel modular forms of weight \(k\) and genus g is denoted \(\mathcal{M}_k(\Sp_{2g}(\Z))\). There is a useful operator know as the Siegel Phi Operator which allows you to lift known modular forms from lower genus to higher genus \cite[5]{bruinier123ModularForms2008}
\[\mathcal{M}_k(\Sp_{2g}(\Z)) \xrightarrow{\Phi} \mathcal{M}_{k}(\Sp_{2(g-1)}(\Z))\]
defined by the limit for \(\tau\in \mathcal{H}_{g-1}\)
\[\Phi(f)(\tau) \defeq \lim_{t\to \infty} f\begin{pmatrix}
    \tau & \\
    & it 
\end{pmatrix}\]
in this context a cusp form is defined to be a Siegel modular form in the kernel of the Siegel \(\Phi\) operator and so it is natural to wonder if there is a constant term that is being taken here. 

\subsection{Automorphising}
Given a Siegel modular form \(f\in \mathcal{M}_k(\Sp_{2g}(\Z))\) we can associate an automorphic form
\[\tilde{f} : \Sp_{2g}(\R)\to\C, \qquad \begin{pmatrix} a & b\\ c & d\end{pmatrix}\mapsto \det(ci+d)^{-k} f\Bigl((ai+b)(ci+d)\inv\Bigr), \]
where \(a,b,c,d\) are \(g\times g\) matrices such that \(\bigl(\begin{smallmatrix} a & b\\ c & d\end{smallmatrix}\bigr)\in\Sp_{2g}(\R)\).Fix the Borel of upper triangular matrices. Now for \(1\leq r\leq g-1\) an integer we have the standard maximal parabolic of \(\Sp_{2g}\), \(P_r = M_rN_r\) such that 
\[M_r \cong \GL_r\times \Sp_{2(g-r)}\]

\begin{Theorem}[Zhang]
	If \(f\) is a classical Siegel modular form of weight \(k\) and degree \(g\), then
	\begin{equation} 
		\tilde f_{P_r}(u\gamma) = \det u^k\cdot (\Phi^{r} f)^\sim(\gamma)
	\end{equation}
	for every element \(\gamma\) of \(\Sp_{2(g-r)}(\R)\) and every element \(u\) of \(\GL_{r}(\R)\).

 In particular 
 \[\tilde{f}_{P_{g-1}}\begin{pmatrix} a & 0 & b & 0\\ 0 & 1 & 0 & 0\\ c & 0 & d & 0\\ 0 & 0 & 0 & 1\end{pmatrix} = (\Phi f)^\sim\begin{pmatrix}
     a & b\\
     c& d
 \end{pmatrix}\]
\end{Theorem}

This shows that perhaps the correct generalisation of the Siegel Phi function is just the constant term that we all know and love. We could also attempt to expand this to Siegel modular forms that are vector valued or not of full level. 

The only other work on generalising the Siegel \(\Phi\) operator that we could find appears in \cite{grenierANALOGUESIEGELXOPERATOR2024}. 
Grenier formulates the \(\Phi\) operator in the language of symmetric spaces \cite[Ch. 2]{terrasHarmonicAnalysisSymmetric2016} and then shows that the analogous definition in the case of ``automorphic forms'' in the sense of the symmetric space \(\mathscr{P}_n/\GL_n(\Z)\) of symmetric positive definite real matrices \cite[1.5.1]{terrasHarmonicAnalysisSymmetric2016} behaves in the same way. Namely his \cite[Thm. 2]{grenierAnalogueSiegelFOperator1992} shows that it sends an automorphic form for \(
GL_n(\Z)\) to an automorphic form for \(\GL_{n-1}(\Z)\). The point is that the \(\Phi\) operator can be defined in the generality of symmetric spaces and Grenier shows that at least in one other case it still preserves the relevant notion of automorphic form. This suggests two things that would be interesting to investigate; using the classification of symmetric spaces is it possible to give a uniform definition of the \(\Phi\) operator following Grenier and does this definition agree with the constant term in the way that the Siegel \(\Phi\) operator does. With my limited knowledge of symmetric spaces this seems to be very tractable.

\subsection{Base Case}
The base case is very instructive, it deals with modular forms. So consider \(f\) a (elliptic) modular form of full level and weight k, which has a Fourier expansion given by 
\[f(z) = \sum_{n\geq 0} a_ne^{2\pi i nz }\]
Then one can verify that
\[\tilde f \begin{pmatrix}
    a & b\\
    c & d
\end{pmatrix} = (ci+d)^{-k} f\Bigl(\frac{ai+b}{ci+d}\Bigr)\]
is an automorphic form on \(\Sp_2\). The only non-trivial parabolic P is the one of upper triangular matricies, with Levi and unipotant given respectively 
\[M = \begin{pmatrix} m & 0\\ 0 & m^{-1}\end{pmatrix}\cong \GL_1 , \;\;\; N = \begin{pmatrix} 1 & b\\ 0 & 1\end{pmatrix} \cong \mathbb{G}_a\]
along which we can now compute the constant term 
\begin{equation*}
    \begin{aligned}
		\tilde f_P(m)
		& = \int_{N(\Z)\backslash N(\R)}\tilde f (mb) \mathrm{d}b\\
		& =  \int_{\Z\backslash\R}\tilde f \begin{pmatrix} m & mb\\ 0 & m^{-1}\end{pmatrix}\mathrm{d}b\\
		& = \int_{\Z\backslash\R} m^k f(m^2i+m^2b) \mathrm{d}b\\
            & = m^k a_0 \\
	\end{aligned}
\end{equation*}
We have chosen normalisation to remove the usual factor of \(1/2\pi\) in the constant term of the Fourier series. Moreover we see that
\[\Phi(f)= \lim_{t\to \infty} f(it) =\lim_{t\to \infty} \sum_{n\geq 0} a_ne^{-2\pi nt }  =  a_0\]

\subsection{Simplifying the Constant Term}
As we saw in \ref{maximal_parabolic} for \(1\leq r\leq g-1\) an integer we have the standard maximal parabolic of \(\Sp_{2g}\), \(P_r = M_rN_r\) such that 
\[M_r \cong \GL_r\times \Sp_{2(g-r)}\]
which can be given the explicit matrix representations 
    \[m(\gamma, A) \defeq \begin{pmatrix}
        A &&& \\
         &a&&b \\
         &&(A^t)\inv& \\
         &c&&d \\
    \end{pmatrix}, \;\;\; A\in \GL_r(F), \; \gamma = \begin{pmatrix}
        a & b\\
        c & d \\
    \end{pmatrix} \in \Sp_{2(g-r)}(F) \]

    and unipotent 
    \[ n(s;h,k) \defeq \begin{pmatrix} 1 & 0 & 0 & h\\ -k^t & 1 & h^t & s+h^t k\\ 0 & 0 & 1 & k\\ 0 & 0 & 0 & 1 \end{pmatrix}, \;\;\; h, k\in\mathrm{Mat}_{(g-r)\times r}(\R),\; s\in\mathrm{Sym}_{r}(\R)\]
We have the following short exact sequence \todo[inline]{prove it}
\[ 1\to \mathrm{Sym}_{r}(\R)\to N_r(\R)\to \mathrm{Mat}_{(g-r)\times r}(\R)\times\mathrm{Mat}_{(g-r)\times r}(\R) \to 1. \]
which we will use to unfold our integral below, for compactness we define \(H_r \defeq \mathrm{Mat}_{(g-r)\times r}\). We will now denote \([G] \defeq G(\Z)\backslash G(\R)\) and compute the constant term
\begin{align}
		\tilde f_{P_r}\bigl(m(\gamma, A)\bigr)
		& = \intof{[N_r]}{\tilde f\bigl(n m(\gamma, A)\bigr)}{n} \notag\\
		& = \intof{[H_r\times H_r]}{\intof{[\mathrm{Sym}_{g-r}]}{\tilde f\bigl(n(s; h, k) m(\gamma, A)\bigr)}{s}}{(h,k)} \notag\\
		& = \intof{[H_r]}{\intof{[H_r]}{\intof{[\mathrm{Sym}_{g-r}]}{\tilde f\bigl(n(s; h, k) m(\gamma, A)\bigr)}{s}}{h}}{k}.
\end{align} 

Now we focus on simplifying the integrand. We want an explicit form of the matrix so we can relate it back to the value of the un-lifted Siegel modular form \(f\); simply multiply the matrices gives, where (all rings are commutative) \(A^{-t} \defeq (A^t)\inv\)
\[
		n(s; h, k) m(\gamma, A) =
		\begin{pmatrix}
			a & 0 & b & h A^{-t}\\
			-k^t a + h^t c & A & -k^t b + h^t d & s A^{-t} + h^t k A^{-t}\\
			c & 0 & d & k A^{-t}\\
			0 & 0 & 0 & A^{-t}
		\end{pmatrix}.
	\]
because \(a,b,c,d \in \mathrm{Mat}_{(g-r)\times (g-r)}, A \in \mathrm{Mat}_{r\times r}\) we see that the \(g\times g\) blocks that we now need to take the determinant of are the \(4\times 4\) corners of this picture, hence the matrices below should all be in \(\mathcal{H}_g\subseteq \mathrm{Mat}_{g\times g}\)

\begin{align*}
    \tilde{f}(n(s; h, k) m(\gamma, A) ) &= \det\left(\begin{pmatrix}
    c & 0 \\
    0 & 0
\end{pmatrix}i+ \begin{pmatrix}
    d & kA^{-t} \\
    0 & A^{-t}
\end{pmatrix} \right)^{-k} \cdot \\
&f\left(         \left(\begin{pmatrix}
    a & 0\\
    -k^ta + h^tc & A
\end{pmatrix}i+\begin{pmatrix}
    b & hA^{-t} \\
    -k^tb + h^td & sA^{-t} + h^tkA^{-t}
\end{pmatrix}\right)   \left(\begin{pmatrix}
    c & 0 \\
    0 & 0
\end{pmatrix}i+ \begin{pmatrix}
    d & kA^{-t} \\
    0 & A^{-t}
\end{pmatrix} \right)\inv      \right) \\
&= \det\left( \begin{pmatrix}
    ic + d & kA^{-t} \\
    0 & A^{-t}
\end{pmatrix}\right)^{-k} \cdot \\
&f\left( \begin{pmatrix}
    ia + b &  hA^{-t}\\
    -k^t(ia +b) + h^t(d + ic) & iA + sA^{-t} + h^tkA^{-t}
\end{pmatrix}  \begin{pmatrix}
    ic + d & kA^{-t} \\
    0 & A^{-t}
\end{pmatrix}\inv      \right) \\
&=\left(\frac{\det(ic + d)}{\det(A)}\right)^{-k} \cdot \\ &f\left( \begin{pmatrix}
    ia + b &  hA^{-t}\\
    -k^t(ia +b) + h^t(d + ic) & iA + sA^{-t} + h^tkA^{-t}
\end{pmatrix}   \begin{pmatrix}(ci+d)^{-1} & -(ci+d)^{-1} k\\ 0 & A^t \end{pmatrix} \right) \\
&= \left(\frac{\det(A)}{\det(ic + d)}\right)^{k} f\begin{pmatrix} \tau & -\tau k + h\\ -k^t \tau + h^t & k^t \tau k + A A^t i + s \end{pmatrix}, \;\;\; \tau \defeq (ai+b)(ci + d)\inv \\
%&= \left(\frac{\det(A)}{\det(ic + d)}\right)^{k} f\bigl(n(s; h, k) m(\gamma,A )(i 1_g)\bigr) I think g here should be 2g but then the multiplication seems wrong IDK what he meant by this..
\end{align*} 
So we have shown that 
\begin{align*}
    \tilde f_{P_r}\bigl(m(\gamma, A)\bigr) &= \intof{[H_r]}{\intof{[H_r]}{\intof{[\mathrm{Sym}_{g-r}]}{  \left(\frac{\det(A)}{\det(ic + d)}\right)^{k} f\begin{pmatrix} \tau & -\tau k + h\\ -k^t \tau + h^t & k^t \tau k + A A^t i + s \end{pmatrix}   }{s}}{h}}{k}\\
     &= \left(\frac{\det(A)}{\det(ic + d)}\right)^{k} \intof{[H_r]}{\intof{[H_r]}{\intof{[\mathrm{Sym}_{g-r}]}{  f\begin{pmatrix} \tau & -\tau k + h\\ -k^t \tau + h^t & k^t \tau k + A A^t i + s \end{pmatrix}   }{s}}{h}}{k}\\
\end{align*}

Again lets focus on this integrand \(f\begin{pmatrix} \tau & -\tau k + h\\ -k^t \tau + h^t & k^t \tau k + A A^t i + s \end{pmatrix}\) and compute its Fourier expansion, see \cite[3.4]{bruinier123ModularForms2008}. Recall that a symmetric matrix \(n\in \GL_g(\Q)\) is called half integral if \(2n\) is integral with even diagonal entries, then a Siegel modular form has a Fourier expansion of the form
\[f(z) = \sum_{n \text{ half integral}}a(n) e^{2\pi i \mathrm{Tr}(nz)} \]
First the space of half integral \(g\times g\) matrices, \(\mathrm{HI}_g\), decomposes as a direct sum via the (additive) group isomorphism \todo[inline]{prove it}
\[ \mathrm{HI}_{g-r} \oplus \tfrac{ 1}{ 2} \mathrm{Mat}_{ r\times (g-r)}(\Z) \oplus \mathrm{HI}_{r}\to\mathrm{HI}_g, \qquad (n, m, l)\mapsto \begin{pmatrix} n & m\\ m^t & l \end{pmatrix}, \]
thus unfolding the (discrete) integral we get 
\begin{align*}
    f\begin{pmatrix} \tau & -\tau k + h\\ -k^t \tau + h^t & k^t \tau k + A A^t i + s \end{pmatrix} &=   \sum_{n\in\mathrm{HI}_{g-r}} \sum_{m\in\frac{1}{2} \mathrm{Mat}_{ r\times (g-r)}(\Z)} \sum_{l\in\mathrm{HI}_{r}} a\begin{pmatrix} n & m\\ m^t & l \end{pmatrix} \\
    &\exp \left(2\pi i \mathrm{Tr} \begin{pmatrix} n & m\\ m^t & l \end{pmatrix}\begin{pmatrix} \tau & -\tau k + h\\ -k^t \tau + h^t & k^t \tau k + A A^t i + s \end{pmatrix} \right)  \\
\end{align*}
because all the block sizes are compatible we can ``block multiply'' the inner matrices and because we are taking the trace we can forget about off diagonal entries
\begin{align*}
    \begin{pmatrix} n & m\\ m^t & l \end{pmatrix}\begin{pmatrix} \tau & -\tau k + h\\ -k^t \tau + h^t & k^t \tau k + A A^t i + s \end{pmatrix} &= 
    \begin{pmatrix} n\tau + m(-k^t \tau + h^t ) & \ast\\ \ast & m^t(-\tau k + h) + l( k^t \tau k + A A^t i + s) \end{pmatrix}
\end{align*}
putting this into our Fourier expansion
\begin{align*}
    f\begin{pmatrix} \tau & -\tau k + h\\ -k^t \tau + h^t & k^t \tau k + A A^t i + s \end{pmatrix} &= \sum_{n} \sum_{m} \sum_{l} a\begin{pmatrix} n & m\\ m^t & l \end{pmatrix} \\
    &\exp \left(2\pi i (\mathrm{Tr} (n\tau) +  \mathrm{Tr} (m(-k^t \tau + h^t )) +  \mathrm{Tr} (m^t(-\tau k + h)) +  \mathrm{Tr} (l( k^t \tau k + A A^t i + s))) \right)  \\
\end{align*}

If we denote \(T_l \defeq \mathrm{Tr} (l( k^t \tau k + A A^t i + s))\) and \todo[inline]{my Tm differs from Chengjing}
\[T_{m} \defeq \mathrm{Tr} (m(-k^t \tau + h^t )) +  \mathrm{Tr} (m^t(-\tau k + h)) = \mathrm{Tr}(-mk^t\tau - m^t\tau k) + \mathrm{Tr}(mh^t + m^th) \defeq T_{m,k} + T_{m,h}\]
we can substitute this back into our constant term\todo[inline]{Converges uniformly a priori on compact sets, well I don't know if I can swap all these sums haha}
\begin{align*}
    \tilde f_{P_r}\bigl(m(\gamma, A)\bigr)
     &= \left(\frac{\det(A)}{\det(ic + d)}\right)^{k} \intof{[H_r]}{\intof{[H_r]}{\intof{[\mathrm{Sym}_{g-r}]}{ \sum_{n} \sum_{m} \sum_{l} a\begin{pmatrix} n & m\\ m^t & l \end{pmatrix}\exp \left(2\pi i (\mathrm{Tr} (n\tau) +  T_m + T_l) \right)
     }{s}}{h}}{k}\\
     &= \left(\frac{\det(A)}{\det(ic + d)}\right)^{k} \sum_{n} \sum_{m} \sum_{l} a\begin{pmatrix} n & m\\ m^t & l \end{pmatrix}e^{2\pi i \mathrm{Tr} (n\tau)}
     \intof{[H_r]}{\intof{[H_r]}{\intof{[\mathrm{Sym}_{g-r}]}{  e^{2\pi i(T_m + T_l)}  }{s}}{h}}{k}\\
     &= \left(\frac{\det(A)}{\det(ic + d)}\right)^{k} \sum_{n} \sum_{m} \sum_{l} a\begin{pmatrix} n & m\\ m^t & l \end{pmatrix}e^{2\pi i \mathrm{Tr} (n\tau)}
     \intof{[H_r]}{\intof{[H_r]}{e^{2\pi iT_m} \intof{[\mathrm{Sym}_{g-r}]}{  e^{2\pi iT_l}  }{s}}{h}}{k}\\
     &= \left(\frac{\det(A)}{\det(ic + d)}\right)^{k} \sum_{n} \sum_{m} \sum_{l} a\begin{pmatrix} n & m\\ m^t & l \end{pmatrix}e^{2\pi i \mathrm{Tr} (n\tau)}
     \intof{[H_r]}{e^{2\pi iT_{m,k}}\intof{[H_r]}{e^{2\pi i T_{m,h}} \intof{[\mathrm{Sym}_{g-r}]}{  e^{2\pi iT_l}  }{s}}{h}}{k}\\
\end{align*}
Now we use that the integration of unitary characters is very simple \ref{integrate_unitary_char} and the fact that 
\[s\mapsto  e^{2\pi iT_l} \]
is a non-trivial unitary character of \(\mathrm{Sym}_{g-r}\) whenever \(l\neq 0\) to get that 
\[\intof{[\mathrm{Sym}_{g-r}]}{  e^{2\pi iT_l}  }{s} = \begin{cases}
    1, & l=0 \\
    0, & l\neq 0
\end{cases}\]
we repeat this trick with the second integral, which enforces that \(m = 0\) and end up with 
\begin{align*}
    \tilde f_{P_r}\bigl(m(\gamma, A)\bigr)
     &=\left(\frac{\det(A)}{\det(ic + d)}\right)^{k} \sum_{n\in\mathrm{HI}_{g-r}} a\begin{pmatrix} n & 0\\ 0 & 0 \end{pmatrix}e^{2\pi i \mathrm{Tr} (n\tau)}\\
\end{align*}
but by \cite[3.5]{bruinier123ModularForms2008} we know that the Fourier expansion of the Siegel Phi operator is 
\[(\Phi^{r} f)(\tau) = \sum_{n\in\mathrm{HI}_{g-r}} a\begin{pmatrix} n & 0\\ 0 & 0 \end{pmatrix} e^{2\pi i\mathrm{Tr}(n \tau)}.\]
hence 
\begin{align*}
    \tilde f_{P_r}\bigl(m(\gamma, A)\bigr)
     &=\left(\frac{\det(A)}{\det(ic + d)}\right)^{k} \Phi^r(f)(\tau)\\
     &= \det(A)^k (\Phi^r(f))^{\sim}(\gamma)
\end{align*}
which concludes the proof.
\begin{FlushRight}
     \(\square\)
\end{FlushRight}