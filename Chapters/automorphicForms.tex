There are different definitions of the words automorphic forms floating around, here we fix one and then explain how they generalize the classical modular forms.

The story starts with the classical modular forms, or functions on the upper half plane that satisfy some invariance conditions and differential equations. This evolves into the notions of Maas form on symmetric spaces and eventually reaches its apotheosis in the concept of automorphic form that we will present here. 

We will present two notions of automorphic form here. In the literature they are both called ``automorphic forms'' however here we will distinguish those that are defined only on the Archimedean points as ``Archimedean automorphic forms'' for clarity.

The first natural question is if there is a special case of automorphic forms which yield modular forms. In fact no, the space of automorphic forms is larger than just modular forms, however it gives the space of Maas forms (or modular and Maas forms, depending on convention). This is well covered in the literature \cite{emertonCLASSICALMODULARFORMS}\cite[3.2]{bumpAutomorphicFormsRepresentations1997}\cite{booherVIEWINGMODULARFORMS}\cite{garrettTransitionEisensteinSeries2016}. We explain modular forms as Archimedean automorphic forms as we think it is where the connection is clearest. We will give an example of modular forms as adelic automorphic forms when we come to the Eisenstein series in section \ref{sec:classic-eisenstein}.




\section{Archimedean Automorphic Form}
Fix a number field \(F\) and a classical group \(G\) defined over \(F\). Let \(\infty\) denote the set of Archimedean places. We denote \(\A_\infty = F_\infty \defeq \prod_{\nu\in\infty} F_\nu\) and note that \(G(F_\infty) \cong \prod_{\nu\in\infty}G(F_\nu)\). Consider \(\nu\in \infty\) one such Archimedean place, then \(F_\nu\) is either \R or \C. In particular (the analytification of) \(G(F_\nu)\) is a Lie group and we call a function, \(\phi: G(F_\nu) \to \C\), \textbf{smooth}  if it is smooth in the sense of manifolds. The collection of such smooth functions on \(G(F_\infty)\) will be denoted \(C^\infty(G(F_\infty))\).

    Because \(G(F_\infty)\) is a Lie group we know how to define its Lie algebra and we now denote \(Z(\mathfrak{g})\) the center of the \textit{universal enveloping algebra} of the \textit{complexification} of \(\mathfrak{g}\), it would be more reasonable to use \(Z(\mathcal{U}(\mathfrak{g}_{\C}))\) but that is too cumbersome so we follow the tradition. 
    A vector in a \(Z(\mathfrak{g})\)-module \(\phi\in V\) is called \(Z(\mathfrak{g})\)-\textbf{finite} if the space \(Z(\mathfrak{g})\phi\) is finite dimensional. 

	Let \(K_\infty\subseteq  G(F_\infty)\) be a maximal compact subgroup. Then again an element of a \(K_\infty\)-module is \(K_\infty\) \textbf{finite} if its orbit is a finite dimensional vector space space (we think here of \(\C[K_\infty]\)-modules).

	To define automorphic forms we look at the representation \(C^\infty(G(F_\infty))\) with the right regular action of \(K_\infty\), i.e. \(g.f(x) = f(xg)\).  In particular the \(Z(\mathfrak{g})\) module structure is induced from the action of \(\mathfrak{g}\) on \(C^\infty(G(F_\infty))\) by \label{lie_algebra_action}
	\[z.F(g) = \Dif{}{t}F(ge^{tz})|_{t=0}.\] 
	
	Finally we want a growth condition. Fix an embedding \(\iota : G\to GL_n\) which gives another embedding \(G\to SL_{2n}\) via
	\[\iota': g\mapsto \begin{pmatrix}
		\iota (g) & \\
		& (\iota (g))^{-t}
	\end{pmatrix}.\]
	We have denoted the inverse of the transpose \(-t\). A function \(\phi : G(F_\infty) \to \C \) is of \textbf{moderate growth} if there are constants \((c,r)\in \R_{>0}\times \R\) such that 
	\[|\phi(g)| \leq c\norm{g}^r = c \left(\prod_{v\in\infty} \sup_{1\leq i, j\leq 2n} |\iota' (g)_{i,j, \nu}|_\nu\right)^r.\]
	This is taking the maximum of the \(2n\times 2n \times |\infty| \) three dimensional matrix. 

	\begin{remark}
		One can define norms on \(G(\A)\) via the linearisation of such groups, i.e. via their representations. Concretely if \(\sigma\) is a finite dimensional complex representation on some Hilbert space with a \(K_\infty\) invariant inner product and \(\ast\) is the adjoint matrix with respect to this Hilbert space structure then a \textbf{norm} on \(G(\A)\) is a function of the form
		\[g\mapsto \big(\mathrm{tr }\; \sigma(g)^*\sigma(g)\big)^{\frac{1}{2}}.\]
		This moderate growth condition is then equivalent to some norm \(\norm{ - }\) existing on \(G(F_\infty)\) such that 
		\[|\phi(x)| \leq C\norm{x}^n,\]
		for some \(C>0, n\in \N\) and all \(x\in G(F_\infty)\). This is also equivalent to all such norms satisfying this condition \cite[1.2]{borelAutomorphicFormsRepresentations1979}.
	\end{remark}

	\begin{Definition}
		Let \(\Gamma\leq G(F_\infty)\) some (arithmetic) subgroup, an \textbf{automorphic form for \(\Gamma\)} is a smooth function of moderate growth 
		\[\phi: G(F_\infty) \to \C,\]
		that is \(K_\infty\) and \(Z(\mathfrak{g})\) finite with a (left) \(\Gamma\) invariance. We denote the set of these ``Archimedean'' automorphic forms by \(\mathcal{A}(\Gamma \backslash G(F_\infty))\).
	\end{Definition}


\section{Adelic Automorphic Form}
Here we follow \cite[I.2.17]{moeglinSpectralDecompositionEisenstein1995}. Fix a Borel \(B\subseteq G\) and a standard parabolic \(B\subseteq P \subseteq G\) with a standard Levi decomposition \(P = MU\). We let \(K\) be a maximal compact subgroup of \(G(\A)\) that is in good position as in section \ref{max_compact_subgroup}.

For \(v\notin \infty\) a non-Archimedean place then we say that a function \(f: G(F_\nu) \to \C\) is smooth if it is locally constant on \(G(F_\nu)\), in the Hausdorff topology. The set of such smooth functions is denoted \(C^\infty(G(F_\nu))\). This suggests the definition of smooth functions on the ``finite adeles'' \(\A_f\) as 
\todo[inline, color=green]{Clarify what this restricted tensor product is}
	\[C^\infty(\mathbb{A}_f) \defeq \bigotimes^{}_{\nu \notin \infty} \phantom{}' C^\infty(G(F_\nu)). \]
	Thus for the full adeles we have the notion of smooth as an element of the following,
	\[C^\infty(\mathbb{A}_F) \defeq   C^\infty(G(\mathbb{A}_f))   \otimes   C^\infty(G(F_\infty)).\]
	Notice that a priori the codomain is an infinite tensor product over \C of copies of \C, this is \textit{canonically} isomorphic to \C, thus we can conflate a smooth function with its composition along this isomorphism and think of them as functions into \C.

	We still consider \(Z(\mathfrak{g})\) to be the center of the universal enveloping algebra of the complexified Lie algebra at the infinite places, exactly as before. We define an action by linearly extending
    \[z.(f\tensor g) = f\tensor (z.g),\]
    i.e. it acts on the archimedean places as in the setting of Archimedean automorphic forms. 
	
	The definition of moderate growth carries over verbatim, however we change the set of places multiplied over to be all of them now.
    
    \begin{remark}[\cite{borelAutomorphicFormsRepresentations1979}, 1.II.3]
        The collection of moderate growth functions is independent of the choices of embedding. 
    \end{remark}

\begin{definition}
    A function \(\phi: U(\A)M(F)\backslash G(\A) \to \C\) is an \textbf{automorphic form} if it is smooth, moderate growth, \(Z(\mathfrak{g})\) and \(K\) finite. We will denote the set of these automorphic forms by \(\mathcal{A}(U(\A)M(F)\backslash G(\A))\).
\end{definition}

\begin{remark}
    It is important that \(M(F)\) is treated as a subgroup of \(M(\A)\) via the diagonal embedding.
\end{remark}
\begin{remark}
	What we have called automorphic forms are sometimes referred to as ``smooth K-finite automorphic forms'' \cite[2.2]{cogdellLecturesLfunctionsConverse}.
\end{remark}
	
    

\section{Modular Forms} \label{sec:modular-forms}

	Recall the definition of a \textbf{modular form of weight k} (of full level and trivial character) \cite[1.1.2]{diamondFirstCourseModular2005} as a function
		\[\phi: \mathcal{H} \to \C,\]
		where \(\mathcal{H}\) is the upper half plane in \C, that is holomorphic, satisfies 
		\[\phi(\gamma.z) = (cz+d)^k\phi(z), \quad \gamma = \begin{pmatrix}
			a &b \\
			c &d
		\end{pmatrix}\in \SL_2(\Z),\]
		and extends holomorphically to \(\infty\).

	We want to think of the upper half plane as a quotient of the \(\Q_\infty = \R\) points of some reductive group. If we have a transitive action of some reductive group then by the orbit stabiliser theorem we would have a bijection of sets.

	\begin{Theorem}
		\[\mathcal{H} \cong  \SL_2(\R) / SO_2(\R) ,\]
		as sets.
	\end{Theorem}
	\proofbar{
		Consider the action 
		\[\SL_2(\R) \curvearrowright \mathcal{H}: \;\; \begin{pmatrix}
			a & b\\
			c & d
		\end{pmatrix}.z = \frac{az + b}{cz + d}.\]
		Then look at the orbit of \(i\), namely 
		\[\begin{pmatrix}
			a & b\\
			 & d
		\end{pmatrix}.i = \frac{ai + b}{d} = a^2i + ab,\]
		which letting \(a, b\in \R\) vary is clearly surjective onto the whole upper half plane. So there is one orbit, and hence by the orbit stabiliser we know that 
		\[\mathcal{H} \cong \SL_2(\R) /stab(i) ,\]
		so we want to find
		\[stab(i) = \left\{ g = \begin{pmatrix}
			a & b\\
			c & d
		\end{pmatrix}\in \SL_2(\R) : g.i = i   \right\},\]
		in particular we solve 
		\begin{equation*}
			\begin{aligned}
				i &= g.i = \frac{ai + b}{ci + d} = (c^2 + d^2)\inv (ac + bd  + i\det g) .\\
			\end{aligned}
		\end{equation*}
		So equating coefficients we have 
		\[\det g (c^2 + d^2)\inv  = 1 \implies c^2 +  d^2 = \det g = 1,\] 
		on the other hand 
		\[ac + bd = 0.\]
		Now the pairs \(c^2 + d^2 = \det g = 1\) are parameterized by \(\theta\in [0, 2\pi)\) using \(c = \sin \theta, d =  \cos\theta\) hence subbing this into the above equation
		\[\frac{-b}{a} = \tan\theta,\]
		and so \(b = -k\sin\theta, a = k\cos\theta\) for some  \(k\in \R\) but the determinant must be \(1\) so \(k = 1\).
		Hence 
		\[stab(i) = \left\{ \begin{pmatrix}
			\cos\theta & -\sin\theta \\
			\sin\theta & \cos\theta
		\end{pmatrix} : \theta \in [0, 2\pi)\right\} = SO_2(\R).\]
		One then has to check that this is all continuous. 
	}
    \begin{remark}
    	This can be beefed up to an isomorphism of complex analytic spaces. 
        Sometimes to make the action of certain (Hecke) operators more apparent this is exhibited as 
        \[\mathcal{H} \cong \GL_2^+(\R)/ A_{\GL_2}SO_2(\R).\]
        This obscures the connection with the reductive group setting however so we avoid it here. 
    \end{remark}

	\(\SL_2\) is a reductive group and \(SO_2(\R)\) is its maximal compact subgroup. This decomposition of the upper half plane suggests that function on it might have some invariance along the maximal compact subgroup of the reductive group \(\SL_2\).
    
		% https://q.uiver.app/#q=WzAsNCxbMCwwLCJcXGJlZ2lue3BtYXRyaXh9eV57MS8yfSAmIHggeV57LTEvMn1cXFxcICYgeV57LTEvMn1cXGVuZHtwbWF0cml4fVNPXzIoXFxSKT1cXFNsXzIoXFxSKSJdLFsyLDAsIlxcU2xfMihcXFIpL1NPXzIoXFxSKSJdLFsyLDIsIlxcU2xfMihcXFopXFxzZXRtaW51c1xcU2xfMihcXFIpIl0sWzQsMCwiXFxtYXRoY2Fse0h9Il0sWzEsMywiXFxzaW0iXSxbMCwxLCJwcm9qIl0sWzEsMywieFxcbWFwc3RvIHguaSIsMl0sWzAsMiwiXFx0ZXh0e2Rlc2NlbmQ/Pz99IiwyXV0=
		
		\vspace{6mm}
		
	\[\begin{tikzcd}[cramped, transform canvas={scale=0.8}]
		{\left\{\begin{pmatrix}y^{1/2} & x y^{-1/2}\\ & y^{-1/2}\end{pmatrix} : x,y\in \R, y\neq 0 \right\} SO_2(\R)=\SL_2(\R)} && {\SL_2(\R)/SO_2(\R)} && {\mathcal{H}} \\
		\\
		&& {\SL_2(\Z)\setminus\SL_2(\R)}
		\arrow["\sim", from=1-3, to=1-5]
		\arrow["\mathrm{project}", from=1-1, to=1-3]
		\arrow["{g\mapsto g.i}"', from=1-3, to=1-5]
		\arrow["{\text{project}}"', from=1-1, to=3-3]
	\end{tikzcd}\]
	
		\vspace{12mm}
		
   	We can lift a function on \(\SL_2(\R) / SO_2(\R)\) to \(\SL_2(\R)\) by composing with the projection, however this is not \(\SL_2(\Z)\) invariant, thus we need to add a pre-factor to ensure this in our associated automorphic form. The algebro-geometric perspective in \cite{emertonCLASSICALMODULARFORMS} can make this seem slightly less ad hoc. 
   	\todo[inline,color=green]{does it? Am I trivialising a line bundle...?}
   	Thus for \(f\) a modular form of weight k the following function on \(\SL_2(\R)\)
	\[F(g) \defeq  (ci + d)^{-k}f(g.i),\]
	we claim is an automorphic form for \(\SL_2(\Z)\). We take for granted its smoothness. The \(\SL_2(\Z)\) invariance is obvious from the modularity condition. It remains to show the three other properties:

	\begin{Lemma}
		\(F(g)\) is of moderate growth.
	\end{Lemma}
	\proofbar{
		Unraveling the definitions we require two constants such that 
		\[|F(g)| = |ci+ d|^{-k}|f(g.i)| \leq c(\sup_{i,j}(g, g\inv))^r,\]
		A direct computation shows that 
		\[Im(g.i) = |ci+ d|^{-2},\]
		hence we require to show
		\[ \mathrm{Im}(g.i)^{k/2}|f(g.i)| \leq c(\sup_{i,j}(g, g\inv))^r.\]
		\textcolor{red}{Somehow invoke polynomial growth...?}
		\todo[inline,color=blue]{Fill}
        but the modularity condition has the growth condition that \(\lim_{x\to \infty}f(xi)\) be bounded. 
	}

	\begin{Lemma}
		\(SO_2(\R)\) is a maximal compact subgroup inside \(\SL_2(\R)\). \(F\) is an \(SO_2(\R)\) finite function.
	\end{Lemma}
	\proofbar{
		Using that \(\kappa \in K = SO_2(\R)\) acts trivially on \(i\), an elementary computation shows that for \(g  \in \SL_2(\R)\),
		\begin{equation*}
			\begin{aligned}
				F(g\kappa) = e^{-ik\theta}F(g). \\
			\end{aligned}
		\end{equation*}
		Hence \(F(g)\) is acted on by \(K\) via a one dimensional irreducible representation. In particular it is finite dimensional.
		}

	\begin{Lemma}
		\(F\) is a \(Z(\mathfrak{sl}_2)\) finite function.
	\end{Lemma}
	\proofbar{ Only a sketch. 

		The center of the universal enveloping algebra of the complexified Lie algebra is generated by the Casimir operators. From \cite{garrettInvariantDifferentialOperators2010} we know that the casimir is 
		\[\Omega = \frac{1}{2}H^2 + XY + YX.\]
		We have the coordinates on \(\begin{pmatrix}y^{1/2} & x y^{-1/2}\\ & y^{-1/2}\end{pmatrix}SO_2(\R)=\SL_2(\R)\) from \cite{bumpAutomorphicFormsRepresentations1997}[1.19 pg 139] in which 
		 the casimir acts as the differential operator
		\[\Delta = y^2\left(\left(\Dif{}{x}\right)^2 +\left(\Dif{}{y}\right)^2\right) - y\Dif{^2}{x\partial \theta},\] 
		\cite{bumpAutomorphicFormsRepresentations1997}[1.29 pg 143 ,Prop 2.2.5 pg 155]. Now we claim that F is an eigenfunction for this operator. 
		An element \((x,y,\theta) \defeq \begin{pmatrix}y^{1/2} & x y^{-1/2}\\ & y^{-1/2}\end{pmatrix}\kappa_\theta \in \SL_2(\R)\) acts on \(i\) by sending it to \(x+ iy\) (elementary computation). The bottom row of the product is \(y^{-1/2}\sin\theta ;y^{-1/2}\cos\theta \) which results in 
		\[F(x,y,\theta) = y^{k/2}e^{-ik\theta}f(x + iy).\]
		It is then a calculus exercise to apply \(\Delta\) to this, using the holomorphicity we also get that \(f_{xx} - f_{yy} = 0\) and \(f_y = if_x\) which cancels away terms and we get that 
		\[\Delta F(x,y,\theta) = \frac{k}{2}\left(\frac{k}{2} - 1\right) F(x,y,\theta).\]
		
		Therefore the dimension of \(Z(\mathfrak{g})F\) is simply one.
	}
    This example makes it clear that the two finiteness conditions for automorphic forms are in some sense functional equations that they must satisfy. 
	There is a nice explanation of how to lift this to the adelic setting in several places, the key is the isomorphism 
	\[\SL_2(\R) \cong Z(\A)\GL_2(\Q) \backslash \GL_2(\A).\]
	The details are quite clear in \cite[2.1]{cogdellLecturesLfunctionsConverse} or \cite{booherVIEWINGMODULARFORMS}. We will revisit this perspective through the example of the Eisenstein series in section \ref{sec:classic-eisenstein}.
