\label{ch:jiang}
Our goal here is to exposit a small example that forms the heart of the work in papers such as \cite{brennerNotesAnalyticProperties2009}\cite{jiangPolesCertainResidual2013}. 

\cite{brennerNotesAnalyticProperties2009} gave an analysis of the residual poles of Eisenstein series attached to \(\Sp_{2n}\), there were some minor errors that were corrected in \cite{jiangPolesCertainResidual2013} where they give essentially the same proof; theirs however works for the other classical groups. To show the pattern we will focus on the case of \(\Sp_{2n}\), as an algebraic group defined over \(F\) a number field. 

Theirs is a proof by induction and we will try to give the details of the base case, which is simply an explicit computation of a constant term.

\section{Residual Eisenstein Series}
So for the rest of the chapter we will fix an \(n\in \N\) and \(G_n = \Sp_{2n}\), then we look at partitions of \(n = r + m\), where \(1\leq r,m \leq n\) and \(r,m\in \Z\). Then as we saw in section \ref{maximal_parabolic} there corresponds a maximal standard (proper) parabolic of \(\Sp_{2n}\), which we denote \(P_r = M_rU_r\), such that the Levi component is 
\[\GL_r\times \Sp_{2m}. \]
As we saw in section \ref{ex:characters} the space of characters \(X^{\Sp_{2n}}_{M_r}\) is one dimensional by the maximality of \(P_r\). We fix a \(\tau\), an irreducible unitary cuspidal automorphic representation of \(\GL_r\). Now we take an irreducible cuspidal automorphic representation \(\sigma\) of \(\Sp_{2m}\), then the tensor product \(\tau \tensor \sigma\) gives a representation of \(\GL_{r}\times \Sp_{2m}\) and hence of the Levi \(M_r\). We now consider the Eisenstein series attached to this representation, namely if 
\[\phi\in \mathcal{A}(U_r(\A)M_r(F)\setminus \Sp_{2n}(\A) )_{\tau\tensor \sigma},\] 
then we have the Eisenstein series
\[E(\phi, s)(g) = \sum_{\gamma\in P_r(F) \setminus \Sp_{2n}(F)} s.\phi(\gamma g),\]
for \(g\in \Sp_{2n}(F) \setminus \Sp_{2n}(\A)\). 
This is the base case of the setup in \cite{jiangPolesCertainResidual2013}.

\section{The Constant Term}
So far we only know how to do one thing with such Eisenstein series and that is compute their constant term. We will compute the constant term along the maximal parabolic \(P_r = M_rU_r\) because by \cite[II.1.7 (ii)]{moeglinSpectralDecompositionEisenstein1995} the others are zero (see theorem \ref{thm:constant_terms}).

By our earlier calculations in section \ref{const_eisenstein}, the fact that the tensor of cuspidal representations is cuspidal (elementary) and \cite{jiangPolesCertainResidual2013} we know that 
     \[E(\phi, s)_{P_r} = \sum_{ w\in W_{M'}\setminus W_G / W_{M}} \sum_{m'\in M_r(F)\cap wP_r(F)w\inv\setminus M_r(F)} \int_{U_r(F)\cap wP_r(F)w\inv \setminus U_r(\A)} \lambda.\phi( w\inv um' x)  du.\] 
     By \cite{jiangPolesCertainResidual2013} the inner integral vanishes for all \(w\neq id, \omega\) where \(\omega\in W_{\Sp_{2n}}\), this element is computed explicitly in \cite{ginzburgDescentMapAutomorphic2011} 
     and is 
     \[\omega \defeq (-1)^r\begin{pmatrix}
     	&& I_r\\
     	&I&\\
     	\pm I_r&&
     \end{pmatrix},\]
     where \(I_a\) is the \(a\times a\) identity matrix. Note that the \(\pm\) is there to make sure the matrix is in \(\Sp_{2n}\) and will in general depend on \(n\) and \(r\). Hence the first sum becomes over two elements and we have 

     \[E(\phi, s)_{P_r} = E(\phi, s)_{P_r, id} + E(\phi, s)_{P_r, \omega},\]
     where 
     \[E(\phi, s)_{P_r, w}(x) =  \sum_{m'\in M_r(F)\cap wP_r(F)w\inv\setminus M_r(F)} \int_{U_r(F)\cap wP_r(F)w\inv \setminus U_r(\A)} s.\phi( w\inv um' x)  du.\]

First the identity term simplifies

     \begin{equation*}
        \begin{aligned}
            E(\phi, s)_{P_r, id} (x)&=  \sum_{m'\in M_r(F)\cap P_r(F)\setminus M_r(F)} \int_{U_r(F)\cap P_r(F) \setminus U_r(\A)} s.\phi( um' x)  du\\
            &= \sum_{m'\in M_r(F)\setminus M_r(F)} \int_{U_r(F)\setminus U_r(\A)} s.\phi( um' x)  du \\
            &=\int_{U_r(F)\setminus U_r(\A)} s.\phi( u x)  du \\
            &= s.\phi(x)_{P_r}.
        \end{aligned}
     \end{equation*}
     Note that because \(\phi\) was an automorphic form that is \(U(\A)\) invariant we have in particular that 
     \[s.\phi(x)_{P_r} = s.\phi(x).\]

     Considering now the \(\omega\) term 
     \[E(\phi, s)_{P_r, \omega}(x) =  \sum_{m'\in M_r(F)\cap \omega P_r(F)\omega\inv\setminus M_r(F)} \int_{U_r(F)\cap \omega P_r(F)\omega\inv \setminus U_r(\A)} s.\phi( \omega\inv um' x)  du.\]

     By \cite[2C]{jiangPolesCertainResidual2013} \(M_r(F)\cap \omega P_r(F)\omega\inv \setminus M_r(F)\) is isomorphic to \(P_0 \setminus \Sp_{2(n-a)}\), but \(P_0\) has Levi \(M_0 = \Sp_{2(n-a)}\) by definition and hence is itself \(\Sp_{2(n-a)}\). Thus the sum is over \(\Sp_{2(n-a)}(F) \setminus \Sp_{2(n-a)}(F)\) and hence is over a point. Therefore we get by definition of the intertwining operator
     \[E(\phi, s)_{P_r, \omega}(x) = \int_{U_r(F)\cap \omega P_r(F)\omega\inv \setminus U_r(\A)} \phi( \omega\inv u x)  du = M(\omega, s)(\phi)(x),\]
     because we took the constant term along the same parabolic as the definition of the Eisenstein series we know that the Levis are (the same) conjugate.
    Thus we have shown that 
    \[E(\phi, s)_{P_r} = s.\phi + M(\omega, s )(\phi).\]
    
    Because \(\phi\) is an automorphic form it has no poles and so we have shown the following:
    \begin{Lemma}[Base case of \cite{jiangPolesCertainResidual2013}, 2.1]
        The poles of \(E(\phi, s)\) are exactly the poles of \(E(\phi,s)_{P_a}\) (see section \ref{const_lemma_3}) which are exactly the poles of \(M(\omega, s)\).
    \end{Lemma}
    
	The poles of the \(M(w, s)\) function of of great interest due to their relation to L--functions. We leave this to future work.
     