Our goal here is to exposit and survey the work in papers such as \cite{brennerNotesAnalyticProperties2009}\cite{jiangPolesCertainResidual2013}\cite{ginzburgTopFourierCoefficients2021} and perhaps give a trivial extension of them. The idea is to locate the poles and zeroes of certain Eisenstein series. 

\cite{brennerNotesAnalyticProperties2009} gave an analysis of the residual poles of Eisenstein series attached to \(\Sp_{2n}\), there were some minor errors that were corrected in \cite{jiangPolesCertainResidual2013} where they give esssentially the same proof; theirs however works for the other classical groups. For our purposes, the case of \(\Sp_{2n}\), as a group defined over \(F\) a number field, is most relevant, and we shall therefore focus exclusively on this case, however it should be noted that this limitation in the non-covering case is artificial, although it does simplify things a little by removing some casework, and we hope also in the covering case to be able to remove it in future work. 

\section{Residual Eisenstein Series}
So for the rest of the chapter we will fix an \(n\in \N\), \(G_n = \Sp_{2n}\) and the Borel of upper triangular matrices in \(\Sp_{2n} \), then we look at partitions of \(n = r + m\), where \todo{what are the ranges of these}. Then as we saw in \ref{maximal_parabolic} there corresponds a maximal standard parabolic of \(\Sp_{2n}\), which we denote \(P_r = M_rN_r\), such that the Levi component is 
\[\GL_r\times \Sp_{2m} \]
As we saw in \ref{ex:characters} the space of characters \(X^{\Sp_{2n}}_{M_r}\) is one dimensional by the maximality of \(P_r\). If we look at the divisors of \(r = ab\) \todo{check the exact ranges, will depend on the range of r}
 and fix a \(\tau\), an irreducible unitary cuspidal automorphic representation of \(\GL_a\), then from \ref{residual_spec} we know that \(\Delta(\tau, b)\) is a residual representation of \(\GL_{ab} = \GL_r\). Now we take an irreducible generic cuspidal automorphic representation \(\sigma\) of \(\Sp_{2m}\), and so their tensor product \(\Delta(\tau, b) \tensor \sigma\) gives a representation of \(\GL_{r}\times \Sp_{2m}\) and hence of the Levi \(M_r\). We now consider the Eisenstein series attached to this representation, namely if 
\[\phi\in \mathcal{A}(N_r(\A)M_r(F)\setminus \Sp_{2n}(\A) )_{\Delta(\tau, b)\tensor \sigma}\] 
then we have the Eisenstein series
\[E(\phi, s)(g) = \sum_{\gamma\in P_r(F) \setminus \Sp_{2n}(F)} s.\phi(\gamma g)\]
for \(g\in \Sp_{2n}(F) \setminus \Sp_{2n}(\A)\). Becuase it is induced from the residual representation \(\Delta(\tau, b)\) we call these residual Eisenstein series.

This is the setup in \cite{jiangPolesCertainResidual2013}, where they prove their results by a sort of induction on \(b\). Here we will focus exclusively on the base case of this induction, leaving the inductive step for future work. So from now on we will fix \(b = 1\). Hence \(n = a + m\). Then fixing a standard parabolic of \(\Sp_{2n}\) we have the maximal standard parabolic \(P_a = M_aN_a\) where \(M_a = \GL_a \times \Sp_{2m}\).  Now if \(\tau\) is irreducible unitary cuspidal automorphic representation of \(\GL_a\) then by definition \todo{reference Brenner or explain in the previous section...}
\[\Delta(\tau, 1)(\phi)(g) = E(\phi,s)(g) = s.\phi(g)\]
where the Eisenstein series is defined via the parabolic induction from the Levi \((\GL_a)^{\times b} \) to \(\GL_{ab}\). Thus we have \(\Delta(\tau, 1) = \tau\).


\begin{comment}
    \section{Our Results}
We consider an almost identical setup but we deal with the metaplectic cover of \(\Sp_{2n}\), again over a number field \(F\), \(\Mp_{2n}\)\todo[inline]{reference the section I discuss this in.}. We again fix the Borel of upper triangular matricies, consider partitions of \(n = r+m\) and look at maximal standard parabolics of \(\Sp_{2n}\), \(P_r = M_rN_r\) such that 
\[M_r = \GL_r \times \Sp_{2m}\]
then if \(r = ab\) we still have that \(\Delta(\tau, b)\tensor \sigma\), for \(\tau\) irreducible unitary cuspidal automorphic representation of \(\GL_a\) and \(\sigma\) irreducible generic cuspidal automorphic representation of \(\Sp_{2m}\), is a representation of \(M_r\). The difference is in the parabolic induction as we now consider 
\[\phi\in \mathcal{A}(N_r(\A)M_r(F)\setminus \Mp_{2n}(\A) )_{\Delta(\tau, b)\tensor \sigma}\] 
and then the Eisenstein series is defined in the same way
\[E(\phi, s)(g) = \sum_{\gamma\in P_r(F) \setminus \Sp_{2n}(F)} s.\phi(\gamma g)\]
for \(g\in \Sp_{2n}(F) \setminus \Mp_{2n}(\A)\) and \(s\in \C \cong X^{\Mp_{2n}}_{M_r}\).

\begin{Lemma}
When \(b = 1\) we have the constant term
    \[E(\phi,s)(g)_{P_a} = \phi(g)_{P_a} + M(\omega, \tau\tensor\sigma)(\phi)(g)\]
\end{Lemma}

\todo[inline]{fill in here as theorems or whatever anything that I end up actually checking....}

\end{comment}


\section{The Constant Term}
So far we only know how to do one thing with such Eisenstein series and that is compute their constant term. \todo{M+W II.1.7, all others are zero..?} We will compute the constant term along the maximal parabolic \(P_a = MN\). 

By our earlier calculations \ref{const_eisenstein} and the cuspidality of the tensor \ref{cuspidality_tensor} and \cite{jiangPolesCertainResidual2013} we know that 
     \[E(\phi, s)_{P} = \sum_{ w} \sum_{m'} \int_{(w\inv N(\A)w \cap M(\A)) \setminus A} \int_{w\inv N(F) w \cap M(F) \setminus w\inv N(\A)w \cap M(\A)} s.\phi( n_1 n_2 w\inv m' x)  dn_1 dn_2,\] 
     and the inner integral vanishes for all \(w\neq id, \omega\) (\(\omega\) as in \cite{jiangPolesCertainResidual2013}). Hence the first sum becomes over two elements and we have 

     \[E(\phi, s)_{P} = E(\phi, s)_{P, id} + E(\phi, s)_{P, \omega}.\]
     where 
     \[E(\phi, s)_{P, w}(x) =  \sum_{m'\in M(F)\cap wP(F)w\inv\setminus M(F)} \int_{N(F)\cap wP(F)w\inv \setminus N(\A)} s.\phi( w\inv nm' x)  dn.\]

First the identity term simplifies

     \begin{equation*}
        \begin{aligned}
            E(\phi, s)_{P', id} (x)&=  \sum_{m'\in M(F)\cap P(F)\setminus M(F)} \int_{N(F)\cap P(F) \setminus N(\A)} s.\phi( nm' x)  dn\\
            &= \sum_{m'\in M(F)\setminus M(F)} \int_{N(F)\setminus N(\A)} s.\phi( nm' x)  dn \\
            &=\int_{N(F)\setminus N(\A)} s.\phi( n x)  dn \\
            &= s.\phi(x)_P.
        \end{aligned}
     \end{equation*}

     Considering now the \(\omega\) term 
     \[E(\phi, s)_{P, \omega}(x) =  \sum_{m'\in M(F)\cap \omega P(F)\omega\inv\setminus M(F)} \int_{N(F)\cap \omega P(F)\omega\inv \setminus N(\A)} s.\phi( \omega\inv nm' x)  dn.\]

     By \cite[2C]{jiangPolesCertainResidual2013} \(M(F)\cap \omega P(F)\omega\inv \setminus M(F)\) is isomorphic to \(P_0 \setminus \Sp_{2(n-a)}\), but \(P_0\) has Levi \(M_0 = \Sp_{2(n-a)}\) by definition and hence is itself \(\Sp_{2(n-a)}\). Thus the sum is over \(\Sp_{2(n-a)}(F) \setminus \Sp_{2(n-a)}(F)\) and hence is over a point. Therefore we get by definition of the intertwining operator
     \[E(\phi, s)_{P, \omega}(x) = \int_{N(F)\cap \omega P(F)\omega\inv \setminus N(\A)} \phi( \omega\inv n x)  dn = M(\omega, s)(\phi)(x)\]
     because we took the constant term along the same parabolic as the definition of the Eisenstein series we know that the Levis are (the same) conjugate.
    Thus we have shown that 
    \[E(\phi, s)_P = s.\phi_P + M(\omega, s )(\phi)\]
    \todo{cite the relevant parts of jiang and brenner..}
    \begin{remark}
        This work was entirely at the level of the terminals and hence generalises verbatim to covering groups.
    \end{remark}
    
    Because \(\phi\) is an automorphic form it has no poles and so we have shown the following:
    \begin{Lemma}
        The poles of \(E(\phi, s)\) are exactly the poles of \(E(\phi,s)_{P_a}\) which are exactly the poles of \(M(\omega, s)\).
    \end{Lemma}

    \section{Analysing the Intertwining Operator}
    Recall from \ref{L_inter} that we know in this case that there is some ratio of \(L\)-functions \(r(s, \omega)\) such that \(M(s,\omega) = r(s, \omega)N(s, \omega)\) where \(N(s, \omega)\) is both holomorphic and non-zero. 
     In the case we are considering the normalising factor \(r\) is given by the equation \cite[4A]{jiangPolesCertainResidual2013}
     \[r(w, s) = \frac{L(s, \tau\times \sigma)L(2s, \tau,\wedge^2)}{L(s+1, \tau\times \sigma)L(2s+1, \tau, \wedge^2)},\]
    where \(\wedge^2\) denotes the exterior second power of the standard representation of \(\GL_a(\C)\). Thus

     \begin{Lemma}
        The Eisenstein series above has pole at \(s\) if and only if \(r(w,s)\) has a pole.
     \end{Lemma}
     The final step is then to use the known properties of \(L\)-functions to conclude when our \(r\)-factor will have poles and of what order those poles will be.

    --------

    \begin{Theorem}[\cite{jiangPolesCertainResidual2013}, 4.1]
        
    \end{Theorem}

    


    \section{The Metaplectic Generalisation}
    We need to restate the setup now in the metaplectic case\todo{ref}. Using the same notation as above, we now also denote \(\mathbf{M}_a\) the pre-image of \(M_a\) in \(\Mp_{2n}\), in particular if \(\mathrm{pr}: \Mp_{2n} \to \Sp_{2n}\) is the defining projection we have that \(\mathbf{M}_a \defeq \mathrm{pr}\inv(M_a)\). As we remarked \todo{ref} we still have that \(X_{\mathbf{M}_r}^{\Mp_{2n}}\) is one dimensional. Now we want to consider representations of 
    \[\mathbf{M}_a = \mathrm{pr}\inv(\GL_a \times \Sp_{2m}) \cong \big(\mathrm{pr}\inv(\GL_a)\times \mathrm{pr}\inv(\Sp_{2m})\big)/\Delta \mu_2,\]
    where \(\mu_2 = \{\pm 1\}\) acts on the product via the diagonal embedding \todo{This is not clear to me....}.
    So now we let \(\tilde{\tau}\) be a generic, cuspidal, irreducible representation of \(\mathrm{pr}\inv(\GL_a)\) and \(\tilde{\sigma}\) a generic, cuspidal, irreducible representation of \(\mathrm{pr}\inv(\Sp_{2m})\). 

    \begin{remark}
        If neither are \textit{genuine} representations of the covers then they both factor through a representation of the Levi of \(\Sp_{2n}\) and hence we can reduce to the case of the classical group itself (instead of its cover).
    \end{remark}

    Because we may \todo{reference, Gan-Ichino Formal degrees....} we assume that \(\tilde{\tau} \cong \chi\tensor \mathrm{pr}^*(\tau)\) and \(\tilde{\sigma} \cong \chi'\tensor \mathrm{pr}^*(\sigma)\), for some characters \(\chi\) \todo{characters of what??}, where \(\sigma, \tau\) are as in the setup above, irreducible generric cuspidal representations of \(\Sp_{2n}, \GL_{2m}\) respectively.
    Finally we can form the Eisenstein series associate to an automorphic form \(\phi\in \mathcal{A}(N_a(\A)M_a(F)\setminus \Mp_{2n}(\A) )_{\tilde{\tau}\tensor \tilde\sigma}\) defined in the same way as before,
    \[E(\phi, s)(g) = \sum_{\gamma\in P_r(F) \setminus \Sp_{2n}(F)} s.\phi(\gamma g),\]
    for \(g\in \Sp_{2n}(F) \setminus \Mp_{2n}(\A)\) and \(s\in \C \cong X^{\Mp_{2n}}_{M_r}\).

    As we remarked earlier the constant term computation applies immediately to this case as well hence we still have:
    \begin{Lemma}
            The poles of \(E(\phi, s)\) are exactly the poles of \(E(\phi,s)_{P_a}\) which are exactly the poles of \(M(\omega, s)\).
    \end{Lemma}
    This is where the general story ends for the metaplectic groups. The results that we used in the classical group setting have not yet been proven (and are far beyond my scope). 

    \subsection{Conditional Result}
     We follow \cite[Assumption 6.1]{wuThetaCorrespondenceSimple2024wuThetaCorrespondenceSimple2024} for the specific form of this conjecture, although as we mentioned earlier \todo{ref earlier} this has many incarnations. 
     \todo[inline]{need to say where N is holomorphic, right half plane, geq 1/2 or 0...}
     \begin{Conj}\label{conjecture_1}
        There exists a holomorphic and non-zero \(N(s,w)\) such that 
     \[M(s, w) = \frac{L(s, \pi\times \tau^{\vee}) L(2s, \tau, \mathrm{Sym}^2)}{L(s+1, \pi\times \tau^\vee)L(2s+1, \tau, \mathrm{Sym}^2)}N(s,w),\]
     where \(\mathrm{Sym}^2\) is the symmetric second power of the standard representation of \(\GL_a(\C)\).
     \end{Conj}
     \textcolor{red}{This is proven for an intertwining operator associated to maximal parabolics (our case) in \cite[Thm. 7.10]{gaoLanglandsShahidiFunctionsBrylinskiDeligne2018} however their operator is not quite the same, I feel maybe their notation is just off?}

     under the hypothesis of this conjecture we again have the following
     \begin{Lemma}
        Given \ref{conjecture_1} then the Eisenstein series above has pole at \(s\) if and only if 
        \[\frac{L(s, \pi\times \tau^{\vee}) L(2s, \tau, \mathrm{Sym}^2)}{L(s+1, \pi\times \tau^\vee)L(2s+1, \tau, \mathrm{Sym}^2)},\]
        has a pole.
     \end{Lemma}
     by \cite[Thm. 35]{kaplanDoublingConstructionsComplete2021a} for \(Re(s)>0, \quad L(s, \pi\times \tau^{\vee})\) is abosolutely convergent. 


    \subsection{Siegel Parabolic}


    Kaplan in a series of recent works with collaborators \cite{kaplanDoublingConstructionsComplete2021a}\cite{kaplanDoublingConstructionsTensor2020}\cite{caiDoublingConstructionsGlobal2024} has supplied some of the key peices for the normalisation of the metaplectic intertwining operators in the case of the Siegel parabolic. Namely if \(n = r+ m\) and we require \(m=0\), then the parabolic associated to the stadard Levi \(M_n = \GL_n\) is called the Siegel parabolic. In this case we have

    \begin{Theorem}[\cite{kaplanDoublingConstructionsComplete2021a}, 4.2]
    \[M(s, w) = r(s,w)N(s,w),\]
    Where N is a non-zero and holomorphic function and \(r\) is a ratio of L-functions.
    \end{Theorem} 

    In this case the relvant normalisation is \cite[Eq. 1.5]{ginzburgTopFourierCoefficients2021}
    \[r(s,\omega) = \frac{ L(2s, \tau, \mathrm{Sym}^2)}{L(2s+1, \tau, \mathrm{Sym}^2)} .\]

    and so we have
    \begin{Lemma}
        The Eisenstein series,  above has pole at \(s\) if and only if \(r(w,s)\) has a pole.
     \end{Lemma}
     Finally we need to once again see what properties of these L-fuctions have been proven. 

    --------------

     \begin{Theorem}[\cite{ginzburgTopFourierCoefficients2021}, ]
        
     \end{Theorem}


     