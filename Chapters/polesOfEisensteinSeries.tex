\label{ch:jiang}
Our goal here is to exposit and survey the work in papers such as \cite{brennerNotesAnalyticProperties2009}\cite{jiangPolesCertainResidual2013}. The idea is to locate the poles and zeroes of certain Eisenstein series. 

\cite{brennerNotesAnalyticProperties2009} gave an analysis of the residual poles of Eisenstein series attached to \(\Sp_{2n}\), there were some minor errors that were corrected in \cite{jiangPolesCertainResidual2013} where they give essentially the same proof; theirs however works for the other classical groups. To show the pattern we will focus on the case of \(\Sp_{2n}\), as a group defined over \(F\) a number field. Theirs is a proof by induction and we will try to give the details of the base case. 

\section{Residual Eisenstein Series}
So for the rest of the chapter we will fix an \(n\in \N\), \(G_n = \Sp_{2n}\) and the Borel of upper triangular matrices in \(\Sp_{2n} \), then we look at partitions of \(n = r + m\), where \(1\leq r,m \leq n\) and \(r,m\in \Z\). Then as we saw in \ref{maximal_parabolic} there corresponds a maximal standard (proper) parabolic of \(\Sp_{2n}\), which we denote \(P_r = M_rU_r\), such that the Levi component is 
\[\GL_r\times \Sp_{2m}. \]
As we saw in \ref{ex:characters} the space of characters \(X^{\Sp_{2n}}_{M_r}\) is one dimensional by the maximality of \(P_r\). We fix a \(\tau\), an irreducible unitary cuspidal automorphic representation of \(\GL_r\). Now we take an irreducible \textit{generic} cuspidal automorphic representation \(\sigma\) of \(\Sp_{2m}\), then the tensor product \(\tau \tensor \sigma\) gives a representation of \(\GL_{r}\times \Sp_{2m}\) and hence of the Levi \(M_r\). We now consider the Eisenstein series attached to this representation, namely if 
\[\phi\in \mathcal{A}(U_r(\A)M_r(F)\setminus \Sp_{2n}(\A) )_{\tau\tensor \sigma},\] 
then we have the Eisenstein series
\[E(\phi, s)(g) = \sum_{\gamma\in P_r(F) \setminus \Sp_{2n}(F)} s.\phi(\gamma g)\]
for \(g\in \Sp_{2n}(F) \setminus \Sp_{2n}(\A)\). 
This is the base case of the setup in \cite{jiangPolesCertainResidual2013}.


\begin{comment}
    \section{Our Results}
We consider an almost identical setup but we deal with the metaplectic cover of \(\Sp_{2n}\), again over a number field \(F\), \(\Mp_{2n}\)\todo[inline]{reference the section I discuss this in.}. We again fix the Borel of upper triangular matricies, consider partitions of \(n = r+m\) and look at maximal standard parabolics of \(\Sp_{2n}\), \(P_r = M_rN_r\) such that 
\[M_r = \GL_r \times \Sp_{2m}\]
then if \(r = ab\) we still have that \(\Delta(\tau, b)\tensor \sigma\), for \(\tau\) irreducible unitary cuspidal automorphic representation of \(\GL_a\) and \(\sigma\) irreducible generic cuspidal automorphic representation of \(\Sp_{2m}\), is a representation of \(M_r\). The difference is in the parabolic induction as we now consider 
\[\phi\in \mathcal{A}(N_r(\A)M_r(F)\setminus \Mp_{2n}(\A) )_{\Delta(\tau, b)\tensor \sigma}\] 
and then the Eisenstein series is defined in the same way
\[E(\phi, s)(g) = \sum_{\gamma\in P_r(F) \setminus \Sp_{2n}(F)} s.\phi(\gamma g)\]
for \(g\in \Sp_{2n}(F) \setminus \Mp_{2n}(\A)\) and \(s\in \C \cong X^{\Mp_{2n}}_{M_r}\).

\begin{Lemma}
When \(b = 1\) we have the constant term
    \[E(\phi,s)(g)_{P_a} = \phi(g)_{P_a} + M(\omega, \tau\tensor\sigma)(\phi)(g)\]
\end{Lemma}

\todo[inline]{fill in here as theorems or whatever anything that I end up actually checking....}

\end{comment}


\section{The Constant Term}
So far we only know how to do one thing with such Eisenstein series and that is compute their constant term. We will compute the constant term along the maximal parabolic \(P_r = M_rU_r\) becuase by \cite[II.1.7 (ii)]{moeglinSpectralDecompositionEisenstein1995} the others are zero.

By our earlier calculations \ref{const_eisenstein}, the fact that the tensor of cuspidal representations is cuspidal (elementary) and \cite{jiangPolesCertainResidual2013} we know that 
     \[E(\phi, s)_{P_r} = \sum_{ w} \sum_{m'} \int_{U_r(F)\cap wP_r(F)w\inv \setminus U_r(\A)} \lambda.\phi( w\inv um' x)  du.\] 
     By \cite{jiangPolesCertainResidual2013} the inner integral vanishes for all \(w\neq id, \omega\) where \(\omega\in W_{\Sp_{2n}}\), this element is computed explicitly in \cite{ginzburgDescentMapAutomorphic2011} however it is not needed here. Hence the first sum becomes over two elements and we have 

     \[E(\phi, s)_{P_r} = E(\phi, s)_{P_r, id} + E(\phi, s)_{P_r, \omega}.\]
     where 
     \[E(\phi, s)_{P_r, w}(x) =  \sum_{m'\in M_r(F)\cap wP_r(F)w\inv\setminus M_r(F)} \int_{U_r(F)\cap wP_r(F)w\inv \setminus U_r(\A)} s.\phi( w\inv um' x)  du.\]

First the identity term simplifies

     \begin{equation*}
        \begin{aligned}
            E(\phi, s)_{P_r, id} (x)&=  \sum_{m'\in M_r(F)\cap P_r(F)\setminus M_r(F)} \int_{U_r(F)\cap P(F) \setminus U_r(\A)} s.\phi( um' x)  du\\
            &= \sum_{m'\in M_r(F)\setminus M_r(F)} \int_{U_r(F)\setminus U_r(\A)} s.\phi( um' x)  du \\
            &=\int_{U_r(F)\setminus U_r(\A)} s.\phi( u x)  du \\
            &= s.\phi(x)_{P_r}.
        \end{aligned}
     \end{equation*}

     Considering now the \(\omega\) term 
     \[E(\phi, s)_{P_r, \omega}(x) =  \sum_{m'\in M_r(F)\cap \omega P_r(F)\omega\inv\setminus M_r(F)} \int_{U_r(F)\cap \omega P_r(F)\omega\inv \setminus U_r(\A)} s.\phi( \omega\inv um' x)  du.\]

     By \cite[2C]{jiangPolesCertainResidual2013} \(M_r(F)\cap \omega P_r(F)\omega\inv \setminus M_r(F)\) is isomorphic to \(P_0 \setminus \Sp_{2(n-a)}\), but \(P_0\) has Levi \(M_0 = \Sp_{2(n-a)}\) by definition and hence is itself \(\Sp_{2(n-a)}\). Thus the sum is over \(\Sp_{2(n-a)}(F) \setminus \Sp_{2(n-a)}(F)\) and hence is over a point. Therefore we get by definition of the intertwining operator
     \[E(\phi, s)_{P_r, \omega}(x) = \int_{U_r(F)\cap \omega P_r(F)\omega\inv \setminus U_r(\A)} \phi( \omega\inv u x)  du = M(\omega, s)(\phi)(x),\]
     because we took the constant term along the same parabolic as the definition of the Eisenstein series we know that the Levis are (the same) conjugate.
    Thus we have shown that 
    \[E(\phi, s)_{P_r} = s.\phi_{P_r} + M(\omega, s )(\phi).\]
    
    Because \(\phi\) is an automorphic form it has no poles and so we have shown the following:
    \begin{Lemma}[Base case of \cite{jiangPolesCertainResidual2013}, 2.1]
        The poles of \(E(\phi, s)\) are exactly the poles of \(E(\phi,s)_{P_a}\) which are exactly the poles of \(M(\omega, s)\).
    \end{Lemma}
    

    \section{Analysing the Intertwining Operator}
    It is at this point that our understanding becomes quite superficial. We can only quote the recent results, as they are completely out of the scope of this thesis. We have summarised what we know in this direction in Appendix \ref{L-functions}
    
    First in \cite[11.1]{cogdellFUNCTORIALITYCLASSICALGROUPS} it is shown that 
    \[M(w, s) = r(w, s)N(w,s)\]
    where \(N(w,s)\) is an intertwining operator that is holomorphic and non-zero for \(\mathrm{Re}(s)\geq 0\) and \(r(w, s)\) is a ratio of L-functions.
    \begin{remark}
    	It is not shown in complete generality but only for intertwining operators associated to tensor products
    	\[\pi\tensor \pi',\]
    	where \(\pi\) is an irreducible admissible unitary generic representation of \(\GL_n(\A)\) and \(\pi'\) is a generic cuspidal automorphic representation of \(\Sp_{2m}(\A)\). This is the case that we are in however so it can be applied. 
    \end{remark}
    
     In the case we are considering the normalising factor \(r\) is given by the equation \cite[4A]{jiangPolesCertainResidual2013}
     \[r(w, s) = \frac{L(s, \tau\times \sigma)L(2s, \tau,\wedge^2)}{L(s+1, \tau\times \sigma)L(2s+1, \tau, \wedge^2)},\]
    where \(\wedge^2\) denotes the exterior second power of the standard representation of \(\GL_r(\C)\). Thus

     \begin{Lemma}
        The Eisenstein series above has pole at \(s\) for \(\mathrm{Re}(s)\geq 0\) if and only if \(r(w,s)\) has a pole at \(s\).
     \end{Lemma}
     The final step is then to use the known properties of \(L\)-functions to conclude when our \(r\)-factor will have poles and of what order those poles will be. \cite{jiangPolesCertainResidual2013} tells us that \(L(s, \tau\times \sigma)\) and \(L(s, \tau, \wedge^2)\) are both holomorphic except for posible simple poles at \(s=0, 1\) and non-zero for \(\mathrm{Re}(s)\geq 1\). 
     
     The denominator is holomorphic and non-zero for \(\mathrm{Re}(s)>0\).
	 The numerator is holomorphic except possible poles at \(s= 1\) or \(s=1/2\). 
	 
	 Moreover we know that these poles will occur only when they occur in the respective L-functions. If \(s=0\) then the simple poles on the numerator cancel with those on the denominator. This is summarised in the following:

    \begin{Theorem}[\cite{jiangPolesCertainResidual2013}, 4.1]
        Let \(\tau\) be an irreducible unitary cuspidal automorphic representation of \(\GL_r\). Let \(\sigma\) be an irreducible generic cuspidal automrophic representation of \(\Sp_{2m}\). The Eisenstein series \(E(\phi, s)\) is holomorphic for all \(s\in \C\) with \(\mathrm{Re}(s)\geq 0\) except at \(s = \frac{1}{2}\) and \(s= 1\) where it has possible simple poles. Moreover 
       	\begin{itemize}
       		\item It has a simple pole at \(s= \frac{1}{2}\) if and only if \(L(s, \tau, \wedge^2)\) has a pole at \(s=1\) and \(L(\frac{1}{2}, \tau\times \sigma)\neq 0\)
       		\item It has a simple pole at \(s=1\) if and only if \(L(s, \tau\time \sigma)\) has a pole at \(s=1\).
       	\end{itemize}
    \end{Theorem}
     