Our goal here is to exposit and survey the work in papers such as \cite{brennerNotesAnalyticProperties2009}\cite{jiangPolesCertainResidual2013}, we were going to generalise some of these results to metaplectic covers however we found these results in \cite{ginzburgTopFourierCoefficients2021}.

\section{Their Results}
\cite{brennerNotesAnalyticProperties2009} gave an analysis of the residual poles of Eisenstein series attached to \(\Sp_{2n}\), there were some minor errors that were corrected in \cite{jiangPolesCertainResidual2013} where they give esssentially the same proof; theirs however works for the other classical groups. For our purposes, the case of \(\Sp_{2n}\), as a group defined over \(F\) a number field, is most relevant, and we shall therefore focus exclusively on this case, however it should be noted that this limitation in the non-covering case is artificial, although it does simplify things a little by removing some casework, and we hope also in the covering case to be able to remove it in future work. 

We fix an \(n\in \N\) and the Borel of upper triangular matricies in \(\Sp_{2n} \), then we look at partitions of \(n = r + m\), where \todo[inline]{what are the ranges of these}. Then as we saw in \todo[inline]{write this up in the classical group section} ---- there corresponds a maximal standard parabolic of \(\Sp_{2n}\), which we denote \(P_r = M_rN_r\), such that the Levi component is 
\[\GL_r\times \Sp_{2m} \]
As we saw in \todo[inline]{explain this in the Eisenstein series section}---- the space of characters \(X^{\Sp_{2n}}_{M_r}\) is one dimensional by the maximality of \(P_r\). If we look at the divisors of \(r = ab\) \todo[inline]{check the exact ranges, will depend on the range of r} and fix a \(\tau\), an irreducible unitary cuspidal automorphic representation of \(\GL_a\), then from \todo[inline]{link to the residual representation section} ---- we know that \(\Delta(\tau, b)\) is a residual representation of \(\GL_{ab} = \GL_r\). Now we take an irreducible generic cuspidal automorphic representation \(\sigma\) of \(\Sp_{2m}\), and so their tensor product \(\Delta(\tau, b) \tensor \sigma\) gives a representation of \(\GL_{r}\times \Sp_{2m}\) and hence of the Levi \(M_r\). We now consider the Eisenstein series attached to this representation, namely if 
\[\phi\in \mathcal{A}(N_r(\A)M_r(F)\setminus \Sp_{2n}(\A) )_{\Delta(\tau, b)\tensor \sigma}\] 
then we have the Eisenstein series
\[E(\phi, s)(g) = \sum_{\gamma\in P_r(F) \setminus \Sp_{2n}(F)} s.\phi(\gamma g)\]
\todo[inline]{explain the s action in Eisenstein series section.}
for \(g\in \Sp_{2n}(F) \setminus \Sp_{2n}(\A)\). Becuase it is induced from the residual representation \(\Delta(\tau, b)\) we call these residual Eisenstein series. The main theorem is then a statement that after some normalisation the poles of this Eisenstein series are limited to a particular set and that they are simple. Details will be given later.


\section{Our Results}
We consider an almost identical setup but we deal with the metaplectic cover of \(\Sp_{2n}\), again over a number field \(F\), \(\Mp_{2n}\)\todo[inline]{reference the section I discuss this in.}. We again fix the Borel of upper triangular matricies, consider partitions of \(n = r+m\) and look at maximal standard parabolics of \(\Sp_{2n}\), \(P_r = M_rN_r\) such that 
\[M_r = \GL_r \times \Sp_{2m}\]
then if \(r = ab\) we still have that \(\Delta(\tau, b)\tensor \sigma\), for \(\tau\) irreducible unitary cuspidal automorphic representation of \(\GL_a\) and \(\sigma\) irreducible generic cuspidal automorphic representation of \(\Sp_{2m}\), is a representation of \(M_r\). The difference is in the parabolic induction as we now consider 
\[\phi\in \mathcal{A}(N_r(\A)M_r(F)\setminus \Mp_{2n}(\A) )_{\Delta(\tau, b)\tensor \sigma}\] 
and then the Eisenstein series is defined in the same way
\[E(\phi, s)(g) = \sum_{\gamma\in P_r(F) \setminus \Sp_{2n}(F)} s.\phi(\gamma g)\]
for \(g\in \Sp_{2n}(F) \setminus \Mp_{2n}(\A)\) and \(s\in \C \cong X^{\Mp_{2n}}_{M_r}\).

\begin{Lemma}
When \(b = 1\) we have the constant term
    \[E(\phi,s)(g)_{P_a} = \phi(g)_{P_a} + M(\omega, \tau\tensor\sigma)(\phi)(g)\]
\end{Lemma}

\todo[inline]{fill in here as theorems or whatever anything that I end up actually checking....}

\section{Setup}
In the setup we used that \(s\in \C \cong X^{\Mp_{2n}}_{M_r}\) the first step is to make sure that this is actually true
\begin{Lemma}
        \( X_{M_r}^{\Mp_{2n}(\A)}\) is at most a one dimensional \C vector space. 
    \end{Lemma}
    \proofbar{
         First of all we have that \cite[I.1.4]{moeglinSpectralDecompositionEisenstein1995}
         \[ X_{M_r}^{\Mp_{2n}(\A)} \subseteq X_{M_r} \cong \mathfrak{a}_{M_r}^*\defeq Rat(M_r) \tensor_\Z \C\]
        thus it is clearly sufficent to bound the dimension of \(\mathfrak{a}_{M_r}^*\) as a \C vector space, moreover this dimension agrees with the dimension of \(Rat(M_r)\) as a free \Z module. 

        Thus we compute \(\dim_\Z(Rat(M_r))\):
        \begin{equation*}
            \begin{aligned}
                Rat(M_r) &= Rat(\GL_r \times \Sp_{2m}) \\
                         &= \Hom(\GL_r \times \Sp_{2m}, \mathbb{G}_m) \\
                         (2)&\cong \Hom(\mathrm{Ab}(\GL_r \times \Sp_{2m}), \mathbb{G}_m) \\
                         (1)&\cong \Hom(\mathrm{Ab}(\GL_r) \times \mathrm{Ab}(\Sp_{2m}), \mathbb{G}_m) \\
                         (3)&\cong \Hom(\mathbb{G}_m \times 1, \mathbb{G}_m) \\
                         &\cong \Z
            \end{aligned}
        \end{equation*}
        in (2) we have used the universal property of the abelianization \(\mathrm{Ab}(G) = \mathcal{D}(G) \setminus G = [G, G] \setminus G \) because \(\mathbb{G}_m\) is abelian. (1) is that the abelianization commutes with direct products (citation as comment in Tex). (3) is because \(\Sp\) is a perfect group.
        %https://groupprops.subwiki.org/wiki/Symplectic_group_is_perfect
        %https://mathoverflow.net/questions/35713/abelianization-of-a-semidirect-product
}\todo[inline]{I havent shown that it is not trivial...}


\section{Lemma 1}
\todo[inline]{The representation is supposed to be of the covering of the Levi///? need to fix this}
We here consider the case that \(b=1\), hence \(n = a + m\). Then fixing a standard parabolic of \(\Sp_{2n}\) we have the maximal standard parabolic \(P_a = M_aN_a\) where \(M_a = \GL_a \times \Sp_{2m}\).  Now if \(\tau\) is irreducible unitary cuspidal automorphic representation of \(\GL_a\) then by definition \todo[inline]{Brenner..}
\[\Delta(\tau, 1)(\phi)(g) = E(\phi,s)(g) = s.\phi(g)\]
where the Eisenstein series is defined via the parabolic induction from the Levi \((\GL_a)^{\times b} \) to \(\GL_{ab}\). Thus we have \(\Delta(\tau, 1) = \tau\). So for the appropriate \(\sigma \) a rep of \(\Sp_{2m}\) we get a rep of the Levi of \(\Sp_{2n}\), \(M_r = M_a = \GL_a\times \Sp_{2m}\) given by \(\tau\tensor \sigma\). To this we associate the Eisenstein series for \(\phi\in \mathcal{A}(N_r(\A)M_r(F)\setminus \Mp_{2n}(\A))_{\tau\tensor \sigma}\) \(E(\phi,s)\) as usual. Now we proceed to calculate the constant term of this Eisenstein series along the parabolic \(P_a = MN\). \todo[inline]{M+W II.1.7, all others are zero..?}

By our calculations\todo[inline]{red}, the cuspidality of the tensor\todo[inline]{ref} and \cite{jiangPolesCertainResidual2013} we know that 
     \[E(\phi, s)_{P} = \sum_{ w} \sum_{m'} \int_{(w\inv N(\A)w \cap M(\A)) \setminus A} \int_{w\inv N(F) w \cap M(F) \setminus w\inv N(\A)w \cap M(\A)} \phi( n_1 n_2 w\inv m' x)  dn_1 dn_2\] 
     and the inner integral vanishes for all \(w\neq id, \omega\) (\(\omega\) as in \cite{jiangPolesCertainResidual2013}). Hence the first sum becomes over two elements and we have 

     \[E(\phi, s)_{P} = E(\phi, s)_{P, id} + E(\phi, s)_{P, \omega}\]
     where 
     \[E(\phi, s)_{P, w} =  \sum_{m'\in M(F)\cap wP(F)w\inv\setminus M(F)} \int_{N(F)\cap wP(F)w\inv \setminus N(\A)} \phi( w\inv nm' x)  dn\]

First the identity term simplifies

     \begin{equation*}
        \begin{aligned}
            E(\phi, s)(x)_{P', id} &=  \sum_{m'\in M(F)\cap P(F)\setminus M(F)} \int_{N(F)\cap P(F) \setminus N(\A)} \phi( nm' x)  dn\\
            &= \sum_{m'\in M(F)\setminus M(F)} \int_{N(F)\setminus N(\A)} \phi( nm' x)  dn \\
            &=\int_{N(F)\setminus N(\A)} \phi( n x)  dn \\
            &= \phi(x)_P
        \end{aligned}
     \end{equation*}
     \todo[inline]{I really need to fix this s thing that I dropped in the constant term computations.}

     Considering now the \(\omega\) term 
     \[E(\phi, s)_{P, \omega} =  \sum_{m'\in M(F)\cap \omega P(F)\omega\inv\setminus M(F)} \int_{N(F)\cap \omega P(F)\omega\inv \setminus N(\A)} \phi( \omega\inv nm' x)  dn\]

     by \cite[2C]{jiangPolesCertainResidual2013} \(M(F)\cap \omega P(F)\omega\inv \setminus M(F)\) is isomorphic to\todo[inline]{this is not clear in their paper im just guessing this is what they mean} \(P_0 \setminus \Sp_{2(n-a)}\) where \(P_0 \setminus \Sp_{2(n-a)}\) by defintion, but \(P_0\) has Levi \(M_0 = \Sp_{2(n-a)}\) by definition and hence is itself \(\Sp_{2(n-a)}\). Thus the sum is over \(\Sp_{2(n-a)}(F) \setminus \Sp_{2(n-a)}(F)\) and hence is over a point. Therefore we get by definintion of the intertwining operator
     \[E(\phi, s)_{P, \omega} = \int_{N(F)\cap \omega P(F)\omega\inv \setminus N(\A)} \phi( \omega\inv n x)  dn = M(\omega, -)(\phi)(x)\]
     becuase we took the constant term along the same parabolic as the definition of the Eisenstein series we know that the Levis are (the same) conjugate.
    Thus we have shown that 
    \[E(\phi, s)_P(x) = \phi(x)_P + M(\omega, - )(\phi)(x)\]

    Notice that the computation takes place completely at the level of the terminals which are indupendent of the fact that we have taken a covering group, hence we have really only reused work from \cite{jiangPolesCertainResidual2013}.
